\documentclass{amsart}

\usepackage{amssymb}
\usepackage{amsmath}
\usepackage{amscd}
\usepackage{eucal}
\usepackage{amsthm}
\usepackage[all]{xy}
\usepackage{pstricks}
\newcommand{\bs}{\boldsymbol}
\newcommand{\mb}{\mathbf}
\renewcommand{\dot}{\centerdot}
\newcommand{\D}{\textrm{-}}
%\pdfshift

\begin{document}


\begin{center}
\begin{Large}
\textbf{Action operads comments to fix}
\end{Large}
\end{center}
\vskip1cm

\section{ Introduction}
\begin{itemize}
\item

\end{itemize}
\section{ Action operads}
\begin{itemize}
\item I put in G0abel (2.3.8) to prove, and we should put in a proof that all the $\pi$'s are surjective or trivial (2.3.4)
% \item Defn 2.3.9 - specify what the $h$'s and $g$'s are.

\end{itemize}

\section{Operads in the category of categories}
\begin{itemize}
\item Start of 3.1: Should the $2$-monad should have $EP(n)$? Compare with 3.3.9.
\item Defn 3.3.7 of cocomplete symmetric monoidal cat
\item After 3.3.9 seems repetitive (essentially description of $\underline{P}$)
\item Prop 3.3.16 The proofs need filling out
\item Should we change $E\Lambda(n) \times X^n/\Lambda(n)$ to be $\left(E\Lambda(n) \times X^n\right)/\Lambda(n)$?
\end{itemize}

\section{ Monoidal structures and multicategories}

\begin{itemize}
\item Intro
\item Use \begin{verbatim} \lmc \end{verbatim} for lambda monoidal categories
% \item `By standard methods' - reference for adjoint equivalence (Mac Lane Chapter 4 Section 4 Theorem 1)
% \item Invertible unit: follows from adjoint equivalence
\item Theorem 4.2.11: `Define $\beta$ by' should have $s_{p_1+r_1,q_1+r_1}$ on the RHS, not $s_{p_1,q_1}$?
% \item `Containment relation': Reference back to wherever this is stated.
% \item Defn 4.3.1: $\pi$ is surjective or each $\pi_n$ is surjective?
\item Lemma 4.3.2: Needs rewording. Is the \textit{underlying set of the free monoid}?
\item Prop 4.3.3: Is $im(\pi)$ defined? What is the underlying permutation operad? Does this mean the symmetrized operad?
% \item Odd sentence before definition of \text{spatial}
% \item In \textit{spatial} defn: use a diagram?
% \item Should there be a string diagram here to demonstrate spatialness
% \item ...on the spatiality of algebra \textit{and so $\Lambda$-monoidal categories}.
\item an $E\Lambda$ or a $\Lambda$-monoidal
\item Lemma 4.3.5: Spacing of equations needs fixing.
\item $\Lambda(2)$ not $G(2)$
\item What is an action morphism?
\item Include the extra steps in the first equations and finish with a period.
\item Get rid of \textit{we're}
\item Make sure sentences around here are finished with a period.
\item Remove $\cdot$ and $\circ$ where not used elsewhere
\item Rewrite equations at the end of the proof (spacing and add some words)
\item Change the word `finally'
\item Change `a bit of new terminology'
\item Do we want another notation to emphasise the underlying monoid?
\item `Then we will also \textit{use}'
\item Lemma 4.3.8: Should be a $\Lambda(n)$, not just $\Lambda$.
\item $\Lambda$ not $G$
\item Spacing of $\alpha$ and extra couple of steps?
\item Defn 4.4.3: Check spacing of the strength maps with subscripts
\item Remark 4.4.4: It's mostly described but not directly shown about the strength axioms?
\item Change sentence to be \textit{further} notation
\item Theorem 4.4.5: overfull hbox on second page of proof
\item Spacing of labels on arrows needs looking at
\item Corollary 4.4.6: Still don't like the terminology `non-symmetric'. Is `plain' operad better?
\item Defn 4.5.1: Odd mix of $\alpha$ and $g$. Think something is mixed up here. ($- \cdot \alpha$)
\end{itemize}



\section{ Invertible objects}

\begin{itemize}
\item The notation in the very first sentence needs to be explained somewhere!
\item Rewrite intro: Need to explain that the goal is to understand some group actions
\item Decide on ELambda algebras or Lambda monoidal categories throughout (we decided the second!)
\item New notation: added earlier (line 905, search beta\_to\_oplus), just need to implement, search for action maps or superscript tensors
\item Fix weakly invertible section
\end{itemize}

Leftover fixes that I'm not sure about:
\begin{itemize}
\item Move comment (QQQ)
\item Fix paragraph; make clear we are determining composition
\item Explain M strategy, include forward refs
\end{itemize}



\section{ Invertibility and group actions}

\begin{itemize}
\item I want to write $\Lambda^{\oplus}$ for the underlying monoid maybe??
\item \textbf{why? This one involves real math}
\item not happy with last section
\end{itemize}



\section{ Computing automorphisms of the unit}

\begin{itemize}




\item 4.1.3 check 2.3.10: need to make sure this is in an earlier section, and ref'ed


\item explain purpose
\item improve proof 4.2.3

\item check commutative Square

\item redo 4.4
\item insert diagram

\item consistent text after 4.5.3
\item move something to earlier


\item highlight that star means the inverse under tensor product for morphisms


\item insert the proof from Ed's email
\item put a short proof
\item check the note
\item change express to describe
\item isomorphism symbol
\item change make sure to ensures
\item remove calculation
\item change we want to do
\end{itemize}



\section{a full description of $L_n $}

\begin{itemize}
\item bad line break
\item remove exposition
\item fix fancy G
\item change G to lambda
\item isomorphism symbol
\item tensor product given component wise
\item check reference
\item rewrite calculation
\item check universal property
\item insert for a simple example
\end{itemize}



\section{ Examples}

\begin{itemize}
\item Actually read this section, fix anything
\end{itemize}
\newpage

\begin{center}
\begin{Large}
\textbf{Comments addressed}
\end{Large}
\end{center}
\vskip1cm


\section{ Invertible objects}
\begin{itemize}
\item Include notation for $\eta$ as the unit here
\item Change to equalizers
\item Change to $(LX)_{inv} = LX$
\item Fix ()s
\item Include triangle NO
\item Uniform gp superscripts
\item Remove actually
\item Ref $\eta$
\item Replace with is, remove parts
\item Remove proof
\item Fix ab superscripts, same as gp
\item qi
\item Under red line: move? make remark? delete some?
\item Where do we say this?
\item Need 2-adjunction: this should follow from Thm 8.6 in the enriched\_sketches paper I saved
\item include forward ref to where we use cref{epi}: I can't find it
\item Get better Eckmann-Hilton ref: don't care anymore
\end{itemize}
\section{ Invertibility and group actions}
\begin{itemize}
\item Forward ref
\item definition env
\item little wording fixes
\item change G to Lambda
\item S vs Sigma for symmetric groups: I picked Sigma
\item Think about free monoid lem again
\item Fix triangle
\item lots of notation issues (e, G, length bars)
\item why splitting
\item missing ref?
\item splits by construction: hmm
\item ref?
\item for v, v' not delta of something
\item inverses for morphisms under comp vs tensor
\item more G's (x2)
\item another missing ref
\item another G
\item include corollary? 
\item forward refs
\item practical?
\end{itemize}

\section{ Computing automorphisms of the unit}
\begin{itemize}
\item in the next two results
\item 4.1.2 two boxes
\item the above following square
\item insert =
\item check 4n or 2n (it is correct in 7.2.1)
\item mentioned Delta, I
\item fixed proof 4.3.2
\item remove functor


\item isomorphism symbol
\item clarify this
\item make sure length and size notation is introduced earlier
\item bad line break at the beginning of 4.5
\item change prove to shows
\item bad line break
\end{itemize}

\section{a full description of $L_n $}
\begin{itemize}
\item 
\end{itemize}
\section{ Examples}
\begin{itemize}
\item 
\end{itemize}


\end{document} 