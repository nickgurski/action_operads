\documentclass[a4paper,12pt,times,numbered,print,index]{Style/PhDThesisPSnPDF}
%\setcounter{tocdepth}{3}
%\setcounter{secnumdepth}{3}

\usepackage{amssymb}
\usepackage{amsthm}
\usepackage{graphicx}
\usepackage{eucal}
\usepackage{tikz-cd}
\usetikzlibrary{decorations.markings}
\usetikzlibrary{shapes,snakes}
\usepackage{pdfsync}
\usepackage{relsize}
\usepackage{afterpage}
\usepackage{comment}
\usepackage[capitalise]{cleveref}
\usepackage[T1]{fontenc}
\usepackage{lmodern}
\usepackage{array}
\usepackage{longtable}
\usepackage{enumitem}

\usepackage[color=orange!80,bordercolor=black,textwidth=3cm,textsize=small,colorinlistoftodos]{todonotes}
\makeatletter \providecommand\@dotsep{5}
\makeatother
\newcommand{\amsartlistoftodos}{\makeatother \listoftodos\relax}

\newenvironment{eq}{\begin{equation}}{\end{equation}}
\newenvironment{eq*}{\begin{equation*}}{\end{equation*}}

\numberwithin{equation}{section}

\newtheorem{thm}{Theorem}[chapter]  
\newtheorem{prop}[thm]{Proposition}
\newtheorem*{prop*}{Proposition}
\newtheorem{lem}[thm]{Lemma}
\newtheorem{cor}[thm]{Corollary}

\newtheoremstyle{example}{\topsep}{\topsep}{}{}{\bfseries}{.}{2pt}{\thmname{#1}\thmnumber{ #2}\thmnote{ #3}}
\theoremstyle{example}

\newtheorem{nota}[thm]{Notation} 
\newtheorem{example}[thm]{Example}
\newtheorem{defn}[thm]{Definition}
\newtheorem*{rem}{Remark}
 
\newtheoremstyle{named}{\topsep}{\topsep}{}{}{\bfseries}{.}{\newline}{\thmname{#1}\thmnumber{ #2}\thmnote{ #3}}
\theoremstyle{named}
\newtheorem{namedprop}[thm]{Proposition}
\newtheorem{namedexample}[thm]{Example}

%% E.P. notes
\newcommand{\epnote}[1]{\todo[color=blue!40,linecolor=blue!40!black,size=\tiny]{#1}}
\newcommand{\epmpar}[1]{\todo[noline,color=blue!40,linecolor=blue!40!black,size=\tiny]{#1}}
\newcommand{\epnoteil}[1]{\todo[inline,color=blue!40,linecolor=blue!40!black,size=\normalsize]{#1}}

%% N.G. notes
\newcommand{\ngnote}[1]{\todo[color=red!40,linecolor=red!40!black,size=\tiny]{#1}}
\newcommand{\ngmpar}[1]{\todo[noline,color=red!40,linecolor=red!40!black,size=\tiny]{#1}}
\newcommand{\ngnoteil}[1]{\todo[inline,color=red!40,linecolor=red!40!black,size=\normalsize]{#1}}

% Cleveref definitions  
\crefname{prop}{Proposition}{Propositions}
\crefname{thm}{Theorem}{Theorems} 
\crefname{defn}{Definition}{Definitions}
\crefname{notn}{Notation}{Notations}
\crefname{construction}{Construction}{Constructions}
\crefname{lem}{Lemma}{Lemmas}
\crefname{rem}{Remark}{Remarks}
\crefname{cor}{Corollary}{Corollaries}
\crefname{scholium}{Scholium}{Scholia}
\crefname{figure}{Figure}{Figures}
\crefname{equation}{Display}{Displays}
\crefname{eq}{Display}{Displays}
\crefname{eqn}{Display}{Displays}

\newcommand{\quotient}[2]{ \raisebox{0.5\height}{$#1$} \mkern-5mu\diagup\mkern-4mu \raisebox{-0.5\height}{$#2$} }
\newcommand{\bigquotient}[2]{ \raisebox{0.75\height}{$#1$} \mkern-12mu\scalebox{2}{$\diagup$}\mkern-10mu \raisebox{-0.5\height}{$#2$} }
\newcommand{\pullback}{\mathbin{\text{\rotatebox{45}{$\mathlarger{\mathlarger{\mathlarger{\mathlarger{\llcorner}}}}$}}}}
\newcommand{\pushout}{\mathbin{\text{\rotatebox{225}{$\mathlarger{\mathlarger{\mathlarger{\mathlarger{\llcorner}}}}$}}}}

\newcommand{\downsim}{\rotatebox{90}{$\sim$}}

\newcolumntype{L}{>{$}l<{$}}
\newcolumntype{C}{>{$}c<{$}}
\newcolumntype{R}{>{$}r<{$}}

\tikzset{->-/.style={decoration={markings,mark=at position #1 with {\arrow{>}}},postaction={decorate}}} 

\pgfdeclarelayer{bg}
\pgfsetlayers{bg,main}

%\excludecomment{proof}  


\title{Action operads and the free $G$-monoidal category on $n$ invertible objects}
%\subtitle{}
\author{Edward G. Prior}
\dept{School of Mathematics and Statistics}
\university{University of Sheffield}
%\crest{\includegraphics{Style/UnivShield}}
 



\begin{document} 
 

\frontmatter

\maketitle  

%\begin{dedication} 
%\end{dedication}

\begin{acknowledgements}    
I would like to thank everyone who I met during my four years who helped me to become a mathematician. I especially want to thank Alex and Rhiannon, who made going through a PhD together so enjoyable. And most of all I would like to thank Nick, for putting up with me through everything, and being such a wonderful supervisor. 
\end{acknowledgements}

\begin{abstract}
We use the theory of action operads and their algebras to study a class of associated monoidal categories, particularly those that are freely generated by some number of invertible objects. We first provide some results about $\mathbb{G}_n$, which is known to be both the free $\mathrm{E}G$-algebra and the free $G$-monoidal category over $n$ objects, for a given action operad $G$. Then we deduce the existence of $L\mathbb{G}_n$, the free algebra on $n$ invertible objects, and show that its objects and connected components arise as a group completion of the data of $\mathbb{G}_n$. In order to determine the rest of $L\mathbb{G}_n$, we will prove that this algebra is the target of a surjective coequaliser $q$ of monoidal categories; that collapsing the tensor product and composition into a single operation forms one half of an adjunction $\mathrm{M}( \, \_ \, )^{\mathrm{ab}} \dashv \mathrm{B}$; that its action operad $G$ embeds into its group completion; and that its morphisms are a semidirect product $(s \times t)(L\mathbb{G}_n) \ltimes L\mathbb{G}_n(I,I)$ of a chosen subgroup by the unit endomorphisms. With these and other assorted results, we will compile a method for constructing $L\mathbb{G}_n$ for most action operads, and from this produce descriptions of the free symmetric, braided, and ribbon braided monoidal categories on invertible objects.
\end{abstract}

\tableofcontents

\chapter{Introduction}
 
The central goal of this paper is to determine how one can construct free monoidal categories over invertible objects, for as many different kinds of monoidal category as possible. This will be achieved by framing the problem in terms of the theory of action operads, and then gradually exploring the features possessed by their algebras.

The motivation for this topic came from earlier work by the author which attempted to produce a classification theorem for 3-groups. In general, $n$-groups are a higher dimensional categorification of the standard notion of a group. While a group can be seen as a monoid in which all elements are invertible, a 2-group is a monoidal category in which all objects and morphisms are invertible in the appropriate sense, a 3-group is a monoidal 2-category with all data invertible, and so on. Much has already been written on the subject of 2-groups \cite{hda5}, including a theorem which classifies them completely in terms of group cohomology. The original intention of the author --- which will hopefully still form the basis of a future paper --- was to generalise this classification theorem to work for 3-groups, by taking each step in the proof and replacing it with a version using concepts from one dimension up. In particular, to replace the sections that involved group cohomology it would be necessary to develop a theory of braided 2-group cohomology. A cohomology of \emph{symmetric} 2-groups already exists \cite{picard}, but proving that it is well-defined involves exploiting certain facts about symmetric monoidal categories, ones that do not immediately transfer to the braided case.   

Thus the key to resolving the whole issue is to understand the behaviour of braided monoidal categories whose objects are all invertible. Indeed, it would suffice to know how to construct the \emph{free} braided monoidal category on $n$ invertible objects, for any value of $n \in \mathbb{N}$, but this in turn is fairly tricky. Over the course of the following chapters we shall see how to accomplish this task, as well as how to find the analogous free entity for a large class of similar structures, what we will call the $G$-monoidal categories. These include the familiar symmetric monoidal categories, but also more unusual cases, such as ribbon braided monoidal categories. 

First, we shall spend most of \cref{actionoperad} covering definitions and results from the existing literature which will be relevant for reaching our objective. After beginning with a quick review of the concepts of monoidal categories and operads, we will introduce the main objects of study for this paper, the so-called `action operads'. First appearing with extra restrictions as `categorical operads' in the thesis of Nathalie Wahl \cite{ribbon1}, before being studied later in full generality by Alex Corner and Nick Gurski \cite{ogge}, action operads are kind of operad which generalise the notion of a group action upon a set. We will see how many common examples of operads-with-extra-structure --- including the founding example of operad theory, the symmetric operad \cite{gils} --- can be united into a single framework by viewing them as $G$-operads, ones that are acted on by some suitable action operad $G$. The translation operad $\mathrm{E}G$ will also be introduced at this point, as a way to categorify certain aspects of a given action operad $G$. Following on from the discussion of $G$-operads will be a look into what the appropriate algebras for these operads should be. In particular, we will see how they differ slightly from the more typical definition of an operad algebra, due to an additional equivariance condition. During this we will see that a certain monoidal structure, present in all action operads $G$, will be inherited by the algebras of both $G$ and $\mathrm{E}G$. Then at last all of the work in this chapter will come to a head in \cref{Gmonthm}, a result of Gurski \cite{operadborel}, where we learn that algebras of the $G$-operad $\mathrm{E}G$ are equivalent to kind of monoidal category, one equipped with extra permutative structure dictated by the nature of the action operad $G$. These are the $G$-monoidal categories, and thus by framing our questions about free braided monoidal categories in the language of action operad algebras, we will be able to produce results which are applicable to a much wider range of situations. 

Next, \cref{initialalgebra} will begin our investigation into the free $\mathrm{E}G$-algebras. We will open with a look at $\mathbb{G}_n$, the free algebra on some number $n \in \mathbb{N}$ of not-necessarily invertible objects. After providing a description for $\mathbb{G}_n$, we will also be able to surmise the existence of free $\mathrm{E}G$-algebra on $n$ invertible objects, denoted $L\mathbb{G}_n$, through the use of some monad theory. Then we shall see how this $L\mathbb{G}_n$ can be viewed as the initial object in a certain comma category of algebras, when paired with the obvious map between free algebras $\eta: \mathbb{G}_n \to L\mathbb{G}_n$. From this initial algebra perspective it will be possible for us to extract several important pieces of information about the structure of $L\mathbb{G}_n$, using a technique where we exploit the properties of adjoint functors. First, by showing that the previously mentioned translation functor $\mathrm{E}$ forms an adjunction with the object monoid functor $\mathrm{Ob}$, we will demonstrate that the objects of $L\mathbb{G}_n$ are the group completion of the objects of $\mathbb{G}_n$. Likewise, forming an adjunction between discrete category functor $\mathrm{D}$ and the connected component functor $\pi_0$ will let us prove that the components of $L\mathbb{G}_n$ are the group completion of $\pi_0(\mathbb{G}_n)$. However, a way of using this method to find the morphisms of $L\mathbb{G}_n$ will remain elusive. The closest we can get is by showing that the delooping functor $\mathrm{B}$ is right adjoint to a certain functor $\mathrm{M}( \, \_ \,)^{\mathrm{ab}} : \mathrm{MonCat} \to \mathrm{CMon}$, which describes what we will call the `collapsed morphisms' of a given monoidal category. In order to salvage this approach, we must therefore try to translate the defining property of $L\mathbb{G}_n$ into one that works solely within the category $\mathrm{MonCat}$, and then also prove that both the algebra structure and the true morphisms of a given $\mathrm{E}G$-algebra can be recovered from these new collapsed morphisms. This task will form the majority of the remaining three chapters. 

\cref{colimalgebra} will bring a couple of new ways for us to think about the algebra $L\mathbb{G}_n$. Instead of viewing it as part of an initial object like in \cref{initialalgebra}, we will instead show that it forms the target of a coequaliser map $q: \mathbb{G}_{2n} \to L\mathbb{G}_n$, whose source now has twice as many generating objects as before. The simplest way to do this involves exhibiting $q$ as the cokernel of an algebra map $\delta: \mathbb{G}_{2n} \to \mathbb{G}_{2n}$, which is designed in such a way that the additional $n$ generators of $\mathbb{G}_{2n}$ will get sent by $q$ onto the inverses of the $n$ generators of $L\mathbb{G}_n$. Through this new perspective we will learn several important facts about the action $\alpha$ of $L\mathbb{G}_n$, including how we will eventually be able to reconstruct it from $L\mathbb{G}_n$'s monoid of morphisms, once we finally understand them. This insight will then indicate how we can subtly change the coequaliser diagram for $q$, so that the preservation of the $\mathrm{E}G$-action is now a consequence of the way that we have built it, rather than just an automatic feature of $q$ being an algebra map. In other words, we will have demonstrated that the underlying monoidal functor of $q$ is also a coequaliser, and thus have found a property which marks the free algebra $L\mathbb{G}_n$ as special within the world of monoidal categories. This is exactly what we need in order to leverage the left adjoint status of the functor $\mathrm{M}( \, \_ \,)^{\mathrm{ab}}$, since it lives over the category $\mathrm{MonCat}$ and commutes with all colimits, like coequalisers. With a little work, our approach will then yield a description of the abelian group of collapsed morphisms $\mathrm{M}(L\mathbb{G}_n)^{\mathrm{gp},\mathrm{ab}}$ as a quotient of the larger group of collapsed morphisms of $\mathbb{G}_{2n}$.

In \cref{morphisms}, we will see how to use the information that we've accumulated up to this point to build the morphisms of $L\mathbb{G}_n$. The idea is that the invertibility of the objects in this category will let us split the monoid $\mathrm{Mor}(L\mathbb{G}_n)$ into two relevant pieces. The first is a subgroup $(s \times t)(L\mathbb{G}_n)$, which encodes all of the ordered pairs of objects that appear as the source and target data of at least one morphism. The fact that there is such a subgroup --- that we can choose a representative morphism for each source/target pair in a way which respects the tensor product of $L\mathbb{G}_n$ --- is a consequence of the way that the morphisms of the free algebra $\mathbb{G}_n$ are structured. Specifically, the source and target monoid $(s \times t)(\mathbb{G}_n)$ is free, which lets us easily construct an inclusion $(s \times t)(\mathbb{G}_n) \to \mathrm{Mor}(L\mathbb{G}_n)$, whose image under the coequaliser $q$ then forms the required inclusion for $(s \times t)(L\mathbb{G}_n)$. By comparison, the second subgroup that we need is much simpler, as it is just the homset of endomorphisms of the unit object, $L\mathbb{G}_n(I, I)$. Together, these two subgroups  shape the whole of $\mathrm{Mor}(L\mathbb{G}_n)$, in the sense that the latter is a semidirect product of the former. Moreover, under certain circumstances which will include all of the motivating examples for this research, this semidirect product is actually direct. This will allow us to easily perform abelianisations, group completions, and repeated quotients of $\mathrm{Mor}(L\mathbb{G}_n)$ until we arrive at the same the collapsed $\mathrm{M}(L\mathbb{G}_n)^{\mathrm{gp}, \mathrm{ab}}$ we had before, after which we will have successfully described a path from the morphisms of the free algebra $\mathbb{G}_{2n}$ to those of the invertible $L\mathbb{G}_n$. The rest of the chapter will then be concerned with simplifying this description, by carrying out some calculations that do not change for different instances of $L\mathbb{G}_n$. This will include an investigation into the way that action operads and the monoids we've built out of them will act under group completion and abelianisation.

Finally, in \cref{mainthm} we will compile all of the major results of the previous chapters into a single account of the free $\mathrm{E}G$-algebra on $n$ invertible objects. The only piece of data still missing at this stage will be the action $\alpha$, but a method for recovering it will have already been established back in \cref{colimalgebra}, so this will not present any further challenges. \cref{freeinvalgG1,freeinvalgc} are the focal point of the thesis, providing a step-by-step construction of the algebra $L\mathbb{G}_n$ for all values of $n \in \mathbb{N}$ and all action operads $G$. The remainder of the paper will then consist of applications of these theorems to specific examples of free $G$-monoidal categories on invertible objects --- the symmetric, the braided, and the ribbon braided.  






















\mainmatter

\chapter{Operads and their algebras} \label{actionoperad} 

Before we can talk about the main focus of this paper, the free $\mathrm{E}G$-algebras on $n$ invertible objects, we will need to work our way through several intermediate concepts. This chapter will cover the background material needed to understand each of these other structures in turn --- monoidal categories, operads, action operads, $G$-operads, and operad algebras. Most of this content is due to other authors, and the reader is encouraged to refer to the given sources if they are interested in a more complete analysis of any of the featured topics. 

\section{Basic definitions}

In this section we shall briefly review some standard definitions from category theory that will be used throughout the paper. Everything in this section can be found in any good introductory text on category theory, such as the foundational `Categories for the Working Mathematician' \cite{cwm} by Saunders Mac Lane.

We will start with the notion of an adjunction.

\begin{defn} Let $C$ and $D$ be categories, and $F: C \to D$, $G: D \to C$ be functors. Then we say that $F$ is \emph{left adjoint} to $G$, and that $G$ is \emph{right adjoint} to $F$, if for any objects $X$ in $C$ and $Y$ in $D$ there exists an isomorphism
\begin{eq*} D\big( \, F(X), \,  Y \, \big) \quad \cong \quad C\big( \, X, \, G(Y) \, \big) \end{eq*}
natural in both variables. Equivalently, $F$ and $G$ are adjoints if there exist natural transformations $\eta: \mathrm{id}_{C} \Rightarrow G \circ F$ and $\epsilon : F \circ G \Rightarrow \mathrm{id}_{D}$ which obey the so-called zig-zag identities,
\begin{eq*} \begin{tikzcd}
F \ar[rr, "\mathrm{id}_F \circ \eta"] \ar[ddrr, "\mathrm{id}_F"'] & & FGF \ar[dd, "\epsilon \circ \mathrm{id}_F"] & \quad & G \ar[rr, "\eta \circ \mathrm{id}_G"] \ar[ddrr, "\mathrm{id}_G"'] & & GFG \ar[dd, "\mathrm{id}_G \circ \epsilon"] \\
& & & & & & \\
& & F & & & & G
\end{tikzcd} \end{eq*}
This \emph{adjunction} is denoted $F \dashv G$.
\end{defn}

First described by Daniel Kan in \cite{adjoint}, adjoint functors are an incredibly common mathematical structure. They appear in group theory, with the forgetful functor $U: \mathrm{Grp} \to \mathrm{Set}$ and its right adjoint the free group functor $F: \mathrm{Set} \to \mathrm{Grp}$, or the inclusion of abelian groups into groups $\mathrm{Ab} \hookrightarrow \mathrm{Grp}$ and its right adjoint abelianisation $\mathrm{ab}: \mathrm{Gp} \to \mathrm{Ab}$. They appear in topology, where the suspension functor $\Sigma$ is left adjoint to the loop space functor $\Omega$, and in logic, where the act of substituting by a variable is left adjoint to universal quantification and right adjoint to existential quantification \cite{aif}. Indeed, the aforementioned `Categories for the Working Mathematician' \cite{cwm} opens by saying that its slogan is "Adjoint functors arise everywhere". For our purposes, the most important feature of adjoint functors is the following:

\begin{prop} Left adjoints preserve colimits. Right adjoints preserve limits. \end{prop}

In particular, \cref{initialalgebra,colimalgebra,morphisms} will all utilise the fact that left adjoints can be commuted past colimits at some point.

Next, this paper will also rely upon the concept of the monoidal category.

\begin{defn} \label{moncat} A \emph{monoidal category} is a category $C$ equipped with
\begin{itemize}\itemsep0.3em
\item a functor $\otimes : C \times C \to C$, called the \emph{tensor product} of $C$
\item an object $I \in C$, called the \emph{unit}
\item a natural isomorphism $a$, called the \emph{associator}, with components
\begin{eq*} a_{x,y,z} \, : \, (x \otimes y) \otimes z \longrightarrow x \otimes (y \otimes z) \end{eq*}
\item two natural isomorphisms $l$ and $r$, called the \emph{left} and \emph{right unitors}, with components
\begin{eq*} l_{x} \, : \, I \otimes x \longrightarrow x, \quad \quad \quad \quad r_{x} \, : \, x \otimes I \longrightarrow x \end{eq*}
\end{itemize}
which satisfy two coherence conditions. The first of these, the pentagon identity, is best displayed as the commutative diagram
\begin{eq*} \begin{tikzcd}
& (w \otimes x) \otimes (y \otimes z) \ar[dr, "a_{w,x,y \otimes z}"] & \\
((w \otimes x) \otimes y) \otimes z \ar[ur, "a_{w\otimes x,y, z}"] \ar[dd, "a_{w,x,y} \otimes \mathrm{id}_{z}"']  & & w \otimes (x \otimes (y \otimes z)) \\
& & \\
(w \otimes (x \otimes y)) \otimes z \ar[rr, "a_{w,x\otimes y, z}"] & & w \otimes ((x \otimes y) \otimes z) \ar[uu, "\mathrm{id}_{z} \otimes a_{x,y,z}"']
\end{tikzcd} \end{eq*}
and shows that the operation $\otimes$ is weakly associative. Likewise the second condition, the triangle identity, corresponds to the diagram
\begin{eq*} \begin{tikzcd}
(x \otimes I) \otimes y \ar[rr, "a_{x,I,y}"] \ar[dr, "r_x \otimes \mathrm{id}_y"'] & & x \otimes (I \otimes y) \ar[dl, "\mathrm{id}_y \otimes l_y"] \\
& x \otimes y & 
\end{tikzcd} \end{eq*}
and represents the fact that $I$ is a weak unit. A monoidal category in which the natural isomorphisms $a, l, r$ are all identities --- and thus the two coherence conditions hold trivially --- is said to be \emph{strictly} monoidal. For contrast, we will therefore sometimes refer to the above kind of category as \emph{weakly} monoidal.
\end{defn} 

While it isn't explicitly stated in \cref{moncat}, notice that the functoriality of $\otimes$ induces the following relationship between the tensor product and composition in $C$:
\begin{eq*} (f' \circ f) \otimes (g' \circ g) \quad = \quad (f' \otimes g') \circ (f \otimes g) \end{eq*}
This is known as the \emph{interchange law} of $C$. We will use this equality frequently in later chapters, especially when investigating its interaction with \emph{invertible objects} --- those objects $x$ in a monoidal category which possess an \emph{inverse} $x^*$ satisfying

\begin{eq*} x \otimes x^* \quad = \quad I \quad = \quad x^* \otimes x \end{eq*}

Monoidal categories are also found everywhere throughout mathematics. Commonly studied examples include the category of sets $\mathrm{Set}$ with the cartesian product $\times$, the category of abelian groups $\mathrm{Ab}$ under direct sum $\oplus$, and the category of $K$-vector spaces $K-\mathrm{Vect}$ with its usual tensor product $\otimes_{K}$. Part of the reason for their ubiquity is that monoidal categories are, in some sense, really degenerate versions of a higher dimensional category, specifically a one-object bicategory. We will not be exploring the concept of higher categories in this paper (see for example \cite{hohc} for a proper treatment), but suffice it to say that there are also other kinds of degenerate $n$-categories which appear to be common kinds of category-with-extra-structure. 

\begin{defn} \label{bdmoncat} A \emph{braided} monoidal category is a monoidal category $C$ equipped with an additional natural isomorphism,
\begin{eq*} \beta_{x,y} \, : \, x \otimes y \longrightarrow y \otimes x \end{eq*}
called the \emph{braiding}, which satisfies the hexagon identities,
\begin{eq*} \begin{tikzcd}
& (x \otimes y) \otimes z \ar[rr, "\beta_{x \otimes y, z}"] \ar[dl, "a_{x,y, z}"'] & & z \otimes (x \otimes y) \ar[dr, "a^{-1}_{z, x,y}"] & \\
x \otimes (y \otimes z) \ar[dr, "\mathrm{id}_x \otimes \beta_{y,x}"'] & & & & (z \otimes x) \otimes y \ar[dl, "\beta_{z, x} \otimes \mathrm{id}_y"] \\
& x \otimes (z \otimes y) \ar[rr, "a^{-1}_{x,z,y}"] & & (x \otimes z) \otimes y &
\end{tikzcd} \end{eq*}
\begin{eq*} \begin{tikzcd}
& x \otimes (y \otimes z) \ar[rr, "\beta_{x, y \otimes z}"] \ar[dl, "a^{-1}_{x,y, z}"'] & & (y \otimes z) \otimes x \ar[dr, "a_{y,z, x}"] & \\
(x \otimes y) \otimes z \ar[dr, "\beta_{x,y} \otimes \mathrm{id}_z"'] & & & & y \otimes (z \otimes x) \ar[dl, "\mathrm{id}_y \otimes \beta_{z, x}"] \\
& (y \otimes x) \otimes z \ar[rr, "a_{y,x,z}"] & & y \otimes (x \otimes z) &
\end{tikzcd} \end{eq*}
\end{defn}

Again, though it isn't directly mentioned in, the above definition also implies another pair of coherence conditions for the unit in $C$, namely
\begin{eq*} \begin{tikzcd}
x \otimes I \ar[rr, "\beta_{x, I}"] \ar[dr, "r_x"'] & & I \otimes x \ar[dl, "l_x"] & & I \otimes x \ar[rr, "\beta_{I,x}"] \ar[dr, "l_x"'] & & x \otimes I \ar[dl, "r_x"] \\
& x & & & & x &
\end{tikzcd} \end{eq*}

\begin{defn} \label{symmoncat} A \emph{symmetric} monoidal category is a braided monoidal category $C$ whose braiding satisfies an extra symmetry condition, $\beta_{x, y}^{-1} = \beta_{y,x}$.
\end{defn}

Braided monoidal categories can be seen as the `same' as doubly-degenerate tricategories, while symmetric monoidal categories `are' triply-degenerate weak 4-categories. For a more thorough explanation of this relationship, see \cite{ptncld1} and \cite{ptncld2}.

Strict symmetric monoidal categories are sometimes known as `permutative categories', and it is not hard to see why. If we set $a, l, r = \mathrm{id}$, then in the symmetric case the diagrams from \cref{bdmoncat} simplify to 
\begin{eq*} \begin{array}{rclcrcl}
			\beta_{x \otimes y, z} & = & (\beta_{z, x} \otimes \mathrm{id}_y) \circ (\mathrm{id}_x \otimes \beta_{y,x}), & \quad \quad & \beta_{x, I} & = & \mathrm{id}_x \\
			\beta_{x, y \otimes z} & = & (\mathrm{id}_y \otimes \beta_{z, x}) \circ (\beta_{y,x} \otimes \mathrm{id}_x), & \quad \quad & \beta_{I,xI} & = & \mathrm{id}_x
		\end{array}
\end{eq*}
Collectively, these identities represent the fact that for any collection of distinct objects $x_1, ..., x_n$ in a strict symmetric monoidal category $X$ and any permutation $\sigma \in \mathrm{S}_n$, there exists a unique isomorphism
\begin{eq*} x_1 \otimes ... \otimes x_n \, \longrightarrow \, x_{\sigma^{-1}(1)} \otimes ... \otimes x_{\sigma^{-1}(n)} \end{eq*}
built out of the symmetries $\beta$. In other words, elements of the symmetric groups $\mathrm{S}_n$ act like $n$-ary operations, which take in an appropriate number of objects and return some data for a strict symmmetric monoidal category. This is a fairly vague statement however; it would be nice if we could make it more rigorous.

\section{Operads} \label{operad}

What we need is the concept of an operad. These were first introduced by Peter May in the book `The Geometry of Iterated Loop Spaces' \cite{gils}, though our usage will be slightly different, for reasons discussed later.

\begin{defn} \label{opdef} An \emph{operad} $O$ in a symmetric monoidal category $(C, \otimes, I)$ is a structure consisting of
\begin{itemize}\itemsep0.3em
\item a family of objects, $O(n)$ for $n \in \mathbb{N}$, 
\item a morphism $1: I \to O(1)$, called the identity
\item a family of morphisms,
\begin{eq*} \mu_{n;k_1,...,k_n} \, : \, O(n) \otimes O(k_1) \otimes ... \otimes O(k_n) \, \longrightarrow \, O(k_1+...+k_n) \end{eq*}
called operadic multiplication.
\end{itemize}
This data is then subject to the unitality conditions
\begin{eq*} \begin{tikzcd}
I \otimes O(n) \ar[dd, "1 \otimes \mathrm{id}_{O(n)}"'] \ar[ddrr, "l_x"] & & & O(n) \otimes I \otimes ... \otimes I \ar[dd, "\mathrm{id}_{O(n)} \otimes 1 \otimes ... \otimes 1"'] \ar[ddrr, "r_{x \otimes I \otimes ... \otimes I} \, \circ ... \circ \, r_{x}"] & & \\
& & & & & \\
O(1) \otimes O(n) \ar[rr, "\mu_{1;n}"'] & & O(n) & O(n) \otimes O(1) \otimes ... \otimes O(1) \ar[rr, "\mu_{n;1,...,1}"'] & & O(n)
\end{tikzcd} \end{eq*}
for all $n \in \mathbb{N}$, and the associativity conditions
\begin{eq*} \begin{tikzcd}
O(n) \otimes \prod O(m_i) \otimes \prod O(k_{1,j}) \otimes ... \otimes \prod O(k_{n, j}) \ar[ddr, "\mu \otimes \mathrm{id}"] \ar[dd, "\beta"'] & \\
& \\
O(n) \otimes \prod \big( \, O(m_i) \otimes \prod O(k_{i,j}) \, \big) \ar[dd, "\mathrm{id} \, \otimes \prod \mu"'] & O(m_1+...+m_n) \otimes \prod O(k_{i,j}) \ar[dd, "\mu"] \\
& \\
O(n) \otimes \prod O(k_{i,1}+...+k_{i,m_i}) \ar[r, "\mu"] & O(k_{1,1}+...+k_{n,m_n})
\end{tikzcd} \end{eq*}
for all $n, m_1, ..., m_n, k_{1,1}, ..., k_{1, m_1}, ..., k_{n,1}, ...,  k_{n, m_n} \in \mathbb{N}$.
\end{defn}

The idea behind operads is that they are supposed to generalise the notion of `operations'. That is, objects $O(n)$ are to be thought of as somehow representing collections of $n$-ary operations, with the identity as a distinguished unary operation. Multiplication in an operad is then motivated by the intuition that we can plug the outputs of $n$ given operations into the inputs of an $n$-ary operation. 
\begin{center} \begin{tabular}{ccccc}
			\begin{tikzpicture}[baseline]
				\fill (-0.3,0.3) to (-0.3,-0.3) to (0.3,0);
				\draw[-] (-0.7,0.2) to (-0.3,0.2);
				\draw[-] (-0.7,0) to (-0.3,0);
				\draw[-] (-0.7,-0.2) to (-0.3,-0.2);
				\draw[-] (0.2,0) to (0.7,0);
			\end{tikzpicture} & & 
			\begin{tikzpicture}[baseline]
				\draw[-] (-0.5,0) to (0.5,0);
			\end{tikzpicture} & & 
			\begin{tikzpicture}[baseline]
				\fill (-1.2,1.2) to (-1.2,0.6) to (-0.6,0.9);
				\fill (-1.2,0.3) to (-1.2,-0.3) to (-0.6,0);
				\fill (-1.2,-1.2) to (-1.2,-0.6) to (-0.6,-0.9);
				\fill (0,1.5) to (0,-1.5) to (1.2,0);
				\draw[-] (-1.6,1) to (-1.2,1);
				\draw[-] (-1.6,0.8) to (-1.2,0.8);
				\draw[-] (-1.6,-0.9) to (-1.2,-0.9);
				\draw[-] (-0.6,0.9) to (0,0.9);
				\draw[-] (-0.6,0) to (0,0);
				\draw[-] (-0.6,-0.9) to (0,-0.9);
				\draw[-] (1.2,0) to (1.6,0);
			\end{tikzpicture} \\
			$n$-ary operation & \quad \quad \quad \quad & Identity & \quad \quad \quad \quad & Operadic multiplication \\
\end{tabular} \end{center}
As an example, if we were to represent some operations pictorially as in the diagram above, then the figure on the right is what is meant by the multiplication $\mu: O(3) \times O(2) \times O(0) \times O(1) \to O(2+0+1)$. Under this interpretation, each of the coherence conditions for an operad represents some obvious fact about how generic $n$-ary operations should interact with one another. For instance, unitality of the identity is simply
\begin{center} \begin{tabular}{ccccc}
			& & & & \\
			\begin{tikzpicture}[baseline]
				\fill (-0.3,0.3) to (-0.3,-0.3) to (0.3,0);
				\draw[-] (-1.7,0.2) to (-0.3,0.2);
				\draw[-] (-1.7,0) to (-0.3,0);
				\draw[-] (-1.7,-0.2) to (-0.3,-0.2);
				\draw[-] (0.2,0) to  (0.7,0);
			\end{tikzpicture} & &
			\begin{tikzpicture}[baseline]
				\fill (-0.3,0.3) to (-0.3,-0.3) to (0.3,0);
				\draw[-] (-0.7,0.2) to (-0.3,0.2);
				\draw[-] (-0.7,0) to (-0.3,0);
				\draw[-] (-0.7,-0.2) to (-0.3,-0.2);
				\draw[-] (0.2,0) to  (0.7,0);
			\end{tikzpicture} & &
			\begin{tikzpicture}[baseline]
				\fill (-0.3,0.3) to (-0.3,-0.3) to (0.3,0);
				\draw[-] (-0.7,0.2) to (-0.3,0.2);
				\draw[-] (-0.7,0) to (-0.3,0);
				\draw[-] (-0.7,-0.2) to (-0.3,-0.2);
				\draw[-] (0.2,0) to  (1.7,0);
			\end{tikzpicture} \\
			& & & & \\
			$\mu( \, x \, ; \, 1, 1, 1 \, )$ & \quad = \quad & $x$ & \quad = \quad & $\mu( \, 1 \, ; \, x \, )$ \\
\end{tabular} \end{center}

As with most mathematical structures, operads naturally form a category, together with a suitable notion of morphisms between operads.

\begin{defn} Given two operads $O, O'$ in a symmetric monoidal category $(C, \otimes, I)$, a \emph{map of operads} between them is a family of maps between their operations which preserve operadic composition. That is, any $f: O \to O'$ is composed of morphisms $f_n : O(n) \to O'(n)$, $n \in \mathbb{N}$ which satisfy
\begin{eq*} \begin{tikzcd}
& I \ar[ddl, "1^O"'] \ar[ddr, "1^{O'}"] & & O(n) \otimes O(k_1) \otimes ... \otimes O(k_n) \ar[dd, "f_n \otimes f_{k_1} \otimes ... \otimes f_{k_n}"'] \ar[r, "\mu^O"] & O(k_1 + ... + k_n) \ar[dd, "f_{k_1 + ... +k_n}"] \\
& & & & \\
O(1) \ar[rr, "f_1"]& & O'(1) & O'(n) \otimes O'(k_1) \otimes ... \otimes O'(k_n) \ar[r, "\mu^{O'}"] & O'(k_1 + ... + k_n)
\end{tikzcd} \end{eq*}
for all $n, k_1, ..., k_n \in \mathbb{N}$. The category of operads and maps of operads in $(C, \otimes, I)$ is denoted $\mathrm{Op}(C)$, though in the case of $\mathrm{Set}$ we will just call it $\mathrm{Op}$. Composition in this category is defined by term-wise composition of families $f_n : O(n) \to O'(n)$, $g_n : O'(n) \to O''(n)$, and the identity morphisms $\mathrm{id}_O : O \to O$ are simply the families $\mathrm{id}_{O(n)}$ from $C$.
\end{defn}

For a far more in depth explanation of operads and their intimate relationship with category theory, see the book `Higher Operads, Higher Categories' \cite{hohc} by Tom Leinster.
 
When we are working with operads in the category of sets, $(\mathrm{Set}, \times, 1)$, the objects $O(n)$ genuinely are collections of elements, with a distinguished identity $1 \in O(1)$. However, these elements still do not have to be operations in any way other than that they satisfy \cref{opdef}, as we will see in the following examples.

\begin{namedexample}[(The symmetric operad)]
There is an operad in $\mathrm{Set}$ whose sets of operations $\mathrm{S}(n)$ are the underlying sets of the symmetric groups $\mathrm{S}_n$. The identity element of this \emph{symmetric operad} $\mathrm{S}$ is the identity permutation of a single object, $e_1 \in \mathrm{S}_1$, and the operadic multiplication is defined in the following way:
\begin{itemize}
\item First, there exist maps $\otimes : \mathrm{S}_m \times \mathrm{S}_n \to \mathrm{S}_{m+n}$ called the \emph{direct sum} or \emph{block sum} of permutations. For any $\sigma \in \mathrm{S}_m$ and $\tau \in \mathrm{S}_n$, these are given by
\begin{eq*} (\sigma \otimes \tau)(i) \quad = \quad \begin{cases}
								\quad \sigma(i) & \quad 1 \le i \le m \\
								\quad \tau(i-m) +m & \quad m+1 \le i \le m+n
							\end{cases}
\end{eq*}
As the name suggests, this direct sum is usually denote by the symbol $\oplus$, but we will stick with $\otimes$ so that our notation here matches all of the other tensor products we will see throughout this paper. Also, notice that the value of these direct sums in general are determined by those specific cases where one of the inputs is an identity permutation:
\begin{eq*} \sigma \otimes \tau \quad = \quad (\sigma \otimes e_n) \cdot (e_m \otimes \tau) \quad = \quad (e_m \otimes \tau) \cdot (\sigma \otimes e_n) \end{eq*}
\item Next, we'll define functions $( \, \_ \, )_{(k_1, ..., k_n)} : \mathrm{S}_n \to \mathrm{S}_{k_1 + ... + k_n}$ for all $n, k_1, ..., k_n \in \mathbb{N}$. These will act by taking a $\sigma$ which permutes $n$ individual objects and sending it onto a $\sigma_{(k_1, ..., k_n)}$ that permutes $n$ blocks of objects of size $k_1, ..., k_n$ in the same way. More concretely, if $k_1 + ... + k_{i-1} < j \le k_1 + ... + k_i$ then
\begin{eq*} \sigma_{(k_1, ..., k_n)}(j) \quad = \quad j - k_1 - ... - k_{i-1} + k_{\sigma^{-1}(1)} + ... + k_{\sigma^{-1}( \, \sigma(i) -1 \, )} \end{eq*}
\item Finally, the multiplication maps $\mu: \mathrm{S}_n \times \mathrm{S}_{k_1} \times ... \times \mathrm{S}_{k_n} \to \mathrm{S}_{k_1 + ... + k_n}$ are given by
\begin{eq*} \begin{array}{rrl} 
			\mu(\sigma; \tau_1, ..., \tau_n) & := & \sigma_{(k_1, ..., k_n)} \cdot (\tau_1 \otimes ... \otimes \tau_n) \\
			& = & (\tau_{\sigma^{-1}(1)} \otimes ... \otimes \tau_{\sigma^{-1}(n)}) \cdot \sigma_{(k_1, ..., k_n)}
		\end{array}
\end{eq*}
In other words, the operadic multiplication of permutations comes from both permutating objects within distinct blocks and also permuting the blocks themselves.
\end{itemize}

If we decide to represent elements of the symmetric operad pictorially --- for example as strings which cross over another according to the appropriate permutation --- then both $\sigma \otimes \tau$ and $\sigma_{(k_1, ..., k_n)}$ have rather nice interpretations. 
\begin{center} \begin{tabular}{ccccc}
			\begin{tikzpicture}[baseline]
				\node(x1) at (-0.5,1){};
				\node(y1) at (0.5,1){};	
				\node(y2) at (-0.5, -1){};
				\node(x2) at (0.5, -1){};
       				\draw[rounded corners](x1.south) to (-0.5,0.5) to (0.5,-0.5) to (x2.north);
				\draw[rounded corners](y1.south) to (0.5, 0.5) to (-0.5, -0.5) to (y2.north);		
			\end{tikzpicture} & \quad $\bigotimes$ \quad \quad &
			\begin{tikzpicture}[baseline]
				\node(x1) at (-0.5,1){};
				\node(y1) at (0.5,1){};	
				\node(x2) at (-0.5, -1){};
				\node(y2) at (0.5, -1){};
				\draw[rounded corners](x1.south) to (x2.north);	
       				\draw[rounded corners](y1.south) to (y2.north);	
			\end{tikzpicture} & \quad $=$ \quad \quad &
			\begin{tikzpicture}[baseline]
				\node(x1) at (-1.5,1){};	
				\node(y1) at (-0.5,1){};
				\node(y2) at (-1.5, -1){};
				\node(x2) at (-0.5, -1){};
				\node(x'1) at (0.5,1){};
				\node(y'1) at (1.5,1){};
				\node(x'2) at (0.5, -1){};
				\node(y'2) at (1.5, -1){};
       				\draw[rounded corners](x1.south) to (-1.5,0.5) to (-0.5,-0.5) to (x2.north);	
				\draw[rounded corners](y1.south) to (-0.5, 0.5) to (-1.5, -0.5) to (y2.north);
				\draw[rounded corners](x'1.south) to (x'2.north);	
       				\draw[rounded corners](y'1.south) to (y'2.north);	
			\end{tikzpicture} \\
			$\sigma$ & & $\tau$ & & $\sigma \otimes \tau$
\end{tabular} \end{center}
The direct sum of two permutations is just the result of placing two permutations `next to' each other, as above, and block permutations are given by expanding each string into some number of parallel strings:
\begin{center} \begin{tabular}{ccc}
			\begin{tikzpicture}[baseline]
				\node(x1) at (-0.5,1){};
				\node(y1) at (0.5,1){};	
				\node(y2) at (-0.5, -1){};
				\node(x2) at (0.5, -1){};
       				\draw[rounded corners](x1.south) to (-0.5,0.5) to (0.5,-0.5) to (x2.north);
				\draw[rounded corners](y1.south) to (0.5, 0.5) to (-0.5, -0.5) to (y2.north);		
			\end{tikzpicture} & \quad $\mapsto$ \quad \quad &
			\begin{tikzpicture}[baseline]
				\node(x1) at (-2,1){};
				\node(x'1) at (-1,1){};
				\node(x''1) at (0,1){};
				\node(y1) at (1,1){};	
				\node(y'1) at (2,1){};
				\node(y2) at (-2, -1){};
				\node(y'2) at (-1, -1){};
				\node(x2) at (0, -1){};
				\node(x'2) at (1, -1){};
				\node(x''2) at (2, -1){};
       				\draw[rounded corners](x1.south) to (-2,0.5) to (0,-0.5) to (x2.north);
       				\draw[rounded corners](x'1.south) to (-1,0.5) to (1,-0.5) to (x'2.north);
       				\draw[rounded corners](x''1.south) to (0,0.5) to (2,-0.5) to (x''2.north);
				\draw[rounded corners](y1.south) to (1, 0.5) to (-2, -0.5) to (y2.north);
				\draw[rounded corners](y'1.south) to (2, 0.5) to (-1, -0.5) to (y'2.north);	
			\end{tikzpicture} \\
			$\sigma$ & & $\sigma_{(3, 2)}$
\end{tabular} \end{center}
With a little work, we can actually replace the functions $( \, \_ \, )_{(k_1, ..., k_n)}$ with an explicit combination of group multiplication and tensor product. This is due to basic fact about the symmetric groups $\mathrm{S}_n$, which is that they possess a presentation in terms of the \emph{elementary transpositions} $(i \,\,\,  i+1)$.

\begin{lem} \label{sympres} The group $\mathrm{S}_n$ is generated by the permutations $(1 \, 2), ..., (n-1 \, \, \, n)$, subject to the relations
\begin{eq*} \begin{array}{rclll}
			(i \, \, \, i+1)^2 & = & e & & \\
			(i-1 \, \, \, i)(i \, \, \, i+1)(i-1 \, \, \, i) & = & (i \, \, \, i+1)(i-1 \, \, \, i)(i \, \, \, i+1) & & \\
			(i \, \, \, i+1)(j \, \, \, j+1) & = & (j \, \, \, j+1)(i \, \, \, i+1), & & i+1 < j
		\end{array}
\end{eq*}
\end{lem}

Thus if $\sigma \in \mathrm{S}_n$ is a permutation with a decomposition $\sigma = \sigma_m \cdot ... \cdot \sigma_1 $ in terms of elementary transpositions $\sigma_i \in \mathrm{S}_n$, we can break down the block permutation $\sigma_{(k_1, ..., k_n)}$ into the $m$ `elementary block transpositions' $(\sigma_i)_{(k_1, ..., k_n)}$:
\begin{eq*} \begin{array}{rll}
			\sigma_{(k_1, ..., k_n)}(j) & = & j - k_1 - ... - k_{i-1} + k_{\sigma^{-1}(1)} + ... + k_{\sigma^{-1}( \, \sigma(i) -1 \, )} \\
			& = & j - k_1 - ... - k_{i-1} \\
			& & + \, k_{\sigma_1^{-1}(1)} + ... + k_{\sigma_1^{-1}( \, \sigma_1(i) -1 \, )} \\
			& & -  \, k_{\sigma_1^{-1}(1)} - ... - k_{\sigma_1^{-1}( \, \sigma_1(i) -1 \, )} \\
			& & + \, k_{(\sigma_2 \sigma_1)^{-1}(1)} + ... + k_{(\sigma_2 \sigma_1)^{-1}( \, \sigma_2\sigma_1(i) -1 \, )} \\
			& & \vdots \\
			& & - \,  k_{(\sigma_{m-1}...\sigma_1^{-1}(1)} - ... - k_{(\sigma_{m-1}...\sigma_1)^{-1}( \, \sigma_{m-1}...\sigma_1(i) - 1 \, )} \\
			& & + \, k_{(\sigma_m...\sigma_1)^{-1}(1)} + ... + k_{(\sigma_m...\sigma_1)^{-1}( \, \sigma_m...\sigma_1(i) - 1 \, )} \\
			& = & \big( \, (\sigma_m)_{(k_1, ..., k_n)} \cdot ... \cdot (\sigma_1)_{(k_1, ..., k_n)} \, \big)(j)
		\end{array}
\end{eq*}
However, since elementary transpositions only really permute two objects, they can be written as a block sum in the operad $\mathrm{S}$ involving the sole transposition of $\mathrm{S}_2$, plus some number of identity permutations.
\begin{eq*} (i \, \, \, i+1) \quad = \quad e_{i-1} \otimes (1 \, 2) \otimes e_{n-i-1} \end{eq*}
This means that the elementary block transpositions are
\begin{eq*} \begin{array}{rll}
			(i \, \, \,  i+1)_{(k_1, ..., k_n)} & = & \quad (e_{i-1} \otimes (1 \, 2) \otimes e_{n-i-1})_{(k_1, ..., k_n)} \\
			& = & e_{k_1 + ... k_{i-1}} \otimes (1 \, 2)_{(k_i, k_{i+1})} \otimes e_{k_{i+1} + ... k_n}
		\end{array}
\end{eq*}
So all we need to know to fully understand the functions $( \, \_ \, )_{(k_1, ..., k_n)}$ are the values they take on the transposition $(1 \, 2)$. These can be defined recursively, via
\begin{eq*} \begin{array}{lclclcl}
			(1 \, 2)_{(0, n)} & = & e_n, & \quad \quad & (1 \, 2)_{(m+m', n)} & = & \big( \, (1 \, 2)_{(m, n)} \otimes e_{m'} \, \big) \cdot \big( \, e_m \otimes (1 \, 2)_{(m', n)} \, \big), \\
			(1 \, 2)_{(m, 0)} & = & e_m, & \quad \quad & (1 \, 2)_{(m, n+n')} & = & \big( \, e_n \otimes (1 \, 2)_{(m, n')} \, \big) \cdot \big( \, (1 \, 2)_{(m, n)} \otimes e_{n'} \, \big) \\
			(1 \, 2)_{(1,1)} & = & (1 \, 2), & & & &		
		\end{array}
\end{eq*}
which all follow from the definition of $( \, \_ \, )_{(k_1, ..., k_n)}$. Therefore all $\sigma_{(k_1, ..., k_n)}$ and hence all $\mu(\sigma; \tau_1, ..., \tau_n)$ can be expressed in terms of group multiplication $\cdot$ and direct sum $\otimes$, and the elementary permutations which constitute $\sigma, \tau_1, ..., \tau_n$.
\end{namedexample}

Something very important to notice about the symmetric operad is that while its sets of operations $\mathrm{S}_n$ are groups, it is \emph{not} an operad in the category of groups, because the operadic multiplication we have just outlined is not a group homomorphism. If it were, then it would obey
\begin{eq*} \mu(\sigma; \tau_1, ..., \tau_n) \cdot \mu(\sigma'; \tau'_1, ..., \tau'_n) \quad = \quad \mu( \, \sigma\sigma' \, ; \, \tau_1\tau'_1, ..., \tau_n\tau'_n \, ) \end{eq*}
for all $\sigma, \sigma' \in \mathrm{S}_n$, $\tau_i, \tau'_i \in \mathrm{S}_{k_i}$, but this is clearly false. As a counterexample, consider the fairly simple case
\begin{eq*} \begin{array}{rcccccl}
			\mu\big( \, (1 \, 2) \, ; \, e_2, e_1 \, \big) & = & (1 \, 2)_{(2, 1)} \cdot (e_2 \otimes e_1) & = & (1 \, 2 \, 3) \cdot e_3 & = & (1 \, 2 \, 3) \\
			\mu\big( \, e_2 \, ; \, (1 \, 2), e_1 \, \big) & = & {(e_2)}_{(2, 1)} \cdot \big(\, (1 \, 2) \otimes e_1 \, \big) & = & e_3 \cdot (1 \, 2) & = & (1 \, 2) \\
			\mu\big( \,  (1 \, 2) \, ; \, (1 \, 2), e_1 \, \big) & = & (1 \, 2)_{(2, 1)} \cdot \big(\, (1 \, 2) \otimes e_1 \, \big) & = & (1 \, 2 \, 3) \cdot (1 \, 2) & &
		\end{array}
\end{eq*}
Then we have
\begin{eq*} \mu\big( \, e_2 \, ; \, (1 \, 2), e_1 \, \big) \cdot \mu\big( \, (1 \, 2) \, ; \, e_2, e_1 \, \big) \quad = \quad (1 \, 2) \cdot (1 \, 2 \, 3) \end{eq*}
which is \emph{not} the same as
\begin{eq*} \mu\big( \, e_2 \cdot (1 \, 2) \, ; \, (1 \, 2) \cdot e_2 , \, e_1 \cdot e_1 \, \big) \quad = \quad \mu\big( \,  (1 \, 2) \, ; \, (1 \, 2), e_1 \, \big) \quad = \quad (1 \, 2 \, 3) \cdot (1 \, 2) \end{eq*}
At first this seems like pretty strange behaviour. After all, the symmetric groups play a central role in the theory of groups, so it would be reasonable to assume that their operad would be similarly crucial for the theory of group operads. But $\mathrm{S}$ is not the only family of groups whose operad is fundamentally set related.

\begin{namedexample}[(The braid operad)]\label{braidop}
The \emph{braid groups} $B_n$ are the family of groups that result from taking the symmetric groups and removing the requirement that everything needs to be self-inverse. That is, the group $B_n$ has a presentation on some \emph{elementary braids} $b_1, ..., b_{n-1}$, given by the relations
\begin{eq*} b_i b_{i+1} b_i \, = \, b_{i+1} b_i b_{i+1}, \quad \quad \quad \quad \quad b_i b_j \, = \, b_j b_i, \quad i+1 < j \end{eq*}
As might be expected, the underlying sets of these groups also form an operad in $\mathrm{Set}$ known as the \emph{braid operad} $B$, and they do so in a way directly analogous to the operad $\mathrm{S}$. That is, the identity element of $B$ is $e_1 \in B_1$, and the operadic multiplication is constructed as follows:
\begin{itemize}
\item Tensor products $\otimes : B_m \times B_n \to B_{m+n}$ are determined by setting 
\begin{eq*} x \otimes y \quad = \quad (x \otimes e_n) \cdot (e_m \otimes y) \quad = \quad (e_m \otimes x) \cdot (y \otimes e_n) \end{eq*}
for all $x \in B_m$, $y \in B_n$, and also
\begin{eq*} b_i \quad = \quad e_{i-1} \otimes b \otimes e_{n-i-1} \end{eq*}
for any elementary braid $b_i \in B_n$, where $b$ is the only elementary braid in $B_2$.
\item The functions $( \, \_ \, )_{(k_1, ..., k_n)} : B_n \to B_{k_1 + ... + k_n}$ are first defined recursively on the elementary braid $b \in B_2$ by
\begin{eq*} \begin{array}{rclcrcl}
			b_{(0, n)} & = & e_n, & \quad \quad & b_{(m+m', n)} & = & (b_{(m, n)} \otimes e_{m'}) \cdot (e_m \otimes b_{(m', n)}) \\
			b_{(m, 0)} & = & e_m, & \quad \quad & b_{(m, n+n')} & = & (e_n \otimes b_{(m, n')}) \cdot (b_{(m, n)} \otimes e_{n'}) \\
			b_{(1,1)} & = & b & & & &				
		\end{array}
\end{eq*}
then on arbitrary elementary braids $b_i \in B_n$ via
\begin{eq*} (b_i)_{(k_1, ..., k_n)} \quad = \quad e_{k_1 + ... k_{i-1}} \otimes b_{(k_i, k_{i+1})} \otimes e_{k_{i+1} + ... k_n} \end{eq*}
and finally on all elements of the braid groups by using their presentation in terms of the $b_i$,
\begin{eq*} \begin{array}{rll}
			x & = & b_{i_m} \cdot ... \cdot b_{i_1} \\
			\implies \quad x_{(k_1, ..., k_n)} & = & (b_{i_m})_{(k_1, ..., k_n)} \cdot ... \cdot (b_{i_1})_{(k_1, ..., k_n)} 
		\end{array}
\end{eq*}
\item Then as in the symmetric case, the multiplication maps $\mu: B_n \times B_{k_1} \times ... \times B_{k_n} \to B_{k_1 + ... + k_n}$ are just
\begin{eq*} \mu(x; y_1, ..., y_n) \quad := \quad x_{(k_1, ..., k_n)} \cdot (y_1 \otimes ... \otimes y_n) \end{eq*}
\end{itemize}

These operations are exactly what they need to be in order for them to possess the same pictorial representations as the operations in $\mathrm{S}$, but with actual braids replacing simple crossings. That is, the tensor product $x \otimes y$ is the braids $x$ and $y$ laid side-by-side,
\begin{center} \begin{tabular}{ccccc}
			\begin{tikzpicture}[baseline]
				\node(x1) at (-0.5,1){};
				\node(y1) at (0.5,1){};	
				\node(y2) at (-0.5, -1){};
				\node(x2) at (0.5, -1){};
				\node(b) at (0,0)[circle,fill=white]{};
       				\draw[rounded corners](x1.south) to (-0.5,0.5) to (0.5,-0.5) to (x2.north);
				\begin{pgfonlayer}{bg}
				\draw[rounded corners](y1.south) to (0.5, 0.5) to (-0.5, -0.5) to (y2.north);
    				\end{pgfonlayer}	
			\end{tikzpicture} & \quad $\bigotimes$ \quad \quad &
			\begin{tikzpicture}[baseline]
				\node(x1) at (-0.5,1){};
				\node(y1) at (0.5,1){};	
				\node(x2) at (-0.5, -1){};
				\node(y2) at (0.5, -1){};
				\draw[rounded corners](x1.south) to (x2.north);	
       				\draw[rounded corners](y1.south) to (y2.north);	
			\end{tikzpicture} & \quad $=$ \quad \quad &
			\begin{tikzpicture}[baseline]
				\node(x1) at (-1.5,1){};	
				\node(y1) at (-0.5,1){};
				\node(y2) at (-1.5, -1){};
				\node(x2) at (-0.5, -1){};
				\node(b) at (-1,0)[circle,fill=white]{};
				\node(x'1) at (0.5,1){};
				\node(y'1) at (1.5,1){};
				\node(x'2) at (0.5, -1){};
				\node(y'2) at (1.5, -1){};
       				\draw[rounded corners](x1.south) to (-1.5,0.5) to (-0.5,-0.5) to (x2.north);	
				\begin{pgfonlayer}{bg}
				\draw[rounded corners](y1.south) to (-0.5, 0.5) to (-1.5, -0.5) to (y2.north);
    				\end{pgfonlayer}
				\draw[rounded corners](x'1.south) to (x'2.north);	
       				\draw[rounded corners](y'1.south) to (y'2.north);	
			\end{tikzpicture} \\
			$x$ & & $y$ & & $x \otimes y$
\end{tabular} \end{center}
and the `block braids' are multiple strings braided together in parallel,
\begin{center} \begin{tabular}{ccc}
			\begin{tikzpicture}[baseline]
				\node(x1) at (-0.5,1){};
				\node(y1) at (0.5,1){};	
				\node(y2) at (-0.5, -1){};
				\node(x2) at (0.5, -1){};
				\node(b) at (0,0)[circle,fill=white]{};
       				\draw[rounded corners](x1.south) to (-0.5,0.5) to (0.5,-0.5) to (x2.north);
				\begin{pgfonlayer}{bg}
				\draw[rounded corners](y1.south) to (0.5, 0.5) to (-0.5, -0.5) to (y2.north);
    				\end{pgfonlayer}		
			\end{tikzpicture} & \quad $\mapsto$ \quad \quad &
			\begin{tikzpicture}[baseline]
				\node(x1) at (-2,1){};
				\node(x'1) at (-1,1){};
				\node(x''1) at (0,1){};
				\node(y1) at (1,1){};	
				\node(y'1) at (2,1){};
				\node(y2) at (-2, -1){};
				\node(y'2) at (-1, -1){};
				\node(x2) at (0, -1){};
				\node(x'2) at (1, -1){};
				\node(x''2) at (2, -1){};
				\node(b1) at (-0.8,-0.1)[circle,fill=white]{};
				\node(b2) at (-0.2,0.1)[circle,fill=white]{};
				\node(b3) at (0.4,0.3)[circle,fill=white]{};
				\node(b4) at (-0.4,-0.3)[circle,fill=white]{};
				\node(b5) at (0.2,-0.1)[circle,fill=white]{};
				\node(b6) at (0.8,0.1)[circle,fill=white]{};
       				\draw[rounded corners](x1.south) to (-2,0.5) to (0,-0.5) to (x2.north);
       				\draw[rounded corners](x'1.south) to (-1,0.5) to (1,-0.5) to (x'2.north);
       				\draw[rounded corners](x''1.south) to (0,0.5) to (2,-0.5) to (x''2.north);
				\begin{pgfonlayer}{bg}
				\draw[rounded corners](y1.south) to (1, 0.5) to (-2, -0.5) to (y2.north);
				\draw[rounded corners](y'1.south) to (2, 0.5) to (-1, -0.5) to (y'2.north);
    				\end{pgfonlayer}	
			\end{tikzpicture} \\
			$x$ & & $x_{(3, 2)}$
\end{tabular} \end{center}
\end{namedexample}

\section{Action operads}

It is not hard to see that the symmetric and braided operads both share certain features which are not otherwise common among operads of sets. This fact has been noticed by several different authors, each of whom proposed a slightly different definition and terminology for these sorts of structures. While older treatments exist --- see for example \cite{ribbon1} and \cite{groupop} --- in this paper we will be following the conventions laid out in \cite{ogge}, since they are the most general.

\begin{defn} \label{actop} An \emph{action operad} $(G, \pi)$ consists of 
\begin{itemize}
\item an operad $G$ in the category of sets, whose $G(n)$ are also all groups
\item a map of operads $\pi : G \to \mathrm{S}$ whose components $\pi_n : G(n) \to \mathrm{S}_n$ are also group homomorphisms
\end{itemize}
where the operadic multiplication of $G$ and the group multiplication of the $G(n)$ are linked via the map $\pi$ in the following way:
\begin{eq*} \mu( \, gg' \, ; \, h_1 h'_1, ...,  h_n h'_n \, ) \quad = \quad \mu(g; h_{\pi(g')^{-1}(1)}, ..., h_{\pi(g')^{-1}(n)}) \cdot \mu(g'; h'_1, ..., h'_n) \end{eq*}
\end{defn}

The element $\pi(g)$ is called the \emph{underlying permutation} of $g$, and as we can see the role it plays is to permute the inputs of an operadic multiplication when two of them are multiplied as group elements. This is exactly the behaviour we observed before with the symmetric operad; for instance, recalling our previous example we see that what we should have had was
\begin{eq*} \begin{array}{rclllll}
			\mu\big( \, (1 \, 2) \, ; \, e_1, e_2 \, \big) & = & (1 \, 2 \, 3) & & & & \\
			\mu\big( \, e_2 \, ; \, e_1, (1 \, 2) \, \big) & = & {(e_2)}_{(1, 2)} \cdot \big(\, e_1 \otimes (1 \, 2) \, \big) & = & e_3 \cdot (2 \, 3) & = & (2 \, 3) \\
			\mu\big( \,  (1 \, 2) \, ; \, (1 \, 2), e_1 \, \big) & = & (1 \, 2 \, 3) \cdot (1 \, 2) & & & &
		\end{array}
\end{eq*}
\begin{eq*} \begin{array}{rll}
			\implies \mu\big( \, e_2 \, ; \, (1 \, 2), e_1 \, \big) \cdot \mu\big( \, (1 \, 2) \, ; \, e_2, e_1 \, \big) & = & (2 \, 3) \cdot (1 \, 2 \, 3) \\
			& = &  (1 \, 2 \, 3) \cdot (1 \, 2) \\
			& = & \mu\big( \,  (1 \, 2) \, ; \, (1 \, 2), e_1 \, \big) \\
			& = & \mu\big( \, e_2 \cdot (1 \, 2) \, ; \, (1 \, 2) \cdot e_2 , \, e_1 \cdot e_1 \, \big)
		\end{array}
\end{eq*}
The effect that this has on the map $\mu$ also mirrors the way that we had to define operadic multiplication for $\mathrm{S}$ and $B$ in stages. Specifically, if for any action operad $G$ we define
\begin{eq*} g_{(k_1, ..., k_n)} \, := \, \mu(g; e_{k_1}, ..., e_{k_n}), \quad \quad \quad g_1 \otimes ... \otimes g_n \, := \, \mu(e_n; g_1, ..., g_n) \end{eq*}
then it follows from \cref{actop} that
\begin{eq*} \begin{array}{rll}
			\mu( \, g \, ; \, h_1, ..., h_n \, ) & = & \mu( \, g \cdot e_n \, ; \, e_{k_1} \cdot h_1, ..., e_{k_n} \cdot h_n \, ) \\
			& = & \mu( \, g \, ; \, e_{\pi(e_n)^{-1}(k_1)} ..., e_{\pi(e_n)^{-1}(k_1)} \, ) \cdot \mu( \, e_n \, ; \, h_1, ..., h_n \, ) \\
			& = & \mu( \, g \, ; \, e_{k_1} ..., e_{k_1} \, ) \cdot \mu( \, e_n \, ; \, h_1, ..., h_n \, ) \\
			& = & g_{(k_1, ..., k_n)} \cdot (h_1 \otimes ... \otimes h_n)
		\end{array}
\end{eq*}
for all $g \in G(n)$, $h_i \in G(k_i)$, $n, k_1, ..., k_n \in \mathbb{N}$. 

Now we can also see the reason why we chose the tensor product notation for the operation $\mu(e_n; \_, ..., \_)$ before. Just like the tensor product of a monoidal category, the definition of this $\otimes$ in $G$ immediately implies an interchange law:
\begin{eq*} \begin{array}{rll}
			(g \cdot g') \otimes (h \cdot h') & = & \mu( \, e_2 \, ; \, gg', \, hh' \, ) \\
			& = & \mu(e_2; g, h) \cdot \mu(e_2; g', h')\\
			& = & (g \otimes h) \cdot (g' \otimes h')
		\end{array}
\end{eq*}
This interaction between the operad and group structures of $G$ places some restrictions on which groups we may build action operads from. One such consequence that we will refer to in later chapters is the following: 

\begin{lem} \label{G0abel} For any action operad $G$, the group $G(0)$ is abelian.  
\end{lem}
\begin{proof}
This lemma is an example of the classic Eckmann-Hilton argument, first put forth in \cite{eckhil}. The idea is that if a set is equipped with two binary operations which obey some form of interchange, and both of them possess the same unit element $e$, then they are in reality a single, commutative operation.

In the case of $G(0)$, we know that it is closed under group multiplication $\cdot$, and the unit of this is the identity element $e_0$. But the operadic multiplication of $G$ includes a map
\begin{eq*} \mu_{n;0,...,0} \, : \, G(n) \times G(0) \times ... \times G(0) \, \longrightarrow \, G(0+...+0) \, = \, G(0) \end{eq*}
which means that tensor products of elements in $G(0)$,
\begin{eq*} g_1 \otimes ... \otimes g_n \quad = \quad \mu_{n;0,...,0}(e_n; g_1, ..., g_n) \end{eq*}
are also in $G(0)$. This $\otimes$ has unit $e_0$ as well; since $(e_2;e_1,e_0)$ is the identity of $G(2)\times G(1)\times G(0)$, the operadic associativity, unitality, and group homomorphism property $\mu$ gives
\begin{eq*} \begin{array}{rll}
			g \otimes e_0 & = & \mu(e_2; g, e_0) \\
			& = & \mu\big( \, e_2 \, ; \, \mu(e_1;g), \, \mu(e_0; -) \, \big) \\
			& = & \mu\big( \, \mu(e_2;e_1,e_0) \, ; \, g \, \big) \\
			& = & \mu(e_1;g) \\
			& = & g
		\end{array}
\end{eq*}
and likewise for $e_0 \otimes g = g$. Moreover, we've just seen that the group multiplication and tensor product of $G$ obey an interchange law. Therefore we can apply the Eckmann-Hilton argument: for any $g, h \in G(0)$,
\begin{eq*} g \otimes h \quad = \quad (g \cdot e_0) \otimes (e_0 \cdot h) \quad = \quad (g \otimes e_0) \cdot (e_0 \otimes h) \quad = \quad g \cdot h \end{eq*}
and also
\begin{eq*} h \otimes g \quad = \quad (e_0 \cdot h) \otimes (g \cdot e_0) \quad = \quad (e_0 \otimes g) \cdot (h \otimes e_0) \quad = \quad g \cdot h \end{eq*}
In other words, tensor product and group multiplication coincide on $G(0)$, and are commutative, so that $G(0)$ is an abelian group.
\end{proof}

Much like standard operads, we can pair action operads with a natural notion of maps between them in order to form a category.

\begin{defn} Given action operads $G, G'$, a \emph{map of action operads} $f: G \to G'$ is a map of operads in $\mathrm{Set}$ whose components $f_n : G(n) \to G'(n)$ are all group homomorphisms, and which preserve all underlying permutations:
\begin{eq*} \begin{tikzcd}
G(n) \ar[rr, "f_n"] \ar[ddr, "\pi^G"'] & & G'(n) \ar[ddl, "\pi^{G'}"] \\
& & \\
& \mathrm{S}_n &
\end{tikzcd} \end{eq*}
The identity maps $\mathrm{id}_G : G \to G$ and the composites of action operad maps $g \circ f : G \to G' \to G''$ in $\mathrm{Op}$ are all well-defined maps of action operads themselves, and so together these constitute a category of action operads and their maps, called $\mathrm{AOp}$.
\end{defn}

There are a couple of operads which trivially have the structure of an action operad. First we have the \emph{terminal operad} $\mathrm{T}$, which has a single operation for each arity, so that $\mathrm{T}(n) = \{ e_n \}$. Each of these sets can be seen as the trivial group, and it follows from this that the $\pi^{\mathrm{T}} : \mathrm{T}(n) \to \mathrm{S}_n$ must be the respective zero maps, the terminal homomorphisms in the category of groups. The action operad condition is then
\begin{eq*} \mu(e_n; e_{k_1}, ..., e_{k_n}) \cdot \mu(e_n; e_{k_1}, ..., e_{k_n}) \quad = \quad \mu(e_n; e_{k_1}, ..., e_{k_n}) \end{eq*}
which is really just
\begin{eq*} e_{k_1 + ... + k_n} \cdot e_{k_1 + ... + k_n} \, = \, e_{k_1 + ... + k_n} \end{eq*}
and hence is trivially true. As its name suggests, the terminal operad is the terminal object in the category $\mathrm{Op}$, but it is also the \emph{initial} object in $\mathrm{AOp}$. This is because for any other $G$ in the category of action operads, the zero homomorphisms $\mathrm{T}(n) \to G(n)$ define the unique map of operads $f: \mathrm{T} \to G$.

On the other hand, the symmetric operad $\mathrm{S}$ itself functions as the terminal object in $\mathrm{AOp}$. Its action operad structure is just given by the standard group multiplications on the $\mathrm{S}_n$, with the identity maps $\mathrm{id}_{\mathrm{S}_n} : \mathrm{S}_n \to \mathrm{S}_n$ functioning as its $\pi_n$. To see terminality, notice that for any other action operad $G$ a valid morphism $f: G \to \mathrm{S}$ in $\mathrm{AOp}$ must obey
\begin{eq*} \pi^{\mathrm{S}} \circ f \, = \, \pi^{G} \quad \implies \quad f \, = \, \pi^{G} \end{eq*}
Thus there is only one map of action operads $G \to \mathrm{S}$: the very underlying permutation structure used to define $G$ in the first place.

There are more interesting examples of action operads we can look at too. For instance, we know that the braid groups $B_n$ have the same presentation as the symmetric groups, except without the relations $b_i^2 = e$. Thus if we take their quotients by these relations we will obtain a sequence of homomorphisms $B_n \to \mathrm{S}_n$, each sending $b_i \mapsto (i \, \, \, i+1)$. This provides a natural way to describe the underlying permutation of any braid, and indeed choosing these maps to form $\pi^B$ gives a valid way of seeing the braid operad as an action operad. Another example can also be built from the so-called ribbon braid groups.

\begin{defn} For each $n \in \mathbb{N}$, the \emph{ribbon braid group} $RB_n$ is the group whose presentation is the same as that of the braid group $B_n$, except with the addition of $n$ new generators $t_1, ..., t_n$, known as the \emph{twists}. These twists all commute with one other, and also commute with all braids except in the following cases:
\begin{eq*} b_i \cdot t_i \quad = \quad t_{i+1} \cdot b_i, \quad \quad \quad \quad \quad b_i \cdot t_{i+1} \quad = \quad t_i \cdot b_i \end{eq*}
The \emph{ribbon braid operad} $RB$ is then the operad made up of these groups in a way that extends the definition of the braid operad. In other words, the identity is still $e_1 \in RB_1$, and the operadic multiplication is built up in stages in exactly the same ways as in \cref{braidop}, but with some additional rules for dealing with twists. With regards to the tensor product, we have that for any twist $t_i \in RB_n$,
\begin{eq*} t_i \quad = \quad e_{i-1} \otimes t \otimes e_{n-i} \end{eq*}
where $t$ is the sole twist in $RB_1$, and for the `block twists' $t_{(m)}$ we again work recursively:
\begin{eq*} t_{(0)} \, = \, e_n, \quad \quad \quad t_{(m+m')} \quad = \quad (t_{(m)} \otimes t_{(m')}) \cdot b_{(m', m)} \cdot b_{(m, m')} \end{eq*}
\end{defn}

Much as the symmetric groups can be represented by crossings of a collection of strings, and the braid groups by braidings of strings, the ribbon braid groups deal with the ways that one can braid together several flat ribbons, including the ability to twist a ribbon about its own axis by 360 degrees.
\begin{center} \begin{tabular}{ccc}
			\begin{tikzpicture}[baseline]
				\node(xl1) at (-0.7,1){};
				\node(xr1) at (-0.3,1){};
				\node(yl1) at (0.3,1){};
				\node(yr1) at (0.7,1){};
				\node(yl2) at (-0.7, -1){};
				\node(yr2) at (-0.3, -1){};
				\node(xl2) at (0.3, -1){};
				\node(xr2) at (0.7, -1){};
				\node(b) at (0,0)[circle,fill=white, minimum size=0.5cm]{};
       				\draw[rounded corners](xl1.north) to (-0.7,0.5) to (0.3,-0.5) to (xl2.south);
       				\draw[rounded corners](xr1.north) to (-0.3,0.5) to (0.7,-0.5) to (xr2.south);
				\begin{pgfonlayer}{bg}
				\draw[rounded corners](yl1.north) to (0.3, 0.5) to (-0.7, -0.5) to (yl2.south);
				\draw[rounded corners](yr1.north) to (0.7, 0.5) to (-0.3, -0.5) to (yr2.south);
    				\end{pgfonlayer}
				\draw(xl1.north) to (xr1.north);
				\draw(xl2.south) to (xr2.south);
				\draw(yl1.north) to (yr1.north);
				\draw(yl2.south) to (yr2.south);
			\end{tikzpicture} & \quad \quad \quad \quad \quad \quad \quad &
			\begin{tikzpicture}[baseline]
				\node(xl1) at (-0.2,1){};
				\node(xr1) at (0.2,1){};	
				\node(xl2) at (-0.2, -1){};
				\node(xr2) at (0.2, -1){};
				\draw[rounded corners](xl1.north) to (-0.2,0.4) to (0.2, 0.3) to (0.2, -0.3) to (-0.2, -0.4) to (xl2.south);	
       				\draw[rounded corners](xr1.north) to (0.2,0.4) to (-0.2, 0.3) to (-0.2, -0.3) to (0.2, -0.4) to (xr2.south);
				\draw(xl1.north) to (xr1.north);
				\draw(xl2.south) to (xr2.south);	
			\end{tikzpicture} \\
			$b$ & & $t$ 
\end{tabular} \end{center}
This operad $RB$ is also clearly an action operad, since we can just define $\pi^{RB} : RB_n \to \mathrm{S}_n$ to act like $\pi^B$ on any braids, at which point the fact that $\pi(t) \in S_1 = \{e_1\}$ will automatically take care of the twists. To learn more about the ribbon braids and their operads, see Natalie Wahl's thesis \cite{ribbon1} on the subject, or her subsequent paper with Paolo Salvatore \cite{ribbon2}.

The fact that the ribbon braid operad seems to contain the whole of the braid operad is the key to easily understanding its operadic structure. We can formalise this kind of relationship in the following way:

\begin{defn} An action operad $G$ is said to be a \emph{sub action operad} of some other action operad $G'$ if for all $n \in \mathbb{N}$ we have
\begin{eq*} G(n) \le G'(n), \quad \quad \quad \mu^{G}(g;h_1, ..., h_n) \, = \, \mu^{G'}(g;h_1, ..., h_n), \quad \quad \quad \pi^{G}(g) \, = \, \pi^{G'}(g) \end{eq*}
\end{defn}

The most important example of sub action operads are those of the symmetric operad, $\mathrm{S}$. This is because \cref{actop} itself makes explicit reference to the symmetric groups, and so every action operad will end up being related to some sub-operad of $\mathrm{S}$:

\begin{defn} For any action operad $G$, the images of the underlying permutation maps $\pi^G_n : G(n) \to \mathrm{S}_n$ naturally form an action operad $\mathrm{im}(\pi^G)$, where
\begin{itemize}
\item the sets of operations are the images of $G$'s sets of operations under the homomorphisms $\pi^G$:
\begin{eq*} \mathrm{im}(\pi^G)(n) \quad := \quad \mathrm{im}(\pi^G_n) \end{eq*}
\item the underlying permutation maps are the evident inclusions:
\begin{eq*} \pi^{\mathrm{im}(\pi^G)}_n \, : \, \mathrm{im}(\pi^G_n) \hookrightarrow \mathrm{S}_n \end{eq*}
\item the operad multiplication is the appropriate restriction of the multiplication of $\mathrm{S}$:
\begin{eq*} \mu^{\mathrm{im}(\pi^G)}( \, g \, ; \, h_1, ..., h_n \, ) \quad := \quad \mu^{\mathrm{S}}( \, g \, ; \, h_1, ..., h_n \, ) \end{eq*}
\end{itemize}
Clearly this $\mathrm{im}(\pi^G)$ is a sub action operad of the symmetric operad $\mathrm{S}$, and so we will call it the \emph{underlying permutation operad} of $G$.
\end{defn} 

For example, consider the action operad $B$ we just saw in \cref{braidop}. For a given $n$, the braid group $B_n$ is generated by $n-1$ elementary braids. But the underlying permutations of these braids are just the $n-1$ elementary transpositions which generate the symmetric group $\mathrm{S}_n$, and so the underlying permutation maps $\pi^{B}_n : B_n \to \mathrm{S}_n$ are all surjective. Thus the underlying permutation operad of $B$ is just the whole symmetric action operad, $\mathrm{im}(\pi^B) = \mathrm{S}$.

It is even easier to see that $\mathrm{S}$ itself will have underlying permutations $\mathrm{S}$, as the maps $\pi^{\mathrm{S}}_n = \mathrm{id} : \mathrm{S}_n \to \mathrm{S}_n$ are obviously surjective. Similarly, the trivial operad $\mathrm{T}$ is also its own underlying permutation action operad, as the image of the homomorphisms $\pi^{\mathrm{T}}_n : \{ e \} \to \mathrm{S}_n$ are trivial. Faced with rather dull examples like these, it might be tempting to try and construct some new action operads with more exotic underlying permutations, like maybe the alternating groups $\mathrm{A}_n \subset \mathrm{S}_n$. But it turns out that this is not possible; when it come to their underlying permutation operad, action operads come in exactly two flavours, as seen in \cite{groupop}.

\begin{defn} Let $G$ be an action operad where $\mathrm{im}(\pi)(n)$ is the trivial group for each $n \in \mathbb{N}$. Then we say that $G$ is \emph{non-crossed}, since its operad multiplication will be a true group homomorphism:
\begin{eq*} \begin{array}{rll}
			\mu( \, gg' \, ; \, h_1 h'_1, ..., h_n h'_n \, ) & = & \mu( \, g \, ; \, h_{\pi(g')^{-1}(1)}, ..., h_{\pi(g')^{-1}(n)} \, ) \mu( \, g' \, ; \, h'_1, ..., h'_n \, ) \\
			& = & \mu( \, g \, ; \, h_1, ..., h_n \, ) \mu( \, g' \, ; \, h'_1, ..., h'_n \, ) \\
		\end{array}
\end{eq*}
Likewise, a \emph{crossed} action operad will refer to any that has a non-trivial underlying permutation operad.
\end{defn}

\begin{lem}\label{surjortriv} An action operad $G$ is crossed if and only if it has surjective underlying permutation maps $\pi_n : G(n) \to \mathrm{S}_n$. In other words, the underlying permutations operad of $G$ must be either the trivial operad $\mathrm{T}$ or the symmetric operad $\mathrm{S}$.
\end{lem}
\begin{proof}
Let $\mathrm{im}(\pi)$ be the underlying permutation operad of $G$, and let us assume that $G$ is crossed, so that $\mathrm{im}(\pi)$ is not the trivial operad. This means that for some natural number $n$, the $n$-ary operations of $\mathrm{im}(\pi)$ include at least one permutation $\sigma$ which is not the identity element of the relevant symmetric group $\mathrm{S}_n$. Put another way, there must be some $\sigma$ and some $1 \le i \le n$ for which $\sigma(i) \neq i$. But now consider evaluating the expression
\begin{eq*} \mu^{\mathrm{im}(\pi)}( \, \sigma \, ; \, e_0, ..., e_0, e_1, e_0, ...., e_0, e_1, e_0, ..., e_0 \, ) \end{eq*}
where the $e_1$'s above are appearing in the $i$th and $\sigma(i)$th coordinates, which we know are distinct. From the definitions of $\mathrm{im}(\pi)(n)$ and of operad multiplication in $\mathrm{S}$, this permutation is really just
\begin{eq*} \mu^{\mathrm{S}}( \, \sigma \, ; \, e_0, ..., e_0, e_1, e_0, ...., e_0, e_1, e_0, ..., e_0 \, ) \quad = \quad (1 \, 2) \end{eq*}
the only non-identity element of $\mathrm{S}_2$. This proves that the map $\pi_2 : G(2) \to \mathrm{S}_2$ is indeed surjective, but more than that it shows that $\mathrm{im}(\pi)$ must contain every possible adjacent transposition, since for any $m \in \mathbb{N}$ we have
\begin{eq*} \begin{array}{rll}
			& & \mu^{\mathrm{im}(\pi)}( \, e_n \, ; \, e_1, ..., e_1, (1 \, 2), e_1, ...., e_1 \, ) \\
			& = & \mu^{\mathrm{S}}( \, e_n \, ; \, e_1, ..., e_1, (1 \, 2), e_1, ...., e_1 \, ) \\
			& = & (m \, \, m+1) \in \mathrm{S}_n
		\end{array}
\end{eq*}
Then because adjacent transpositions generate the symmetric groups $\mathrm{S}_n$, it follows that every permutation is actually an operation in $\mathrm{im}(\pi)$, so that it is really just the full symmetric operad $\mathrm{S}$. Thus by only assuming that our action operad $G$ was crossed, we have shown that all of the maps $\pi_n$ must be surjective.
\end{proof} 

\section{$G$-Operads}

The most important feature of action operads, and the reason for giving them that name in the first place, is that they are able to `act' on other operads. The way that this is done for operads in the category of sets is a direct generalisation of the more familiar notion of group actions on sets. Before we begin then, we should recall what exactly is meant by an action of a group on a set.

\begin{defn} For any set $S$ and group $H$, a \emph{(right) action} of $H$ on $S$ is a function $\cdot : S \times H \to S$ which respects the group multiplication of $H$. That is,
\begin{eq*} x \cdot e \, = \, x, \quad \quad \quad \quad x \cdot (hh') \, = \, (x \cdot h) \cdot h' \end{eq*}
for any $x \in S$, $h,h' \in H$, and $e$ the identity of $H$. The set $S$ equipped with this action is known as an \emph{$H$-set}.
\end{defn}

In more categorical terms, an $H$-set is simply a functor $\mathrm{B}H \to \mathrm{Set}$. Here the notation $\mathrm{B}H$ refers to the category that has a single object $\ast$, and a homset $\mathrm{B}H(\ast, \ast)$ which is just isomorphic to the group $H$ when viewed as a monoid under composition. The bridge between these two perspectives is that if the functor $\mathrm{B}H \to \mathrm{Set}$ sends $\ast \mapsto S$, then the rest of the functor constitutes a monoid homomorphism $H \to \mathrm{Set}(S, S)$. We can then see this as a special kind of function $S \times H \to S$, via the (right) tensor-hom adjunction for the category of sets:
\begin{eq*} \mathrm{Set}(A \times B, C) \quad \cong \quad \mathrm{Set}\big( \, B, \, \mathrm{Set}(A,C) \, \big) \end{eq*}

Now we take the idea of $H$-sets and generalise it to the domain of operads and action operads.

\begin{defn} \label{Gopset} Let $G$ be an action operad. Then a \emph{$G$-operad} in the category of sets is an operad $O$ in $\mathrm{Set}$, equipped with an action of the group $G(n)$ on the set $O(n)$ for each $n \in \mathbb{N}$, which respect the operadic multiplications of $G$ and $O$ in the following sense:
\begin{eq*} \mu^O( \, x \cdot g \, ; \, y_1 \cdot h_1, ..., y_n \cdot h_n \, ) \quad = \quad \mu^O(x; y_{\pi(g)^{-1}(1)}, ..., y_{\pi(g)^{-1}(n)}) \cdot \mu^G(g; h_1, ..., h_n) \end{eq*}
Additionally, if a map of operads $f: O \to O'$ between two $G$-operads preserves all of the actions, so that the diagrams
\begin{eq*} \begin{tikzcd}
O(n) \times G(n) \ar[rr, "f_n \times \mathrm{id}_{G(n)}"] \ar[dd] & & O'(n) \times G(n) \ar[dd] \\
& & \\
O(n) \ar[rr, "f_n"] & & O'(n)
\end{tikzcd} \end{eq*}
commute for each $n \in \mathbb{N}$, then we say that $f$ is a \emph{map of $G$-operads} in $\mathrm{Set}$. Together $G$-operads of sets and their maps form a category, which we shall call $G\mbox{-}\mathrm{Op}$.
\end{defn}

It is a well-known fact that every group $H$ can itself be seen as an $H$-set, with the action $H \times H \to H$ being given by multiplication on the right. The equivariance axiom above has been chosen in such a way that we can immediately conclude an analogous result about operads. That is, any action operad $G$ is also $G$-operad with actions $G(n) \times G(n) \to G(n)$ given by multiplication on the right, because under those conditions the defining equation of a $G$-operad simply becomes the defining equation for an action operad.

For certain specific $G$, the $G$-operads are already well-studied objects. If we take our action operad $G$ to be the symmetric operad $\mathrm{S}$, then since the map $\pi^{\mathrm{S}}$ is trivial we arrive at a rather straightforward variety of $G$-operads, those whose equivariance is given by
\begin{eq*} \mu^O( \, x \cdot \sigma \, ; \, y_1 \cdot \tau_1, ..., y_n \cdot \tau_n \, ) \quad = \quad \mu^O(x; y_{\sigma^{-1}(1)}, ..., y_{\sigma^{-1}(n)}) \cdot \mu^{\mathrm{S}}(\sigma; \tau_1, ..., \tau_n) \end{eq*}
These $\mathrm{S}$-operads are nowadays generally known as \emph{symmetric operads}, or sometimes \emph{permutative operads}. However, May's original definition \cite{gils} for `operads' was actually this symmetric version, and so some authors prefer to reserve that term for these structures, instead calling the subject of \cref{opdef} `planar operads', or `operads without permutation'. This should give an idea of just how important these symmetric operads really are. Prominent examples include the `little cubes', `little discs', and similar operads which helped motivate the development of operad theory.
There are also \emph{braided operads}, which are $B$-operads for the braid operad $B$ --- these appear in the work of Zbigniew Fiedorowicz \cite{braidedop}.

As one might expect, the notion of $G$-operads can be extended from $\mathrm{Set}$ to work in other symmetric monoidal categories $(C, \otimes, I)$, by instead working with the operads within that category. Since we are aiming to connect action operads to symmetric and braided monoidal categories, the particular context we will be interested in is $\mathrm{Cat}$, the category of (small) categories. Here the concept of a group action is particularly simple --- it is just like a group action on sets, applied to both the objects and morphism of a category.

\begin{defn} Let $X$ be a category, and $H$ a group which we will also think of as a discrete category. Then a \emph{(right) action} of $H$ on $X$ is a functor $\cdot : X \times H \to X$ which respects the group multiplication of $H$. That is,
\begin{eq*} \begin{array}{rclcrcl}
			x \cdot e & = & x, & \quad \quad & x \cdot (hh') & = & (x \cdot h) \cdot h'  \\
			f \cdot \mathrm{id}_{e} & = & f, & \quad \quad & f \cdot \mathrm{id}_{hh'} & = & (f \cdot\mathrm{id}_{h}) \cdot \mathrm{id}_{h'} 
		\end{array}
\end{eq*}
for any objects $x$ and morphisms $f$ of $X$, and elements $h,h' \in H$ with $e$ the identity.
\end{defn}

As before, we can view a group action like this as a functor $\mathrm{B}H \to \mathrm{Cat}$ where the sole object $\ast$ of $\mathrm{B}H$ is sent to the category $X$ in question. This is because these are equivalent to monoid homomorphisms $H \to \mathrm{Cat}(X, X)$, which we can see as functors $X \times H \to X$ using the fact that $\mathrm{Cat}$ is copowered (on the right) over $\mathrm{Set}$:
\begin{eq*} \mathrm{Cat}(X \times S, Y) \quad \cong \quad \mathrm{Set}\big( \, S, \, \mathrm{Cat}(X,Y) \, \big) \end{eq*}
Here $S$ is a set which again we identify with a discrete category.

\begin{defn} \label{Gopcat} Let $G$ be an action operad. Then a \emph{$G$-operad} in $\mathrm{Cat}$ is an operad $O$ in $(\mathrm{Cat}, \times, 1)$, equipped with an action of the group $G(n)$ on the category $O(n)$ for each $n \in \mathbb{N}$, which respect the operadic multiplications of $G$ and $O$ via a higher dimensional version of the equation in \cref{Gopset}. Specifically, we require that the diagram
\begin{eq*} \begin{tikzcd}
O(n) \times G(n) \times \prod \big( \, O(k_i) \times G(k_i) \, \big) \ar[d, "\mathrm{id} \times {(\pi, \mathrm{id})} \times \mathrm{id}"'] \ar[rr] & & O(n) \times \prod O(k_i) \ar[ddd, "\mu^O"] \\
O(n) \times \mathrm{S}_n \times G(n) \times \prod \big( \, O(k_i) \times G(k_i) \, \big) \ar[d, "\beta"'] & & \\
O(n) \times \mathrm{S}_n \times \prod O(k_i) \times G(n) \times \prod G(k_i) \ar[d, "\tilde{\mu}^O \times \mu^G"'] \ar[d, "\tilde{\mu}^O \times \mu^G"'] & & \\
O(k_1 + ... + k_n) \times G(k_1 + ... + k_n) \ar[rr] & & O(k_1 + ... + k_n)
\end{tikzcd} \end{eq*}
commutes for all $n, k_1, ..., k_n \in \mathbb{N}$. Here we are using $\tilde{\mu}^O$ to refer to the obvious functor $\mathrm{S}_n \times O(n) \times \prod O(k_i) \to O(k_1 + ... + k_n)$ which acts like $\mu^O$ but with suitably permuted inputs:
\begin{eq*} \tilde{\mu}^O( \, \sigma, x \, ; \, y_1, ..., y_n \, ) \quad := \quad \mu^O( \, x \, ; \, y_{\sigma^{-1}(1)}, ..., y_{\sigma^{-1}(1)} \, ) \end{eq*}
\end{defn}

The easiest way to produce examples of $G$-operads in $\mathrm{Cat}$ is to simply build them out of existing operads in the category of sets. In particular, if we design our categories of operations so that the morphisms are determined entirely by their source and target, then a single operad in $\mathrm{Set}$ will suffice to create one of these new higher dimensional operads.

\begin{defn}\label{Edef} For any monoid $M$, we will define its \emph{translation category} $\mathrm{E}M$ to be the (monoidal) category whose objects are the elements of the monoid $M$, and whose morphisms consist of a unique isomorphism between each pair of objects. Also, for any monoid homomorphisms $h: M \to M'$ we can define a functor
\begin{eq*} \begin{array}{rlrll}
		\mathrm{E}h & : & \mathrm{E}M & \to & \mathrm{E}M' \\
		& : & m & \mapsto & h(m) \\
		& : & m \to m' & \mapsto & h(m) \to h(m')
		\end{array}
\end{eq*}
This definition of $\mathrm{E}h$ obviously respects composition and identities, and so together with $\mathrm{E}M$ it describes a functor $\mathrm{E}: \mathrm{Mon} \to \mathrm{Cat}$.

Likewise, for any operad $O$ in the category of sets we can define its \emph{translation operad} $\mathrm{E}O$ to be the operad in $\mathrm{Cat}$ given by the data
\begin{eq*} (\mathrm{E}O)(n) \, := \, \mathrm{E}\big( O(n) \big), \quad \quad \quad 1^{\mathrm{E}O} \, = \, \mathrm{E}(1^{O}), \quad \quad \quad \mu^{\mathrm{E}O} \, = \, \mathrm{E}(\mu^O) \end{eq*}
For each of the coherence conditions which $\mathrm{E}O$ must satisfy in order to be a well-defined operad in $\mathrm{Cat}$, we can obtain them from the corresponding conditions that make $O$ an operad in $\mathrm{Set}$, by simply applying the functor $\mathrm{E}$ everywhere.
 \end{defn}

$\mathrm{E}O$ can be seen as a categorified version of the operad $O$. That is, it may live in $\mathrm{Cat}$ rather than $\mathrm{Set}$, but in many other respects it behaves the same way that $O$ does. Of particular interest to us is what this means in the case when $O$ is really an action operad $G$. We saw earlier that any $G$ is always a $G$-operad in the category of sets, with an action given by group multiplication. The categorified variant of this statement is the following:

\begin{lem} For any action operad $G$, the translation operad $\mathrm{E}G$ is a $G$-operad in $\mathrm{Cat}$, with actions
\begin{eq*} \begin{array}{rll}
		\mathrm{E}G(n) \times G(n) & \to & \mathrm{E}G(n) \\
		(g, h) & \mapsto & gh \\
		(g \to g', \mathrm{id}_h) & \mapsto & gh \to g'h
		\end{array}
\end{eq*}
\end{lem}

The proof of this fact can be found in \cite{operadborel}.

\section{Operad algebras}

As with many mathematical structures, we are not merely interested in operads for their own sake, but also for their algebras.

\begin{defn} \label{opalg} Let $O$ be an operad in the symmetric monoidal category $(C, \otimes, I)$. Then an \emph{algebra} of $O$ is an object $X$ in $C$, equipped with a family of morphisms $\alpha_n : O(n) \otimes X^{\otimes n} \to X$, $n \in \mathbb{N}$ called the action of $O$ on $X$, which obey axioms that mirror those needed to define an operad. In other words, we a unitality condition
\begin{eq*} \begin{tikzcd}
I \otimes  X \ar[dd, "1 \otimes \mathrm{id}_{X}"'] \ar[ddrr, "l_X"] & & \\
& & \\
O(1) \otimes X \ar[rr, "\alpha_{1}"'] & &  X
\end{tikzcd} \end{eq*}
and then for all $n, k_1, ...,  k_n \in \mathbb{N}$ we have an associativity condition,
\begin{eq*} \begin{tikzcd}
O(n) \otimes \prod O(k_i) \otimes \prod X^{\otimes k_i} \ar[ddr, "\mu \otimes \mathrm{id}"] \ar[dd, "\beta"'] & \\
& \\
O(n) \otimes \prod \big( \, O(k_i) \otimes X^{\otimes k_i} \, \big) \ar[dd, "\mathrm{id} \, \otimes \prod \alpha"'] & O(k_1+...+k_n) \otimes X^{\otimes (k_1 + ... + k_n)} \ar[dd, "\alpha"] \\
& \\
O(n) \otimes X^{\otimes n} \ar[r, "\alpha"] & X
\end{tikzcd} \end{eq*}
As one might expect, a \emph{map of algebras} $f: (X, \alpha^X) \to (Y, \alpha^Y)$ between two algebras of $O$ is then simply a map between their underlying objects, $f: X \to Y$, which preserves this algebra structure:
\begin{eq*} \begin{tikzcd}
O(n) \times X^n \ar[rr, "\mathrm{id}^{O(n)} \times f^n"] \ar[dd, "\alpha^X"'] & & O(n) \times Y^n \ar[dd, "\alpha^Y"] \\
& & \\
X \ar[rr, "f"] & & Y
\end{tikzcd} \end{eq*}
Together these form the category $O\mathrm{Alg}$ of all $O$-algebras and their maps.
\end{defn}

When $O$ is an operad in $\mathrm{Set}$, an algebra of $O$ is simply a realisation of the elements of the $O(n)$ as actual $n$-ary operations on some set. A similar statement is true in any concrete category, though with extra structure or restrictions depending on the nature of $(C, \otimes, I)$. Also when $O$ is an operad in $\mathrm{Cat}$ we can upgrade the category $O\mathrm{Alg}$ into a 2-category, by simply adding in monoidal natural transformations as the 2-morphisms between algebra maps. We will use the notation $O\mathrm{Alg}_{S}$ for this 2-dimensional structure to indicate that everything involved is still strict, unlike the weaker pseudoalgebras which have been studied elsewhere \cite{ogge}.

As we've seen many times already, when the operad we are working with is actually an action operad, the presence of the additional group structure will cause something more interesting to happen. Specifically, the operadic multiplication $\mu^G(e_n; \, \_ \, , ..., \, \_ \,)$ can be interpreted as a tensor product, and so the operad algebras of $G$ will end up inheriting a monoidal structure of their own.

\begin{lem} Let $G$ be an action operad, and $X$ an algebra of $G$ in the category of sets. Then $X$ is a monoid with respect to the operation
\begin{eq*} x_1 \otimes ... \otimes x_n \quad := \quad \alpha(e_n; x_1, ..., x_n) \end{eq*}
and there exists a forgetful functor $G\mathrm{Alg} \to \mathrm{Mon}$ sending the algebras of $G$ to this underlying monoid structure.

Similarly, let $Y$ be an algebra of $\mathrm{E}G$ in $\mathrm{Cat}$. Then $Y$ is a strict monoidal category with respect to the operation
\begin{eq*} y_1 \otimes ... \otimes y_n \quad := \quad \alpha(e_n; y_1, ..., y_n), \quad \quad \quad f_1 \otimes ... \otimes f_n \quad := \quad \alpha(e_n; f_1, ..., f_n) \end{eq*}
and there is a forgetful 2-functor $\mathrm{E}G\mathrm{Alg}_{S} \to \mathrm{MonCat}_{S}$ sending these algebras to their underlying strict monoidal structure.
\end{lem}
\begin{proof}
We'll start by checking that above definition on $X$ makes sense. Firstly, for any element $x \in X$ we want the one-fold tensor product of $x$ with itself to just be $x$ again. This is ensured by the unitality of the action $\alpha^X$, which says that $\alpha(e_1; x) = x$. Next, we need to make sure that the tensor products for each arity are all compatible with each other, which follows from the associativity axiom for $\alpha^X$:
\begin{eq*} \begin{array}{rl}
			& (x_1 \otimes ... \otimes x_{k_1}) \otimes ... \otimes (x_{k_1 + ... + k_{n-1}+1} \otimes ... \otimes x_{k_1 + ... + k_n}) \\
			= & \alpha\big( \, e_n \, ; \, \alpha(e_{k_1}; x_1, ..., x_{k_1}), \, ..., \, \alpha(e_{k_n}; x_{k_1 + ... + k_{n-1}+1}, ... , x_{k_1 + ... + k_n}) \, \big) \\
			= & \alpha\big( \, \mu(e_n; e_{k_1}, ..., e_{k_n}) \, ; \, x_1, ..., x_{k_1}, ..., x_{k_1 + ... + k_{n-1}+1}, ... , x_{k_1 + ... + k_n} \, \big) \\
			= & \alpha( \, e_{k_1+...+k_n} \, ; \, x_1, ... , x_{k_1 + ... + k_n} \, ) \\
			= & x_1 \otimes ... \otimes x_{k_1 + ... + k_n}
		\end{array}
\end{eq*}
Perhaps unsurprising, this means that the associativity axiom also forces the tensor product to be associative. Finally, a special case of the above --- where $n=2$ and the $k_i$ are $0$ and $1$ --- shows that the empty tensor product $\alpha(e_0; -)$ acts as the unit of $\otimes$. Thus $X$ is indeed a well-defined monoid under the tensor product that comes from its action. Moreover, since all algebra maps $f: X \to X'$ preserve actions they will also preserve this monoid structure.
\begin{eq*} f(x \otimes x') \quad = \quad f\big( \, \alpha^X(e_2; x, x') \, \big) \quad = \quad \alpha^{X'}\big( \, e_2 \, ; \, f(x), f(x') \, \big) \quad = \quad f(x) \otimes f(x') \end{eq*}
Therefore if we forget all of the features of our $G$-algebras other than the tensor product, what we are left with are monoids and monoid homomorphisms, and this defines an obvious functor $G\mathrm{Alg} \to \mathrm{Mon}$.

Turning now to the category $Y$, if we think of all of the functors in the unitality and associativity axioms for $\alpha^Y$ as acting just on objects, the exact same arguments we employed above will show that $(\mathrm{Ob}(Y), \otimes)$ is well-defined monoid. Likewise, restricting our view to morphisms will let us prove that $(\mathrm{Mor}(Y), \otimes)$ is a monoid, and then functoriality of $\alpha^Y$ tells us that we can stitch these two tensor products together into a single functor $\otimes : Y \times Y \to Y$.
\begin{eq*} \begin{array}{rll}
			(f : x \to y) \otimes (f': x' \to y') & = & \alpha(e_2 ; f, f') : \alpha(e_2; x, x') \to \alpha(e_2; y, y') \\
			& = & f \otimes f' : x \otimes x' \to y \otimes y'
		\end{array}
\end{eq*}
Thus $Y$ as a whole has a tensor product, and because it comes from a monoid at both levels it will be \emph{strictly} associative and unital. Therefore $\mathrm{E}G$-algebras are strict monoidal categories, and since any algebra map $F: Y \to Y'$ will preserve this monoidal structure for the same reason we had before, there is an associated forgetful 2-functor $\mathrm{E}G\mathrm{Alg}_S \to \mathrm{MonCat}_{S}$ onto the 2-category of strict monoidal categories.
\end{proof}

In general, the algebras of $G$ and $\mathrm{E}G$ will have a lot more structure to them than just this tensor product. For example, any algebra for the symmetric operad $\mathrm{S}$ will have an extra binary operation coming from the elementary permutation in $\mathrm{S}_2$:
\begin{eq*} \begin{array}{rll}
		X \times X & \to & X \\
		(x, x') & \mapsto & \alpha\big( \, (1 \, 2) \, ; \, x, x' \, \big) \\
		\end{array}
\end{eq*}
However, the rules that govern operad algebras do not put any extra constraints on these operations, which makes the category $\mathrm{SAlg}$ far too broad to say anything useful about. The problem is that by using the concept of a standard operad algebra, we are ignoring the group multiplication of our action operads, since this is not something that every operad of sets can be expected to have.

What we need is a notion for algebras of a $G$-operad. Of course, as operads themselves any $G$-operad will already have algebras in the sense of \cref{opalg}, but in general these won't respect the $G$-operadic actions, which anything worthy of the name `$G$-operad algebra' should do. We can fix this by simply demanding that the action $\alpha$ coequalises certain maps, chosen in a way which will force the equivariance to hold.

\begin{defn} \label{Gopalgdef} For any operad $O$ in $\mathrm{Set}$ or $\mathrm{Cat}$, a \emph{$G$-operad algebra} $X$ of $O$ is just an operad algebra of $O$ whose actions $\alpha_n : O(n) \times X^n \to X$ coequalise two maps $O(n) \times G(n) \times X^n \to O(n) \times X^n$, one coming from the action of $G(n)$ on $O(n)$, and the other from the reordering of $X^n$ by the underlying permutations of $G(n)$.

More precisely, recall that the symmetric monoidal structures of $(\mathrm{Set}, \times, 1)$ and $(\mathrm{Cat}, \times, 1)$ provide us with several different isomorphisms $\beta : X^n \to X^n$. Indeed, there will be one for each permutation in $\mathrm{S}_n$, and this gives rise to a natural embedding of monoids $\mathrm{S}_n \to \mathrm{Set}(X^n, X^n)$ or $\mathrm{S}_n \to \mathrm{Cat}(X^n, X^n)$. We can then use the (left) copower isomorphisms of the given categories to turn these embeddings into maps $\tilde{\beta} : \mathrm{S}_n \times X^n \to X^n$. With this notation, we define a $G$-operad algebra of $O$ to be any operad algebra $X$ of $O$ for which the following two composites are equal:
\begin{eq*} \begin{tikzcd}
O(n) \times G(n) \times X^n \ar[ddr, "\mathlarger{\cdot} \, \times \mathrm{id}_{X^n}"'] \ar[rr, "\mathrm{id}_{O(n)} \times \pi \times \mathrm{id}_{X^n}"] & & O(n) \times \mathrm{S}_n \times X^n \ar[ddl, "\mathrm{id}_{O(n)} \times \tilde{\beta}"] \\
& & \\
& O(n) \times X^n \ar[dd, "\alpha"] & \\
& & \\
& X &
\end{tikzcd} \end{eq*}
Since we don't really have a reason to care about the non-$G$-operad algebras of a $G$-operad $O$, from now on we will use the notation $O\mathrm{Alg}$ to refer to this new category instead.
\end{defn}

So, what are the algebras of an action operad $G$ like in this context? Unfortunately, this version of $G\mathrm{Alg}$ is even less interesting than the one before; it is simply the category of monoids, $\mathrm{Mon}$. To see this, notice that if we view an action operad $G$ as a $G$-operad with multiplication for its action, then the equivariance condition for an algebra $X$ will become
\begin{eq*} \begin{array}{rll}
			\alpha(g;x_1, ..., x_n) & = & \alpha( \, e_n \cdot g \, ; \, x_1, ..., x_n \, ) \\
			& = & \alpha( \, e_n \, ; \, x_{\pi(g)^{-1}(1)}, ..., x_{\pi(g)^{-1}(n)} \, ) \\
			& = & x_{\pi(g)^{-1}(1)} \otimes ... \otimes x_{\pi(g)^{-1}(n)}
		\end{array}
\end{eq*}
That is, the action $\alpha^X$ is entirely determined by the tensor product of $X$, and as we've already seen that this is unrestrained by the axioms for an operad algebra, so $X$ is just an undecorated monoid. 

However, the 2-category $\mathrm{E}G\mathrm{Alg}_S$ is far more exciting. Sure, the same argument we've just used for $G$ will ensure that the action reduces to the tensor product on objects, but on morphisms the underlying permutative structure $\pi$ will finally come into play. As an example, for the symmetric operad $\mathrm{S}$ we know that any $\mathrm{ES}$-algebra $X$ must contain action morphisms of the form
\begin{eq*} \begin{array}{rcrcl}
			\alpha\big( \, e_2 \to (1 \, 2) \, ; \, \mathrm{id}_x, \mathrm{id}_y \, \big) & : & \alpha\big( \, e_2 \, ; \, x, y \, \big) & \to & \alpha\big( \, (1 \, 2) \, ; \, x, y \, \big) \\
			& : & x \otimes y & \to & y \otimes x
			\end{array}
\end{eq*}
for all objects $x$,$y$. Indeed, it is not to difficult to see that these morphisms are really the symmetries $\beta_{x,y}$ for a strict symmetric monoidal category. For instance, the relation $\beta_{y,x} \circ \beta_{x,y} = \mathrm{id}_{x \otimes y}$ comes from the $\mathrm{S}$-operad algebra equivariance axiom, and the fact that the functor $\alpha$ preserves composition:
\begin{eq*} \begin{array}{rl}
			& \alpha\big( \, e_2 \to (1 \, 2) \, ; \, \mathrm{id}_y, \mathrm{id}_x \, \big) \circ \alpha\big( \, e_2 \to (1 \, 2) \, ; \, \mathrm{id}_x, \mathrm{id}_y \, \big) \\[\medskipamount]
			= & \alpha\big( \, (1 \, 2) \cdot (1 \, 2) \to e_2 \cdot (1 \, 2) \, ; \, \mathrm{id}_y, \mathrm{id}_x \, \big) \circ \alpha\big( \, e_2 \to (1 \, 2) \, ; \, \mathrm{id}_x, \mathrm{id}_y \, \big) \\[\medskipamount]
			= & \alpha\big( \, (1 \, 2) \to e_2 \, ; \, \mathrm{id}_x, \mathrm{id}_y \, \big) \circ \alpha\big( \, e_2 \to (1 \, 2) \, ; \, \mathrm{id}_x, \mathrm{id}_y \, \big) \\[\medskipamount]
			= & \alpha\big( \, e_2 \to (1 \, 2) \to e_2 \, ; \, \mathrm{id}_x \circ \mathrm{id}_x, \mathrm{id}_y \circ \mathrm{id}_y \, \big) \\[\medskipamount]
			= & \alpha( \mathrm{id}_{e_2} ; \mathrm{id}_x, \mathrm{id}_y ) \\
			= & \mathrm{id}_{\alpha(e_2 ; x, y )} \\
			= & \mathrm{id}_{x \otimes y}
			\end{array}
\end{eq*} 
The questions that should follow from this observation are obvious. What about the braid operad $B$? Are the objects of $\mathrm{E}B\mathrm{Alg}_S$ braided monoidal categories, in the same way that those of $\mathrm{ES}\mathrm{Alg}_S$ are symmetric monoidal? What about the algebras of the ribbon braid operad, what sort of monoidal category are they? And do the $\mathrm{S}$-operad algebras of $\mathrm{ES}$ have any additional structure, other than their symmetries?

It turns out that there is a theorem which answers all of these questions at once, for all possible $G$. To properly state it though, we'll need some new terminology. 

\begin{defn} \label{GRmon} A \emph{$(\mathcal{G}, \mathcal{R})$-monoidal category} is a strict monoidal category $X$, equipped with a set of natural isomorphisms
\begin{eq*} \mathcal{G} \, = \, \big\{ \, (f; \pi_f) \, : \, x_1 \otimes ... \otimes x_n \xrightarrow{f}  x_{\pi_f^{-1}(1)} \otimes ... \otimes x_{\pi_f^{-1}(n)} \, \big\} \end{eq*}
which are subject to some set of relations $\mathcal{R}$. Each of the relations $r \in \mathcal{R}$ will be of the form
\begin{eq*} r \, \, : \, \, (f_{1, 1} \otimes ... \otimes f_{1, k_1}) \circ ... \circ (f_{n, 1} \otimes ... \otimes f_{n, k_n})  \, = \, (f'_{1, 1} \otimes ... \otimes f'_{1, k'_1}) \circ ... \circ (f'_{n', 1} \otimes ... \otimes f'_{n', k'_{n'}}) \end{eq*}
for its own collection of elements $f_{1, 1}, ..., f_{n, k_n}, f'_{1, 1}, ..., f'_{n', k_n'} \in \mathcal{G}$ and indexing variables $n, n', k_1, ..., k_n, k'_1, ..., k'_n \in \mathbb{N}$. 
\end{defn}

\begin{defn} \label{Gmon} Let $G$ be an action operad. Then a \emph{$G$-monoidal category} will refer to the notion of $(\mathcal{G}, \mathcal{R})$-monoidal category we get from $G$ by setting
\begin{eq*} \mathcal{G} \, = \, \big\{ \, \big( \, g \, ; \, \pi(g) \, \big) \, : \, \forall \, g \in G \, \big\} \end{eq*}
and having $\mathcal{R}$ contain one element $(r; n, n', \underline{k}, \underline{k'}, \underline{g}, \underline{g'})$ for each relation
\begin{eq*} r \, \, : \, \, (g_{1, 1} \otimes ... \otimes g_{1, k_1}) \cdot ... \cdot (g_{n, 1} \otimes ... \otimes g_{n, k_n})  \, = \, (g'_{1, 1} \otimes ... \otimes g'_{1, k'_1}) \cdot ... \cdot (g'_{n', 1} \otimes ... \otimes g'_{n', k'_{n'}}) \end{eq*}
satisfied by the action operad $G$.
\end{defn}

\begin{thm} \label{Gmonthm} For any action operad $G$, the algebras of $\mathrm{E}G$ are precisely the $G$-monoidal categories. Furthermore, any given notion of $(\mathcal{G}, \mathcal{R})$-monoidal category is equivalent to the $G$-monoidal categories for some action operad $G$, and thus also the $\mathrm{E}G$-algebras.
\end{thm} 
\begin{proof}
See \cite{operadborel}, Theorem 3.11 and Corollary 3.12.
\end{proof} 

This powerful result lets us to freely move back and forth between the worlds of action operads and strict monoidal categories, allowing us to reframe our questions about the latter into ones concerning the former. For instance, it is not difficult to see that the action operad corresponding to braided monoidal categories is the braid operad $B$. Thus if we want to describe certain kinds of free braided monoidal category, we can instead choose to look for the same sorts of free $\mathrm{E}B$-algebra. Moreover, this equivalence reveals a way to generate new examples of either structure. Using an earlier example, we know that the ribbon braid groups form an action operad $RB$, and so we can immediately conclude that there exists some notion of ribbon braided monoidal category \cite{ribbon1}, sometimes also known as balanced monoidal categories \cite{graphicalmon}. Conversely, if we had already known about these ribbon categories then we could have surmised from their strict versions that the ribbon braid groups formed an action operad.

Also, \cref{Gmonthm} will lead to a simplification for how we describe the action $\alpha$ of an $\mathrm{E}G$-algebra $X$. First, from now on we will generally only speak of the action as an operation that can be applied to the morphisms of $X$, because while $\alpha$ is really a functor its effect on objects is already covered by any discussion of the tensor product $\otimes$. Secondly, when the $\mathrm{E}G$ coordinate of $\alpha$ contains the unique morphism $g \to h$, we can always use the action of $G$ on $\mathrm{E}G$ to rewrite things so that we have the morphism $e \to hg^{-1}$ instead. We saw this briefly in the symmetric example we looked at, but the definition of $\mathcal{G}$ in \cref{Gmon} along with \cref{Gmonthm} shows that this shift to a single variable will never cause any problems or additional considerations. Thus from now on we will freely identify the morphism $g \to h$ in $\mathrm{E}G$ with the element $hg^{-1}$.   

\chapter{Free $\mathrm{E}G$-algebras} 
\label{initialalgebra}

From here on out, everything we do in this paper will be geared towards goal of describing the free $\mathrm{E}G$-algebras on $n$ invertible objects, for each action operad $G$. By \cref{Gmonthm}, this will then tell us how to construct the equivalent free structure for whole host of strict monoidal categories. Specifically, we will proceed by showing that such algebras are the initial objects of a particular comma category, in accordance with some well known properties of adjunctions and their units. Using this initial object perspective will allow us to recover all of the data associated with the objects of a given free invertible algebra --- what those objects are, how they act under tensor product, and which pairs of objects form the source and target of at least one morphism. Unfortunately, a concrete description of the morphisms themselves will ultimately remain elusive. We can get tantalisingly closer though, and an examination of the exact way that this method fails will provide the necessary insight to motivate a more successful approach in later chapters. 

\section{The free $\mathrm{E}G$-algebra on $n$ objects}

Before we attempt any of this though, it is crucial that we understand a much simpler case, where we do not require that our objects be invertible.

\begin{prop}\label{freealg} There exists a free $\mathrm{E}G$-algebra on $n$ objects. That is, there is an $\mathrm{E}G$-algebra $Y$ such that for any other $\mathrm{E}G$-algebra $X$, we have an isomorphism of categories
\begin{eq*} \mathrm{E}G\mathrm{Alg}_S(Y, X) \quad \cong \quad X^n \end{eq*}
\end{prop}

The proof of this fact is fairly standard. There is an obvious forgetful 2-functor $U: \mathrm{E}G\mathrm{Alg}_S \to \mathrm{Cat}$, sending each $\mathrm{E}G$-algebra to its underlying category and each algebra map to its underlying functor. As most forgetful functors do, this $U$ has a left adjoint, which we call the free $\mathrm{E}G$-algebra 2-functor $F : \mathrm{Cat} \to \mathrm{E}G\mathrm{Alg}_S $. It follows immediately that
\begin{eq*}\begin{array}{rll}
		U(X)^n & = & \mathrm{Cat}(\{z_1, ..., z_n\}, U(X) ) \\
		& \cong & \mathrm{E}G\mathrm{Alg}_S( F(\{z_1, ..., z_n\}), X) 
		\end{array}
\end{eq*}
where $\{z_1, ..., z_n\}$ is any set with $n$ distinct elements. Since $X$ and $U(X)$ are obviously isomorphic as categories, this shows that $F(\{z_1, ..., z_n\})$ is the required free algebra. It is also not to difficult to describe which category this $F(\{z_1, ..., z_n\})$ is.

\begin{defn} Let $G$ be an action operad. Then for any category $X$ and $k \in \mathbb{N}$, we will denote by $\mathrm{E}G(k) \times_{G(k)} X^k$ the coequaliser of the two functors $\mathrm{E}G(k) \times G(k) \times X^k \to \mathrm{E}G(k) \times X^k$ from \cref{Gopalgdef}:
\begin{eq*} \begin{tikzcd}
\mathrm{E}G(k) \times G(k) \times X^k \ar[ddr, "\mathlarger{\cdot} \, \times \mathrm{id}_{X^k}"'] \ar[rr, "\mathrm{id}_{\mathrm{E}G(k)} \times \pi \times \mathrm{id}_{X^k}"] & & \mathrm{E}G(k) \times \mathrm{S}_k \times X^k \ar[ddl, "\mathrm{id}_{\mathrm{E}G(k)} \times \tilde{\beta}"] \\
& & \\
& \mathrm{E}G(k) \times X^k \ar[dd] & \\
& & \\
& \mathrm{E}G(k) \times_{G(k)} X^k &
\end{tikzcd} \end{eq*}
\end{defn} 

\begin{prop} \label{Gndef} Let $\{ z_1, ..., z_n \}$ be an $n$-object set, which can also be considered as a discrete category. Then the free $\mathrm{E}G$-algebra on $n$ objects is the algebra $\mathbb{G}_n$ whose underlying category is 
\begin{eq*} \mathbb{G}_n \quad := \quad \coprod_{k \in \mathbb{N}} \, \mathrm{E}G(k) \times_{G(k)} \{ z_1, ..., z_n \}^k \end{eq*}
where for all $m, k_1, ..., k_m \in \mathbb{N}$, $g \in G(m)$, $x_i \in \{z_1, ..., z_n \}$ the action is given by
\begin{eq*} \alpha\big( \, g \, ; \, (h_1; \mathrm{id}_{x_1}, ..., \mathrm{id}_{x_{k_1}}), \, ..., \, (h_m; \mathrm{id}_{x_1}, ..., \mathrm{id}_{x_{k_m}}) \, \big) \, = \, \big( \, \mu(g;h_1, .., h_m) \, ; \, \mathrm{id}_{x_1}, \, ..., \, \mathrm{id}_{x_{k_m}} \, \big) \end{eq*}
In other words, for any $\mathrm{E}G$-algebra $X$,
\begin{eq*} \mathrm{E}G\mathrm{Alg}_S(\mathbb{G}_n, X) \quad \cong \quad \mathrm{Cat}(\{ z_1, ..., z_n \}, X) \quad \cong \quad X^n \end{eq*}
\end{prop}
 
Again, this is something already covered by the work of Gurski \cite{ogge}, so we won't go through all of the details here. The basic idea is that since the actions $\alpha_m : \mathrm{E}G(m) \times X^m \to X$ of any $\mathrm{E}G$-algebra coequalise the diagram from \cref{Gopalgdef}, the universal property of $\mathrm{E}G(k) \times_{G(k)} X^k$ will allow us to factor them uniquely though some $\alpha'$,
\begin{eq*} \begin{tikzcd}
\mathrm{E}G(k) \times X^k \ar[dd] \ar[ddrr, "\alpha_k"] & & \\
& & \\
\mathrm{E}G(k) \times_{G(k)} X^k \ar[rr, dashed, "\alpha'_k"'] & & X
\end{tikzcd} \end{eq*}
This then lets us upgrade any choice $f : \{ z_1, ..., z_n \} \to X$ of $n$ objects from $X$ into an algebra map $\mathbb{G}_n \to X$:
\begin{eq*} \begin{tikzcd}
\coprod_{k \in \mathbb{N}} \, \mathrm{E}G(k) \times_{G(k)} \{ z_1, ..., z_n \}^k \ar[rr, "\coprod \mathrm{id} \times f^k"] & & \coprod_{k \in \mathbb{N}} \, \mathrm{E}G(k) \times_{G(k)} X^k \ar[rr, "\coprod \alpha'_k"] & & X
\end{tikzcd} \end{eq*}
\cref{Gndef} serves as a fairly opaque definition of $\mathbb{G}_n$ at first, so we'll spend a little time now unpacking it. Recall that $\coprod_{k \in \mathbb{N}} \, \mathrm{E}G(k) \times_{G(k)} \{ z_1, ..., z_n \}^k$ is the coequaliser of the maps
\begin{eq*} \begin{tikzcd}
\coprod_{k \in \mathbb{N}} \mathrm{E}G(k) \times G(k) \times \{ z_1, ..., z_n \}^k \ar[r, shift left] \ar[r, shift right] & \coprod_{k \in \mathbb{N}} \mathrm{E}G(k) \times \{ z_1, ..., z_n \}^k
\end{tikzcd} \end{eq*}
that come from the action of $G(k)$ on $\mathrm{E}G(k)$ by multiplication on the right,
\begin{eq*} \begin{array}{rll}
			\mathrm{E}G(k) \times G(k) & \to & \mathrm{E}G(k) \\
			(g, h) & \mapsto & gh \\
			( \, !: g \to g', \mathrm{id}_h \, ) & \mapsto & !: gh \to g'h
		\end{array}
\end{eq*}
and the action of $G(k)$ on $\{ z_1, ..., z_n \}^k$ by underlying permutations,
\begin{eq*} \begin{array}{rll}
			G(k) \times \{ z_1, ..., z_n \}^k & \to & \{ z_1, ..., z_n \}^k \\
			( \, h \, ; \, x_1, ..., x_k \, ) & \mapsto & (x_{\pi(h^{-1})(1)}, ..., x_{\pi(h^{-1})(k)}) \\
			 \, (\mathrm{id}_h \, ; \, \mathrm{id}_{(x_1, ..., x_k)} \, ) & \mapsto & \mathrm{id}_{(x_{\pi(h^{-1})(1)}, ..., x_{\pi(h^{-1})(k)})}
		\end{array}
\end{eq*}
Thus objects in this algebra are equivalence classes of tuples $(g; x_1, ..., x_m)$, for some $g \in G(m)$ and $x_i \in \{z_1, ..., z_n\}$, under the relation
\begin{eq*} ( \, gh \, ; \, x_1, \, ..., \, x_m \, ) \sim ( \, g \, ; \, x_{\pi(h)^{-1}(1)}, \, ..., \, x_{\pi(h)^{-1}(m)} \, )\end{eq*}
But we can use this relation to rewrite any $(g;x_1, ..., x_m)$ uniquely in the form $(e_m; x'_1, ..., x'_m) = x'_1 \otimes ... \otimes x'_m$ where $x'_i = x_{\pi(g)(i)}$, and this means that each such equivalence class is just a tensor product for some unique sequence of generators $z_i$. More concretely, we have:

\begin{lem} \label{Gnobj} $\mathrm{Ob}(\mathbb{G}_n)$ is the free monoid on $n$ generators, which is $\mathbb{N}^{\ast n}$, the free product of $n$ copies of $\mathbb{N}$. \end{lem}

Similarly, the morphisms of $\mathbb{G}_n$ are all of the form
\begin{eq*} (g ; \mathrm{id}_{x_1},...,\mathrm{id}_{x_m}) \, : \, x_1 \otimes ... \otimes x_m \, \to \, x_{\pi(g^{-1})(1)} \otimes ... \otimes x_{\pi(g^{-1})(m)} \end{eq*}
for some $g \in G(m)$ and $x_i \in \{z_1, ..., z_n\}$. However, notice the definition of the action $\alpha$ of $\mathbb{G}_n$, we can rewrite these as
\begin{eq*} \begin{array}{rll}
			(g ; \mathrm{id}_{x_1},...,\mathrm{id}_{x_m}) & = & \big( \, \mu(g;e_1, ..., e_1) \, ; \, \mathrm{id}_{x_1},...,\mathrm{id}_{x_m} \, \big) \\
			& = & \alpha\big( \, g \, ; \, (e_1; \mathrm{id}_{x_1}), \, ..., \, (e_1; \mathrm{id}_{x_m}) \, \big) \\
			& = & \alpha( g ; \, \mathrm{id}_{(e_1;x_1)}, ..., \mathrm{id}_{(e_1; x_m)} ) \\
			& = & \alpha( g ; \, \mathrm{id}_{x_1}, ..., \mathrm{id}_{x_m} )
		\end{array}
\end{eq*}
That is, the free $\mathrm{E}G$-algebra $\mathbb{G}_n$ does not have any objects or morphisms that do not arise straightforwardly from the tensor product and action.

\begin{lem} \label{Gnmapsaction} Every morphism of $\mathbb{G}_n$ can be expressed uniquely as an action morphism 
\begin{eq*} \alpha( \, g \, ; \, \mathrm{id}_{x_1}, ..., \mathrm{id}_{x_m} \, ) \, : \, x_1 \otimes ... \otimes x_m \, \to \, x_{\pi(g)^{-1}(1)} \otimes ... \otimes x_{\pi(g)^{-1}(m)} \end{eq*}
for some $g, g' \in G(m)$ and $x_i \in \{z_1, ..., z_n \}$. \end{lem}

As an immediate consequence of this, the source and target of any given morphism in $\mathbb{G}_n$ must be related to one another via some permutation of the form $\pi(g)$. In other words, the connected components of $\mathbb{G}_n$ will depend upon the underlying permutation operad of $G$, in the following way:

\begin{prop}\label{Gnconcomp} Considered as a monoid under tensor product,
\begin{eq*} \pi_0(\mathbb{G}_n) \quad = \quad \begin{cases}
							\quad \mathbb{N}^n & \text{if $G$ is crossed} \\
							\quad \mathbb{N}^{\ast n} & \text{otherwise}
							\end{cases}
 \end{eq*} 
Also, the canonical homomorphism sending objects in $\mathbb{G}_n$ to their connected component,
\begin{eq*} [ \, \_ \, ] \, : \, \mathrm{Ob}(\mathbb{G}_n) \to \pi_0(\mathbb{G}_n) \end{eq*}
is the quotient map of abelianisation
\begin{eq*} \mathrm{ab} \, : \, \mathbb{N}^{*n} \to (\mathbb{N}^{*n})^{\mathrm{ab}} \, = \, \mathbb{N}^n \end{eq*}
when $G$ is crossed, and the identity map $\mathrm{id}_{\mathbb{N}^{*n}}$ otherwise.
\end{prop}
\begin{proof}
By \cref{Gnmapsaction}, all morphisms in $\mathbb{G}_n$ can be written uniquely as $\alpha(g; \mathrm{id}_{x_1}, ..., \mathrm{id}_{x_m})$, for some $g \in G(m)$ and $x_i \in \{z_1, ..., z_n \}$. Since maps of this form have source $x_1 \otimes ... \otimes x_m$ and target $x_{\pi(g^{-1})(1)} \otimes ... \otimes x_{\pi(g^{-1})(m)}$, we see that the only pairs of object which might have a morphism between them are those that can be expanded as tensor products that differ by some permutation. 

If our action operad $G$ is crossed, then for any two objects like this --- say source $x_1 \otimes ... \otimes x_m$ and target $x_{\sigma^{-1}(1)} \otimes ... \otimes x_{\sigma^{-1}(m)}$ for an arbitrary $\sigma \in \mathrm{S}_m$ --- we can always find a map $\alpha(g; \mathrm{id}_{x_1}, ..., \mathrm{id}_{x_m})$ between them, because by \cref{surjortriv} the underlying permutations maps $\pi_m: G(m) \to S_m$ are all surjective and so there must exist at least one $g$ with $\pi(g) = \sigma$. In particular, for any two generating objects $z_i$ and $z_j$ of $\mathbb{G}_n$ there must exist at least morphism between $z_i \otimes z_j$ and $z_j \otimes z_i$, and therefore
\begin{eq*} [z_i] \otimes [z_j] \quad = \quad [z_i \otimes z_j] \quad = \quad [z_j \otimes z_i] \quad = \quad [z_j] \otimes [z_i] \end{eq*}
Thus the canonical map $[ \, \_ \, ] : \mathrm{Ob}(\mathbb{G}_n) \to \pi_0(\mathbb{G}_n)$ is the one that makes the free product of $\mathbb{N}^{*n}$ commutative; that is, the quotient map for the abelianisation $\mathrm{ab} : \mathbb{N}^{*n} \to (\mathbb{N}^{*n})^{\mathrm{ab}}$. Hence $\pi_0(\mathbb{G}_n) = \mathbb{N}^n$.

Conversely, if $G$ is non-crossed then its underlying permutation operad $\mathrm{im}(\pi)$ is trivial, and so the only morphisms we have in $\mathbb{G}_n$ will be those of the form
\begin{eq*} \alpha( \, e_m \, ; \, \mathrm{id}_{x_1}, ..., \mathrm{id}_{x_m} \, ) \quad = \quad \mathrm{id}_{x_1} \otimes ... \otimes \mathrm{id}_{x_m} \quad = \quad \mathrm{id}_{x_1 \otimes ... \otimes x_m} \end{eq*}
Therefore the map $[ \, \_ \,]$ just sends each object to its identity morphism, and since that function is one-to-one and onto it follows that
\begin{eq*} \pi_0(\mathbb{G}_n) \quad = \quad \mathrm{Ob}(\mathbb{G}_n) \quad = \quad \mathbb{N}^{\ast n}, \quad \quad \quad \quad \quad [ \, \_ \,] \quad = \quad \mathrm{id}_{\mathbb{N}^{*n}} \end{eq*}
by \cref{Gnobj}.
\end{proof}

\cref{Gnconcomp} is not the only way that the behaviour of $\mathbb{G}_n$ is contingent on whether $G$ is crossed. Consider the following common property of monoidal categories:

\begin{defn} A monoidal category $X$ is said to be \emph{spacial} if all of its identity morphisms commute with the endomorphisms of the unit object: 
\begin{eq*} f \otimes \mathrm{id}_x \, = \, \mathrm{id}_x \otimes f, \quad \quad \quad x \in \mathrm{Ob}(X), f \in X(I,I) \end{eq*}
\end{defn}

The motivation for the name `spacial' comes from the context of string diagrams \cite{graphicalmon}. In a string diagram, the act of tensoring two strings together is represented by placing those strings side by side. Since the defining feature of the unit object is that tensoring it with other objects should have no effect, the unit object is therefore represented diagrammatically by the absence of a string. An endomorphism of the unit thus appears as an entity with no input or output strings, detached from the rest of the diagram. In a real-world version of these diagrams, made out of physical strings arranged in real space, we could use this detachedness to grab these endomorphisms and slide them over or under any strings we please, without affecting anything else in the diagram. This ability is embodied algebraically by the equation above, and hence categories which obey it are called `spacial'.

It turns out that the crossedness of an action operad has a direct effect on the spaciality of algebras.

\begin{lem}\label{spacial} If $G$ is a crossed action operad, then all $\mathrm{E}G$-algebras are spacial. \end{lem}
\begin{proof}
Let $G$ be a crossed action operad, let $X$ be a $\mathrm{E}G$-algebra, and fix $x \in \mathrm{Ob}(X)$ and \( f: I \to I \). From \cref{surjortriv} we know that \( \pi : G(2) \to S_2 \) is surjective, so that the set $\pi^{-1}( \, (1 \, 2) \, )$ is non-empty, and from the rules for composition of action morphisms we see that for any such $g \in \pi^{-1}( \, (1 \, 2) \, )$,
\begin{eq*}\begin{array}{rll}
		\alpha( \, g \, ; \, \mathrm{id}_x, \, \mathrm{id}_I \, ) \circ \alpha( \, e_2 \, ; \, \mathrm{id}_x, \, f \, ) & = & \alpha( \, g \, ; \, \mathrm{id}_x, \, f \, ) \\
		& = & \alpha( \, e_2 \, ; \, f, \, \mathrm{id}_x \, ) \circ \alpha( \, g \, ; \, \mathrm{id}_x, \, \mathrm{id}_I \, ) \\
		\end{array}
\end{eq*}
Thus in order to obtain the result we're after, it will suffice to find a particular $g \in \pi^{-1}( \, (1 \, 2) \, )$ for which
\begin{eq*}\alpha( \, g \, ; \, \mathrm{id}_x, \, \mathrm{id}_I \, ) = \mathrm{id}_x \end{eq*}
However, since
\begin{eq*}\begin{array}{rll}
		\alpha( \, g \, ; \, \mathrm{id}_x, \, \mathrm{id}_I \, ) & = & \alpha( \, g \, ; \, \mathrm{id}_x, \, \alpha( e_0; - ) \, ) \\
		& = & \alpha( \, \mu(g; e_1, e_0) \, ; \, \mathrm{id}_x \, )
		\end{array}
\end{eq*}
all we really need is to find a $g \in \pi^{-1}( \, (1 \, 2) \, )$ for which
\begin{eq*} \mu(g; e_1, e_0) = e_1 \end{eq*}
To this end, choose an arbitrary element $h \in \pi^{-1}( \, (1 \, 2) \, )$. This $h$ probably won't obey the above equation, but we can use it to construct a new element $g$ which does. Specifically, define
\begin{eq*} k \, := \, \mu( \, h \ ; \, e_1, \, e_0 \, ) \end{eq*}
and then consider
\begin{eq*} g \, := \, h \cdot \mu(e_2; k^{-1}, e_1) \end{eq*} 
To see that this is the correct choice of $g$, first note that we must have \( \pi(k) = e_1 \), since this is the only element of $S_1$. Following from that, we have 
\begin{eq*}\begin{array}{rll}
		\pi \big( \, \mu(e_2; k^{-1}, e_1) \, \big) & = & \mu \big( \, \pi(e_2) \ ; \, \pi(k^{-1}), \, \pi(e_1) \, \big) \\
		& = & \mu \big( \, e_2  \ ; \, e_1, \, e_1 \, \big) \\
		& = & e_2
		\end{array}
\end{eq*}
and hence
\begin{eq*}\begin{array}{rll}
		\pi(g) & = & \pi \big( h \cdot \mu(e_2; k^{-1}, e_1) \big) \\
		& = & \pi(h) \cdot \pi \big(\mu(e_2; k^{-1}, e_1) \big) \\
		& = & (1 \, 2) \cdot e_2 \\
		& = & (1 \, 2)
		.\end{array}
\end{eq*}
So $g$ is indeed in $\pi^{-1}( \, (1 \, 2) \, )$, and furthermore
\begin{eq*}\begin{array}{rll}
		\mu(g; e_1, e_0) & = & \mu \big( \, h \cdot \mu(e_2; k^{-1}, e_1) \ ; \, e_1, \, e_0 \, \big) \\
		& = & \mu( \, h \ ; \, e_1, \, e_0 \, ) \cdot \mu \big( \, \mu(e_2; k^{-1}, e_1) \ ; \, e_1, \, e_0 \, \big) \\
		& = & \mu( \, h \ ; \, e_1, \, e_0 \, ) \cdot \mu \big( \, e_2 \ ; \, \mu(k^{-1}; e_1), \, \mu(e_1; e_0) \, \big) \\
		& = & \mu( \, h \ ; \, e_1, \, e_0 \, ) \cdot \mu( \, e_2 \ ; \, k^{-1}, e_0 \, ) \\
		& = & k \cdot k^{-1} \\
		& = & e_1
		\end{array}
\end{eq*}
Therefore, $h \cdot \mu(e_2; k^{-1}, e_1)$ is exactly the $g$ we were looking for, and so working backwards through the proof we obtain the required result:
\begin{eq*} \begin{array}{rll}
		\mu(g; e_1, e_0) & = & e_1 \\
		\implies \quad \alpha( \, g \, ; \, \mathrm{id}_x, \, \mathrm{id}_I \, ) & = & \mathrm{id}_x \\
		& & \\
		\alpha( \, g \, ; \, \mathrm{id}_x, \, \mathrm{id}_I \, ) \circ \alpha( \, e_2 \, ; \, \mathrm{id}_x, \, f \, ) & = & \alpha( \, e_2 \, ; \, f, \, \mathrm{id}_x \, ) \circ \alpha( \, g \, ; \, \mathrm{id}_x, \, \mathrm{id}_I \, ) \\
		\implies \quad \alpha( \, e_2 \, ; \, \mathrm{id}_x, \, f \, ) & = & \alpha( \, e_2 \, ; \, f, \, \mathrm{id}_I \, )
		\end{array}
\end{eq*}
\end{proof} 

Finally, \cref{Gnmapsaction} also gives a complete description of how the morphisms of $\mathbb{G}_n$ interact as a monoid under tensor product, though to best express this we need a bit of new terminology.

\begin{defn} Let $G$ be an action operad. Then we will also the notation $G$ to denote the \emph{underlying monoid} of this action operad. This is the natural way to consider $G$ as a monoid, with its element set being all of its elements together, $\bigsqcup_m G(m)$, and with tensor product as its binary operation, $g \otimes h = \mu(e_2; g, h)$.

Also, note that this monoid comes equipped with a homomorphism $| \, \_ \, | : G \to \mathbb{N}$, sending each $g \in G$ to the natural number $m$ if and only if $g$ is an element of the group $G(m)$. We'll call this number $|g|$ the \emph{length} of $g$.
\end{defn}

\begin{defn}\label{lengthdef} Let $S$ be a set and $F(S)$ the free monoid on $S$, the monoid whose elements are strings of elements of $S$ and whose binary operation is concatenation. Then we will denote by
\begin{eq*} | \, \_ \, | : F(S) \to \mathbb{N} \end{eq*}
the monoid homomorphism defined by sending each element of $S \subseteq F(S)$ to 1, and therefore also each concatenation of $n$ elements of $S$ to the natural number $n$. Again, we will call $|x|$ the \emph{length} of $x \in F(S)$.
\end{defn}

\begin{lem} \label{Gnmor} The monoid of morphisms of the algebra $\mathbb{G}_n$ is
\begin{eq*} \mathrm{Mor}(\mathbb{G}_n) \quad \cong \quad G \times_{\mathbb{N}} \mathbb{N}^{\ast n} \end{eq*}
where this is a pullback taken over the respective length homomorphisms,
\begin{eq*} \begin{tikzcd}
G \times_{\mathbb{N}} \mathbb{N}^{\ast n} \ar[dd, shift left=4] \ar[rr] \ar[ddrr, phantom, "\lrcorner", very near start, shift left] & & \mathbb{N}^{\ast n} \ar[dd, "| \, \_ \, |"] & \\
& & & \\
\quad \quad G \ar[rr, "| \, \_ \, |"] & & \mathbb{N} &
\end{tikzcd} \end{eq*}
using the fact that $\mathbb{N}^{\ast n}$ is the free monoid $F\big( \, \{z_1, ..., z_n\} \, \big)$.
\end{lem}
\begin{proof}
An element of $G \times_{\mathbb{N}} F( \, \{z_1, ..., z_n\} \, )$ is just an element $g \in G(m)$ for some $m$, together with an $m$-tuple of objects $(x_1, ..., x_m)$ from the set of generators $\{z_1, ..., z_n\}$. Thus the action on $\mathbb{G}_n$ defines an obvious function 
\begin{eq*} \begin{array}{rlrll}
			\alpha & : & G \times_{\mathbb{N}} F\big( \, \{z_1, ..., z_n\} \, \big) & \to & \mathrm{Mor}(\mathbb{G}_n) \\
			& : & (g;x_1, ..., x_m) & \mapsto & \alpha(g; \mathrm{id}_{x_1}, ..., \mathrm{id}_{x_m})
		\end{array}
\end{eq*}
But by \cref{Gnmapsaction}, each element of $\mathrm{Mor}(\mathbb{G}_n)$ can be expressed in the form $\alpha(g; \mathrm{id}_{x_1}, ..., \mathrm{id}_{x_m})$ for a unique collection $(g;x_1, ..., x_m)$, and so this function $\alpha$ is actually a bijection of sets. Furthermore, this function preserves tensor product, since
\begin{eq*} \begin{array}{rll}
			\alpha\big( \, (g;f_1, ..., f_m) \otimes (g';f'_1, ..., f'_m) \, \big) & = & \alpha( \, g \otimes g' \, ; \, f_1, ..., f_m, f'_1, ..., f'_m \, ) \\
			& = & \alpha( \, g \, ; \, f_1, ..., f_m \, ) \otimes \alpha( \, g' \, ; \, f'_1, ..., f'_m \, )
		\end{array}
\end{eq*}
and hence it is a monoid isomorphism, as required.
\end{proof}

\section{The free $\mathrm{E}G$-algebra on $n$ invertible objects}

We saw in \cref{freealg} that the existence of a free $\mathrm{E}G$-algebra on $n$ objects can be proven by taking the left adjoint of a 2-functor which forgets about the algebra structure. Now we want to extend this idea into the realm of algebras on invertible objects. For the analogous approach, we will need to find a new 2-functor that lets us forget about non-invertible objects, and then hopefully we can find its left adjoint too, and use it to freely add inverses to $\mathbb{G}_n$. First though, we need to make this concept of `forgetting non-invertible objects' a little more precise.

\begin{defn} Given an $\mathrm{E}G$-algebra $X$, we'll denote by $X_{\mathrm{inv}}$ the sub-$\mathrm{E}G$-algebra containing all invertible objects in $X$ and the isomorphisms between them. \end{defn}

Note that this is indeed a well-defined $\mathrm{E}G$-algebra. If $f_1, ..., f_m$ are isomorphisms from invertible objects $x_1, ..., x_m$ to invertible objects $y_1, ..., y_m$, then $\alpha(g; f_1, ..., f_m)$ is a map from the invertible object $\alpha(g; x_1, ..., x_m)$ to the invertible object $\alpha(g; y_1, ..., y_m)$, and it has an inverse $\alpha(g^{-1}; f_{\pi(g)(1)}^{-1}, ..., f_{\pi(g)(m)}^{-1})$, since
\begin{eq*} \begin{array}{ll}
		& \alpha\big( \, g^{-1} \, ; \, f_{\pi(g)(1)}^{-1}, \, ..., \, f_{\pi(g)(m)}^{-1} \, \big) \, \circ \, \alpha( \, g \, ; \, f_1, ..., f_m \,) \\[\medskipamount]
		= & \alpha\big( \, g^{-1}g \, ; \, f_1^{-1} f_1, \, ..., \, f_m^{-1} f_m \, \big) \\[\medskipamount]
		= & \mathrm{id}_{x_1 \otimes ... \otimes x_m} \\
		& \\
		& \alpha( \, g \, ; \, f_1, ..., f_m \,) \, \circ \, \alpha\big( \, g^{-1} \, ; \, f_{\pi(g)(1)}^{-1}, \, ..., \, f_{\pi(g)(m)}^{-1} \, \big) \\[\medskipamount]
		= & \alpha\big( \, gg^{-1} \, ; \, f_{\pi(g)(1)} f_{\pi(g)(1)}^{-1}, \, ..., \, f_{\pi(g)(m)} f_{\pi(g)(m)}^{-1} \, \big) \\[\medskipamount]
		= & \mathrm{id}_{y_{\pi(g)(1)} \otimes ... \otimes y_{\pi(g)(m)}}
		\end{array}
\end{eq*}
Clearly then, $X_{\mathrm{inv}}$ is the correct algebra for our new forgetful 2-functor to send $X$ to. Knowing this, we can construct the rest of the functor fairly easily.

\begin{prop} \label{invprop} The assignment $X \mapsto X_{\mathrm{inv}}$ can be extended to a 2-functor $(\_)_{\mathrm{inv}}: \mathrm{E}G\mathrm{Alg}_S \to \mathrm{E}G\mathrm{Alg}_S$.
\end{prop}
\begin{proof}
Let $F: X \to Y$ be a (strict) map of $\mathrm{E}G$-algebras. If $x$ is an invertible object in $X$ with inverse $x^*$, then $F(x)$ is an invertible object in $Y$ with inverse $F(x^*)$, by
\begin{eq*} \begin{array}{rcccccl}
			F(x) \otimes F(x^*) & = & F(x \otimes x^*) & = & F(I) & = & I \\
			 F(x^*) \otimes F(x) & = & F(x^* \otimes x) & = & F(I) & = & I 
		\end{array}
\end{eq*}
Since $F$ sends invertible objects to invertible objects, it will also send isomorphisms of invertible objects to isomorphisms of invertible objects. In other words, the map $F: X \to Y$ can be restricted to a map $F_{\mathrm{inv}} : X_{\mathrm{inv}} \to Y_{\mathrm{inv}}$. Moreover, we have that
\begin{eq*} \begin{array}{rcccl}
			(F \circ G)_{\mathrm{inv}}(x) & = & F \circ G(x) & = & F_{\mathrm{inv}} \circ G_{\mathrm{inv}}(x) \\
			(F \circ G)_{\mathrm{inv}}(f) & = & F \circ G(f) & = & F_{\mathrm{inv}} \circ G_{\mathrm{inv}}(f) 
		\end{array}
\end{eq*}
and so the assignment $F \mapsto F_{\mathrm{inv}}$ is clearly functorial. Next, let $\theta : F \Rightarrow G$ be a monoidal natural transformation. Choose an invertible object $x$ from $X$, and consider the component map of its inverse, $\theta_{x^*} : F(x^*) \to G(x^*)$. Since $\theta$ is monoidal, we have $\theta_{x^*} \otimes \theta_x = \theta_I = I$ and $\theta_x \otimes \theta_{x^*} = I$, or in other words that $\theta_{x^*}$ is the monoidal inverse of $\theta_x$. We can use this fact to construct a compositional inverse as well, namely $\mathrm{id}_{F(x)} \otimes \theta_{x^*} \otimes \mathrm{id}_{G(x)}$, which can be seen as follows:
\begin{eq*}  \begin{array}{rcccl}
		\big( \mathrm{id}_{F(x)} \otimes \theta_{x^*} \otimes \mathrm{id}_{G(x)} \big)  \circ \theta_x & = & \theta_x \otimes \theta_{x^*} \otimes \mathrm{id}_{G(x)} & = &  \mathrm{id}_{G(x)} \\
		&& \\
		\theta_x \circ  \big( \mathrm{id}_{F(x)} \otimes \theta_{x^*} \otimes \mathrm{id}_{G(x)} \big) & = & \mathrm{id}_{F(x)} \otimes \theta_{x^*} \otimes \theta_x & = &  \mathrm{id}_{F(x)} \\
		\end{array} 
\end{eq*}
Therefore, we see that all the components of our transformation on invertible objects are isomorphisms, and hence we can define a new transformation $\theta_{\mathrm{inv}}: F_{\mathrm{inv}} \Rightarrow G_{\mathrm{inv}}$ whose components are just $(\theta_{\mathrm{inv}})_x = \theta_x$. The assignment $\theta \mapsto \theta_{\mathrm{inv}}$ is also clearly functorial, and thus we have a complete 2-functor $(\_)_{\mathrm{inv}}: \mathrm{E}G\mathrm{Alg}_S \to \mathrm{E}G\mathrm{Alg}_S$.
\end{proof}

Now we just need to show that this $(\_)_{\mathrm{inv}}$ forms the right-hand part of an adjunction. The easiest way to do this kind of thing is with an \emph{adjoint functor theorem}. These are a collection of similar results, each of which provides some sufficient conditions for the existence of a left adjoint to a given functor. The first such theorem, what is now known as the `General Adjoint Functor Theorem', is due Peter Freyd \cite{aft}, and a discussion of this and other versions can be found in \cite{cwm}. The variation we will be using comes from the work of Adámek and Rosicky \cite{lpac}, and concerns locally finitely presentable categories.

\begin{defn} A \emph{filtered diagram} is a diagram $D$ where every finite subdiagram has a cocone in $D$. That is, $D$ is non-empty and within it we know that:
\begin{itemize}
\item for each pair of objects $x, y$, there exists at least one object $z$ equipped with morphisms $x \to z$ and $y \to z$
\item for each pair of parallel morphisms $f,g: x \to y$, there exists at least one morphism $h: y \to z$ for which $h \circ f = h \circ g$
\end{itemize}
A colimit over a filtered diagram is called a \emph{filtered colimit}.
\end{defn}

\begin{defn} Let $X$ and $Y$ be categories and $F: X \to Y$ a functor. The we say that
\begin{itemize}
\item an object $x$ in $X$ is \emph{finitely presented} if the functor $\mathrm{Hom}_{X}(x, -) : X \to \mathrm{Set}$ preserves filtered colimits
\item $X$ is \emph{finitely accessible} if it has all finite filtered colimits and every object in $X$ is finitely presented
\item $F: X \to Y$ is \emph{finitely accessible} if both $X$ and $Y$ are finitely accessible and $F$ preserves filtered colimits between them
\item $X$ is \emph{locally finitely presentable} if it is accessible and has all finite colimits
\end{itemize}
\end{defn}

\begin{namedprop}[(The AFT for LFP categories)] \label{aftlfp}
Let $X$ and $Y$ be locally finitely presentable categories. Then a functor $F: X \to Y$ has a left adjoint if and only if it is accessible and preserves all finite limits. 
\end{namedprop}

Now, one might ask why we would choose to use this adjoint functor theorem in particular, when we don't even know whether $\mathrm{E}G\mathrm{Alg}_S$ is locally finitely presentable. The answer is that all of the work needed to prove this fact has already been done for us elsewhere. To see this though, we are going to need to use a little bit of the theory of 2-monads. We won't be doing much more than mentioning certain concepts here, but if the reader is interested in exploring this topic more thoroughly they can refer to \cite{monad1} \cite{monad2} for background on monads and \cite{2monad} for 2-monads.

\begin{defn} A \emph{monad} on a category $X$ is an endofunctor $T: X \to X$ along with natural transformations $\eta: \mathrm{id}_{X} \Rightarrow T$ and $\mu: T \circ T \Rightarrow T$ which satisfy the coherence conditions
\begin{eq*} \mu \circ \mu T \, = \, \mu \circ T\mu, \quad \quad \quad \mu \circ \eta T \, = \, \mathrm{id}_{T} \, = \, \mu \circ T\eta \end{eq*}
Similarly, a \emph{2-monad} on a 2-category $X$ is a 2-functor $T: X \to X$ together with 2-natural transformations $\eta: \mathrm{id}_{X} \Rightarrow T$ and $\mu: T \circ T \Rightarrow T$ which obey the same coherence conditions before, but this time only up to isomorphism, with those isomorphisms then obeying their own set of coherence conditions. The 2-monad is said to be strict if these new isomorphisms are actually still identities.

These monads come with their own notion of algebras, each of which forms a category $T\mathrm{Alg}$ or $T\mathrm{Alg}_S$.
\end{defn}

There is a strong link between these structures and the ones we have been working with so far, proven in \cite{ogge}:

\begin{prop} Let $G$ be an action operad, and let $O$ be a $G$-operad in the category $\mathrm{Set}$. Then there exists a monad $\underline{O}: \mathrm{Set} \to \mathrm{Set}$ whose category of algebras $\underline{O}\mathrm{Alg}$ is isomorphic to the category $O\mathrm{Alg}$.

Likewise, if $O$ is $G$-operad in $\mathrm{Cat}$, then there exists a 2-monad $\underline{O}: \mathrm{Cat} \to \mathrm{Cat}$ whose strict algebras $\underline{O}\mathrm{Alg}_S$ are isomorphic to $O\mathrm{Alg}_S$.
\end{prop}

Because of this, if we want to show that $\mathrm{E}G\mathrm{Alg}_S$ is a locally finitely presentable category, it will suffice to show the same thing for $\underline{\mathrm{E}G}\mathrm{Alg}_S$. Luckily, from the very same paper we also learn the following:

\begin{prop} For any $G$-operad $O$, the associated $\underline{O}$ preserves filtered colimits. \end{prop}

Since $\mathrm{Cat}$ is finitely accessible, this means that the 2-monad $\underline{\mathrm{E}G}: \mathrm{Cat} \to \mathrm{Cat}$ is as well. Finally, to see what impact this has on its category of algebras, we can use a result from \cite{lpac}:

\begin{prop} If $T: X \to X$ is a finitely accessible monad, then $T\mathrm{Alg}$ is locally finitely presentable. \end{prop}

When everything is kept strict this carries through to the 2-monad case as well, so at last we see that $\mathrm{E}G\mathrm{Alg}_S$ really is a locally finitely presentable category. Obtaining our left adjoint functor is now a simple matter of applying the adjoint functor theorem.

\begin{prop} \label{invadj} The 2-functor $(\_)_{\mathrm{inv}}: \mathrm{E}G\mathrm{Alg}_S \to \mathrm{E}G\mathrm{Alg}_S$ has a left adjoint, $L: \mathrm{E}G\mathrm{Alg}_S \to \mathrm{E}G\mathrm{Alg}_S$.
\end{prop}
\begin{proof} Since we already know that $\mathrm{E}G\mathrm{Alg}_S$ is locally finitely presentable, the conditions for \cref{aftlfp} amount to showing that $(\_)_{\mathrm{inv}}$ preserves both limits and filtered colimits.
\begin{itemize}
\item Given an indexed collection of $\mathrm{E}G$-algebras $X_i$, the $\mathrm{E}G$-action of their product $\prod X_i$ is defined componentwise. In particular, this means that the tensor product of two objects in $\prod X_i$ is just the collection of the tensor products of their components in each of the $X_i$. An invertible object in $\prod X_i$ is thus simply a family of invertible objects from the $X_i$ --- in other words, $(\prod X_i)_{\mathrm{inv}} = \prod (X_i)_{\mathrm{inv}}$.
\item Given maps of $\mathrm{E}G$-algebras $F: X \to Z$, $G : Y \to Z$, the $\mathrm{E}G$-action of their pullback $X \times_Z Y$ is also defined component-wise. It follows that an invertible object in $X \times_Z Y$ is just a pair of invertible objects $(x, y)$ from $X$ and $Y$, such that $F(x) = G(y)$. But this is the same as asking for a pair of objects $(x, y)$ from $X_{\mathrm{inv}}$ and $Y_{\mathrm{inv}}$ such that $F_{\mathrm{inv}}(x) = G_{\mathrm{inv}}(y)$, and hence $(X \times_Z Y)_{\mathrm{inv}} = X_{\mathrm{inv}} \times_{Z_{\mathrm{inv}}} Y_{\mathrm{inv}}$.
\item Given a filtered diagram $D$ of $\mathrm{E}G$-algebras, the $\mathrm{E}G$-action of its colimit $\mathrm{colim}(D)$ is defined in the following way: use filteredness to find an algebra which contains (representatives of the classes of) all the things you want to act on, then apply the action of that algebra. In the case of tensor products this means that $[x]\otimes[y] = [x \otimes y]$, and thus an invertible object in $\mathrm{colim}(D)$ is just (the class of) an invertible object in one of the algebras of $D$. In other words, $\mathrm{colim}(D)_{\mathrm{inv}} = \mathrm{colim}(D_{\mathrm{inv}})$.
\end{itemize}
Preservation of products and pullbacks give preservation of limits, and preservation of limits and filtered colimits give the result.
\end{proof}

With this new 2-functor $L: \mathrm{E}G\mathrm{Alg}_S \to \mathrm{E}G\mathrm{Alg}_S$, we now have the ability to `freely add inverses to objects' in any $\mathrm{E}G$-algebra we want. The algebra $L\mathbb{G}_n$ is then a clear candidate for our free algebra on $n$ invertible objects, and indeed the proof of this is very simple.

\begin{thm} There exists a free $\mathrm{E}G$-algebra on $n$ invertible objects. Specifically, the algebra $L\mathbb{G}_n$ is such that for any other $\mathrm{E}G$-algebra $X$, we have an isomorphism of categories
\begin{eq*} \mathrm{E}G\mathrm{Alg}_S(L\mathbb{G}_n, X) \quad \cong \quad (X_{\mathrm{inv}})^n \end{eq*}
\end{thm}
\begin{proof}
Using the adjunction from \cref{invadj} along with the one from \cref{freealg}, we see that
\begin{eq*}\begin{array}{rll}
		 U(X_{\mathrm{inv}})^n & = & \mathrm{Cat}(\{z_1, ..., z_n\}, U(X_{\mathrm{inv}}) ) \\
		& \cong & \mathrm{E}G\mathrm{Alg}_S( F(\{z_1, ..., z_n\}), X_{\mathrm{inv}}) \\
		& \cong & \mathrm{E}G\mathrm{Alg}_S( LF(\{z_1, ..., z_n\}), X)
\end{array}
 \end{eq*}
$X_{\mathrm{inv}}$ and $U(X_{\mathrm{inv}})$ are obviously isomorphic as categories, and so \( LF(\{z_1, ..., z_n\}) = L\mathbb{G}_n \) satisfies the requirements for the free algebra on $n$ invertible objects.
\end{proof}

\section{$L\mathbb{G}_n$ as an initial object}
 
We have proven that a free $\mathrm{E}G$-algebra on $n$ invertible objects indeed exists, but this fact on its own is not very helpful. To be able to actually use the free algebra $L\mathbb{G}_n$, we need to know how to construct it explicitly, in terms of its objects and morphisms. We could do this by finding a detailed characterisation of the 2-functor $L$, and then applying this to our explicit description of $\mathbb{G}_n$ from \cref{Gndef}. However, this would probably take far more effort than is required, since it would involve determining the behaviour of $L$ in many situations that we aren't interested in. Also, we wouldn't be leveraging $\mathbb{G}_n$'s status as a free algebra to make the calculations any easier. We will try a different strategy instead, one that begins by noticing a special property of the functor $L$.

\begin{prop} \label{linveql} For any $\mathrm{E}G$-algebra $X$, we have $L(X)_{\mathrm{inv}} = L(X)$.
\end{prop}
\begin{proof}
From the definition of adjunctions, the isomorphisms
\begin{eq*}\mathrm{E}G\mathrm{Alg}_S(LX , Y) \quad \cong \quad \mathrm{E}G\mathrm{Alg}_S(X, Y_{\mathrm{inv}}) \end{eq*}
are subject to certain naturality conditions. Specifically, given $F: X' \to X$ and $G: Y \to Y'$ we get a commutative diagram
\begin{eq*} \begin{tikzcd}
\mathrm{E}G\mathrm{Alg}_S(LX , Y) \ar[dd, "G \circ \_ \circ LF"'] \ar[r, "\sim"] & \mathrm{E}G\mathrm{Alg}_S(X, Y_{\mathrm{inv}}) \ar[dd, "G_{\mathrm{inv}} \circ \_ \circ F"] \\
& \\
\mathrm{E}G\mathrm{Alg}_S(LX' , Y') \ar[r, "\sim"] & \mathrm{E}G\mathrm{Alg}_S(X', Y'_{\mathrm{inv}})
\end{tikzcd} \end{eq*}
Consider the case where $F$ is the identity map $\mathrm{id}_X : X \to X$ and $G$ is the inclusion $j: L(X)_{\mathrm{inv}} \to L(X)$. Note that because $j$ is an inclusion, the restriction $j_{\mathrm{inv}}: (L(X)_{\mathrm{inv}})_{\mathrm{inv}} \to L(X)_{\mathrm{inv}}$ is also an inclusion, but since $((\_)_{\mathrm{inv}})_{\mathrm{inv}} = (\_)_{\mathrm{inv}}$, we have that $j_{\mathrm{inv}} = \mathrm{id}$. It follows that
\begin{eq*} \begin{tikzcd}
\mathrm{E}G\mathrm{Alg}_S(LX , LX_{\mathrm{inv}}) \ar[dd, "j \circ \_"'] \ar[r, "\sim"] & \mathrm{E}G\mathrm{Alg}_S(X, LX_{\mathrm{inv}}) \ar[dd, equal] \\
& \\
\mathrm{E}G\mathrm{Alg}_S(LX , LX) \ar[r, "\sim"] & \mathrm{E}G\mathrm{Alg}_S(X, LX_{\mathrm{inv}})
\end{tikzcd} \end{eq*}
Therefore, for any map $f: LX \to LX$ there exists a unique $g: LX \to LX_{\mathrm{inv}}$ such that $j \circ g =f$. But this means that for any such $f$, we must have $\mathrm{im}(f) \subseteq L(X)_{\mathrm{inv}}$, and so in particular $L(X) = \mathrm{im}(\mathrm{id}_{LX}) \subseteq L(X)_{\mathrm{inv}}$. Since $L(X)_{\mathrm{inv}} \subseteq L(X)$ by definition, we obtain the result.
\end{proof}

This result is not especially surprising. Intuitively, it just says that when you freely add inverses to an algebra, every object ends up with an inverse. But the upshot of this is that we now have another way of thinking about $L(X)$: as the target object of the unit of our adjunction, $\eta_X: X \to L(X)_{\mathrm{inv}}$. This means that we don't really need to know the entirety of $L$ in order to determine the free algebra $L\mathbb{G}_n$, just its unit. To find this unit directly, we can turn to the following fact about adjunctions, for which a proof can be found in Lemma 2.3.5 of Leinster's \textit{Basic Category Theory} \cite{bct}.

\begin{prop}\label{initial} Let $F \dashv G: A \to B$ be an adjunction with unit $\eta$. For any object $a$ in $A$, let $(a \downarrow G)$ denote the comma category whose objects are pairs $(b, f)$ consisting of an object $b$ from $B$ and a morphism $f: a \to G(b)$ from $A$, and whose morphisms $h: (b, f) \to (b', f')$ are morphisms $f: b \to b'$ from $B$ such that $G(f) \circ f = f'$. Then the pair $\big(F(a), \eta_a: a \to GF(a) \big)$ is an initial object of $(a \downarrow G)$.
\end{prop}

\begin{cor} $\eta_{\mathbb{G}_n}: \mathbb{G}_n \to (L\mathbb{G}_n)_{\mathrm{inv}} = L\mathbb{G}_n$ is an initial object of $(\mathbb{G}_n \downarrow \mathrm{inv})$.
\end{cor}

Being able to view $L\mathbb{G}_n$ as the initial object in the comma category $(\mathbb{G}_n \downarrow \mathrm{inv})$ is pretty useful. This is because it lets us think about the properties of $L\mathbb{G}_n$ in terms of maps $\psi: \mathbb{G}_n \to X_{\mathrm{inv}}$, and this is exactly the context where we can exploit $\mathbb{G}_n$'s status as a free algebra. As a result, it is worth taking some time to think about what exactly this map $\eta_{\mathbb{G}_n}$ is.

\begin{lem} The initial object $\eta_{\mathbb{G}_n}: \mathbb{G}_n \to L\mathbb{G}_n$ is the obvious map from the free $\mathrm{E}G$-algebra on $n$ objects into the free $\mathrm{E}G$-algebra on $n$ \emph{invertible} objects. That is, $\eta_{\mathbb{G}_n}$ is the algebra map defined by
\begin{eq*} \begin{array}{rrrcl}
			\eta_{\mathbb{G}_n} & : & \mathbb{G}_n & \to & L\mathbb{G}_n \\
			& : & F(\{z_1, ..., z_n\}) & \to & LF(\{z_1, ..., z_n\}) \\
			& : & z_i & \mapsto & z_i
		\end{array}
\end{eq*}
\end{lem}
\begin{proof}
Consider the $n$-tuple $(z_1, ..., z_n)$ in $(\mathbb{G}_n)^n$. Clearly the image of $(z_1, ..., z_n)$ under the functor $L$ is just the object $(z_1, ..., z_n)$ in the algebra 
\begin{eq*} L\big( \, (\mathbb{G}_n)^n \, \big) \quad = \quad (L\mathbb{G}_n)^n \quad = \quad LF(\{z_1, ..., z_n\})^n \end{eq*}
But the image of $(z_1, ..., z_n) \in (\mathbb{G}_n)^n$ under the isomorphism
\begin{eq*} \mathrm{E}G\mathrm{Alg}_S( \, \mathbb{G}_n , \mathbb{G}_n \, ) \quad \cong \quad (\mathbb{G}_n)^n \end{eq*}
is just the identity map $\mathrm{id}_{\mathbb{G}_n}$. Thus by functoriality of $L$, the map $L(\mathrm{id}_{\mathbb{G}_n}) = \mathrm{id}_{L\mathbb{G}_n}$ must be the one which corresponds to the $n$-tuple $(z_1, ..., z_n) \in (L\mathbb{G}_n)^n$ image via the isomorphism
\begin{eq*} \mathrm{E}G\mathrm{Alg}_S( \, L\mathbb{G}_n , L\mathbb{G}_n \, ) \quad \cong \quad (L\mathbb{G}_n)^n \end{eq*}
Furthermore, the $\mathbb{G}_n$ component of the unit $\eta$ is by definition the image of the identity map $\mathrm{id}_{L\mathbb{G}_n}$ under the isomorphism
\begin{eq*}\mathrm{E}G\mathrm{Alg}_S( \, L\mathbb{G}_n , L\mathbb{G}_n \, ) \quad \cong \quad \mathrm{E}G\mathrm{Alg}_S( \, \mathbb{G}_n, L\mathbb{G}_n \, ) \end{eq*}
Hence it follows that $\eta_{\mathbb{G}_n}$ is the map that corresponds to $(z_1, ..., z_n)$ via
\begin{eq*} \mathrm{E}G\mathrm{Alg}_S( \, \mathbb{G}_n, L\mathbb{G}_n \, ) \quad \cong \quad (L\mathbb{G}_n)^n \end{eq*}
which is exactly the definition given in the statement of the lemma.
\end{proof}

This incredibly simple description makes the map $\eta_{\mathbb{G}_n}$ very easy to work with. For example, we immediately obtain the following property, one which we will use frequently throughout the rest of the paper:

\begin{cor} \label{epi} $\eta_{\mathbb{G}_n}$ is an epimorphism in $\mathrm{E}G\mathrm{Alg}_S$.
\end{cor}
\begin{proof}
Let $\phi, \psi: L\mathbb{G}_n \to X$ be a pair of algebra maps for which $\phi \circ \eta_{\mathbb{G}_n} = \psi \circ \eta_{\mathbb{G}_n}$. Then on the generators of $L\mathbb{G}_n$ we have
\begin{eq*} \phi(z_i) \quad = \quad \phi\eta_{\mathbb{G}_n}(z_i) \quad = \quad \psi\eta_{\mathbb{G}_n}(z_i) \quad = \quad \psi(z_i) \end{eq*}
and thus also in the restricted case $\phi_{\mathrm{inv}}(z_i) = \psi_{\mathrm{inv}}(z_i)$. But $L\mathbb{G}_n$ is the free $\mathrm{E}G$-algebra on $n$ invertible objects, so maps $L\mathbb{G}_n \to X_{\mathrm{inv}}$ are determined uniquely by where they those generating objects. It follows that $\phi_{\mathrm{inv}} = \psi_{\mathrm{inv}}$, and if $i: X_{\mathrm{inv}} \to X$ is the obvious inclusion,
\begin{eq*} \phi \quad = \quad i \phi_{\mathrm{inv}} \quad = \quad i \psi_{\mathrm{inv}} \quad = \quad \psi \end{eq*}
\end{proof}

Before moving on, we'll make a small change in notation. From now on, rather than writing objects in $(\mathbb{G}_n \downarrow \mathrm{inv})$ as maps $\psi: \mathbb{G}_n \to Y_{\mathrm{inv}}$, we will instead just let $X = Y_{\mathrm{inv}}$ and speak of maps $\psi: \mathbb{G}_n \to X$. This is purely to prevent the notation from becoming cluttered, and shouldn't be a problem so long as we always remember that the targets of these maps only ever contain invertible objects and morphisms. We'll also drop the subscript from $\eta_{\mathbb{G}_n}$, since it is the only component of the unit we'll ever use.

\section{The objects of $L\mathbb{G}_n$}

So $L\mathbb{G}_n$ is an initial object in the category $(\mathbb{G}_n \downarrow \mathrm{inv})$. But what does this actually tell us? After all, we do not currently have a method for finding initial objects in an arbitrary collection of $\mathrm{E}G$-algebra maps. Because of this, we'll have to approach the problem step-by-step, using the initiality of $\eta$ to extract different pieces of information about the algebra $L\mathbb{G}_n$ as we go. We'll begin by trying to find its objects.

\begin{defn}\label{Obdef} Denote by $\mathrm{Ob}: \mathrm{E}G\mathrm{Alg}_S \to \mathrm{Mon}$ the functor that sends $\mathrm{E}G$-algebras $X$ to their monoid of objects $\mathrm{Ob}(X)$, and algebra maps $F: X \to Y$ to their underlying monoid homomorphism $\mathrm{Ob}(F): \mathrm{Ob}(X) \to \mathrm{Ob}(Y)$. \end{defn}

In order to find $\mathrm{Ob}(L\mathbb{G}_n)$, we'll need to make use of an important result about the nature of $\mathrm{Ob}$ --- it is part of an adjunction.

\begin{defn}\label{Edef2} Recall from \cref{Edef} that given a monoid $M$, the monoidal category $\mathrm{E}M$ is the one whose monoid of objects is $M$ and which has a unique isomorphism between any two objects. We can view $\mathrm{E}M$ as not just a category but an $\mathrm{E}G$-algebra, by letting the action on morphisms take the only possible values it can, given the required source and target. Then for any monoid homomorphisms $h: M \to M'$, the definition of $\mathrm{E}h: \mathrm{E}M \to \mathrm{E}M'$ given in \cref{Edef} must be a well-defined map of $\mathrm{E}G$-algebras, by functoriality. Thus we can also view $\mathrm{E}$ as a functor $\mathrm{Mon} \to \mathrm{E}G\mathrm{Alg}_S$.
 \end{defn}

\begin{prop}\label{Obadj} $\mathrm{E}$ is a right adjoint to the functor $\mathrm{Ob}$. 
\end{prop}
\begin{proof}
For any $\mathrm{E}G$-algebra $X$, a map $F: X \to \mathrm{E}M$ is determined entirely by its restriction to objects, the monoid homomorphism $\mathrm{Ob}(F) : \mathrm{Ob}(X) \to M$. This is because functoriality of $F$ ensures that any map $x \to x'$ in $X$ must be sent to a map $F(x) \to F(x')$ in $\mathrm{E}M$, and by the definition of $\mathrm{E}$ there is always exactly one of these to choose from. In other words, we have an isomorphism between the homsets
\begin{eq*} \mathrm{E}G\mathrm{Alg}_S( \, X, \, \mathrm{E}M \, ) \quad \cong \quad \mathrm{Mon}( \, \mathrm{Ob}(X), \, M \, ) \end{eq*}
Additionally, this isomorphism is natural in both coordinates. That is, for any $G: X \to X'$ in $\mathrm{E}G\mathrm{Alg}_S$ and $h : M \to M'$ in $\mathrm{Mon}$, the diagram
\begin{eq*} \begin{tikzcd}
\mathrm{E}G\mathrm{Alg}_S(X, \mathrm{E}M) \ar[dd, "\mathrm{E}h \circ \_ \circ G"'] \ar[r, "\sim"] & \mathrm{Mon}(\mathrm{Ob}(X), M) \ar[dd, "h \circ \_ \circ \mathrm{Ob}(G)"] \\
& \\
\mathrm{E}G\mathrm{Alg}_S(X', \mathrm{E}M') \ar[r, "\sim"] & \mathrm{Mon}(\mathrm{Ob}(X'), M')
\end{tikzcd} \end{eq*}
commutes, because
\begin{eq*} \mathrm{Ob}( \, \mathrm{E}h \circ F \circ G \, ) \quad = \quad \mathrm{Ob}(Eh) \circ \mathrm{Ob}(F) \circ \mathrm{Ob}(G) \quad = \quad h \circ \mathrm{Ob}(F) \circ \mathrm{Ob}(G) \end{eq*}
Therefore, $\mathrm{Ob} \dashv \mathrm{E}$.
\end{proof}

What \cref{Obadj} is essentially saying is that the functor $\mathrm{Ob}$ provides a way for us to move back and forth between the categories $\mathrm{E}G\mathrm{Alg}_S$ and $\mathrm{Mon}$. By applying this reasoning to the universal property of the initial object $\eta$, we can then determine the value of $\mathrm{Ob}(L\mathbb{G}_n)$ in terms of a new universal property of $\mathrm{Ob}(\eta)$ in the category $\mathrm{Mon}$. In particular, the algebras in $(\mathbb{G}_n \downarrow \mathrm{inv})$ are those whose objects are all invertible, and so the induced property of $\mathrm{Ob}(\eta)$ will end up saying something about the relationship between $\mathrm{Ob}(\mathbb{G}_n)$ and groups --- those monoids whose elements are all invertible.

\begin{defn} Let $M$ be a monoid, $M^{\mathrm{gp}}$ a group, and $i: M \to M^{\mathrm{gp}}$ a monoid homomorphism between them. Then we say that $M^{\mathrm{gp}}$ is the \emph{group completion} of $M$ if for any other group $H$ and homomorphism $h: M \to H$, there exists a unique homomorphism $u: M^{\mathrm{gp}} \to H$ such that $u \circ i = h$.
\end{defn}

There are several different ways to actually calculate the group completion of a monoid. One is to use that fact that $M^{\mathrm{gp}}$ is the group whose group presentation is the same as the monoid presentation of $M$. That is, if $M$ is the quotient of the free monoid on generators $\mathcal{G}$ by the relations $\mathcal{R}$, then $M^{\mathrm{gp}}$ is the quotient of the free \emph{group} on generators $\mathcal{G}$ by relations $\mathcal{R}$. This makes finding the completion of free monoids particularly simple.

\begin{prop}\label{Zobj} The object monoid of $L\mathbb{G}_n$ is $\mathbb{Z}^{*n}$, the group completion of the object monoid of $\mathbb{G}_n$. The restriction of $\eta$ on objects, $\mathrm{Ob}(\eta)$, is then the obvious inclusion $\mathbb{N}^{*n} \hookrightarrow \mathbb{Z}^{*n}$.
\end{prop}
\begin{proof}
Let $H$ be a group, and $h: \mathrm{Ob}(\mathbb{G}_n) \to H$ a monoid homomorphism. By \cref{Obadj} we have an isomorphism of homsets
\begin{eq*} \mathrm{E}G\mathrm{Alg}_S( \, \mathbb{G}_n, \, \mathrm{E}H \, ) \quad \cong \quad \mathrm{Mon}( \, \mathrm{Ob}(\mathbb{G}_n), \, H \, ) \end{eq*}
Denote by $h': \mathbb{G}_n \to \mathrm{E}H$ the map of $\mathrm{E}G$-algebras corresponding to $h$ under this isomorphism. Since $H$ is a group, every object in $\mathrm{E}H$ is invertible, and so $h'$ is an object of $(\mathbb{G}_n \downarrow \mathrm{inv})$. Thus, by initiality of $\eta$, there must exist a unique map $u: L\mathbb{G}_n \to \mathrm{E}G$ making the left-hand triangle below commute:
\begin{eq*} \begin{tikzcd}
\mathbb{G}_n \ar[dd, "\eta"'] \ar[ddrr, "h'"] & & & & \mathrm{Ob}(\mathbb{G}_n) \ar[dd, "\mathrm{Ob}(\eta)"'] \ar[ddrr, "h"] & & \\
& & & & & & \\
L\mathbb{G}_n \ar[rr, "u"'] & & \mathrm{E}H & & \mathrm{Ob}(L\mathbb{G}_n) \ar[rr, "\mathrm{Ob}(u)"'] & & H
\end{tikzcd} \end{eq*}
It follows that the righthand triangle --- which is the image of the first under $\mathrm{Ob}$ --- also commutes. Hence for any group $H$ and homomorphism $h: \mathrm{Ob}(\mathbb{G}_n) \to H$, there is at least one map which factors $h$ through $\mathrm{Ob}(\eta)$.

But now recall from \cref{epi} that $\eta$ is an epimorphism. Left adjoint functors preserve epimorphisms, which means that $\mathrm{Ob}(\eta)$ is one too, and so for any $v: \mathrm{Ob}(L\mathbb{G}_n) \to H$,
\begin{eq*} \begin{array}{rllcrll}
			v \circ \mathrm{Ob}(\eta) & = & h & \implies & v \circ \mathrm{Ob}(\eta) & = & \mathrm{Ob}(u) \circ \mathrm{Ob}(\eta) \\
			& & & \implies & v & = & \mathrm{Ob}(u)
		\end{array}
\end{eq*}
Thus there is actually only one possible map which factors $h$ through $\mathrm{Ob}(\eta)$, and therefore every homomorphism from $\mathrm{Ob}(\mathbb{G}_n)$ onto a group factors uniquely through the group $\mathrm{Ob}(L\mathbb{G}_n)$. In other words, $\mathrm{Ob}(L\mathbb{G}_n)$ is the group completion $\mathrm{Ob}(\mathbb{G}_n)^{\mathrm{gp}}$. Since by \cref{Gnobj} the object monoid of $\mathbb{G}_n$ is $\mathbb{N}^{\ast n}$, the free monoid on $n$ generators, we can conclude that
\begin{eq*} \mathrm{Ob}(L\mathbb{G}_n) \quad = \quad \mathrm{Ob}(\mathbb{G}_n)^{\mathrm{gp}} \quad = \quad (\mathbb{N}^{\ast n})^{\mathrm{gp}} \quad = \quad \mathbb{Z}^{\ast n} \end{eq*}
the free group on $n$ generators. Moreover, the map $\mathrm{Ob}(\eta)$ is then the inclusion of $\mathrm{Ob}(\mathbb{G}_n)$ into its completion, which is just $\mathbb{N}^{*n} \hookrightarrow \mathbb{Z}^{*n}$.
\end{proof}

\section{The connected components of $L\mathbb{G}_n$}

The core result of \cref{Zobj} --- that $\mathrm{Ob}(L\mathbb{G}_n)$ is the group completion of $\mathrm{Ob}(\mathbb{G}_n)$ --- makes concrete the sense in which the functor $L$ represents `freely adding inverses' to objects. Extending this same logic to connected components as well, it would seem reasonable to expect that $\pi_0(L\mathbb{G}_n)$ is also the group completion of $\pi_0(\mathbb{G}_n)$. This is indeed the case, and the proof proceeds in a way completely analogous to \cref{Zobj}. 

First, we want to show that the process of taking connected components forms part of an adjunction. To do this we are going to need a category from which we can draw the kind of structures that can act as the components of an $\mathrm{E}G$-algebra. Exactly which category this should be will depend on our choice of action operad $G$, or more precisely its underlying permutations.

\begin{defn} For a given action operad $G$, denote by $\mathrm{im}(\pi)\mbox{-}\mathrm{Mon}$ the full subcategory of $\mathrm{Mon}$ on those monoids whose multiplication is invariant under the permutations in $\mathrm{im}(\pi)$. That is, a monoid $M$ is in $\mathrm{im}(\pi)\mbox{-}\mathrm{Mon}$ if and only if
\begin{eq*} m_1, ..., m_n \in M, \, g \in G(n) \quad \implies \quad m_1 \cdot ... \cdot m_n \, = \, m_{\pi(g)^{-1}(1)} ... m_{\pi(g)^{-1}(n)} \end{eq*}
\end{defn}

Of course, by \cref{surjortriv} there are really only two examples of such an $\mathrm{im}(\pi)\mbox{-}\mathrm{Mon}$. If the underlying permutations of $G$ are trivial, then $\mathrm{im}(\pi)\mbox{-}\mathrm{Mon}$ is just the whole of the category $\mathrm{Mon}$; if  instead $G$ is crossed then we are asking for monoids whose multiplication is invariant under arbitrary permutations from $\mathrm{S}$, and so $\mathrm{im}(\pi)\mbox{-}\mathrm{Mon}$ is just the category of \emph{commutative} monoids, $\mathrm{CMon}$. Regardless, when we are working with an arbitrary action operad $G$, the category $\mathrm{im}(\pi)\mbox{-}\mathrm{Mon}$ is exactly the collection of possible connected components that we were looking for.

\begin{lem}\label{pi0} Let $G$ be an action operad and $\mathrm{im}(\pi)$ its underlying permutation action operad. Then there is a functor
\begin{eq*} \pi_0: \mathrm{E}G\mathrm{Alg}_S \to \mathrm{im}(\pi)\mbox{-}\mathrm{Mon} \end{eq*}
which sends each algebra $X$ to its monoid of connected components $\pi_0(X)$, and sends each map of algebras $F: X \to Y$ to its restriction to connected components $\pi_0(F): \pi_0(X) \to \pi_0(Y)$.
\end{lem}
\begin{proof}
Let $x_1, ..., x_n$ be an arbitrary collection of objects from the algebra $X$, and $g$ an element of the group $G(n)$. Then the action of $G$ guarantees the existence of a morphism
\begin{eq*} \alpha(g; \mathrm{id}_{x_1}, ..., \mathrm{id}_{x_n}) \, : \, x_1 \otimes ... \otimes x_n \to x_{\pi(g^{-1})(1)} \otimes ... \otimes x_{\pi(g^{-1})(n)} \end{eq*}
By definition the source and target of this morphism belong to the same connected component, and hence
\begin{eq*} \begin{array}{rll}
			[ \, x_1 \otimes ... \otimes x_n \, ] & = & [ \, x_{\pi(g^{-1})(1)} \otimes ... \otimes x_{\pi(g^{-1})(n)} \, ] \\
			\implies \quad [x_1] \otimes ... \otimes [x_n] & = & [x_{\pi(g^{-1})(1)}] \otimes ... \otimes [x_{\pi(g^{-1})(n)}]
		\end{array} 
\end{eq*}
But since the $x_i$ are just arbitrary objects of $X$, the components $[x_i]$ are an arbitrary collection of elements from $\pi_0(X)$, and likewise for the group element $g$ and the permutation $\pi(g)$. Therefore multiplication in the monoid $\pi_0(X)$ is invariant under all permutations in the images of the homomorphisms $\pi_n: G(n) \to S_n$, and thus $\pi_0(X)$ is an object of $\mathrm{im}(\pi)\mbox{-}\mathrm{Mon}$, as required. Well-definedness of the functor $\pi_0$ on morphisms then follows immediately from the fullness of $\mathrm{im}(\pi)\mbox{-}\mathrm{Mon}$.
\end{proof}

Now that we have a functor which represents the act of finding the connected component monoid of an algebra, we need another functor heading in the opposite direction, so that we can construct an adjunction between them.

\begin{defn} There exists an inclusion of 2-categories $\mathrm{D}: \mathrm{Set} \hookrightarrow \mathrm{Cat}$ which allows us to view any set $S$ as a \emph{discrete category}, one whose objects are just the elements of $S$ and whose morphisms are all identities. If the given set also happens to be a monoid $M$, then there is an obvious way to see the discrete category $\mathrm{D}M$ as a monoidal category, and so we have a similar inclusion $\mathrm{D}: \mathrm{Mon} \hookrightarrow \mathrm{MonCat}$. Finally, for any action operad $G$ and object $M$ of the category $\mathrm{im}(\pi)\mbox{-}\mathrm{Mon}$, there is a unique way to assign an $\mathrm{E}G$-action to the discrete category $\mathrm{D}M$. This works because for any elements $m_1, ..., m_n \in M$ and $g \in G(n)$, the morphism $\alpha(g; \mathrm{id}_{m_1}, ..., \mathrm{id}_{m_n})$ must have source and target 
\begin{eq*} m_1 \otimes ... \otimes m_n  \quad = \quad m_{\pi(g^{-1})(1)} \otimes ... \otimes m_{\pi(g^{-1})(m)} \end{eq*}
and therefore it can only be the morphism $\mathrm{id}_{m_1 \otimes ... \otimes m_n}$. When $G$ is crossed this yields one last inclusion $\mathrm{CMon} \hookrightarrow \mathrm{E}G\mathrm{Alg}_S$, which we shall also call $\mathrm{D}$. \end{defn}

\begin{prop}\label{concompadj} $\mathrm{D}$ is a right adjoint to the functor $\pi_0$. 
\end{prop}
\begin{proof}
Consider a map of $F: X \to \mathrm{D}C$ from some $\mathrm{E}G$-algebra $X$ onto the discrete $\mathrm{E}G$-algebra for a monoid $M$ in $\mathrm{im}(\pi)\mbox{-}\mathrm{Mon}$. For any $f: x \to x'$ in $X$, the morphism $F(f)$ must be an identity map in $\mathrm{D}M$, since these are the only morphisms that $\mathrm{D}M$ has. It follows that $x$ and $x'$ being in the same connected component will imply $F(x) = F(x')$, and so $F$ is determined entirely by its restriction to connected components, the monoid homomorphism $\pi_0(F) : \pi_0(X) \to M$. In other words, we have an isomorphism between the homsets
\begin{eq*} \mathrm{E}G\mathrm{Alg}_S( \, X, \mathrm{D}M \, ) \quad \cong \quad \mathrm{im}(\pi)\mbox{-}\mathrm{Mon}( \, \pi_0(X), M \, ) \end{eq*}
This isomorphism is natural in both coordinates, since for any $G: X \to X'$ in $\mathrm{E}G\mathrm{Alg}_S$ and $h : M \to M'$ in $\mathrm{im}(\pi)\mbox{-}\mathrm{Mon}$, 
\begin{eq*} \pi_0( \, \mathrm{D}h \circ F \circ G \, ) \quad = \quad \pi_0(\mathrm{D}h) \circ \pi_0(F) \circ \pi_0(G) \quad = \quad h \circ \pi_0(F) \circ \pi_0(G) \end{eq*}
and so the diagram
\begin{eq*} \begin{tikzcd}
\mathrm{E}G\mathrm{Alg}_S(X, \mathrm{D}M) \ar[dd, "\mathrm{D}h \circ \_ \circ G"'] \ar[rr, "\sim"] & & \mathrm{im}(\pi)\mbox{-}\mathrm{Mon}\big( \, \pi_0(X), M \, \big) \ar[dd, "h \circ \_ \circ \pi_0(G)"] \\
& & \\
\mathrm{E}G\mathrm{Alg}_S(X', \mathrm{D}M') \ar[rr, "\sim"] & & \mathrm{im}(\pi)\mbox{-}\mathrm{Mon}\big( \, \pi_0(X'), M' \, \big) 
\end{tikzcd} \end{eq*}
commutes. Therefore, $\pi_0 \dashv \mathrm{D}$.
\end{proof}

Now we can utilise \cref{concompadj} to draw out a universal property of $\pi_0(L\mathbb{G}_n)$, just as we did with $\mathrm{Ob}(L\mathbb{G}_n)$ in \cref{Obadj}.

\begin{prop}\label{Zconcomp} The connected components of $L\mathbb{G}_n$ are the group completion of the connected components of $\mathbb{G}_n$. Also, the restriction of $\eta$ onto connected components, $\pi_0(\eta)$, is the canonical map $\pi_0(\mathbb{G}_n) \to \pi_0(\mathbb{G}_n)^{\mathrm{gp}}$ associated with that group completion.
\end{prop}
\begin{proof}
Let $H$ be a group which is also an object of $\mathrm{im}(\pi)\mbox{-}\mathrm{Mon}$, and let $h: \pi_0(\mathbb{G}_n) \to H$ be a monoid homomorphism. By \cref{concompadj} there is a homset isomorphism
\begin{eq*} \mathrm{E}G\mathrm{Alg}_S( \, \mathbb{G}_n, \, \mathrm{D}H \, ) \quad \cong \quad \mathrm{im}(\pi)\mbox{-}\mathrm{Mon}( \, \pi_0(\mathbb{G}_n), \, H \, ) \end{eq*}
and thus some $\mathrm{E}G$-algebra map $h': \mathbb{G}_n \to \mathrm{D}H$ corresponding to $h$. As $H$ is a group, every object of $\mathrm{D}H$ is invertible, and so $h'$ is an object of $(\mathbb{G}_n \downarrow \mathrm{inv})$. It follows that there exists a unique map $u: L\mathbb{G}_n \to \mathrm{D}M$ which factors $h'$ through the initial object $\eta$:
\begin{eq*} \begin{tikzcd}
\mathbb{G}_n \ar[dd, "\eta"'] \ar[ddrr, "h'"] & & & & \pi_0(\mathbb{G}_n) \ar[dd, "\pi_0(\eta)"'] \ar[ddrr, "h"] & & \\
& & & & & & \\
L\mathbb{G}_n \ar[rr, "u"'] & & \mathrm{D}H & \quad & \pi_0(L\mathbb{G}_n) \ar[rr, "\pi_0(u)"'] & & H
\end{tikzcd} \end{eq*}
Applying the functor $\pi_0$ everywhere, we see that $\pi_0(u)$ must also factor $h$ through the homomorphism $\pi_0(\eta)$. Moreover, since $\eta$ is an epimorphism and $\pi_0$ a left adjoint functor, $\pi_0(\eta)$ is an epimorphism too, and so $\pi_0(u)$ is the only map with this property. Therefore, any monoid homomorphism $\pi_0(\mathbb{G}_n) \to H$ will factor uniquely through $\pi_0(L\mathbb{G}_n)$, so long as $H$ is in $\mathrm{im}(\pi)\mbox{-}\mathrm{Mon}$.  

Now consider another monoid homomorphism $k: \pi_0(\mathbb{G}_n) \to K$, where this time $K$ is still a group but not necessarily in $\mathrm{im}(\pi)\mbox{-}\mathrm{Mon}$. From \cref{pi0}, we know that $\pi_0(\mathbb{G}_n)$ is still an object of $\mathrm{im}(\pi)\mbox{-}\mathrm{Mon}$, and from this we can conclude that the image $\mathrm{im}(k)$ will be too:
\begin{eq*} \begin{array}{rrcl}
			& x_1, ..., x_m & \in & \pi_0(\mathbb{G}_n), \, g \in G(n) \\
			\implies & x_1 \otimes ... \otimes x_m & = & x_{\pi(g)(1)} \otimes ... \otimes x_{\pi(g)(m)} \\
			\implies & k( \, x_1 \otimes ... \otimes x_m \, ) & = & k( \, x_{\pi(g)(1)} \otimes ... \otimes x_{\pi(g)(m)} \, ) \\
			\implies & k(x_1) \otimes ... \otimes k(x_m) & = & k(x_{\pi(g)(1)}) \otimes ... \otimes k(x_{\pi(g)(m)})
		\end{array}
\end{eq*}
Also, since $\mathrm{im}(k)$ is a submonoid of the group $K$, it is a group as well. Thus if we denote by $k_{\mathrm{im}}: \mathrm{Ob}(\mathbb{G}_n) \to \mathrm{im}(k)$ the restriction of $k$ to it image, then $k_{\mathrm{im}}$ is a map in $\mathrm{im}(\pi)\mbox{-}\mathrm{Mon}$ out of $\mathrm{Ob}(\mathbb{G}_n)$ and onto a group, and therefore by what we showed earlier there exists a unique homomorphism $v: \mathrm{Ob}(L\mathbb{G}_n) \to \mathrm{im}(k)$ with the property $v \circ \pi_0(\eta) = k_{\mathrm{im}}$. Composing this $v$ with the inclusion $i: \mathrm{im}(k) \hookrightarrow K$, we see that
\begin{eq*} i \circ v \circ \pi_0(\eta) \, = \, i \circ k_{\mathrm{im}} \, = \, k \end{eq*}
and $i \circ v$ must be the only map for which this is true, for restricting this equation back onto $\mathrm{im}(k)$ yields the unique property of $v$ again. Thus $\pi_0(\eta)$ will actually take any homomorphism from $\mathrm{Ob}(\mathbb{G}_n)$ onto a group and factor it through $\pi_0(L\mathbb{G}_n)$ in a unique way, not just those homomorphisms in $\mathrm{im}(\pi)\mbox{-}\mathrm{Mon}$. In other words, 
\begin{eq*} \pi_0(L\mathbb{G}_n) \quad = \quad \pi_0(\mathbb{G}_n)^{\mathrm{gp}} \end{eq*}
and $\pi_0(\eta)$ is the canonical map of this group completion.
\end{proof}

As we've said before, this result is a reflection of the fact that the functor $L$ is trying to add inverses the objects of $\mathbb{G}_n$ freely, that is, with as little effect on the rest of the algebra as possible. Indeed, if we happen to know whether or not our action operad $G$ is crossed then we can now calculate exactly what the effect on the components will be.

\begin{cor}\label{crossconcomp} If $G$ is a crossed action algebra then
\begin{itemize} \itemsep0em
\item the connected components of $L\mathbb{G}_n$ are the monoid $\mathbb{Z}^n$
\item the restriction of $\eta$ to components is the obvious inclusion $\mathbb{N}^n \hookrightarrow \mathbb{Z}^n$
\item the assignment of objects to their component is given by the quotient map of abelianisation $\mathrm{ab}: \mathbb{Z}^{\ast n} \to \mathbb{Z}^n$
\end{itemize}
If instead $G$ is non-crossed, then
\begin{itemize} \itemsep0em
\item the connected components of $L\mathbb{G}_n$ are the monoid $\mathbb{Z}^{\ast n}$
\item the restriction of $\eta$ to components is the obvious inclusion $\mathbb{N}^{\ast n} \hookrightarrow \mathbb{Z}^{\ast n}$
\item the assignment of objects to their component is $\mathrm{id}_{\mathbb{Z}^{\ast n}}$
\end{itemize}
\end{cor}
\begin{proof}
Combining \cref{Zconcomp,Gnconcomp}, we see that
\begin{eq*} \pi_0(L\mathbb{G}_n) \quad = \quad \pi_0(\mathbb{G}_n)^{\mathrm{gp}} \quad = \quad \begin{cases}
													\quad (\mathbb{N}^n)^{\mathrm{gp}} \quad = \quad \mathbb{Z}^n & \text{if $G$ is crossed} \\
													\quad (\mathbb{N}^{\ast n})^{\mathrm{gp}} \quad = \quad \mathbb{Z}^{\ast n} & \text{otherwise}
														\end{cases}
\end{eq*}
Moreover, \cref{Zconcomp} says that restriction of $\eta$ to connected components, $\pi_0(\eta)$, will be the homomorphism associated with these group completion, which means the inclusion $\mathbb{N}^n \hookrightarrow \mathbb{Z}^n$ when $G$ is crossed and $\mathbb{N}^{\ast n} \hookrightarrow \mathbb{Z}^{\ast n}$ when it is not.

Next, by \cref{Gnconcomp} we know that the map $[ \, \_ \, ] : \mathrm{Ob}(\mathbb{G}_n) \to \pi_0(\mathbb{G}_n)$ sending objects of $\mathbb{G}_n$ to their connected component is either the quotient map of abelianisation $\mathbb{N}^{\ast n} \to \mathbb{N}^n$ or the identity on $\mathbb{N}^{\ast n}$, depending on whether or not it is crossed. If we also use $[ \, \_ \, ]$ to denote the map sending objects of $L\mathbb{G}_n$ to their components, it then follows from functoriality of $\eta$ that the corresponding choice of the following two diagrams will commute:
\begin{eq*} \begin{tikzcd}
\mathbb{N}^{\ast n} \ar[dd, hookrightarrow, "\lbrack \, \_ \, \rbrack"'] \ar[rr, hookrightarrow, "\mathrm{Ob}(\eta)"] & & \mathbb{Z}^{\ast n} \ar[dd, "\lbrack \, \_ \, \rbrack"] & \quad & \mathbb{N}^{\ast n} \ar[dd, equals, "\lbrack \, \_ \, \rbrack"'] \ar[rr, hookrightarrow, "\mathrm{Ob}(\eta)"] & & \mathbb{Z}^{\ast n} \ar[dd, "\lbrack \, \_ \, \rbrack"] \\
& & & & \\
\mathbb{N}^n \ar[rr, hookrightarrow, "\pi_0(\eta)"'] & & \mathbb{Z}^n & & \mathbb{N}^{\ast n} \ar[rr, hookrightarrow, "\pi_0(\eta)"'] & & \mathbb{Z}^{\ast n}
\end{tikzcd} \end{eq*}
Using the values of $[ \, \_ \, ]$ from \cref{Gnconcomp}, $\mathrm{Ob}(\eta)$ from \cref{Zobj}, and $\pi_0(\eta)$ from earlier in this proof, it follows that for any generator $z_i$ of $\mathbb{Z}^{\ast n}$, 
\begin{eq*} [z_i] \quad = \quad [\mathrm{Ob}(\eta)(z_i)] \quad = \quad \pi_0(\eta)([z_i]) \quad = \quad \pi_0(\eta)(z_i) \quad = \quad z_i \end{eq*}
But this description of $[ \, \_ \, ]: \mathrm{Ob}(L\mathbb{G}_n) \to \pi_0(L\mathbb{G}_n)$ on generators is either the definition of the quotient map $\mathrm{ab}: \mathbb{Z}^{\ast n} \to (\mathbb{Z}^{\ast n})^{\mathrm{ab}}$ or the identity $\mathrm{id}: \mathbb{Z}^{\ast n} \to \mathbb{Z}^{\ast n}$, depending on the value of target monoid, as required.
\end{proof}

\section{The collapsed morphisms of $L\mathbb{G}_n$}  

Now that we understand the objects and connected components of the algebra $L\mathbb{G}_n$, the next most obvious thing to look for are its morphisms, $\mathrm{Mor}(L\mathbb{G}_n)$. It would be nice to construct this collection in the same way we constructed $\mathrm{Ob}(L\mathbb{G}_n)$ and $\pi_0(L\mathbb{G}_n)$, by applying the left adjoint of some adjunction to the initial map $\eta$. Before we can do this however, we need to ask ourselves a question. What sort of mathematical object is $\mathrm{Mor}(L\mathbb{G}_n)$, exactly?

Given a pair of morphisms $f: x \to y, f': y' \to z$ in an $\mathrm{E}G$-algebra $X$, there are two basic binary operations we can perform. First, we can take their tensor product $f \otimes f'$, and this together with the unit map $\mathrm{id}_{I}$ imbues $\mathrm{Mor}(X)$ with the structure of a monoid. Second, if we have $y = y'$ then we can form the composite morphism $f' \circ f$. However, these two operations are not as different as they first appear.

\begin{lem} \label{tenscomp} Let $f: x \to y$ and $f': y \to z$ be morphisms in some strict monoidal category, and $y$ is an invertible object of that category. Then
\begin{eq*} f' \circ f \quad = \quad f' \otimes \mathrm{id}_{y*} \otimes f \end{eq*}
\end{lem}
\begin{proof}
By the interchange law for monoidal categories,
\begin{eq*}\begin{array}{rll} 
			f' \circ f & = & (f' \otimes \mathrm{id}_I) \circ (\mathrm{id}_I \otimes f) \\
			& = & (f' \otimes \mathrm{id}_{y*} \otimes \mathrm{id}_y) \circ (\mathrm{id}_y \otimes \mathrm{id}_{y*} \otimes f) \\
			& = & (f' \circ \mathrm{id}_y) \otimes (\mathrm{id}_{y*} \circ \mathrm{id}_{y*}) \otimes (\mathrm{id}_y \circ f) \\
			& = & f' \otimes \mathrm{id}_{y*} \otimes f 
		\end{array}
\end{eq*}
\end{proof} 

In other words, composition along invertible objects in $X$ can always be restated in terms of the tensor product. Thus in cases where every object of $X$ is invertible, the monoidal structure together with knowledge of each morphism's source and target will be enough to determine $X$ uniquely. Since all objects in $L\mathbb{G}_n$ are invertible, this means that we could choose to ignore composition of elements of $\mathrm{Mor}(L\mathbb{G}_n)$ for the time being, and focus on its status as a monoid under tensor product.

However, we are trying to extract information about the morphisms of $L\mathbb{G}_n$ by building some sort of left adjoint functor. Presumably we will also be able to apply it to other $\mathrm{E}G$-algebras, some of which won't have all of their objects invertible, and so we can't just use $\mathrm{Mor}(-): \mathrm{E}G\mathrm{Alg}_S \to \mathrm{Mon}$. What we need is a way to modify the morphism monoid of a category so that both composition and tensor product are recoverable from a single operation. Of course, there is one very easy method for achieving this --- simply force $\otimes$ and $\circ$ to be equal.

\begin{defn} Let $\mathrm{M} : \mathrm{MonCat} \to \mathrm{Mon}$ be the functor which sends monoidal categories $X$ to the quotient of their monoid of morphisms by the relation that sets $\otimes = \circ$.  
\begin{eq*} \mathrm{M}X \quad = \quad \bigquotient{\mathrm{Mor}(X)}{f' \circ f \sim f' \otimes f}\end{eq*}
Each monoidal functors $F: X \to Y$ is then sent to the monoid homomorphism
\begin{eq*} \begin{array}{rlrll}
			\mathrm{M}(F) & : & \mathrm{M}X & \to & \mathrm{M}Y \\
			& : & \mathrm{M}(f) & \mapsto & \mathrm{M}\big( \, F(f) \, \big) \\
		\end{array}
\end{eq*}
where $\mathrm{M}(f)$ refers to the equivalence class of the map $f$ under the quotient $\mathrm{Mor}(X) \to \mathrm{M}(X)$. This homomorphism is well-defined, since it respects the relation $\otimes = \circ$:
\begin{longtable}{RLL}
	\mathrm{M}(F)( \, f' \circ f \, ) & = & \mathrm{M}\big( \, F(f' \circ f) \, \big) \\
	& = & \mathrm{M}\big( \, F(f') \circ F(f) \, \big) \\
	& = & \mathrm{M}\big( \, F(f') \, \big) \circ \mathrm{M}\big( \, F(f) \, \big) \\
	& = & \mathrm{M}\big( \, F(f') \, \big) \otimes \mathrm{M}\big( \, F(f) \, \big) \\
	& = & \mathrm{M}\big( \, F(f') \otimes F(f) \, \big) \\
	& = & \mathrm{M}\big( \, F(f' \otimes f) \, \big) \\
	& = & \mathrm{M}(F)( \, f' \otimes f \, )
\end{longtable}
We will call $\mathrm{M}X$ the \emph{collapsed} morphisms of the $X$.
\end{defn}

From now on we will generally refer to the single operation in $\mathrm{M}X$ as $\otimes$ rather than $\circ$, unless we are focusing on some aspect best understood using composition. This convention makes it easier to remember that because the tensor product is defined between all pairs of morphisms in $X$,  the equivalence class $\mathrm{M}(f') \otimes \mathrm{M}(f)$ will always contain the morphism $f' \otimes f$, but not necessarily $f' \circ f$, as it might fail to exist.

Now we need a candidate for the right adjoint to the functor $\mathrm{M}$.

\begin{defn} For a given monoid $M$, let $\mathrm{B}M$ represent the one-object category whose morphisms are the elements of $M$, with monoid multiplication as composition. This is known as the \emph{delooping} of $M$, for reasons that come from homotopy theory. Likewise, for any monoid homomorphism $h: M \to M'$, denote by $\mathrm{B}h : \mathrm{B}M \to \mathrm{B}M'$ the obvious monoidal functor which acts like $h$ on morphisms. This defines a delooping functor $\mathrm{B}: \mathrm{Mon} \to \mathrm{Cat}$ from the category of monoids onto the category of small categories.

Moreover, let $C$ be a commutative monoid. Then we can view $\mathrm{B}C$ as a monoidal category, with the tensor product also given by the multiplication in $C$, and the sole object as the unit $I$. Clearly for any homomorphism between commutative monoids $h : C \to C'$ the corresponding functor $\mathrm{B}h : \mathrm{B}C \to \mathrm{B}C'$ will preserve this monoidal structure, as it is already preserving it as composition. Thus the restriction of $\mathrm{B}$ to commutative monoids also gives a functor $\mathrm{CMon} \to \mathrm{MonCat}$, which we will still call $\mathrm{B}$.
\end{defn}

The reason that commutativity is required in order for $\mathrm{B}C$ to be a well-defined monoidal category is because we need its operations $\circ$ and $\otimes$ to obey the interchange law for monoidal categories:
\begin{eq*}\begin{array}{rrll}
			& (\mathrm{id}_I \circ f) \otimes (f' \otimes \mathrm{id}_I) & = & (\mathrm{id}_I \otimes f') \circ (f \otimes \mathrm{id}_I) \\
			\implies & \mathrm{id}_I \cdot f \cdot f' \cdot \mathrm{id}_I & = & \mathrm{id}_I \cdot f' \cdot f \cdot \mathrm{id}_I \\
			\implies & f \cdot f' & = & f' \cdot f
		\end{array}
\end{eq*}

\begin{prop}\label{Moradj} $\mathrm{B}$ is a right adjoint to the functor $\mathrm{M}(\, \_ \,)^{\mathrm{ab}} : \mathrm{MonCat} \to \mathrm{CMon}$.
\end{prop}
\begin{proof}
Let $X$ be a monoidal category, $C$ a commutative monoid, and $F: X \to \mathrm{B}C$ a monoidal functor. For any $f: x \to x'$ in $X$, the morphism $F(f)$ is just an element of the monoid $C$, and so $F$ can be used to define a function
\begin{eq*} \begin{array}{rlrll}
			F' & : & \mathrm{M}(X)^{\mathrm{ab}} & \to & C \\
			& : & \mathrm{ab} \circ \mathrm{M}(f) & \mapsto & F(f) \\
		\end{array}
\end{eq*}
where $\mathrm{ab}$ is the quotient map of abelianisation $\mathrm{M}(X) \to \mathrm{M}(X)^{\mathrm{ab}}$. This $F'$ is a well-defined monoid homomorphism; it preserves multiplication and respects the relation $\otimes = \circ$ because the monoid multiplication of $C$ is acts as both tensor product and composition in $\mathrm{B}C$.
\begin{eq*} \begin{array}{rll}
			F'\big( \, \mathrm{ab}\mathrm{M}(f' \circ f ) \, \big) & = & F(f' \circ f) \\
			& = & F(f') \circ F(f)  \\
			& = & F(f') \cdot F(f) \\
			& = & F(f') \otimes F(f) \\
			& = & F(f' \otimes f) \\
			& = & F' \big( \, \mathrm{ab}\mathrm{M}(f' \otimes f) \, \big)
		\end{array}
\end{eq*}
Conversely, if $h: \mathrm{M}(X)^{\mathrm{ab}} \to C$ is a monoid homomorphism, we can define from it a monoidal functor
\begin{eq*} \begin{array}{rlrll}
			h' & : & X & \mapsto & \mathrm{B}C \\
			& : & x & \mapsto & I \\
			& : & f: x \to y & \mapsto & h\big( \, \mathrm{ab}\mathrm{M}(f) \, \big) : I \to I
		\end{array}
\end{eq*}
Yet again, the monoidal functor $h'$ is well-defined because the fact that $\otimes = \circ$ in $\mathrm{B}C$ forces $h'$ to respect that relation.
\begin{eq*} \begin{array}{rll}
			h'(f' \circ f ) & = & h\big( \, \mathrm{ab}\mathrm{M}(f' \circ f ) \, \big) \\
			& = & h\big( \, \mathrm{ab}\mathrm{M}(f') \circ \mathrm{M}(f') \, \big) \\
			& = & h\big( \, \mathrm{ab}\mathrm{M}(f') \, \big) \circ h\big( \, \mathrm{ab}\mathrm{M}(f') \, \big) \\
			& = & h\big( \, \mathrm{ab}\mathrm{M}(f') \, \big) \cdot h\big( \, \mathrm{ab}\mathrm{M}(f') \, \big) \\
			& = & h\big( \, \mathrm{ab}\mathrm{M}(f') \, \big) \otimes h\big( \, \mathrm{ab}\mathrm{M}(f') \, \big) \\
			& = & h\big( \, \mathrm{ab}\mathrm{M}(f') \otimes \mathrm{ab}\mathrm{M}(f') \, \big) \\
			& = & h\big( \, \mathrm{ab}\mathrm{M}(f' \otimes f') \, \big) \\
			& = & h'(f' \otimes f)
		\end{array}
\end{eq*}
But these assignments $F \mapsto F'$ and $h \mapsto h'$ are clearly inverse to one another. For any $F: X \to \mathrm{B}C$ applying them twice gives
\begin{eq*} \begin{array}{rlrllll}
			F'' & : & X & \to & \mathrm{B}C & &\\
			& : & x & \mapsto & I & & \\
			& : & f: x \to y & \mapsto & F'\big( \, \mathrm{ab}\mathrm{M}(f) \, \big) : I \to I & = & F(f)
		\end{array}
\end{eq*}
and similarly for $h: \mathrm{M}X \to C$ we get
\begin{eq*} \begin{array}{rlrllll}
			h'' & : & \mathrm{M}(X)^{\mathrm{ab}} & \to & C & & \\
			& : & \mathrm{ab}\mathrm{M}(f) & \mapsto & h'(f) & = & h\big( \, \mathrm{ab}\mathrm{M}(f) \, \big)
		\end{array}
\end{eq*}
In other words, we have an isomorphism between the homsets
\begin{eq*} \mathrm{MonCat}( \, X, \mathrm{B}C \, ) \quad \cong \quad \mathrm{CMon}( \, \mathrm{M}(X)^{\mathrm{ab}}, C \, ) \end{eq*}
This isomorphism is natural in both coordinates, as for any monoidal functor $G: X \to X'$ and homomorphism $h : C \to C'$ between commutative monoids,
\begin{eq*} \mathrm{ab}\mathrm{M}( \, \mathrm{B}h \circ F \circ G \, ) \quad = \quad \mathrm{ab}\mathrm{M}(\mathrm{B}h) \circ \mathrm{ab}\mathrm{M}(F) \circ \mathrm{ab}\mathrm{M}(G) \quad = \quad h \circ \mathrm{ab}\mathrm{M}(F) \circ \mathrm{ab}\mathrm{M}(G) \end{eq*}
and so the diagram
\begin{eq*} \begin{tikzcd}
\mathrm{MonCat}(X, \mathrm{B}C) \ar[dd, "\mathrm{B}h \circ \_ \circ G"'] \ar[rr, "\sim"] & & \mathrm{CMon}\big( \, \mathrm{M}(X)^{\mathrm{ab}}, C \, \big) \ar[dd, "h \circ \_ \circ \mathrm{ab}\mathrm{M}G"] \\
& & \\
\mathrm{MonCat}(X', \mathrm{B}C') \ar[rr, "\sim"] & & \mathrm{CMon}\big( \, \mathrm{M}(X')^{\mathrm{ab}}, M' \, \big) 
\end{tikzcd} \end{eq*}
commutes. Therefore, $\mathrm{M}(\, \_ \,)^{\mathrm{ab}} \dashv \mathrm{B}$.
\end{proof}

\cref{Moradj} seems at first glance very similar to \cref{Obadj,concompadj}. However, our goal was to discover the relationship between the morphisms of $\mathbb{G}_n$ and $L\mathbb{G}_n$, paralleling what we did in \cref{Zobj,Zconcomp}, and in that regard $\mathrm{M}$ falls short in two very important ways. 

\begin{enumerate}
\item What we really wanted to have was an adjunction involving $\mathrm{E}G\mathrm{Alg}_S$, not $\mathrm{MonCat}$. This is because our previous methodology involved applying our left adjoint functors to $\eta$ and then using its initial property to factor various maps through $L\mathbb{G}_n$. But $\eta$ is an initial object in $(\mathbb{G}_n \downarrow \mathrm{inv})$, and so we only know how to use it to factor \emph{algebra} maps $\mathbb{G}_n \to X_{\mathrm{inv}}$, and not general monoidal functors. 
\item Even if we do find a way to use this adjunction to extract information about $L\mathbb{G}_n$, it will not be the monoid $\mathrm{Mor}(L\mathbb{G}_n)$ we were originally after, only a strange abelianised version where tensor product and composition coincide.  
\end{enumerate}

Unfortunately, this adjunction seems to be the best that we can do. The only general method for assigning an $\mathrm{E}G$-action to the monoidal category $\mathrm{B}C$ for arbitrary $C$ is to set all of its action morphisms $\alpha(g; \mathrm{id}_I, ..., \mathrm{id}_I)$ to be $\mathrm{id}_I$. This would then cause the homomorphism $\mathrm{M}X \to C$ corresponding to any algebra map $X \to \mathrm{B}C$ to be the zero map if $X$ has only action morphisms. Given \cref{Gnmapsaction}, this is clearly no use. However, it turns out that this approach is fixable. To that end, we will spend the bulk of the next two chapters directly addressing problems 1 and 2. 

For now though, we will make one last small alteration to our plan going forward. Instead of working directly with the functor $\mathrm{M}(\, \_ \,)^{\mathrm{ab}}: \mathrm{MonCat} \to \mathrm{CMon}$, we will instead focus on its composite with the group completion functor, $( \, \_ \, )^{\mathrm{gp}} : \mathrm{CMon} \to \mathrm{Ab}$. It may not be clear yet why we would choose to do this, but over the next couple of chapters we will frequently find ourselves having to forming quotients of certain algebraic objects. If we were to stick with the functor $\mathrm{M}$ these would all be commutative monoid quotients, whereas by making the switch to $\mathrm{M}(\, \_ \,)^{\mathrm{gp},\mathrm{ab}}$ they will be abelian groups instead, which are far easier to work with. Also, notice that since the process of group completion is left adjoint to the forgetful functor $\mathrm{Ab} \to \mathrm{CMon}$, its composite with the left adjoint $\mathrm{M}(\, \_ \,)^{\mathrm{ab}}$ will be a left adjoint functor too. Thus with this new functor we will be able use all of the same important properties that we would have done with $\mathrm{M}(\, \_ \,)^{\mathrm{ab}}$, such as the preservation of colimits. Moreover, while we won't prove this for some time, it turns out that the morphisms of $L\mathbb{G}_n$ actually form a group under tensor product. This means that whatever method we would have used to recover $\mathrm{Mor}(L\mathbb{G}_n)$ from $\mathrm{M}(L\mathbb{G}_n)^{\mathrm{ab}}$ will still let us recover $\mathrm{Mor}(L\mathbb{G}_n) = \mathrm{Mor}(L\mathbb{G}_n)^{\mathrm{gp}}$ from $\mathrm{M}(L\mathbb{G}_n)^{\mathrm{gp},\mathrm{ab}}$.

Before we move on, we should spend a little time thinking about this new functor $\mathrm{M}(\, \_ \,)^{\mathrm{gp},\mathrm{ab}}$. Specifically, we might ask in what order do we have to carry out its constituent parts: the collapsing of $\circ$ and $\otimes$ into a single operation, group completion, and abelianisation. It is a well known fact that group completion and abelianisation commute:
\begin{eq*} \begin{tikzcd}
\mathrm{Mon} \ar[rr, "(\, \_ \,)^{\mathrm{gp}}"] \ar[d, "(\, \_ \,)^{\mathrm{ab}}"'] & & \mathrm{Grp} \ar[d, "(\, \_ \,)^{\mathrm{ab}}"] \\
\mathrm{CMon} \ar[rr, "(\, \_ \,)^{\mathrm{gp}}"] & & \mathrm{Ab}
\end{tikzcd} \end{eq*}
Indeed, we already assumed this when talking of `the' canonical map $\mathrm{M}(X)^{\mathrm{gp},\mathrm{ab}}$. But a more interesting question is whether it matters if we choose to group complete or abelianise the tensor product of a monoidal category before or after we collapse its morphisms.

\begin{lem}\label{Morder} For any monoidal category $X$, define
\begin{eq*} \begin{array}{rll} 
			\mathrm{M}_{\mathrm{gp}}(X) & \cong & \bigquotient{\mathrm{Mor}(X)^{\mathrm{gp}}}{\mathrm{gp}(f' \circ f) \sim \mathrm{gp}(f' \otimes f)} \\[\bigskipamount]
			\mathrm{M}_{\mathrm{ab}}(X) & \cong & \bigquotient{\mathrm{Mor}(X)^{\mathrm{ab}}}{\mathrm{ab}(f' \circ f) \sim \mathrm{ab}(f' \otimes f)}
		\end{array}
\end{eq*} 
Then
\begin{eq*} \mathrm{M}_{\mathrm{gp}}(X) \quad = \quad \mathrm{M}(X)^{\mathrm{gp}}, \quad \quad \quad \mathrm{M}_{\mathrm{ab}}(X) \quad = \quad \mathrm{M}(X)^{\mathrm{ab}} \end{eq*}
\end{lem}
\begin{proof}
Consider the following commutative diagram
\begin{eq*} \begin{tikzcd}
& \mathrm{M}(X) \ar[rr, "\mathrm{gp}"] \ar[ddrr, dashed, "v", near start] & & \mathrm{M}(X)^{\mathrm{gp}} \ar[dd, shift left, dashed, "u'"] \\
\mathrm{Mor}(X) \ar[ru, "\mathrm{M}"] \ar[rd, "\mathrm{gp}"'] & & & \\
& \mathrm{Mor}(X)^{\mathrm{gp}} \ar[rr, "\mathrm{M}"'] \ar[rruu, dashed, "u"', near start] & & \mathrm{M}_{\mathrm{gp}}(X) \ar[uu, shift left, dashed, "v'"]
\end{tikzcd} \end{eq*}
Here all of the solid arrows are the respective canonical homomorphisms.

Starting from the left, the top edge of the diagram is a map coming out of $\mathrm{Mor}(X)$ and going into a group, and so by the universal property of the group completion there is a unique homomorphism $u$ factoring it through $\mathrm{Mor}(X)^{\mathrm{gp}}$. But now this $u$ is a map out of $\mathrm{Mor}(X)^{\mathrm{gp}}$ and into group where tensor product and composition are equal, and so by the universal property of the quotient this factors once more through the map $u'$. On the other hand, the bottom edge of the diagram will factor through the map $v$ because of the collapsed morphisms property, and then through the map $v'$ due to the group completion property. Then this diagram says that
\begin{eq*} \begin{array}{rll}
			v' \circ u' \circ \mathrm{gp} \circ \mathrm{M} & = & v' \circ u' \circ u \circ \mathrm{gp} \\
			& = & v' \circ \mathrm{M} \circ \mathrm{gp} \\
			& = & u \circ \mathrm{gp} \\
			& = & \mathrm{gp} \circ \mathrm{M}
		\end{array}
\end{eq*}
But $\mathrm{M}: \mathrm{Mor}(X) \to \mathrm{M}(X)$ is the map associated with a quotient, and so it is an epimorphism. Thus we can cancel it out on the right, leaving just
\begin{eq*} v' \circ u' \circ \mathrm{gp} \quad = \quad \mathrm{gp} \end{eq*}
Then from this we can conclude that for any $\mathrm{M}(f) \in \mathrm{M}(X)$,
\begin{eq*} \begin{array}{rcccl}
			v'u'\big( \, \mathrm{gp}\mathrm{M}(f) \, \big) & = & \mathrm{gp}\mathrm{M}(f) \\
			v'u'\big( \, \mathrm{gp}\mathrm{M}(f)^* \, \big) & = & v'u'\big( \, \mathrm{gp}\mathrm{M}(f) \, \big)^* & = & \mathrm{gp}\mathrm{M}(f)^*
		\end{array}
\end{eq*} 
All elements of $\mathrm{M}(X)^{\mathrm{gp}}$ can be written as $\mathrm{gp}\mathrm{M}(f)$ or $\mathrm{gp}\mathrm{M}(f)^*$ for at least one $f$, so this really says that $v' \circ u'$ is the identity homomorphisms on $\mathrm{M}(X)^{\mathrm{gp}}$. 

A completely analogous argument can also be by made starting from the bottom edge of the diagram instead, and then concluding that $u' \circ v' = \mathrm{id}_{\mathrm{M}_{\mathrm{gp}}(X)}$. Furthermore, we can construct another diagram using the universal property of the abelianisation,
\begin{eq*} \begin{tikzcd}
& \mathrm{M}(X) \ar[rr, "\mathrm{ab}"] \ar[ddrr, dashed, "v''", near start] & & \mathrm{M}(X)^{\mathrm{ab}} \ar[dd, shift left, dashed, "u'''"] \\
\mathrm{Mor}(X) \ar[ru, "\mathrm{M}"] \ar[rd, "\mathrm{ab}"'] & & & \\
& \mathrm{Mor}(X)^{\mathrm{ab}} \ar[rr, "\mathrm{M}"'] \ar[rruu, dashed, "u''"', near start] & & \mathrm{M}_{\mathrm{ab}}(X) \ar[uu, shift left, dashed, "v'''"]
\end{tikzcd} \end{eq*}
and then through another series of analogous arguments conclude that $v''' \circ u''' = \mathrm{id}_{\mathrm{M}(X)^{\mathrm{ab}}}$ and $u''' \circ v''' = \mathrm{id}_{\mathrm{M}_{\mathrm{ab}}(X)}$. All together, these yield the two isomorphisms given in the statement of the proposition.
\end{proof}

In other words, we do not need to worry about order of operations when using the left adjoint functor $\mathrm{M}(\, \_ \,)^{\mathrm{gp},\mathrm{ab}}$. This is very convenient, and later on when we actually need to evalute particular $\mathrm{M}(X)^{\mathrm{gp},\mathrm{ab}}$, we will use this fact to carry out the calculation in whichever order proves easiest. 

\chapter{Free invertible algebras as colimits}
\label{colimalgebra} 

In the previous chapter, we made progress towards understanding the structure of $L\mathbb{G}_n$ by showing that the algebra was an initial object in a certain comma category. Specifically, we saw that the map $\eta: \mathbb{G}_n \to L\mathbb{G}_n$ is initial among all $\mathrm{E}G$-algebra maps $\mathbb{G}_n \to X_{\mathrm{inv}}$. This fact is the rigorous way of expressing a fairly obvious intuition about $L\mathbb{G}_n$ --- that we should expect the free algebra on $n$ invertible objects to be like the free algebra on $n$ objects, except that its objects are invertible.

However, this not the only way of thinking about $L\mathbb{G}_n$. Consider for a moment the free $\mathrm{E}G$-algebra on $2n$ objects, $\mathbb{G}_{2n}$. Intuitively, if we were to take this algebra and then enforce upon it the extra relations $z_{n+1} = z_1^*, ..., z_{2n} = z_n^*$, then we would be changing it from a structure with $2n$ independent generators into one with $n$ independent generators and their inverses. That is, there seems to be a natural way to think about $L\mathbb{G}_n$ as a quotient of the larger algebra $\mathbb{G}_{2n}$. In this chapter we will work towards making this idea precise, and then examine some of its consequences, the most important of which will be allowing us to describe the group $\mathrm{M}(L\mathbb{G}_n)^{\mathrm{gp},\mathrm{ab}}$.

\section{$L\mathbb{G}_n$ as a cokernel in $\mathrm{E}G\mathrm{Alg}_S$} 

We'll begin with some definitions.

\begin{defn}\label{qdef} Let $\delta$ be the map of $\mathrm{E}G$-algebras defined on generators by
\begin{eq*} \begin{array}{rlrlll}
			\delta & : & \mathbb{G}_{2n} & \to & \mathbb{G}_{2n} \\
			& : & z_{i} & \mapsto & z_i \otimes z_{n+i} \\
			& : & z_{n+i} & \mapsto & z_{n+i} \otimes z_i			
		\end{array}
\end{eq*}
for $1 \le i \le n$. We will also denote by $q: \mathbb{G}_{2n} \to Q$ the cokernel this map.
\end{defn}

Note that the above definition does actually make sense. The given description of $\delta$ is enough to specify it uniquely because $\mathbb{G}_{2n}$ is the free $\mathrm{E}G$-algebra on $2n$ objects, and hence algebra maps $\mathbb{G}_{2n} \to \mathbb{G}_{2n}$ are canonically isomorphic to functions $\{z_1, ..., z_{2n}\} \to \mathrm{Ob}(\mathbb{G}_{2n})$. Also we can be sure that the map $q$ exists, because $\mathrm{E}G\mathrm{Alg}_S$ is a locally finitely presentable category and thus has all finite colimits.

The goal of this approach will be show that $Q$ is in fact that same algebra as $L\mathbb{G}_n$. In order to do this, it would help if we could easily compare $q: \mathbb{G}_{2n} \to Q$ to our initial object $\eta: \mathbb{G}_{n} \to L\mathbb{G}_n$. We really want to show that the composite of $q$ with the inclusion $\mathbb{G}_n \hookrightarrow \mathbb{G}_{2n}$ is an object of $(\mathbb{G}_n \downarrow \mathrm{inv})$ --- in other words, that $Q$ has only invertible objects. This can be done using the adjunction we found in \cref{Obadj}.

\begin{prop}\label{Qobj} The object monoid of $Q$ is $\mathbb{Z}^{*n}$, and the restriction of $q$ to objects $\mathrm{Ob}(q): \mathrm{Ob}(\mathbb{G}_{2n}) \to \mathrm{Ob}(Q)$ is the monoid homomorphism defined on generators as
\begin{eq*} \begin{array}{rlrlll}
			\mathrm{Ob}(q) & : & \mathbb{N}^{\ast 2n} & \to & \mathbb{Z}^{\ast n} \\
			& : & z_i & \mapsto & z_i  \\
			& : & z_{n+i} & \mapsto & z_i^*		
		\end{array}
\end{eq*}
\end{prop}
\begin{proof}
Consider $\mathrm{Ob}(\delta)$, the restrictions on objects of the algebra maps $\delta: \mathbb{G}_{2n} \to \mathbb{G}_{2n}$. By \cref{Gnobj}, this is a monoid homomorphism $\mathbb{N}^{\ast 2n} \to \mathbb{N}^{\ast 2n}$, and since $\mathrm{Mon}$ is cocomplete it too must have a cokernel. This will be a new homomorphism whose source is $\mathbb{N}^{\ast 2n}$ and whose target is the quotient of $\mathbb{N}^{\ast 2n}$ by the relations $\mathrm{Ob}(\delta)(x) = I$. Remembering \cref{qdef}, and that $\mathbb{N}^{\ast 2n}$ is the free monoid on $2n$ generators, this quotient monoid will have the following presentation:
\begin{eq*}\begin{array}{ll}
			\text{Generators:} & z_1, \, ..., \, z_{2n} \\
			\text{Relations:} & z_i \otimes z_{n+i} = I, \\
			& z_{n+i} \otimes z_i = I
		\end{array}
\end{eq*}
This is just the same as
\begin{eq*}\begin{array}{ll}
			\text{Generators:} & z_1, \, ..., \, z_{2n} \\
			\text{Relations:} & z_{n+i} = z_i^*, \\
		\end{array}
\end{eq*}
which is the presentation of $\mathbb{Z}^{\ast n}$. 

But by \cref{Obadj}, $\mathrm{Ob}$ is a left adjoint and hence preserves all colimits. Thus the cokernel of $\mathrm{Ob}(\delta)$ is just the underlying homomorphism of the cokernel of $\delta$. Therefore $\mathrm{Ob}(Q) = \mathbb{Z}^{\ast n}$, and $\mathrm{Ob}(q)$ is the quotient map $\mathbb{N}^{\ast 2n} \to \mathbb{Z}^{\ast n}$ sending $z_i \mapsto z_i$ and $z_{n+i} \mapsto z_i^*$ for $1 \le i \le n$.
\end{proof}

Thus every object of the cokernel algebra $Q$ is invertible. Thus $q: \mathbb{G}_{2n} \to Q$ can be composed with an inclusion to give a well-defined object of the category $(\mathbb{G}_n \downarrow \mathrm{inv})$, and hence we can use the initiality of $\eta$ to determine the following result:

\begin{prop}\label{coker} Let $i: \mathbb{G}_n \to \mathbb{G}_{2n}$ be the inclusion of $\mathrm{E}G$-algebras defined on generators by $i(z_i) = z_i$. Then $i \circ q$ is an initial object of $(\mathbb{G}_n \downarrow \mathrm{inv})$. In particular, this means that
\begin{eq*} Q \quad \cong \quad L\mathbb{G}_n \end{eq*}
\end{prop}
\begin{proof}
Let $\psi: \mathbb{G}_n \to X$ be an arbitrary object of $(\mathbb{G}_n \downarrow \mathrm{inv})$. Since $\mathbb{G}_n$ is the free $\mathrm{E}G$-algebra on $n$ objects, we can use it and $\psi$ to define a new map, $\psi^*: \mathbb{G}_n \to X$, which takes the values
\begin{eq*} \psi^*(z_i) \quad := \quad \psi(z_i)^* \end{eq*}
on generators. Using these two functors we can define a new map, $\psi + \psi^*$, via the universal property of the coproduct:
\begin{eq*} \begin{tikzcd}
& \mathbb{G}_n + \mathbb{G}_n \ar[dd, dashed, "\psi + \psi^*"] & \\
\mathbb{G}_n \ar[ur, hookrightarrow, "i"] \ar[dr, "\psi"'] & & \mathbb{G}_n \ar[ul, hookrightarrow, "i'"'] \ar[dl, "\psi^*"] \\
& X & 
\end{tikzcd} \end{eq*}
But because $\mathbb{G}_n$ is the free algebra on $n$ objects, and the free functor $F : \mathrm{Cat} \to \mathrm{E}G\mathrm{Alg}_S$ is a left adjoint and thus preserves colimits, we must have
\begin{eq*} \begin{array}{rll}
		\mathbb{G}_n + \mathbb{G}_n & = & F(\{ z_1, ..., z_n\}) + F(\{ z'_1, ..., z'_n\}) \\
		& = & F( \, \{ z_1, ..., z_n\} + \{ z'_1, ..., z'_n\} \, ) \\
		& = & F(\{ z_1, ..., z_{2n} \}) \\
		& = & \mathbb{G}_{2n} 
		\end{array}
\end{eq*}
This means that we can compose $\psi + \psi^*: \mathbb{G}_{2n} \to X$ with the map $\delta: \mathbb{G}_{2n} \to  \mathbb{G}_{2n}$, though we need to be careful to specify exactly which inclusions we used in the definition of $\psi + \psi^*$. Suppose that the left-hand inclusion is $i$, the one given in the statement of the proposition, and the other is defined by the assignment $z_i \mapsto z_{i+n}$. Then for $1 \leq i \leq n$,
\begin{eq*} \begin{array}{rll}
			(\psi + \psi^*)\delta(z_i) & = & (\psi + \psi^*)(z_i \otimes z_{n+i}) \\
			& = & \psi(z_i) \otimes \psi(z_i)^* \\
			& = & I
		\end{array}
\end{eq*}
\begin{eq*} \begin{array}{rll}
			(\psi + \psi^*)\delta(z_{n+i}) & = & (\psi + \psi^*)(z_{n+i} \otimes z_i) \\
			& = & \psi(z_i)^* \otimes \psi(z_i) \\
			& = & I
		\end{array}
\end{eq*}
That is, $(\psi + \psi^*) \circ \delta = I$. But we've already defined $q: \mathbb{G}_{2n} \to Q$ to be the cokernel of $\delta$, the universal map with this property, and so there must exist a unique $\mathrm{E}G$-algebra map $u: Q \to X$ making the right-hand triangle below diagram commute:
\begin{eq*} \begin{tikzcd}
\mathbb{G}_n \ar[rr, hookrightarrow, "i"] \ar[ddrr, "\psi"'] & & \mathbb{G}_{2n} \ar[rr, "q"] \ar[dd, "\psi + \psi^*", near start] & & Q \ar[ddll, "u"] \\
& & & & \\ 
& & X & &
\end{tikzcd} \end{eq*}
The other triangle commutes by the definition of $\psi + \psi^*$, and so together the diagram tells us that for any object $\psi$ of $(\mathbb{G}_n \downarrow \mathrm{inv})$, there exists at least one morphism $u$ in $(\mathbb{G}_n \downarrow \mathrm{inv})$ going from $q \circ i$ to $\psi$. 

Next, let $v: Q \to X$ be an arbitrary morphism $q \circ i \to \psi$ in $(\mathbb{G}_n \downarrow \mathrm{inv})$. By definition, this means that
\begin{eq*}\begin{array}{rll}
			\psi & = & vqi \\
			\implies \quad \psi + \psi^* & = & vqi + (vqi)^* 
		\end{array}
\end{eq*}
Also, for $1 \leq i \leq n$ we have
\begin{eq*}\begin{array}{rcrllcccl}
			q(z_i) \otimes q(z_{n+i}) & = & q(z_i \otimes z_{n+i}) & = & q\delta(z_i) & = &  I \\
			q(z_{n+i}) \otimes q(z_i) & = & q(z_{n+i} \otimes z_i) & = & q\delta(z_{n+i}) & = & I \\
			& \implies & q(z_{n+i}) & = & q(z_i)^* & & & &
		\end{array}
\end{eq*}
Therefore,
\begin{eq*}\begin{array}{rll}
			(\psi + \psi^*)(z_i) & = & \big( vqi + (vqi)^* \big)(z_i) \\
			& = & vqi(z_i) \\
			& = & vq(z_i) \\
		\end{array}
\end{eq*}
\begin{eq*} \begin{array}{rll}
			(\psi + \psi^*)(z_{n+i}) & = & \big( vqi + (vqi)^* \big)(z_{n+i}) \\
			& = & vqi(z_i)^* \\
			& = & v \big( q(z_i)^* \big) \\
			& = & vq(z_{n+i})
		\end{array}
\end{eq*}
or in other words $\psi + \psi^* = v \circ q$ for any morphism $v: q \circ i \to \psi$ in $(\mathbb{G}_n \downarrow \mathrm{inv})$. But this is the property that the map $u$ was supposed to satisfy uniquely, and thus it must be the only morphism $q \circ i \to \psi$ in $(\mathbb{G}_n \downarrow \mathrm{inv})$. Therefore $q \circ i$ is an initial object, and hence it is isomorphic in $(\mathbb{G}_n \downarrow \mathrm{inv})$ to any other initial object, such as $\eta$. It follows that the targets of these two maps, $Q$ and $L\mathbb{G}_n$ respectively, are isomorphic as $\mathrm{E}G$-algebras.
\end{proof}

It's worth noting that we have not given a method for actually taking cokernels in $\mathrm{E}G\mathrm{Alg}_S$, and so \cref{coker} doesn't immediately provide an explicit description for the whole of $L\mathbb{G}_n$. However, it does offer us another way to extract partial information, like what we were doing in \cref{initialalgebra}. Consider \cref{Qobj}; now that we know that $Q$ is actually $L\mathbb{G}_n$, the statement of this proposition is just the same as that of \cref{Zobj}. But the proof of the former uses the ability of cokernels to preserve left adjoint functors, rather than any of the initial algebra and group completion properties that appear in the latter.

Of course, by \cref{coker} the fact that $q$ is a cokernel is equivalent to it being initial, and so while they may not look it at first glance, these two approaches are secretly the same. Thus from now on whenever we are trying to determine some aspect of $L\mathbb{G}_n$, we will make sure to take a look at both methods, just in case there are some properties of our free algebra which are more readily apparent from one description than another.

\section{$L\mathbb{G}_n$ as a surjective coequaliser} \label{surjcoeq}

An immediate consequence our new cokernel perspective of $L\mathbb{G}_n$ is that, since left adjoint functor all preserve colimits, \cref{Obadj,concompadj} now both imply results about the partial surjectivity of this new map $q$. The former says that since $\mathrm{Ob}(q)$ is a cokernel map of monoids, and hence that every object of $L\mathbb{G}_n$ is the image under $q$ of some object of $\mathbb{G}_{2n}$; the latter says a similar thing for connected components. From this one might guess that $q$ will just turn out to be a surjective map of $\mathrm{E}G$-algebras, and indeed this is the case.

Unfortunately, we can not go about proving that $q$ is surjective on morphisms by a similar adjunction technique, since the best we have is the one from \cref{Moradj} and it will only tell us about the map $\mathrm{M}(q)^{\mathrm{gp},\mathrm{ab}}$. However, there is a general result about the coequalisers of $\mathrm{E}G$-algebras that we can prove to get around this.

\begin{prop}\label{coeqsurj} Let $\phi, \phi' : X \to Y$ be a pair of parallel $\mathrm{E}G$-algebra maps, and $k: Y \to Z$ their coequaliser in $\mathrm{E}G\mathrm{Alg}_S$. If the monoid $\mathrm{Ob}(Z)$ is also a group, then the functor $k$ is surjective.
\end{prop}
\begin{proof}
We begin by mirroring the proof of \cref{Qobj}. We know that the functor $\mathrm{Ob} : \mathrm{E}G\mathrm{Alg}_S \to \mathrm{Mon}$ is a left adjoint, by \cref{Obadj}, and thus preserves all colimits. It follows that the monoid homomorphism $\mathrm{Ob}(k): \mathrm{Ob}(Y) \to \mathrm{Ob}(Z)$ is the coequaliser of the parallel pair $\mathrm{Ob}(\phi), \mathrm{Ob}(\phi') : \mathrm{Ob}(X) \to \mathrm{Ob}(Y)$ in $\mathrm{Mon}$, or in other words
\begin{eq*} \mathrm{Ob}(Z) \quad = \quad \bigquotient{\mathrm{Ob}(Y)}{\sim}\end{eq*}
where $\sim$ is the relation defined by
\begin{eq*}\mathrm{Ob}(\phi)(y) \sim \mathrm{Ob}(\phi')(y), \quad \quad \quad a \sim a', b \sim b' \implies ab \sim a'b' \end{eq*}
The map $\mathrm{Ob}(k): \mathrm{Ob}(Y) \to \mathrm{Ob}(Y)/\sim$ is then clearly surjective.

Next, let $f: v \to w$ and $f' : w' \to v'$ be any two morphisms of the algebra $Y$ for which $k(f)$ and $k(f')$ are composable in $Z$. Since these maps are composable we know that $k(w)$ and $k(w')$ must be the same object of $Z$, and since $Z$ is a group we know this object has an inverse $k(w)^* = k(w')^*$. So by the surjectivity of $k$ we can find another object $y$ of $Y$ for which $k(y) = k(w)^*$. Using this, define the morphism $h: x \to x'$ to be the tensor product $f' \otimes \mathrm{id}_y \otimes f$. Then
\begin{eq*} \begin{array}{rll}
		k(h) & = & k(f' \otimes \mathrm{id}_y \otimes f) \\
		& = & k(f') \otimes \mathrm{id}_{k(y)} \otimes k(f) \\
		& = & k(f') \otimes \mathrm{id}_{k(w)^*} \otimes k(f)
		\end{array}
\end{eq*}
But by \cref{tenscomp}, this is really just the composite $k(f') \circ k(f)$. Thus the set of morphisms of $Z$ which are images of morphisms of $Y$ is closed under composition. 

So now consider $k(Y)$, the subcategory of $Z$ that contains every object $x'$ for which there exists $x$ in $Y$ with $k(x) = x'$, and every morphism $f'$ for which there exists $f$ in $Y$ with $q(f) = f'$. We know that the morphisms of $k(Y)$ are closed under composition, and so this is indeed a well-defined category. Moreover, for any collection of morphisms $f'_1, ..., f'_m$ of $k(Y)$ we'll have
\begin{eq*} \begin{array}{rll}
 			\alpha^{Z}(g; f'_1, ..., f'_m) & = & \alpha^Z\big( \, g \, ; \, k(f_1), ..., k(f_m) \, \big) \\
			& = & k \big( \, \alpha^{Y}(g; f_1, ..., f_m) \, \big) \\
			& \in & k(Y) 
		\end{array}
\end{eq*}
for some $f_1, ..., f_m$, since $k$ is a map of $\mathrm{E}G$-algebras. Thus $k(Y)$ is also a well-defined sub-$\mathrm{E}G$-algebra of $Z$. There is also clearly a canonical map $k': Y \to k(Y)$, the unique surjective map of $\mathrm{E}G$-algebras with the property that $k'(x) = k(x)$ for any object $x$ and $k'(f) = k(f)$ for any morphism $f$. If we denote by $i$ the evident inclusion of algebras $i: k(Y) \hookrightarrow Z$, then these maps are related by the fact that $i \circ k' = k$.
\begin{eq*} \begin{tikzcd}
& & X \ar[dd, bend right, "\phi"'] \ar[dd, bend left, "\phi'"] & & \\
& & & & \\
& & Y \ar[ddll, "k'"'] \ar[dd, "k"] \ar[ddrr, "j"] & & \\ 
& & & & \\
k(Y) \ar[rr, hookrightarrow, "i"] & & Z \ar[rr, "u"] & & U
\end{tikzcd} \end{eq*}
Given all of this, let $j: Y \to U$ be any map of $\mathrm{E}G$-algebras with the property that $j \circ \phi = j \circ \phi'$. Since $h$ is the coequaliser of $\phi$ and $\phi'$, it follows that there exists a unique map $u:  Y \to U$ such that $j = u \circ k$. This means that $j = u \circ i \circ k'$, and hence there is obviously at least one map, $u \circ i$, which lets us factors $j$ through $k'$. But for any other map $v: k(Y) \to U$ that factors $j$ like this, we'll have
\begin{eq*} \begin{array}{rrll}
			& v \circ k' & = & j \\
			& & = & u \circ i \circ k' \\
			\implies \quad & v & = & u \circ i
		\end{array}
\end{eq*}
because $k'$ is surjective, and thus $u \circ i$ is the unique map with this property. That is, $k'$ is also a coequaliser of $\phi$ and $\phi'$. But colimits are always unique up to a unique isomorphism, and so there should be a unique invertible map $k(Y) \to Z$ factoring $k$ through $k'$. This is clearly just the inclusion $i$, and as a result $k(Y) = Z$ and $k' = k$. In other words, the coequaliser map $k$ is surjective. 
\end{proof}

Because the cokernel of a morphism is just its coequaliser with the zero map, and since we know that the objects of $L\mathbb{G}_n$ form a group, we can immediately apply this result to the functor $q$.

\begin{cor}\label{qsurj} The cokernel map $q: \mathbb{G}_{2n} \to L\mathbb{G}_n$ is surjective.
\end{cor}

This is probably the single most important step in our effort to determine the morphisms of $L\mathbb{G}_n$, in the sense of how many of the results hereafter rely on this relatively simple property. Indeed this result is so strong that after a cursory glance, one might be forgiven for thinking that it will immediately provide for us the main thing we have been working towards this chapter --- the value of $\mathrm{M}(L\mathbb{G}_n)^{\mathrm{gp},\mathrm{ab}}$.

After all, every surjective functor is an epimorphism in the category $\mathrm{MonCat}$. We know that left adjoint functors preserve epimorphisms, and that $\mathrm{M}(\, \_ \,)^{\mathrm{gp},\mathrm{ab}}$ is a left adjoint, so from \cref{qsurj} we can surmise that $\mathrm{M}(q)^{\mathrm{gp},\mathrm{ab}}$ is also an epimorphism, this time in $\mathrm{Ab}$. But an epimorphic map of abelian groups is nothing other than a surjective homomorphism, and thus we may apply the First Isomorphism Theorem of groups to get the following:
\begin{eq*} \mathrm{M}(L\mathbb{G}_n)^{\mathrm{gp},\mathrm{ab}} \quad = \quad \bigquotient{\mathrm{M}(\mathbb{G}_{2n})^{\mathrm{gp},\mathrm{ab}}}{\mathrm{ker}\big( \, \mathrm{M}(q)^{\mathrm{gp},\mathrm{ab}} \, \big)} \end{eq*}
So if we knew what the kernel of $\mathrm{M}(q)^{\mathrm{gp},\mathrm{ab}}$ was, we would be done. And it seems like we \emph{should} know this; $q$ was defined to be the cokernel of $\delta$, and by preservation of this colimits means that $\mathrm{M}(q)^{\mathrm{gp},\mathrm{ab}}$ is the cokernel of $\mathrm{M}(\delta)^{\mathrm{gp},\mathrm{ab}}$. Then since we are working with abelian groups, kernels and cokernels interact in a nice way:
\begin{eq*} \mathrm{ker} \, \mathrm{coker}\big( \, \mathrm{M}(\delta)^{\mathrm{gp},\mathrm{ab}} \, \big) \quad = \quad \mathrm{im}\big( \, \mathrm{M}(\delta)^{\mathrm{gp},\mathrm{ab}} \, \big) \end{eq*}
However, this last step doesn't actually work --- $q$ was defined to be $\mathrm{coker}(\delta)$, but only in the category of $\mathrm{E}G$-algebras. In general this will \emph{not} be the same thing as the cokernel of $\delta$ in $\mathrm{MonCat}$, which is what we would really need in order for $\mathrm{M}(\, \_ \,)^{\mathrm{gp, ab}}$ to preserve it.

Still, this is a pretty reasonable guess for what $\mathrm{M}(L\mathbb{G}_n)^{\mathrm{gp, ab}}$ is, and provides an indication of how we should proceed in order to find its true value. We will pick up on this idea again in \cref{colimmoncat}.

\section{Action morphisms of $L\mathbb{G}_n$} \label{actmorLGn}

One important consequence of the surjectivity of $q$ is that it will allow us to import certain results about the free algebra $\mathbb{G}_{2n}$ into the free invertible algebra $L\mathbb{G}_n$. In fact, we have done this once already; looking back at \cref{Qobj} with our current knowledge that $Q = L\mathbb{G}_n$, we can see that it is a direct analogue of \cref{Gnobj}, using the fact that $q$ is surjective on objects. 

In that same vein, one might ask if we can take \cref{Gnmapsaction}, a statement about the morphisms $\mathbb{G}_{2n}$, and extend it to an analagous result on $L\mathbb{G}_n$, using surjectivity of $q$ on morphisms instead. That is, since every morphism of $\mathbb{G}_{2n}$ is an action morphism, and since $\mathrm{E}G$-algebra maps always send action morphisms to action morphisms, we should be able to use $q$ to identify every morphism of $L\mathbb{G}_n$ as an action morphism. This is indeed pretty simple to show.

\begin{lem} \label{allmapsaction} Every morphism in $L\mathbb{G}_n$ can be expressed as $\alpha^{L\mathbb{G}_n}(g; \mathrm{id}_{x_1}, ..., \mathrm{id}_{x_m})$, for some $g \in G(m)$ and $x_i \in \{z_1, ..., z_n, z_1^*, ..., z_n^* \}$.
\end{lem}
\begin{proof}
Let $f$ be an arbitrary morphism in $L\mathbb{G}_n$. By surjectivity of $q$, there must exist at least one morphism $f'$ in $\mathbb{G}_{2n}$ such that $q(f') = f$, and from \cref{Gnmapsaction} we know that this $f'$ can be expressed uniquely as $\alpha(g; \mathrm{id}_{x'_1}, ..., \mathrm{id}_{x'_m})$ for some $g \in G(m)$ and $x'_i \in \{z_1, ..., z_{2n} \}$. Thus, because $q$ is a map of $\mathrm{E}G$-algebras, we will have
\begin{eq*}\begin{array}{rll}
			f & = & q(f') \\
			& = & q\big( \, \alpha^{\mathbb{G}_{2n}}( \, g \, ; \, \mathrm{id}_{x'_1}, ..., \mathrm{id}_{x'_m} \, ) \, \big) \\
			& = & \alpha^{L\mathbb{G}_n}( \, g \, ; \, \mathrm{id}_{q(x'_1)}, ..., \mathrm{id}_{q(x'_m)} \, ) 
		\end{array}
\end{eq*}
Therefore there is at least one collection of $x_i = q(x'_i)$ for which the statement of the proposition holds.
\end{proof}

\cref{allmapsaction} formalises a certain intuition about how the functor $L$ should act on algebras, the idea that a `free' structure really shouldn't have any `superfluous' components, only whatever data is absolutely required for it to be well-defined. In the case of $L\mathbb{G}_n$, we have proven that the only morphisms contained in the free $\mathrm{E}G$-algebra on invertible objects are $\mathrm{E}G$-action morphisms. However, while this is very similar to what we have in the non-invertible case, it should be stressed that \cref{allmapsaction} does \emph{not} prove that the morphisms of $L\mathbb{G}_n$ have \emph{unique} representations $\alpha(g; \mathrm{id}_{w_1}, ..., \mathrm{id}_{w_m})$, as morphisms of $\mathbb{G}_n$ do.

Also, notice that when we eventually find a complete description of $L\mathbb{G}_n$ as a monoidal category, we will be able to use the surjective algebra map $q$ to determine it's $\mathrm{E}G$-action as well. This follows from the same reasoning we used to prove \cref{allmapsaction}, but in reverse:
\begin{eq*}\begin{array}{rll}
			\alpha^{L\mathbb{G}_n}( \, g \, ; \, \mathrm{id}_{x_1}, ..., \mathrm{id}_{x_m} \, ) & = & \alpha^{L\mathbb{G}_n}( \, g \, ; \, \mathrm{id}_{q(x'_1)}, ..., \mathrm{id}_{q(x'_m)} \, ) \\
			& = & q\big( \, \alpha^{\mathbb{G}_{2n}}( \, g \, ; \, \mathrm{id}_{x'_1}, ..., \mathrm{id}_{x'_m} \, ) \, \big)
		\end{array}
\end{eq*}
In fact, since we do know that $q$ is a cokernel of the map $\delta$, we can even extract some information about this action right away, before we have built an understanding of the morphisms of $L\mathbb{G}_n$.

\begin{lem} \label{noscalar} For any element $g \in G(m), m \in \mathbb{N}$ of an action operad $G$,
\begin{eq*} \alpha^{L\mathbb{G}_n}( \, g \, ; \, \mathrm{id}_I, ..., \mathrm{id}_I \, ) \quad = \quad \mathrm{id}_I \end{eq*}
Equivalently, for any element $h \in G(0)$,
\begin{eq*} \alpha^{L\mathbb{G}_n}( \, h \, ; \, - \, ) \quad = \quad \mathrm{id}_I \end{eq*}
\end{lem}
\begin{proof}
First, let $g \in G(m)$. Then because $q$ is the cokernel of $\delta$ in $\mathrm{E}G\mathrm{Alg}_S$,
\begin{eq*}\begin{array}{rll}
			\alpha^{L\mathbb{G}_n}( \, g \, ; \, \mathrm{id}_I, ..., \mathrm{id}_I \, ) & = & \alpha^{L\mathbb{G}_n}( \, g \, ; \, \mathrm{id}_{q(I)}, ..., \mathrm{id}_{q(I)} \, ) \\
			& = & q\big( \, \alpha^{\mathbb{G}_{2n}}( \, g \, ; \, \mathrm{id}_I, ..., \mathrm{id}_I \, ) \, \big) \\
			& = & q\big( \, \alpha^{\mathbb{G}_{2n}}( \, g \, ; \, \mathrm{id}_{\delta(I)}, ..., \mathrm{id}_{\delta(I)} \, ) \, \big) \\
			& = & q \delta \big( \, \alpha^{\mathbb{G}_{2n}}( \, g \, ; \, \mathrm{id}_I, ..., \mathrm{id}_I \, ) \, \big) \\
			& = & \mathrm{id}_I
		\end{array}
\end{eq*}
Clearly this result implies that
\begin{eq*} \alpha^{L\mathbb{G}_n}( \, h \, ; \, - \, ) \quad = \quad \mathrm{id}_I \end{eq*}
for any element $h \in G(0)$, but the implication also goes the other way, since
\begin{eq*}\begin{array}{rll}
			\alpha( \, g \, ; \, \mathrm{id}_I, ..., \mathrm{id}_I \, ) & = & \alpha\big( \, g \, ; \, \alpha(e_0;-), ..., \alpha(e_0;-) \, \big) \\
			& = & \alpha\big( \, \mu(g;e_0, ..., e_0) \, ; \, - \, \big) \\
		\end{array}
\end{eq*}
and $\mu(g;e_0, ..., e_0) \in G(0)$.
\end{proof}

This is a pretty interesting result. By \cref{Gnmapsaction}, morphisms of the form $\alpha^{\mathbb{G}_n}(g; \mathrm{id}_I, ..., \mathrm{id}_I)$ make up the entirety of the homset $\mathbb{G}_n(I,I)$. Now we see that their image under the algebra map $\eta: \mathbb{G}_n \to L\mathbb{G}_n$ is always $\mathrm{id}_I$, and so it follows that the unit endomorphisms of free algebras are wholly unrelated to the unit endomorphisms of the corresponding free \emph{invertible} algebras. In particular, when constructing $L\mathbb{G}_n$ it seems that it should not matter whether our chosen action operad $G$ has nontrivial $G(0)$, since all morphisms $\alpha^{L\mathbb{G}_n}(g; - )$ for $g \in G(0)$ are going to end up as the identity regardless. In order to state this idea more concretely though, we need some way of `removing' the group $G(0)$ from $G$.

\begin{prop} \label{G0quot} Let $G$ be a crossed action operad. Then there exists another crossed action operad $G'$ which has $G'(m) = G(m)/G(0)$ for all $m \in \mathrm{N}$.
\end{prop}
\begin{proof}
For any elements $g \in G(m)$ and $h \in G(0)$, their tensor product $h \otimes g := \mu(e_2; h, g)$ is also an element of $G(m)$. This defines a map $G(0) \times G(m) \to G(m)$, which is both a group homomorphism and a group action:
\begin{eq*} \begin{array}{rllcrll}
			(hh') \otimes (gg') & = & \mu( \, e_2 \, ;  \, hh', gg' \, ) & & e_0 \otimes g & = & g \\
			& = & \mu( \, e_2 \, ;  \, h, g \, ) \cdot \mu( \, e_2 \, ;  \, h', g' \, ) & & & & \\
			& = & (h \otimes g) \cdot (h' \otimes g') 	& & h' \otimes (h \otimes g) & = & (h' \otimes h) \otimes g \\
			& & & = & (h'h) \otimes g
		\end{array} 
\end{eq*}
The last step here uses the fact that tensor product and group multiplication coincide on $G(0)$, by \cref{G0abel}. We can thus take the quotient of each $G(m)$ by the action of $G(0)$, which will amount to quotienting out the image in $G(m)$ of the subgroup $G(0) \cong G(0) \times \{ e_m \} \subseteq G(0) \times G(m)$. 

In order for these new groups $G'(m) = G(m)/G(0)$ to form an action operad, we'll need operadic multiplication maps $\mu^{G'}$ and underlying permutation maps $\pi^{G'}$. These will be defined from $\mu^{G}$ and $\pi^{G}$ using the universal property of the quotient. Specifically, let $h, h_1, ..., h_m \in G(0)$ and $k_1, ..., k_m \in \mathbb{N}$. Then we have
\begin{eq*} \begin{array}{rll}
		\mu^{G}( \, h \otimes e_m \, ; \, h_1 \otimes e_{k_1}, ..., h_m \otimes e_{k_m} \, ) & = & \mu^{G}\big( \, \mu^{G}(e_2; h, e_m) \, ; \, h_1 \otimes e_{k_1}, ..., h_m \otimes e_{k_m} \, \big) \\
		& = & \mu^{G}\big( \, e_2 \, ; \, \mu^{G}(h;-), \mu^{G}(e_m; h_1 \otimes e_{k_1}, ..., h_m \otimes e_{k_m}) \, \big) \\
		& = & \mu^{G}(h;-) \otimes \mu^{G}(e_m; h_1 \otimes e_{k_1}, ..., h_m \otimes e_{k_m})  \\
		& = & h \otimes h_1 \otimes e_{k_1} \otimes ... \otimes h_m \otimes e_{k_m} \\
		& = & e_{k_1} \otimes ... \otimes e_{k_m} \otimes h \otimes h_1 ... \otimes h_m \\
		& = & e_{k_1+...+k_m} \otimes h \otimes h_1 ... \otimes h_m
		\end{array}
\end{eq*}
Here we've used that $\mathbb{G}_1$ is spacial by \cref{spacial}, and so since its morphisms are just elements of $G$, the $e_k$ commute with elements of $G(0)$. 
\begin{eq*} \begin{tikzcd}
G(0) \times G(0) \times ... \times ... G(0) \ar[rr, "\otimes"] \ar[dd, hookrightarrow] & & G(0) \ar[dd, hookrightarrow] \\
& & \\
G(m) \times G(k_1) \times ... \times G(k_m) \ar[rr, "\quad \quad \mu^{G}_m"] \ar[dd, "\lbrack \, \_ \, \rbrack \times ... \times \lbrack \, \_ \, \rbrack"'] & & G(k_1 + ... + k_m) \ar[dd, "\lbrack \, \_ \, \rbrack"] \\
& & \\
\quotient{G(m)}{G(0)} \times \quotient{G(k_1)}{G(0)} \times ... \times \quotient{G(k_m)}{G(0)} \ar[rr, "\mu^{G'}_m"] & & \quotient{G(k_1 + ... + k_m)}{G(0)}
\end{tikzcd} \end{eq*}
In other words, we know that the upper square in the diagram above commutes. Now, the composite on the right-hand side of the diagram is by definition the zero map, and so too is its composite with the $(m+1)$-fold tensor product $G(0)^{m+1} \to G(0)$. Using commutativity of the upper square, it follows that the composite of the inclusion on the left and the upper-right path in the bottom square is also zero, and so this upper-right path will factor uniquely through the quotient of that inclusion. The resulting homomorphism $\mu^{G'}_m$ is then exactly the operadic multiplication map we are looking for; the and associativity condition is immediate consequence of the corresponding conditions for $\mu^{G}$,
\begin{eq*} \begin{array}{rl}
			& \mu^{G'}\Big( \, [g] \, ; \, \mu^{G'}\big( \, [g_1] \, ; \, [h_{1,1}], ..., [h_{1,k_1}] \, \big), ..., \mu^{G'}\big( \, [g_m] \, ; \, [h_{m,1}], ..., [h_{m,k_m}] \, \big) \, \Big) \\[\medskipamount]
			= & \mu^{G'}\Big( \, [g] \, ; \, \big[  \, \mu^{G}(g_1; h_{1,1}, ..., h_{1,k_1}) \, \big], ...,\big[ \, \mu^{G}(g_m; h_{m,1}, ..., h_{m,k_m}) \, \big] \, \Big) \\[\medskipamount]
			= & \Big[ \, \mu^{G}\big( \, g \, ; \, \mu^{G}(g_1; h_{1,1}, ..., h_{1,k_1}), ..., \mu^{G}(g_m; h_{m,1}, ..., h_{m,k_m}) \, \big) \, \Big] \\[\medskipamount]
			= & \Big[ \, \mu^{G}\big( \, \mu^{G}(g; g_1, ..., g_m) \, ; \, h_{1,1}, ..., h_{1,k_1}, ..., h_{m,1}, ..., h_{m,k_m})\, \big) \, \Big] \\[\medskipamount]
			= & \mu^{G'}\Big( \, \big[ \, \mu^{G}(g; g_1, ..., g_m) \, \big] \, ; \, [h_{1,1}], ..., [h_{1,k_1}], ..., [h_{m,1}], ..., [h_{m,k_m}] \, \Big) \\[\medskipamount]
			= & \mu^{G'}\Big( \, \mu^{G'}\big( \, [g] \, ; \, [g_1], ..., [g_m] \, \big) \, ; \, [h_{1,1}], ..., [h_{1,k_1}], ..., [h_{m,1}], ..., [h_{m,k_m}] \, \Big)
		\end{array}
\end{eq*}
and likewise for unitality,
\begin{eq*} \begin{array}{rrcccl}
			& \mu^{G}( \, g \, ; \, e_1, ..., e_1 \, ) & = & g & = & \mu^{G}( \, e_1 \, ; \, g \, ) \\
			\implies & \big[ \, \mu^{G}( \, g \, ; \, e_1, ..., e_1 \, ) \, \big] & = & [g] & = & \big[ \, \mu^{G}( \, e_1 \, ; \, g \, ) \, \big] \\
			\implies & \mu^{G'}\big( \, [g] \, ; \, [e_1], ..., [e_1] \, \big) & = & [g] & = & \mu^{G'}\big( \, [e_1] \, ; \, [g] \, \big)
		\end{array}
\end{eq*}
Similarly, for any $h \in G(0)$ and $m \in \mathbb{N}$ we know that 
\begin{eq*} \pi^{G}(h \otimes e_m) \quad = \quad \pi^{G}(h) \otimes \pi^{G}(e_m) \quad = \quad e_0 \otimes e_m \quad = \quad e_m \end{eq*}
and so the top square in the diagram below will commute:
\begin{eq*} \begin{tikzcd}
G(0) \ar[rr] \ar[d, hookrightarrow] & & S_0 \ar[d, hookrightarrow] \\
G(m) \ar[rr, "\pi^{G}_m"] \ar[d, "\lbrack \, \_ \, \rbrack"'] & & S_m \ar[d, equals] \\
\quotient{G(m)}{G(0)} \ar[rr, "\pi^{G'}_m"] & & S_m
\end{tikzcd} \end{eq*}
Using the same reasoning as before this will define the homomorphisms $\pi^{G'}_m$ uniquely, and the conditions for them to be underlying permutation maps of an action operad follow from those of $\pi^{G}$.
\begin{eq*} \pi^{G'}\big( \, [e_1] \, \big) \quad = \quad \pi^{G}(e_1) \quad = \quad e_1 \end{eq*}
\begin{eq*} \begin{array}{rcl}
			\pi^{G'}\Big( \, \mu^{G'}\big( \, [g] \, ; \, [h_1], ..., [h_m] \, \big) \, \Big) & = & \pi^{G'}\Big( \, \big[ \, \mu^{G}(g;h_1, ...,h_m) \, \big] \, \Big) \\[\medskipamount]
			& = & \pi^{G}\big( \, \mu^{G}(g; h_1, ..., h_m) \, \big) \\[\medskipamount]
			& = & \mu^{S}\big( \, \pi^{G}(g) \, ; \, \pi^{G}(h_1), ..., \pi^{G}(h_m) \, \big) \\[\medskipamount]
			& = & \mu^{S}\Big( \, \pi^{G'}\big( \, [g] \, \big) \, ; \, \pi^{G'}\big( \, [h_1] \, \big), ..., \pi^{G'}\big( \, [h_m] \, \big) \, \Big) \\[\medskipamount]
			& &
		\end{array}
\end{eq*}
\begin{eq*} \begin{array}{rcl}
			& \mu^{G'}\big( \, [g] \, ; [h_1], ..., [h_m] \, \big) \cdot \mu^{G'}\big( \, [g'] \, ; [h'_1], ..., [h'_m] \, \big) \\[\medskipamount]
			= & \big[ \, \mu^{G}(g; h_1, ..., h_m) \, \big] \cdot \big[ \, \mu^{G}(g'; h'_1, ..., h'_m) \, \big] \\[\medskipamount]
			= & \big[ \, \mu^{G}(g; h_1, ..., h_m) \cdot \mu^{G}(g'; h'_1, ..., h'_m) \, \big] \\[\medskipamount]
			= & \Big[ \, \mu^{G}\big( \, gg' \, ; h_{\pi^{G}(g')(1)} h'_1, ..., h_{\pi^{G}(g')(m)} h'_m \, \big) \, \Big] \\[\medskipamount]
			= & \mu^{G'}\big( \, [gg'] \, ; [h_{\pi^{G}(g')(1)} h'_1], ..., [h_{\pi^{G}(g')(m)} h'_m] \, \big) \\[\medskipamount]
			= & \mu^{G'}\big( \, [g] \cdot [g'] \, ; [h_{\pi^{G}(g')(1)}] \cdot [h'_1], ..., [h_{\pi^{G}(g')(m)}] \cdot [h'_m] \, \big) \\[\medskipamount]
			= & \mu^{G'}\big( \, [g] \cdot [g'] \, ; [h_{\pi^{G'}( \, [g'] \, )(1)}] \cdot [h'_1], ..., [h_{\pi^{G}(g')(m)}] \cdot [h'_m] \, \big) \\
		\end{array}
\end{eq*}

Thus $G'$  really is a well-defined action operad.
\end{proof}

For crossed $G$, this notion of quotient by $G(0)$ does exactly what we wanted it to do --- remove certain information which is unnecessary for forming the algebra $L\mathbb{G}_n$.

\begin{prop} \label{noscalarcross} Let $G$ be a crossed action operad, and let $G'$ be the action operad with $G'(m) = G(m)/G(0)$ for all $m \in \mathrm{N}$. Then for any $n \in \mathrm{N}$,
\begin{eq*} L\mathbb{G}'_n \quad \cong \quad L\mathbb{G}_n \end{eq*}
both as $\mathrm{E}G$-algebras and as $\mathrm{E}G'$-algebras. That is, every free invertible algebra over a crossed action operad is the same as one over an action operad with trivial $G(0)$. 
\end{prop}
\begin{proof}
It is fairly easy to see that the maps $[\, \_ \, ]: G(m) \to G(m)/G(0)$ sending elements to their equivalence class under the quotient must be surjective. Because of this, we will be able to use the action $\alpha^{L\mathbb{G}'_n}$ of $L\mathbb{G}'_n$ not just as an $\mathrm{E}G'$-action, but also as an $\mathrm{E}G$-action, which we'll call $\tilde{\alpha}^{L\mathbb{G}'_n}$ for the same of keeping the two concepts distinct. That is,
\begin{eq*} \tilde{\alpha}^{L\mathbb{G}'_n}( \, g \, ; \, \mathrm{id}_{x_1}, ..., \mathrm{id}_{x_m} \, ) \quad := \quad \alpha^{L\mathbb{G}'_n}\big( \, [g] \, ; \, \mathrm{id}_{x_1}, ..., \mathrm{id}_{x_m} \, \big) \end{eq*}
Likewise, the $\mathrm{E}G$-action of $L\mathbb{G}_n$ is also an $\mathrm{E}G'$-action, via
\begin{eq*} \tilde{\alpha}^{L\mathbb{G}_n}\big( \, [g] \, ; \, \mathrm{id}_{x_1}, ..., \mathrm{id}_{x_m} \, \big) \quad := \quad \alpha^{L\mathbb{G}_n}( \, g \, ; \, \mathrm{id}_{x_1}, ..., \mathrm{id}_{x_m} \, ) \end{eq*}
\cref{noscalar} ensures that this statement makes sense; whenever we have $[g] = [g']$ it is because there is some $h \in G(0)$ for which $g' = h \otimes g$, and so
\begin{eq*} \begin{array}{rll} 
			\alpha^{L\mathbb{G}_n}( \, g' \, ; \, \mathrm{id}_{x_1}, ..., \mathrm{id}_{x_m} \, ) & = & \alpha^{L\mathbb{G}_n}( \, h \otimes g \, ; \, \mathrm{id}_{x_1}, ..., \mathrm{id}_{x_m} \, ) \\
			& = & \alpha^{L\mathbb{G}_n}\big( \, \mu(e_2; h, g) \, ; \, \mathrm{id}_{x_1}, ..., \mathrm{id}_{x_m} \, \big) \\
			& = & \alpha^{L\mathbb{G}_n}\big( \, e_2 \, ; \, \alpha^{L\mathbb{G}_n}(h;-), \alpha^{L\mathbb{G}_n}(g;\mathrm{id}_{x_1}, ..., \mathrm{id}_{x_m}) \, \big) \\
			& = & \alpha^{L\mathbb{G}_n}(h;-) \otimes \alpha^{L\mathbb{G}_n}(g;\mathrm{id}_{x_1}, ..., \mathrm{id}_{x_m}) \\
			& = & \mathrm{id}_I \otimes \alpha^{L\mathbb{G}_n}(g;\mathrm{id}_{x_1}, ..., \mathrm{id}_{x_m}) \\
			& = & \alpha^{L\mathbb{G}_n}(g;\mathrm{id}_{x_1}, ..., \mathrm{id}_{x_m}) \\
		\end{array}
\end{eq*}
By \cref{Zobj} we already know that $L\mathbb{G}_n$ and $L\mathbb{G}'_n$ have isomorphic object sets, and so by using the universal properties of $\mathbb{G}_n$ and $\mathbb{G}'_n$ we can produce maps
\begin{eq*} \mathbb{G}_n \longrightarrow L\mathbb{G}'_n \quad \quad \quad \text{and} \quad \quad \quad \mathbb{G}'_n \longrightarrow L\mathbb{G}_n \end{eq*}
which correspond to the same choices of $n$ invertible objects that the maps $\eta^G$ and $\eta^{G'}$ do. The universal properties of $L\mathbb{G}_n$ and $L\mathbb{G}'_n$ will then make these new maps factor through the respective $\eta$'s, and so there must exist an $\mathrm{E}G$-algebra map
\begin{eq*} \begin{array}{rll}
			L\mathbb{G}_n & \to & L\mathbb{G}'_n \\
			x & \mapsto & x \\
			\alpha^{L\mathbb{G}_n}(g;\mathrm{id}_{x_1}, ..., \mathrm{id}_{x_m}) & \mapsto & \tilde{\alpha}^{L\mathbb{G}'_n}(g;\mathrm{id}_{x_1}, ..., \mathrm{id}_{x_m}) \\
			& & = \alpha^{L\mathbb{G}'_n}\big( \, [g] \, ; \, \mathrm{id}_{x_1}, ..., \mathrm{id}_{x_m} \, \big)
		\end{array}
\end{eq*}
and an $\mathrm{E}G'$-algebra map
\begin{eq*} \begin{array}{rll}
			L\mathbb{G}'_n & \to & L\mathbb{G}_n \\
			x & \mapsto & x \\
			\alpha^{L\mathbb{G}'_n}\big( \, [g] \, ;\mathrm{id}_{x_1}, ..., \mathrm{id}_{x_m} \, \big) & \mapsto & \tilde{\alpha}^{L\mathbb{G}_n}\big( \, [g] \, ; \, \mathrm{id}_{x_1}, ..., \mathrm{id}_{x_m} \, \big) \\
			& & = \alpha^{L\mathbb{G}_n}(g;\mathrm{id}_{x_1}, ..., \mathrm{id}_{x_m})
		\end{array}
\end{eq*}
These functors are clearly inverses, and also algebra maps for both $G$ and $G'$. Therefore
\begin{eq*} L\mathbb{G}'_n \quad \cong \quad L\mathbb{G}_n \end{eq*}
in both senses, as required.   
\end{proof}

For non-crossed $G$ we cannot so easily remove the group $G(0)$ like this, as without being spacial we have no way to draw its elements out from in between elements of the higher $G(m)$. Still, there is one more thing about the morphisms of $L\mathbb{G}_n$ that we can deduce from \cref{noscalar}.

\begin{defn} Let $G$ be a non-crossed action operad in which every element of each $G(m)$ can be written as $\mu(g;e_m)$ for some $G \in G(1)$. Then we say that $G$ is a \emph{$G(1)$-generated} action operad. \end{defn}

\begin{lem} \label{noscalarnoncross} If $G$ is a $G(1)$-generated action operad, then $L\mathbb{G}_n(I,I)$ is the trivial group.
\end{lem}
\begin{proof}
First we need to check that this claim makes sense, that elements of the required form are indeed closed under operadic multiplication so that they may make up a valid $G$. This is the case, as we have
\begin{longtable}{RLL}
			\mu\big( \, \mu(g;e_m) \, ; \, \mu(h_1;e_{k_1}), ..., \mu(h_1;e_{k_m}) \, \big) & = & \mu\Big( \, g \, ; \, \mu\big( \, e_m \, ; \,  \mu(h_1;e_{k_1}), ..., \mu(h_m;e_{k_m}) \, \big) \, \Big) \\
			& = & \mu\Big( \, g \, ; \, \mu\big( \, \mu(e_m;h_1, ..., h_m) \, ; \, e_{k_1}, ..., e_{k_m} \, \big) \, \Big) \\
			& = & \mu\Big( \, g \, ; \, \mu\big( \, \mu(h;e_m) \, ; \, e_{k_1}, ..., e_{k_m} \, \big) \, \Big) \\
			& = & \mu\Big( \, g \, ; \, \mu\big( \, h \, ; \, \mu(e_m;e_{k_1}, ..., e_{k_m}) \, \big) \, \Big) \\
			& = & \mu\big( \, g \, ; \, \mu( \, h \, ; \, e_{k_1+...+k_m} \, ) \, \big) \\
			& = & \mu\big( \, \mu(g;h) \, ; \, e_{k_1+...+k_m} \, \big) \\
\end{longtable}
where $\mu(h;e_m)$ is any way of writing $\mu(e_m;h_1, ..., h_m) = h_1 \otimes ... \otimes h_m$ in the required form.

Now let $f$ be an arbitrary element of $L\mathbb{G}_n(I,I)$. By \cref{allmapsaction} there must be some objects $x_1, ..., x_m$ such that $f = \alpha(g; \mathrm{id}_{x_1}, ..., \mathrm{id}_{x_m})$. Then by assumption there must also exist some $h \in G(1)$ for which $g = \mu(h;e_m)$. With this in mind, we see that
\begin{eq*} \begin{array}{rll}
			\alpha(g; \mathrm{id}_{x_1}, ..., \mathrm{id}_{x_m}) & = & \alpha\big( \, \mu(h;e_m) \, ; \, \mathrm{id}_{x_1}, ..., \mathrm{id}_{x_m} \, \big) \\
			& = & \alpha\big( \, h \, ; \, \mu(e_m;\mathrm{id}_{x_1}, ..., \mathrm{id}_{x_m}) \, \big) \\
			& = & \alpha( \, h \, ; \, \mathrm{id}_{x_1 \otimes ... \otimes x_m}) \\
		\end{array}
\end{eq*}
But this is supposed to be a morphism $f:I\to I$, so we know that $x_1 \otimes ... \otimes x_m = I$, and therefore by \cref{noscalar}
\begin{eq*} f \quad = \quad \alpha( \, h \, ; \, \mathrm{id}_I) \quad = \quad \mathrm{id}_I \end{eq*}
As $f$ was chosen arbitrarily, it follows that $L\mathbb{G}_n(I,I) = \{ \mathrm{id}_I \}$.
\end{proof}

Ultimately, we will see that there is very little we can say for sure about the unit endomorphisms of $L\mathbb{G}_n$ when $G$ is not crossed, other than \cref{noscalarnoncross}. For this reason, the main theorems of this paper, \cref{freeinvalgG1,freeinvalgc}, will end up describing only those invertible $\mathrm{E}G$-algebras whose action operads are either $G(1)$-generated or crossed, respectively.

\section{$L\mathbb{G}_n$ as a coequaliser in $\mathrm{MonCat}$} \label{colimmoncat}

Looking back at the proof of \cref{coeqsurj}, notice that we never needed to use the fact that $\phi$, $\phi'$ and $k$ were maps of $\mathrm{E}G$-algebras, only that they were monoidal functors. We did at one point have to show that the category $k(Y)$ was an algebra so that we could then use the universal property of $k$ in $\mathrm{E}G\mathrm{Alg}_S$, because we had assumed from the beginning that we were working in that category, but if $k$ had just been a coequaliser in $\mathrm{MonCat}$ from the start then this part would not have been necessary. We also had to invoke \cref{Obadj} --- which says that $\mathrm{Ob}: \mathrm{E}G\mathrm{Alg}_S \to \mathrm{Mon}$ is a left adjoint --- so that we could exploit preservation of colimits. But since $\mathrm{Ob}$ clearly doesn't care about the morphisms of an algebra, it doesn't really matter whether we are applying it to an algebra in the first place. The actions of $X$, $Y$ and $Z$ just never came into play.

With that in mind, we can co-opt all of these previous proofs about $\mathrm{E}G$-algebra maps to prove the analogous statements about monoidal functors.

\begin{prop}\label{Obadjmon} Let the functors 
\begin{eq*} \mathrm{Ob} \, : \, \mathrm{MonCat} \to \mathrm{Mon}, \quad \quad \quad \mathrm{E} \, : \, \mathrm{Mon} \to \mathrm{MonCat} \end{eq*}
be defined exactly as those from \cref{Obdef,Edef}, except without the requirement that the monoidal categories be $\mathrm{E}G$-algebras. Then $\mathrm{E}$ is a right adjoint to the functor $\mathrm{Ob}$. 
\end{prop}
\begin{proof}
The same as the proof of \cref{Obadj}.
\end{proof}

\begin{prop} \label{coeqsurjmon} Let $\phi, \phi' : X \to Y$ be a pair of parallel monoidal functors, and $k: Y \to Z$ their coequaliser in $\mathrm{MonCat}$. If the monoid $\mathrm{Ob}(Z)$ is also a group, then the functor $k$ is surjective.
\end{prop}
\begin{proof}
The same as the proof of \cref{coeqsurj}, but with \cref{Obadjmon} in place of \cref{Obadj}, and no reference to $k(Y)$ being a sub-$\mathrm{E}G$-algebra.
\end{proof}

Further, these new propositions prove a surjectivity statement just like \cref{qsurj}. 

\begin{defn}\label{Cdef} Let the monoidal functor $c: \mathbb{G}_{2n} \to C$ onto some monoidal category $C$ be the cokernel of the underlying monoidal functor of $\delta$ in $\mathrm{MonCat}$. This map definitely exists because $\mathrm{MonCat}$ is cocomplete, and like with $q$ we can show that its target has a group of objects.
\end{defn}

\begin{prop}\label{Cobj} The object monoid of $C$ is $\mathbb{Z}^{*n}$, and the restriction of $c$ to objects $\mathrm{Ob}(c): \mathrm{Ob}(\mathbb{G}_{2n}) \to \mathrm{Ob}(C)$ is the monoid homomorphism defined on generators as
\begin{eq*} \begin{array}{rlrlll}
			\mathrm{Ob}(c) & : & \mathbb{N}^{\ast 2n} & \to & \mathbb{Z}^{\ast n} \\
			& : & z_i & \mapsto & z_i  \\
			& : & z_{n+i} & \mapsto & z_i^*		
		\end{array}
\end{eq*}
\end{prop}
\begin{proof}
The same as the proof of \cref{Qobj}, but with $c: \mathbb{G}_{2n} \to C$ in place of $q: \mathbb{G}_{2n} \to Q$ and \cref{Obadjmon} in place of \cref{Obadj}.
\end{proof}

\cref{coeqsurjmon,Cobj} then immediately combine to give:

\begin{cor}\label{csurj} The cokernel map $c: \mathbb{G}_{2n} \to C$ is surjective.
\end{cor}

This statement is actually pretty unusual. In \cref{qsurj} it made sense that $q$ would be surjective, but that was because its source and target were special. $\mathbb{G}_{2n}$ is the free $\mathrm{E}G$-algebra on $2n$ objects, and $L\mathbb{G}_n$ is the free $\mathrm{E}G$-algebra on $n$ objects and their $n$ inverses, and so intuitively the map identifying those sets' generators would tell us everything we need to know about the algebra structure of $L\mathbb{G}_n$. And since by freeness we expect algebra maps to be all there really is to $L\mathbb{G}_n$, it was a safe bet that $q$ was going to be surjective.

But none of that is true for $c$. The underlying monoidal category of $\mathbb{G}_{2n}$ is not anything special in $\mathrm{MonCat}$, and neither is $C$. So what is going on here? The answer is that category $C$ is \emph{almost} the algebra $L\mathbb{G}_n$, and likewise the functor $c$ is \emph{almost} the map $q$. To see this, consider the following naive method for assigning an $\mathrm{E}G$-action $\alpha^C$ to $C$:
\begin{eq*} \alpha^C( \, g \, ; \, c(f_1), ..., c(f_m) \, ) \quad := \quad c \big( \, \alpha^{\mathbb{G}_{2n}}( \, g \, ; \, f_1, ..., f_m \, ) \, \big) \end{eq*}
Any action on $C$ that made $c$ into a map of $\mathrm{E}G$-algebras would have to satify this condition, of course. But because $c$ is surjective, every collection of morphisms in $C$ can be written as $c(f_1), ..., c(f_m)$, and this forces $\alpha^C$ to take a unique value everywhere, assuming it is well-defined. Then, since the the cokernel of $\delta$ in $\mathrm{MonCat}$ would be an $\mathrm{E}G$-algebra map, we could conclude that it was also the cokernel of $\delta$ in $\mathrm{E}G\mathrm{Alg}_S$ too. However, `assuming it is well-defined' is where the problems lie. In particular, since $c$ is not injective on objects we can find $w_1, ..., w_m$ and $w'_1, ..., w'_m$ in $\mathbb{G}_{2n}$ for which $c(w_i) = c(w'_i)$, and so $\alpha^C$ would only be well-defined if
\begin{eq*} c \big( \, \alpha^{\mathbb{G}_{2n}}( \, g \, ; \, \mathrm{id}_{w_1}, ..., \mathrm{id}_{w_m} \, ) \, \big) \quad = \quad c \big( \, \alpha^{\mathbb{G}_{2n}}( \, g \, ; \, \mathrm{id}_{w'_1}, ..., \mathrm{id}_{w'_m} \, ) \, \big) \end{eq*}
which we have no reason to believe is true. 

To fix this issue, what we need is a way of describing the map $q$ as a colimit of a slightly different diagram in $\mathrm{E}G$-algebras, one whose colimit in $\mathrm{MonCat}$ will have all of the same properties that $c$ does but will also satisfy the condition above. To that end, consider the following $\mathrm{E}G$-algebra maps:
 
\begin{defn} \label{coprodmapdef} Let $\tilde{\delta} := \mathrm{id}_{\mathbb{G}_{2n}}+\delta$ be the map defined from $\delta$ and the identity by using the universal property of the coproduct $\mathbb{G}_{4n} = \mathbb{G}_{2n} + \mathbb{G}_{2n}$ in $\mathrm{E}G\mathrm{Alg}_S$. That is, $\tilde{\delta}$ is the map of $\mathrm{E}G$-algebras which acts on generators by
\begin{eq*} \begin{array}{rlrlll}
			\tilde{\delta} & : & \mathbb{G}_{4n} & \to & \mathbb{G}_{2n} \\
			& : & z_i & \mapsto & z_i  \\
			& : & z_{n+i} & \mapsto & z_{n+i} \\
			& : & z_{2n+i} & \mapsto & z_i \otimes z_{n+i} \\
			& : & z_{3n+i} & \mapsto & z_{n+i} \otimes z_i			
		\end{array}
\end{eq*}
for $1 \le i \le n$. Similarly, let $\tilde{I} := \mathrm{id}_{\mathbb{G}_{2n}}+I$ be the $\mathrm{E}G$-algebra map defined in the same way but from the constant map on the unit $I$ instead of $\delta$:
\begin{eq*} \begin{array}{rlrlll}
			\tilde{I} & : & \mathbb{G}_{4n} & \to & \mathbb{G}_{2n} \\
			& : & z_i & \mapsto & z_i  \\
			& : & z_{n+i} & \mapsto & z_{n+i} \\
			& : & z_{2n+i} & \mapsto & I \\
			& : & z_{3n+i} & \mapsto & I
		\end{array} 
\end{eq*}
\end{defn}

\begin{lem} $q$ is the coequaliser of $\tilde{\delta}$ and $\tilde{I}$ in $\mathrm{E}G\mathrm{Alg}_S$.
\end{lem}
\begin{proof}
Let $\psi: \mathbb{G}_{2n} \to X$ be an map of $\mathrm{E}G$-algebras. Then
\begin{eq*} \begin{array}{rcl}
			\psi \circ (\mathrm{id}_{\mathbb{G}_{2n}}+\delta)(z_i) & = & \psi \circ (\mathrm{id}_{\mathbb{G}_{2n}}+I) \\
			& \iff & \\
			\psi \circ \mathrm{id}_{\mathbb{G}_{2n}} \quad = \quad \psi \circ \mathrm{id}_{\mathbb{G}_{2n}}, & & \psi \circ \delta \quad = \quad \psi \circ I
		\end{array}
\end{eq*}
and hence
\begin{eq*} \mathrm{coeq}( \, \mathrm{id}_{\mathbb{G}_{2n}}+\delta, \, \mathrm{id}_{\mathbb{G}_{2n}}+I \, ) \quad = \quad \mathrm{coeq}(\delta, I) \quad = \quad \mathrm{coker}(\delta) \quad = \quad q\end{eq*}
\end{proof}

While this proof may seem rather trivial, notice that it does rely on the fact that the $+$ here represents the coproduct in the category of $\mathrm{E}G$-algebras. There is no reason to expect that the coequaliser of the underlying monoidal functors of these maps would also be equal the cokernel of the underlying monoidal functor of $\delta$. Thus these new maps will give rise to a new map which is distinct from the cokernel functor $c$, yet possesses many of the same properties.

\begin{defn} \label{C'def} Denote by $\tilde{c}: \mathbb{G}_{2n} \to \tilde{C}$ the coequaliser of $\tilde{\delta}$ and $\tilde{I}$ in the category $\mathrm{MonCat}$. \end{defn}

\begin{lem} \label{C'obj} The object monoid of $\tilde{C}$ is
\begin{eq*} \mathrm{Ob}(\tilde{C}) \quad = \quad \mathrm{Ob}(C) \quad = \quad \mathbb{Z}^{*n} \end{eq*}
and the restriction of $\tilde{c}$ to objects $\mathrm{Ob}(\tilde{c}): \mathrm{Ob}(\mathbb{G}_{2n}) \to \mathrm{Ob}(\tilde{C})$ is just the monoid homomorphism $\mathrm{Ob}(c): \mathbb{N}^{*2n} \to \mathbb{Z}^{*n}$ from \cref{Cobj}.
\end{lem}
\begin{proof}
Consider the monoid homomorphisms $\mathrm{Ob}(\tilde{\delta}): \mathbb{N}^{\ast 4n} \to \mathbb{N}^{\ast 2n}$ and $\mathrm{Ob}(\tilde{I}): \mathbb{N}^{\ast 4n} \to \mathbb{N}^{\ast 2n}$. These are fully determined by the descriptions of the corresponding algebra maps in \cref{coprodmapdef}, and as such they are obviously just
\begin{eq*} \begin{array}{rclcrcl}
			\mathrm{Ob}(\mathrm{id}_{\mathbb{G}_{2n}}+\delta) & = & \mathrm{id}_{\mathbb{N}^{\ast 2n}}+\mathrm{Ob}(\delta) & \quad \quad \quad \quad & \mathrm{Ob}(\mathrm{id}_{\mathbb{G}_{2n}}+I) & = & \mathrm{id}_{\mathbb{N}^{\ast 2n}}+\mathrm{Ob}(I) \\
			& & & & & = & \mathrm{id}_{\mathbb{N}^{\ast 2n}}+I
		\end{array}
\end{eq*}
where the $+$ on the righthand side of the equations means the coproduct in the category of monoids. Therefore
\begin{eq*} \mathrm{coeq}\big( \, \mathrm{Ob}(\mathrm{id}_{\mathbb{G}_{2n}}+\delta), \, \mathrm{Ob}(\mathrm{id}_{\mathbb{G}_{2n}}+I) \, ) \quad = \quad \mathrm{coeq}\big( \, \mathrm{Ob}(\delta), I \, \big) \quad = \quad \mathrm{Ob}(c) \end{eq*}
and thus $\mathrm{Ob}(\tilde{C}) = \mathrm{Ob}(C)$.
\end{proof}

\begin{cor} \label{c'surj} The coequaliser map $\tilde{c}: \mathbb{G}_{2n} \to \tilde{C}$ is surjective.
\end{cor}
\begin{proof}
\cref{C'obj} says that the monoid $\tilde{C}$ is a group, so we may apply \cref{coeqsurjmon}.
\end{proof}

So why bother with any of this? What features do $\tilde{\delta}$ and $\tilde{I}$ have that will make an action possible on $\tilde{C}$ when it wasn't on $C$? The answer is that unlike $\delta$ and $I$, these new maps form a \emph{reflexive pair} --- a parallel pair of functors which share a right-inverse.

\begin{lem} \label{sect} Let $\iota: \mathbb{G}_{2n} \to \mathbb{G}_{4n}$ be the inclusion of algebras defined on generators by $z_i \mapsto z_i$. Then $\iota$ is a right-inverse of both $\tilde{\delta}$ and $\tilde{I}$. \end{lem} 
\begin{proof}
For $1 \le i \le 2n$,
\begin{eq*}\begin{array}{rcccccccl}
			\tilde{\delta} \iota(z_i) & = & \tilde{\delta}(z_i) & = & z_i & = & \tilde{I}(z_i) & = & \tilde{I} \iota(z_i) \\
			\\
			\implies & & \tilde{\delta} \circ \iota & = & \mathrm{id}_{\mathbb{G}_{2n}} & = & \tilde{I} \circ \iota & & 
		\end{array}
\end{eq*}
\end{proof} 

In other words, $\tilde{c}$ is a \emph{reflexive coequaliser} in the category $\mathrm{MonCat}$. This is the key difference which will eventually let us prove that $\tilde{c}$ respects action morphisms in the way that we need it to. First though, we will need a few intermediate results.

\begin{defn}\label{decompdef} If $w$ is an element of $\mathbb{N}^{\ast m}$, then we can use the definition of the free product of groups to decompose it uniquely as a tensor product of the $m$ generators $z_1, ..., z_m$. We'll denote this by
\begin{eq*} w \quad =: \quad \bigotimes_{i=1}^{|w|} \, d(w, i), \quad \quad \quad d(w, i) \in \{ z_1, ..., z_m \} \end{eq*}
If instead $w$ is an element of $\mathbb{Z}^{\ast m}$, then we can use the definition of the free product of groups to decompose $x$ uniquely as a tensor product, but this time one made up of the $m$ generators $z_1, ..., z_m$ and their inverses $z_1^*, ..., z_m^*$. As before we'll denote this by
\begin{eq*} w \quad = \quad \bigotimes_{i=1}^{|w|} \, d(w, i) \end{eq*}
where $d(w, i) \in \{ z_1, ..., z_m, z_1^*, ..., z_m^* \}$, and also for any $1 \le i < |w|$ we will always have $d(w, i+1) \neq d(w, i)^*$. By analogy with \cref{lengthdef}, we will call the upper bound of this tensor product the \emph{length} of the element $w$, and denote it by $|w|$, but be aware that this number is the one that comes from the \emph{monoid} homomorphism
\begin{eq*} F\big( \, \{ z_1, ..., z_m, z_1^*, ..., z_m^* \} \, \big) \to \mathbb{N} \end{eq*}
that sends each generator to 1, and not the perhaps more obvious \emph{group} homomorphism
\begin{eq*} F\big( \, \{ z_1, ..., z_m \} \, \big) \to \mathbb{Z} \end{eq*}
\end{defn}

\begin{prop}\label{c'alg1} Let $w$ be an object of $\mathbb{G}_{2n}$. Then there exist objects $w^{(1)}, ..., w^{(k)}$ in $\mathbb{G}_{2n}$ and $u^{(1)}, ..., u^{(k)}$ in $\mathbb{G}_{4n}$, for some value of $k \in \mathbb{N}$, such that
\begin{eq*} w^{(1)} \,= \, w, \quad \quad u^{(k)} \, = \, \iota(w^{(k)}), \quad \quad \quad \tilde{I}(u^{(i-1)}) \quad = \quad w^{(i)} \quad = \quad \tilde{\delta}(u^{(i)}) \end{eq*}
for $1 \le i \le k$, and for any object $u$ of $\mathbb{G}_{4n}$,
\begin{eq*} \tilde{\delta}(u) \, = \, w^{(k)} \quad \iff \quad u \, = \, u^{(k)} \end{eq*}
\end{prop}
\begin{proof}
From \cref{lengthdef,coprodmapdef}, we know that for any generator $z_i$ of $\mathbb{G}_{4n}$,
\begin{eq*}\begin{array}{rllll}
				 | \tilde{\delta}(z_i) |  & = & \left. \begin{cases}
								\quad 1 & \text{if} \quad 1 \le i \le 2n \\
								\quad 2 & \text{if} \quad 2n+1 \le i \le 4n
							\end{cases} \quad \right \rbrace & \ge 1 \\
				& & \\
				| \tilde{I}(z_i) |  & = & \left. \begin{cases}
								\quad 1 & \text{if} \quad 1 \le i \le 2n \\
								\quad 0 & \text{if} \quad 2n+1 \le i \le 4n
							\end{cases} \quad \right \rbrace & \le 1 
		\end{array}
\end{eq*}
Also these lengths are additive across tensor products, since $| \, \_ \, |$ is a monoid homomorphism $\mathbb{G}_{2n} \to \mathbb{N}$. Thus for any object $u$ in $\mathbb{G}_{4n}$, we can conclude that
\begin{eq*}\begin{array}{rllllllll}
			| \tilde{\delta}(u) | & = & | \, \tilde{\delta}\big( \, \mathlarger{\bigotimes_{i=1}^{|u|} d(u, i)} \, \big) \, | & = & \mathlarger{\sum_{i=1}^{|u|} | \, \tilde{\delta} \big( \, d(u, i) \, \big) \, |} & \ge & \mathlarger{\sum_{i=1}^{|u|} 1} & = & |u| \\[\bigskipamount]
			| \tilde{I}(u) | & = & | \, \tilde{I}\big( \, \mathlarger{\bigotimes_{i=1}^{|u|} d(u, i)} \, \big) \, | & = & \mathlarger{\sum_{i=1}^{|u|} | \, \tilde{I} \big( \, d(u, i) \, \big) \, |} & \le & \mathlarger{\sum_{i=1}^{|u|} 1} & = & |u|
		\end{array}
\end{eq*}
Also, since the only generators that have $| \tilde{\delta}(z_i) | = | \tilde{I}(z_i) | = 1$ are those from the $\mathbb{G}_{2n}$ subalgebra associated with $\iota$, the inequalities above becomes equalities if and only if $u$ is in the image of $\iota$. That is,
\begin{eq*} | \tilde{I}(u) | \, = \, |u| \, = \, |\tilde{\delta}(u)|  \quad \iff \quad \exists \, v \in \mathbb{N}^{\ast 2n} \, : \, u \, = \, \iota(v) \end{eq*}

Next, consider the set
\begin{eq*} \tilde{\delta}^{-1}(w) \quad := \quad \{ \, u \in \mathbb{N}^{\ast 4n} : \tilde{\delta}(u) = w \, \} \end{eq*}
of all objects in $\mathbb{G}_{4n}$ which $\tilde{\delta}$ sends to $w$. This set is always nonempty, since by \cref{sect} $\iota$ is a right-inverse of $\delta$:
\begin{eq*} \tilde{\delta} \iota(w) \, = \, w \quad \implies \quad \iota(w) \in \tilde{\delta}^{-1}(w) \end{eq*}
Moreover, $\iota(w)$ is the only element of $\tilde{\delta}^{-1}(w)$ which can be expessed as $\iota(v)$ for some object $v$ in $\mathbb{G}_{2n}$, because
\begin{eq*} \tilde{\delta} \big( \, \iota(v) \, \big) \, = \, w \quad \implies \quad v \, = \, w \end{eq*}

With all of this now in place, we can begin constructing the sequences $w^{(1)}, ..., w^{(k)}$ and $u^{(1)}, ..., u^{(k)}$. Start by setting $w^{(1)} = w$ and $i=1$, then apply the following algorithm:
\begin{enumerate}
\item If $\tilde{\delta}^{-1}(w^{(i)})$ is just the set $\{ \iota(w^{(i)}) \}$, choose $u^{(i)} = \iota(w^{(i)})$, set $k$ to be the current value of $i$, and terminate.
\item Otherwise, choose $u^{(i)}$ to be any element of $\tilde{\delta}^{-1}(w^{(i)})$ other than $\iota(w^{(i)})$.
\item Set $w^{(i+1)} = \tilde{I}(u^{(i)})$.
\item Increase the value of $i$ by 1, then return to step 1.
\end{enumerate}
By design, none of the $u^{(i)}$ produced by this process can be expressed as $u^{(i)} = \iota(v)$ for some $v$ in $\mathbb{G}_{2n}$, with the possible exception of $u_k$ if the algorithm terminates. This is because $\iota(w^{(i)})$ is the only element of $\delta^{-1}(w^{(i)})$ that can be expressed that way, and the above process will terminate the first time it has to pick $u^{(i)} = \iota(w^{(i)})$, at which point $i$ is set equal to $k$. Thus given what we found earlier in the proof, for any $i \neq k$ we must have the following \emph{strict} inequalities:
\begin{eq*} |w^{(i+1)}| \quad = \quad | \tilde{I}(u^{(i)}) | \quad < \quad |u^{(i)}| \quad < \quad |\tilde{\delta}(u^{(i)})| \quad = \quad |w^{(i)}| \end{eq*}
That is, the $w^{(i)}$ produced by this algorithm form a sequence with strictly decreasing length. However, it is impossible to have a infinite sequence of strictly decreasing natural numbers, and hence we can be sure that this process will terminate at some finite $k$. 

But in order for the algorithm to terminate, it must be the case that 
\begin{eq*} \tilde{\delta}^{-1}(w^{(k)}) \quad = \quad \{ \iota(w^{(k)}) \} \end{eq*}
and hence
\begin{eq*} \tilde{\delta}(u) \, = \, w^{(k)} \quad \iff \quad u \, = \, \iota(w^{(k)}) \, = \, u^{(k)} \end{eq*}
Thus the sequences $w^{(1)}, ..., w^{(k)}$ and $u^{(1)}, ..., u^{(k)}$ satisfy all of the conditions in the statement of the lemma.
\end{proof}

The intuition behind \cref{c'alg1} is that we are successively removing parts of the object $w$, without changing its image under $\tilde{c}$. The map $\tilde{\delta}$ sends $z_{2n+i} \mapsto z_i \otimes z_{n+i}$ and $z_{3n+i} \mapsto z_{n+1} \otimes z_i$ while $\tilde{I}$ sends these all to $I$, and so for any $u$ in $\mathbb{G}_{4n}$ the object $\tilde{I}(u)$ will look like $\tilde{\delta}(u)$ except missing some number of $z_i \otimes z_{n+i}$ or $z_{n+1} \otimes z_i$ substrings. But since $\tilde{c}$ sends $z_{n+i} \mapsto z_i^*$, these are exactly the sort of omissions which the coequaliser doesn't care about. If we repeat this process then it will eventually terminate at $u^{(k)} = \iota(w^{(k)})$, so we really have a method for removing \emph{all} of the relevant substrings from objects of $\mathbb{G}_{2n}$. In other words, $w^{(k)}$ has the smallest possible length while still having $\tilde{c}(w^{(k)}) = \tilde{c}(w)$. In fact, we will show that it is the unique shortest object of $\mathbb{G}_{2n}$ with this property.

\begin{prop}\label{c'alg2} Let $w$, $w'$ be objects of $\mathbb{G}_{2n}$ such that $\tilde{c}(w) = \tilde{c}(w')$. If $w^{(1)}, ..., w^{(k)}$ and $u^{(1)}, ..., u^{(k)}$ are the sequences generated from $w$ via \cref{c'alg1}, and likewise $w'^{\, (1)}, ..., w'^{\, (k')}$ and $u'^{\, (1)}, ..., u'^{\, (k')}$ from $w'$, then $w^{(k)} = w'^{\, (k')}$ and $u^{(k)} = u'^{\, (k')}$.
\end{prop}
\begin{proof}
Consider the decomposition of the object $w^{(k)} \in \mathbb{N}^{\ast 2n}$ as in \cref{decompdef}. Assume, for the sake of contradiction, that there exist $1 \le j < |w^{(k)}|$ and $1 \le m \le n$ such that
\begin{eq*} d(w^{(k)}, j) \, = \, z_m, \quad \quad d(w^{(k)}, j+1) \, = \, z_{n+m} \end{eq*}
Then we can use $j$ and $m$ to contruct a new element $u \in \mathbb{N}^{\ast 4n}$, defined by
\begin{eq*} |u| \, = \, |w| - 1, \quad \quad d(u, i) \, = \, \begin{cases}
									\quad \iota \big( \, d(w^{(k)}, i) \, ) & \text{if} \quad 1 \le i < j \\
									\quad z_{2n + m} & \text{if} \quad i = j \\
									\quad \iota \big( \, d(w^{(k)}, i+1) \, ) & \text{if} \quad j < i \le |u|
								\end{cases}
\end{eq*}
This $u$ will then have the property that
\begin{eq*} \begin{array}{rll}
			\tilde{\delta}(u) & = & \tilde{\delta} \big( \, \mathlarger{\bigotimes_{i=1}^{|u|} \, d(u, i)} \, \big) \\[\bigskipamount]
			& = & \mathlarger{\bigotimes_{i=1}^{|u|} \, \tilde{\delta} \big( \, d(u, i) \, \big)} \\[\bigskipamount]
			& = & \mathlarger{\bigotimes_{i=1}^{j-1} \, \tilde{\delta} \iota \big( \, d(w^{(k)}, i) \, )} \, \otimes \tilde{\delta}(z_{2n + m}) \otimes \, \mathlarger{\bigotimes_{i=j+1}^{|u|} \, \tilde{\delta} \iota \big( \, d(w^{(k)}, i+1) \, \big)} \\[\bigskipamount]
			& = & \mathlarger{\bigotimes_{i=1}^{j-1} \, d(w^{(k)}, i)} \, \otimes z_m \otimes z_{n + m} \otimes \, \mathlarger{\bigotimes_{i=j+2}^{|u|+1} \, d(w_k, i)} \\[\bigskipamount]
			& = & \mathlarger{\bigotimes_{i=1}^{j-1} \, d(w^{(k)}, i)} \, \otimes d(w^{(k)}, j) \, \otimes d(w^{(k)}, j+1) \otimes \mathlarger{\bigotimes_{i=j+2}^{|u|+1} \, d(w_k, i)} \\[\bigskipamount]
			& = & w^{(k)}
		\end{array}
\end{eq*}
But this is impossible, since by \cref{c'alg1} $u^{(k)}$ is the only object of $\mathbb{G}_{4n}$ whose image under $\tilde{\delta}$ is $w^{(k)}$, and this $u$ we have constructed is manifestly not $w^{(k)}$. Thus we can conclude that there are no values of $j$ and $m$ for which
\begin{eq*} d(w^{(k)}, j) \, = \, z_m, \quad \quad d(w^{(k)}, j+1) \, = \, z_{n+m} \end{eq*}
An analogous line of reasoning --- using $z_{3n + m}$ rather than $z_{2n + m}$ in the definition of $u$ --- demonstrates that there are also no $j, m$ with
\begin{eq*} d(w^{(k)}, j) \, = \, z_{n+m}, \quad \quad d(w^{(k)}, j+1) \, = \, z_m \end{eq*}
As a result, for all $1 \le i < |w^{(k)}|$
\begin{eq*} \tilde{c} \big( \, d(w^{(k)}, i+1) \, \big) \quad \neq \quad \tilde{c} \big( \, d(w^{(k)}, i) \, \big)^* \end{eq*}
and this combined with the fact that
\begin{eq*} \bigotimes_{i=1}^{|w^{(k)}|} \, \tilde{c} \big( \, d(w^{(k)}, i) \, \big) \quad = \quad \tilde{c} \big( \, \bigotimes_{i=1}^{|w^{(k)}|} \, d(w^{(k)}, i) \, \big) \quad = \quad \tilde{c}(w^{(k)}) \end{eq*}
shows that the unique decomposition of $\tilde{c}(w^{(k)}) \in \mathbb{Z}^{\ast n}$ as in \cref{decompdef} is given by
\begin{eq*} |\tilde{c}(w^{(k)})| \, = \, |w^{(k)}|, \quad \quad d\big( \, \tilde{c}(w^{(k)}), i \, \big) \quad = \quad \tilde{c} \big( \, d(w^{(k)}, i) \, \big) \end{eq*}
Next, let $r$ be a function --- not a homomorphism --- defined by
\begin{eq*} \begin{array}{rllll}
			r & : & \mathbb{Z}^{\ast n} & \to & \mathbb{N}^{\ast 2n} \\
			& : & z_i & \mapsto & z_i \\
			& : & z_i^* & \mapsto & z_{n+i} \\
			& : & x & \mapsto & \bigotimes_{i=1}^{|x|} \, r \big( \, d(x, i) \, \big)
		\end{array}
\end{eq*}
Then for $1 \le i \le n$,
\begin{eq*} r\tilde{c}(z_i) \, = \, r(z_i) \, = \, z_i, \quad \quad r\tilde{c}(z_{n+i}) \, = \, r(z_i^*) \, = \, z_{n+i} \end{eq*}
and so it follows that
\begin{eq*}	r\tilde{c}(w^{(k)}) \quad = \quad \bigotimes_{i=1}^{|w^{(k)}|} \, r\tilde{c} \big( \, d(w^{(k)}, i) \, \big) \quad = \quad \bigotimes_{i=1}^{|w^{(k)}|} \, d(w^{(k)}, i) \quad = \quad w^{(k)} \end{eq*}
Finally, notice that the exact same logic as we've used above will work for $w'^{\, (k')}$ as well, so that $r\tilde{c}(w'^{\, (k')}) = w'^{\, (k')}$. 

Therefore, putting everything together tells us that
\begin{longtable}{RCCCCCC}
	w_k & = & r\tilde{c}(w^{(k)}) & = & r\tilde{c}\tilde{I}(u^{(k-1)}) & = & r\tilde{c}\tilde{\delta}(u^{(k-1)}) \\
	& = & r\tilde{c}(w^{(k-1)}) & = & \vdots & = & \vdots  \\
	& \vdots & & & & & \\
	& = & r\tilde{c}(w^{(1)}) & & & & \\
	& = & r\tilde{c}(w) & & & &  \\
	& = & r\tilde{c}(w') & & & & \\
	& = & r\tilde{c}(w'^{\, (1)}) & = & r\tilde{c}\tilde{\delta}(u'^{\, (1)}) & = &  r\tilde{c}\tilde{I}(u'^{\, (1)}) \\
	& = & r\tilde{c}(w'^{\, (2)}) & = & \vdots & = & \vdots  \\
	& \vdots & \\
	& = & r\tilde{c}(w'^{\, (k')}) \\
	& = & w'^{\, (k')}			
\end{longtable}
as required.
\end{proof}

It is this special property --- shared by all $w$, $w'$ for which $\tilde{c}(w) = \tilde{c}(w')$ --- that will now let us prove that the coequaliser $\tilde{c}$ satisfies the condition which we couldn't prove about the cokernel $c$. In other words, with \cref{c'alg1,c'alg2} we can now construct a valid $\mathrm{E}G$-action on the monoidal category $\tilde{C}$.

\begin{prop}\label{c'alg} There is a unique action $\alpha^{\tilde{C}}$ making the category $\tilde{C}$ into $\mathrm{E}G$-algebra and the functor $\tilde{c}: \mathbb{G}_{2n} \to \tilde{C}$ into a map of $\mathrm{E}G$-algebras.  
\end{prop}
\begin{proof}
We will try to affix an action to $\tilde{C}$ in the same way we thought about doing with the category $C$. In order for the functor $\tilde{c} : \mathbb{G}_{2n} \to \tilde{C}$ to be an $\mathrm{E}G$-algebra map with respect to some $\alpha^{\tilde{C}}$, it must satisfy
\begin{eq*} \tilde{c} \big( \, \alpha^{\mathbb{G}_{2n}}( \, g \, ; \, f_1, ..., f_m \, ) \, \big) \quad = \quad \alpha^{\tilde{C}}( \, g \, ; \, \tilde{c}(f_1), ..., \tilde{c}(f_m) \, ) \end{eq*}
for all morphisms $f_1, ..., f_m$ in $\mathbb{G}_{2n}$, though given \cref{Gnmapsaction} it will be enough to have
\begin{eq*} \begin{array}{rll}
			\tilde{c} \big( \, \alpha^{\mathbb{G}_{2n}}( \, g \, ; \, \mathrm{id}_{w_1}, ..., \mathrm{id}_{w_m} \, ) \, \big) & = & \alpha^{\tilde{C}}( \, g \, ; \, \tilde{c}(\mathrm{id}_{w_1}), ..., \tilde{c}(\mathrm{id}_{w_m}) \, ) \\
			& = & \alpha^{\tilde{C}}( \, g \, ; \, \mathrm{id}_{\tilde{c}(w_1)}, ..., \mathrm{id}_{\tilde{c}(w_m)} \, )
		\end{array}
\end{eq*}
But since we know from \cref{c'surj} that $\tilde{c}$ is surjective, this condition will actually suffice as a definition for $\alpha^{\tilde{C}}$, provided that we can prove it to be well-defined. 

To that end, let $w_1, ..., w_m$ and $w'_1, ..., w'_m$ be any two sequences of objects in $\mathbb{G}_{2n}$ that have $\tilde{c}(w_i) = \tilde{c}(w'_i)$ for all $1 \le i \le m$. Then using \cref{c'alg1}, let $w^{(1)}_i, ..., w^{(k)}_i$ and $u^{(1)}_i, ..., u^{(k)}_i$ be the sequences we get from each $w_i$ and $w'^{\, (1)}_i, ..., w'^{\, (k')}_i$, $u'^{\, (1)}_i, ..., u'^{\, (k')}_i$ those we get from $w'_i$. It follows that
\begin{eq*} \begin{array}{rll}
			\tilde{c}\big( \, \alpha^{\mathbb{G}_{2n}}( \, g \, ; \, \mathrm{id}_{w^{(i)}_1}, ..., \mathrm{id}_{w^{(i)}_m} \, ) \, \big) & = & \tilde{c}\big( \, \alpha^{\mathbb{G}_{2n}}( \, g \, ; \, \mathrm{id}_{\tilde{\delta}(u^{(i)}_1)}, ..., \mathrm{id}_{\tilde{\delta}(u^{(i)}_m)} \, ) \, \big)\\
			& = & \tilde{c}\tilde{\delta}\big( \, \alpha^{\mathbb{G}_{2n}}( \, g \, ; \, \mathrm{id}_{u^{(i)}_1}, ..., \mathrm{id}_{u^{(i)}_m} \, ) \, \big)\\
			& = & \tilde{c}\tilde{I}\big( \, \alpha^{\mathbb{G}_{2n}}( \, g \, ; \, \mathrm{id}_{u^{(i)}_1}, ..., \mathrm{id}_{u^{(i)}_m} \, ) \, \big)\\
			& = & \tilde{c}\big( \, \alpha^{\mathbb{G}_{2n}}( \, g \, ; \, \mathrm{id}_{\tilde{I}(u^{(i)}_1)}, ..., \mathrm{id}_{\tilde{I}(u^{(i)}_m)} \, ) \, \big)\\
			& = & \tilde{c}\big( \, \alpha^{\mathbb{G}_{2n}}( \, g \, ; \, \mathrm{id}_{w^{(i+1)}_1}, ..., \mathrm{id}_{w^{(i+1)}_m} \, ) \, \big)
		\end{array} 
\end{eq*}
and likewise for the $w'$. Thus from \cref{c'alg2} we can conclude that
\begin{eq*} \begin{array}{rll}
			\tilde{c}\big( \, \alpha^{\mathbb{G}_{2n}}( \, g \, ; \, \mathrm{id}_{w_1}, ..., \mathrm{id}_{w_m} \, ) \, \big)& = & \tilde{c}\big( \, \alpha^{\mathbb{G}_{2n}}( \, g \, ; \, \mathrm{id}_{w^{(1)}_1}, ..., \mathrm{id}_{w^{(1)}_m} \, ) \, \big)\\
			& = & \tilde{c}\big( \, \alpha^{\mathbb{G}_{2n}}( \, g \, ; \, \mathrm{id}_{w^{(2)}_1}, ..., \mathrm{id}_{w^{(2)}_m} \, )\, \big) \\
			& \vdots & \\
			& = & \tilde{c}\big( \, \alpha^{\mathbb{G}_{2n}}( \, g \, ; \, \mathrm{id}_{w^{(k)}_1}, ..., \mathrm{id}_{w^{(k)}_m} \, ) \, \big)\\
			& = & \tilde{c}\big( \, \alpha^{\mathbb{G}_{2n}}( \, g \, ; \, \mathrm{id}_{w'^{\, (k')}_1}, ..., \mathrm{id}_{w'^{(\, k')}_m} \, ) \, \big)\\
			& \vdots & \\
			& = & \tilde{c}\big( \, \alpha^{\mathbb{G}_{2n}}( \, g \, ; \, \mathrm{id}_{w'_1}, ..., \mathrm{id}_{w'_m} \, ) \, \big)
		\end{array} 
\end{eq*}
Thus the value of $\alpha^{\tilde{C}}(g; \mathrm{id}_{\tilde{c}(w_1)}, ..., \mathrm{id}_{\tilde{c}(w_m)})$ we gave earlier does not depend on our particular choice of $w_i$. Therefore $\alpha^{\tilde{C}}$ is indeed a well-defined $\mathrm{E}G$-action on $\tilde{C}$, and the coequaliser $\tilde{c}$ from $\mathrm{MonCat}$ is a map of $\mathrm{E}G$-algebras with respect to $\alpha^{\tilde{C}}$.
\end{proof}

\section{Extracting $\mathrm{M}(L\mathbb{G}_n)^{\mathrm{gp},\mathrm{ab}}$ from $\mathbb{G}_{2n}$}

We are now finally ready to address problem 1 from the end of the previous chapter: how can we deal with the fact that our adjunction $\mathrm{M}(\, \_ \,)^{\mathrm{gp},\mathrm{ab}} \dashv C$ involves monoidal categories rather than full $\mathrm{E}G$-algebras? It turns out that this is all we really needed, as despite us originally conceiving of $L\mathbb{G}_n$ as a colimit in $\mathrm{E}G\mathrm{Alg}_S$ it can equally be viewed as a slightly more complicated colimit in $\mathrm{MonCat}$.

\begin{prop} \label{c'=q} The coequaliser functor $\tilde{c}: \mathbb{G}_{2n} \to \tilde{C}$ defined in \cref{C'def} is isomorphic as a map of $\mathrm{E}G$-algebras to $q: \mathbb{G}_{2n} \to L\mathbb{G}_n$, the cokernel of $\delta$ in $\mathrm{E}G\mathrm{Alg}_S$.
\end{prop}
\begin{proof}
First, consider what we know of the functor $\tilde{c}$. By definition it has the property for any $1 \le i \le 2n$
\begin{eq*} \tilde{c}\delta(z_i) \quad = \quad \tilde{c} \tilde{\delta}(z_{2n+i}) \quad = \quad \tilde{c} \tilde{I}(z_{2n+i}) \quad = \quad \tilde{c}(I) \quad = \quad I \end{eq*}
so that $\tilde{c} \circ \delta$ is the constant functor on the unit object $I$. Moreover, given what we saw in \cref{c'alg} we know that $\tilde{c}$ is map of $\mathrm{E}G$-algebras which has this property. But the cokernel map $q$ is universal among maps like these, and so it follows that there must exist a unique map of $\mathrm{E}G$-algebras $u: L\mathbb{G}_n \to \tilde{C}$ factoring $\tilde{c}$ through $q$. Conversely, the algebra map $q$ is a monoidal functor for which $q \circ \delta = I$, while $\tilde{c}$ is the universal map in $\mathrm{MonCat}$ with this property. Thus there also exists a unique monoidal functor $v : \tilde{C} \to L\mathbb{G}_n$ which factors $q$ through $\tilde{c}$.

Putting these facts together with the surjectivity of $q$ and $\tilde{c}$ (from \cref{qsurj,c'surj} respectively), we can conclude that the maps $u$ and $u'$ form a isomorphism of monoidal categories:
\begin{eq*} \begin{array}{rccclcrcl}
			u \circ v \circ \tilde{c} & = & u \circ q & = & \tilde{c} & \quad \implies \quad  & u \circ v & = & \mathrm{id}_{\tilde{C}} \\
			v \circ u \circ q & = & v \circ \tilde{c} & = & q & \quad \implies \quad & v \circ u & = & \mathrm{id}_{L\mathbb{G}_n}
		\end{array}
\end{eq*}
Furthermore, not only is $u$ an algebra map, but $v$ is one too. To see this, use the surjectivity of $\tilde{c}$ to find for any morphism $f_i$ in $\tilde{C}$ a corresponding $f'_i$ in $\mathbb{G}_{2n}$ with $\tilde{c}(f'_i) = f_i$. Then
\begin{eq*} \begin{array}{rll}
			v \big( \, \alpha^{\tilde{C}}( \, g \, ; \, f_1, ..., f_m \, ) \, \big) & = & v \big( \, \alpha^{\tilde{C}}( \, g \, ; \,\tilde{c}(f'_1), ..., \tilde{c}(f'_m) \, ) \, \big) \\
			& = & v \tilde{c} \big( \, \alpha^{\mathbb{G}_{2n}}( \, g \, ; \, f'_1, ..., f'_m \, ) \, \big) \\
			& = & q \big( \, \alpha^{\mathbb{G}_{2n}}( \, g \, ; \, f'_1, ..., f'_m \, ) \, \big) \\
			& = & \alpha^{L\mathbb{G}_{n}}\big( \, g \, ; \, q(f'_1), ..., q(f'_m) \, \big) \\
			& = & \alpha^{L\mathbb{G}_{n}}\big( \, g \, ; \, v\tilde{c}(f'_1), ..., v\tilde{c}(f'_m) \, \big) \\
			& = & \alpha^{L\mathbb{G}_{n}}\big( \, g \, ; \, v(f_1), ..., v(f_m) \, \big)
		\end{array}
\end{eq*}
Therefore $(u,v)$ is also an isomorphism of $\mathrm{E}G$-algebras $\tilde{C} \cong L\mathbb{G}_n$, and up to this isomorphism the algebra maps $q$ and $\tilde{c}$ are the same.
\end{proof}

With our newfound ability to express the map $q: \mathbb{G}_{2n} \to L\mathbb{G}_n$ as a colimit of monoidal categories, we can now set about using the adjunction from \cref{Moradj} to calculate $\mathrm{M}(L\mathbb{G}_n)^{\mathrm{gp},\mathrm{ab}}$. The most obvious way to do this is to mimic what we did in \cref{Qobj} --- apply the left adjoint functor to $q$ and then commute it with the colimit to get a formula in terms of the known monoid $\mathrm{Mor}(\mathbb{G}_{2n})$.

\begin{prop}\label{Zmor2} Let $\Delta$ be the subgroup of $\mathrm{M}(\mathbb{G}_{2n})^{\mathrm{gp, ab}}$ generated by elements of the form
\begin{eq*} \mathrm{M}(\tilde{\delta})^{\mathrm{gp, ab}}(f) \, \otimes \, \mathrm{M}(\tilde{I})^{\mathrm{gp, ab}}(f)^*, \quad \quad \quad f \in \mathrm{M}(\mathbb{G}_{4n})^{\mathrm{gp, ab}} \end{eq*}
Then the abelianisation of the group completion of the collapsed morphisms of $L\mathbb{G}_n$ is 
\begin{eq*} \mathrm{M}(L\mathbb{G}_n)^{\mathrm{gp, ab}} \quad = \quad \bigquotient{{\mathrm{M}(\mathbb{G}_{2n})}^{\mathrm{gp, ab}}}{\Delta} \end{eq*}
with $\mathrm{M}(q)^{\mathrm{gp, ab}}$ acting as the appropriate quotient map. 
\end{prop}
\begin{proof}
From \cref{Moradj}, we know that $\mathrm{M}(\, \_ \,)^{\mathrm{gp, ab}}: \mathrm{MonCat} \to \mathrm{Ab}$ is a left adjoint functor. This means that it preserves all colimits in $\mathrm{MonCat}$, including the coequaliser use to define $\tilde{c}$, which from \cref{c'=q} we now know is really $q$.  Thus
\begin{eq*} \mathrm{coeq}\big( \, \mathrm{M}(\tilde{\delta})^{\mathrm{gp, ab}}, \, \mathrm{M}(\tilde{I})^{\mathrm{gp, ab}} \, \big) \quad = \quad \mathrm{M}\big( \, \mathrm{coeq}(\tilde{\delta}, \tilde{I}) \, \big)^{\mathrm{gp, ab}} \quad = \quad \mathrm{M}(q)^{\mathrm{gp, ab}} \end{eq*}
or in other words, the following is a coequaliser diagram in the category of abelian groups:
\begin{eq*} \begin{tikzcd}
\mathrm{M}(\mathbb{G}_{2n})^{\mathrm{gp, ab}} \ar[rrr, shift left, "\mathrm{M}(\tilde{\delta})^{\mathrm{gp, ab}}"] \ar[rrr, shift right, "\mathrm{M}(\tilde{I})^{\mathrm{gp, ab}}"'] & & &
\mathrm{M}(\mathbb{G}_{2n})^{\mathrm{gp, ab}} \ar[rrr, "\mathrm{M}(c)^{\mathrm{gp, ab}}"] & & &
\mathrm{M}(L\mathbb{G}_{n})^{\mathrm{gp, ab}}
\end{tikzcd} \end{eq*} 
But the coequaliser of two abelian group homomorphisms is just the quotient of their common target by the image of their difference. Hence in this case we have
\begin{eq*} \mathrm{M}(L\mathbb{G}_{n})^{\mathrm{gp, ab}} \quad = \quad \bigquotient{{\mathrm{M}(\mathbb{G}_{2n})}^{\mathrm{gp, ab}}}{\mathrm{im}\big( \, {\mathrm{M}(\tilde{\delta})}^{\mathrm{gp, ab}} - {\mathrm{M}(\tilde{I})}^{\mathrm{gp, ab}} \, \big)} \quad = \quad \bigquotient{{\mathrm{M}(\mathbb{G}_{2n})}^{\mathrm{gp, ab}}}{\Delta}  \end{eq*}
\end{proof} 

Notice that the subgroup $\Delta$ contains all elements of the group $\mathrm{im}(\mathrm{M}(\delta)^{\mathrm{gp, ab}})$, but in general these two are not the same subgroup of $\mathrm{M}(\mathbb{G}_{2n})^{\mathrm{gp, ab}}$. This means that the naive approach we could have taken at the end of \cref{surjcoeq} was indeed a mistake, and thus all of the effort we have put into circumventing it has been worthwhile.  

Now, at some point later on we will actually want to evaluate the quotient in \cref{Zmor2} for particular values of action operad $G$. This would be fairly tricky without an explicit description of the elements of $\Delta$, so we need to take a moment to think about what we really mean when we say $\mathrm{M}(\tilde{\delta})^{\mathrm{gp, ab}}(f) \otimes \mathrm{M}(\tilde{I})^{\mathrm{gp, ab}}(f)^*$.

\begin{lem} $\Delta$ is the subgroup of $\mathrm{M}(\mathbb{G}_{2n})^{\mathrm{gp, ab}}$ whose elements are tensor products of equivalence classes
\begin{eq*} \begin{array}{c}
			\big[ \, \alpha^{\mathbb{G}_{2n}}\big( \, \mu( \, g \, ; \, e_{|\tilde{\delta}(x_1)|}, ..., e_{|\tilde{\delta}(x_m)|} \, ) \, ; \, \mathrm{id}_{x'_1}, ...,  \mathrm{id}_{x'_{m'}} \, \big) \, \big] \\
			\, \otimes \, \\
			\big[ \, \alpha^{\mathbb{G}_{2n}}\big( \, \mu( \, g \, ; \, e_{|\tilde{I}(x_1)|}, ..., e_{|\tilde{I}(x_m)|} \, ) \, ; \, \mathrm{id}_{x''_1}, ...,  \mathrm{id}_{x''_{m''}} \, \big) \, \big]^*
		\end{array}
\end{eq*} 
where $g \in G(m)$, the $x_i$ are generators of $\mathbb{N}^{\ast 4n}$, the $x'_i, x''_i$ are generators of $\mathbb{N}^{\ast 2n}$, and
\begin{eq*} \begin{array}{rll}
			\tilde{\delta}( x_1 \otimes ... \otimes x_m) & = & x'_1 \otimes ... \otimes x'_{m'} \\
			\tilde{I}( x_1 \otimes ... \otimes x_m) & = & x''_1 \otimes ... \otimes x''_{m''}
		\end{array}
\end{eq*}
\end{lem}
\begin{proof}  
Let $f$ be an element of $\mathrm{M}(\mathbb{G}_{4n})^{\mathrm{gp, ab}}$. By definition this means that $f$ is an equivalence class of morphisms from $\mathbb{G}_{4n}$, and so by \cref{Gnmapsaction} there must exist $g \in G(m)$ and $x_1, ..., x_m \in \{ z_1, ..., z_{4n} \}$ for which
\begin{eq*} f \quad = \quad [ \, \alpha^{\mathbb{G}_{4n}}(g; \mathrm{id}_{x_1}, ..., \mathrm{id}_{x_m}) \, ] \end{eq*}
Thus
\begin{eq*} \begin{array}{rll}
			\mathrm{M}(\tilde{\delta})^{\mathrm{gp, ab}}(f) & = & \mathrm{M}(\tilde{\delta})^{\mathrm{gp, ab}} \big( \, [ \, \alpha^{\mathbb{G}_{4n}}(g; \mathrm{id}_{x_1}, ..., \mathrm{id}_{x_m}) \, ] \, \big) \\
			& = & \big[ \, \tilde{\delta}\big( \, \alpha^{\mathbb{G}_{4n}}(g; \mathrm{id}_{x_1}, ..., \mathrm{id}_{x_m}) \, \big) \, \big]  \\
			& = & [ \, \alpha^{\mathbb{G}_{2n}}(g; \mathrm{id}_{\tilde{\delta}(x_1)}, ..., \mathrm{id}_{\tilde{\delta}(x_m)}) \, ]
		\end{array}
\end{eq*}
But again using \cref{Gnmapsaction}, we know it must be possible to express the action morphism $\alpha^{\mathbb{G}_{2n}}(g; \mathrm{id}_{\tilde{\delta}(x_1)}, ..., \mathrm{id}_{\tilde{\delta}(x_m)})$ as an action morphism on the identities of generators. Since the source of this map is
\begin{eq*} \tilde{\delta}(x_1) \otimes ... \otimes \tilde{\delta}(x_m) \quad = \quad \tilde{\delta}(x_1 \otimes ... \otimes x_m) \quad =: \quad x'_1 \otimes ... \otimes x'_{m'}  \end{eq*}
clearly the $x'_i$ are the generators we want, and so by expanding the $\tilde{\delta}(x_i)$ as tensor products of these we find that
\begin{eq*} [ \, \alpha^{\mathbb{G}_{2n}}(g; \mathrm{id}_{\tilde{\delta}(x_1)}, ..., \mathrm{id}_{\tilde{\delta}(x_m)})  \, ] \quad = \quad \big[ \, \alpha^{\mathbb{G}_{2n}}\big( \, \mu( \, g \, ; \, e_{|\tilde{\delta}(x_1)|}, ..., e_{|\tilde{\delta}(x_m)|} \, ) \, ; \, \mathrm{id}_{x'_1}, ..., \mathrm{id}_{x'_{m'}} \, \big) \, \big] \end{eq*}
For analogous reasons we also get
\begin{eq*} \begin{array}{rll}
			\mathrm{M}(\tilde{I})^{\mathrm{gp, ab}}(f) & = & [ \, \alpha^{\mathbb{G}_{2n}}(g; \mathrm{id}_{\tilde{I}(x_1)}, ..., \mathrm{id}_{\tilde{I}(x_m)}) \, ]  \\
			& = &  \big[ \, \alpha^{\mathbb{G}_{2n}}\big( \, \mu( \, g \, ; \, e_{|\tilde{I}(x_1)|}, ..., e_{|\tilde{I}(x_m)|} \, ) \, ; \, \mathrm{id}_{x''_1}, ...,  \mathrm{id}_{x''_{m''}} \, \big) \, \big]
		\end{array}
\end{eq*}
and using these equations the lemma follows immediately from the definition of $\Delta$.
\end{proof} 

\chapter{Morphisms of free invertible algebras}
\label{morphisms}
 
The goal of this chapter will be to show that we can reconstruct all of the morphisms of $L\mathbb{G}_n$ from the abelian group $\mathrm{M}(L\mathbb{G}_n)^{\mathrm{gp, ab}}$, and therefore that we can actually use the adjunction from \cref{Moradj} to help find a description of the free $\mathrm{E}G$-algebra on $n$ invertible objects. 

The first step towards this goal will involve splitting $\mathrm{Mor}(L\mathbb{G}_n)$ up as the product of two other monoids. The first of these will encode all of the possible combinations of source and target data for morphisms in $L\mathbb{G}_n$, while the second will just be the endomorphisms of the unit object, $L\mathbb{G}_n(I, I)$. In other words, we will see that the monoid $\mathrm{Mor}(L\mathbb{G}_n)$ can be broken down into a context where source and target are the only thing that matter, and another where they are irrelevant. 

Once we have done this, we can then use the fact that $L\mathbb{G}_n(I, I)$ is always an abelian group to rewrite $\mathrm{Mor}(L\mathbb{G}_n)$ in terms of its abelian group completion, $\mathrm{Mor}(L\mathbb{G}_n)^{\mathrm{gp, ab}}$. This is not quite the same thing as $\mathrm{M}(L\mathbb{G}_n)^{\mathrm{gp, ab}}$, but they are close enough that we can find a simple equation linking the two, which will in turn allow us to frame the former in terms of the quotient of $\mathrm{M}(\mathbb{G}_{2n})^{\mathrm{gp, ab}}$ we described last chapter. All together, this will constitute an expression for $\mathrm{Mor}(L\mathbb{G}_n)$ that is built up of pieces which we know how to calculate.

\section{Sources and targets in $L\mathbb{G}_n$}   

To get things started, we will spend this section considering the source and target information of morphisms in $L\mathbb{G}_n$. 

\begin{defn}\label{st} For any $\mathrm{E}G$-algebra $X$, denote by $s: \mathrm{Mor}(X) \to \mathrm{Ob}(X)$ and $t: \mathrm{Mor}(X) \to \mathrm{Ob}(X)$ the monoid homomorphisms which send each morphism of $X$ to its source and target, respectively. That is,
\begin{eq*} s( \, f: x \to y) \, = \, x, \quad \quad t( \, f: x \to y) \, = \, y \end{eq*}
\end{defn}

If we use the universal property of products, we can combine these source and target homomorphisms into a single map, $s \times t: \mathrm{Mor}(X) \to \mathrm{Ob}(X) \times \mathrm{Ob}(X)$. The monoid we are interested in finding is the image $L\mathbb{G}_n$ under its instance of this map, which can be described using a pullback as follows:

\begin{lem}\label{stmon} Let $X$ be an $\mathrm{E}G$-algebra, and $s \times t: \mathrm{Mor}(X) \to \mathrm{Ob}(X)^2$ the map built from $s$ and $t$ using the universal property of products. Then the image of this map is
\begin{eq*} (s \times t)(X) \, = \, \mathrm{Ob}(X) \times_{\pi_0(X)} \mathrm{Ob}(X) \end{eq*}
where this pullback is taken over the canonical maps sending objects of $X$ to their connected components:
\begin{eq*} \begin{tikzcd}
\mathrm{Ob}(X) \times_{\pi_0(X)} \mathrm{Ob}(X) \ar[dd, shift left=12] \ar[rr] \ar[ddrr, phantom, "\lrcorner", near start, shift left=4] & & \mathrm{Ob}(X) \ar[dd, "\lbrack \, \_ \, \rbrack"] & \\ 
& & & \\
\quad \quad \quad \quad \quad \quad \mathrm{Ob}(X) \ar[rr, "\lbrack \, \_ \, \rbrack"] & & \pi_0(X)
\end{tikzcd} \end{eq*}
\end{lem} 
\begin{proof}
By definition, there exists a morphism $f: x \to y$ between objects $x, y$ of $X$ if and only if they are in the same connected component, $[x] = [y]$. Thus
\begin{eq*} \begin{array}{rll}
		(x, y) \, \in \, (s \times t)(X) & \iff & \exists \, f \, : \quad s(f) \, = \, x, \quad t(f) \, = \, y \\
		& \iff & [x] = [y] \\
		& \iff & (x, y) \, \in \, \mathrm{Ob}(X) \times_{\pi_0(X)} \mathrm{Ob}(X)
		\end{array}
\end{eq*}
as required.
\end{proof}

Recalling \cref{Gnobj,Gnconcomp,Zobj,crossconcomp}, we can immediately conclude the following:

\begin{cor} \label{stpullback}
\begin{eq*} \begin{array}{rll} 
		(s \times t)(\mathbb{G}_n) & = & \begin{cases}
								\quad \mathbb{N}^{\ast n} \times_{\mathbb{N}^n} \mathbb{N}^{\ast n} & \text{if $G$ is crossed}\\
								\quad \mathbb{N}^{\ast n} & \text{otherwise}
							\end{cases} \\
		& & \\
		(s \times t)(L\mathbb{G}_n) & = & \begin{cases}
								\quad \mathbb{Z}^{\ast n} \times_{\mathbb{Z}^n} \mathbb{Z}^{\ast n}  & \text{if $G$ is crossed}\\
								\quad \mathbb{Z}^{\ast n} & \text{otherwise}
							\end{cases} \\
		\end{array}
\end{eq*}
where the pullbacks are taken over the quotients of abelianisation for $(\mathbb{N}^{\ast n})^{\mathrm{ab}} = \mathbb{N}^n$ and $(\mathbb{Z}^{\ast n})^{\mathrm{ab}} = \mathbb{Z}^n$ respectively.
\end{cor}

Next, we want to show that this $(s \times t)(L\mathbb{G}_n)$ we have described is in fact a submonoid of $\mathrm{Mor}(L\mathbb{G}_n)$. This is a little tricky though, since we don't currently know what the morphisms of $L\mathbb{G}_n$ even are. We will sidestep this problem by first proving the analogous statement for all $\mathbb{G}_n$, and then recovering the $L\mathbb{G}_n$ version from it later.

Now, by \cref{Gnmor} we know that wanting $(s \times t)(\mathbb{G}_n)$ to be a submonoid of $\mathrm{Mor}(\mathbb{G}_n)$ is the same as asking if we can find an injective homomorphism $\mathbb{N}^{\ast n} \times_{\mathbb{N}^n} \mathbb{N}^{\ast n} \to G \times_{\mathbb{N}} \mathbb{N}^{\ast n}$, assuming $G$ is crossed, or $\mathbb{N}^{\ast n} \to G \times_{\mathbb{N}} \mathbb{N}^{\ast n}$ if it is not. The latter case is pretty obvious, so we'll focus on crossed $G$ for the moment. Creating a injective \emph{function} $\mathbb{N}^{\ast n} \times_{\mathbb{N}^n} \mathbb{N}^{\ast n} \to G \times_{\mathbb{N}} \mathbb{N}^{\ast n}$ is not especially hard. For any pair $(w, w') \in \mathbb{N}^{\ast n} \times_{\mathbb{N}^n} \mathbb{N}^{\ast n}$, the image of $w$ and $w'$ in the abelian group $\mathbb{N}^n$ is the same, which is to say that if $x_1, ..., x_m \in \{z_1, ..., z_n\}$ are the collection of generators for which 
\begin{eq*} w \quad = \quad x_1 \otimes ... \otimes x_m \end{eq*}
and there exists at least one permutation $\sigma \in S_m$ such that
\begin{eq*} w' \quad = \quad x_{\sigma(1)} \otimes ... \otimes x_{\sigma(m)} \end{eq*}
Then since the underlying permutation maps $\pi : G(m) \to \mathrm{S}_m$ of a crossed action operad $G$ are all surjective, we can always find an element of $g \in G(m)$ for which $\pi(g) = \sigma$. Thus in order to make our injective function all we need to do is make a choice $g =: \rho(w, w')$ like this to represent each $(w, w')$, and then set
\begin{eq*} \begin{array}{rll}
			\mathbb{N}^{\ast n} \times_{\mathbb{N}^n} \mathbb{N}^{\ast n} & \to & G \times_{\mathbb{N}} \mathbb{N}^{\ast n} \\
			(w, w') & \mapsto & ( \, \rho(w, w'), w \, )
		\end{array}
\end{eq*}
Injectivity follows because given a specific $( g, w )$, the only element that could map onto it is $(w, \pi(g)(w))$. 

So how do we know if we can choose these representatives $\rho(w, w')$ in such a way that the resulting function $i$ is also a monoid homomorphism? If we could find a presentation of $\mathbb{N}^{\ast n} \times_{\mathbb{N}^n} \mathbb{N}^{\ast n}$ in terms of generators and relations then this would help a little, since we would only need to pick a $\rho(z, z')$ for each generator $(z, z')$, and then define all other $\rho(w, w')$ by way of tensor products:
\begin{eq*} \rho(v \otimes w, v' \otimes w') \quad = \quad \rho(v, v') \otimes \rho(w, w') \end{eq*}
But then we would still need make sure that our choice of $\rho(z, z')$ obeyed the necessary relations on the generators of $\mathbb{N}^{\ast n} \times_{\mathbb{N}^n} \mathbb{N}^{\ast n}$. Luckily for us though, this turns out to be no problem at all. 

\begin{prop}\label{freemon} $\mathbb{N}^{\ast n} \times_{\mathbb{N}^n} \mathbb{N}^{\ast n}$ is a free monoid.
\end{prop}
\begin{proof}
Given an element $(w, w')$ of the monoid $\mathbb{N}^{\ast n} \times_{\mathbb{N}^n} \mathbb{N}^{\ast n}$, let $D(w, w')$ be the following set:
\begin{eq*} D(w, w') \, = \, \left\{ \begin{array}{rlrll}
							& & (w, w') & = & (u, u') \otimes (v, v'), \\
							(u, u'), (v, v') \in \mathbb{N}^{\ast n} \times_{\mathbb{N}^n} \mathbb{N}^{\ast n} & : & (u, u') & \neq & (I, I), \\
							& & (v, v') & \neq & (I,I)
					\end{array} \right\} 
\end{eq*}
We can use these sets to recursively define a decomposition of any element $(w, w')$ as a product of other elements of $\mathbb{N}^{\ast n} \times_{\mathbb{N}^n} \mathbb{N}^{\ast n}$. Specifically, if $D(w, w')$ is empty then we say that the decomposition of $(w, w')$ is just $(w, w')$ itself, and otherwise we choose any $\big( \, (u, u'), (v, v') \, \big) \in D(w, w')$ and say that the decomposition of $(w, w')$ is the concatenation of the decomposition of $(u, u')$ with the decomposition of $(v, v')$. Note that when we look at the lengths of these elements, as defined in \cref{lengthdef}, $|u|$ and $|v|$ are always strictly smaller that $|w|$, and any strictly decreasing sequence of natural numbers is finite, so this process definitely terminates.,

Of course, we need to check that this decomposition of $(w, w')$ is well-defined, which amounts to checking that the choice of $(u, u'), (v, v')$ we make at each stage won't change the eventual output. To that end, suppose for the sake of contradiction that $(u_1, u'_1), ..., (u_m, u'_m)$ and $(v_1, v'_1), ..., (v_m', v'_{m'})$ are distinct decompositions of $(w, w')$ we could arrive at using the above process. Notice that we can assume without loss of generality that $|u_1| < |v_1|$. If instead $|u_1| > |w_1|$, we can just swap the labels of the sequences, and if $|u_1| = |v_1|$ then we can just discard those elements and  instead consider the decompositions $(u_2, u'_2), ..., (u_m, u'_m)$ and $(v_2, v'_2), ..., (v_m', v'_{m'})$ of $(u_2, u'_2) \otimes ... \otimes (u_m, u'_m) = (v_2, v'_2) \otimes ... \otimes (v_m', v'_{m'})$. Since $(u_1, u'_1), ..., (u_m, u'_m)$ and $(v_1, v'_1), ..., (v_m', v'_{m'})$ were distinct decompositions of $(w, w')$, in this way we will eventually reach some subsequences whose first elements are different; once we have, we can relabel them so that $|u_1| < |v_1|$. Then by definition,
\begin{eq*} u_1 \otimes \big( \, \bigotimes_{i=2}^m u_i \, ) \, = \, w \, = \, v_1 \otimes \big( \, \bigotimes_{i=2}^{m'} v_i \, )\end{eq*}
But $w, u_1, v_1, \bigotimes_{i=2}^m u_i, \bigotimes_{i=2}^{m'} v_i$ are all elements of $\mathbb{N}^{\ast n}$, which is a free monoid, and so they each have a unique decomposition as products of the generators $\{ z_1, ..., z_n \}$, and these all respect tensor products. Therefore, since $|u_1| < |v_1|$, there must exist some element $a$ of $\mathbb{N}^{\ast n}$ such that
\begin{eq*} w \, = \, u_1 \otimes a \otimes \big( \, \bigotimes_{i=2}^{m'} v_i \, )  \quad \implies \quad v_1 \, = \, u_1 \otimes a \end{eq*}
Since
\begin{eq*} |u'_1| \, = \, |u_1| \, < \, |v_1| \, = \, |v'_1| \end{eq*}
we can also use exactly the same reasoning to find an $a'$ in $\mathbb{N}^{\ast n}$ with $v'_1 = u'_1 \otimes a'$, and hence $(v_1, v'_1) = (u_1, u'_1) \otimes (a, a')$. Moreover, this $(a, a')$ is an element of $\mathbb{N}^{\ast n} \times_{\mathbb{N}^n} \mathbb{N}^{\ast n}$, because
\begin{eq*}\begin{array}{rrcccl}
			& v_1 & = & u_1 \otimes a & & \\
			\implies \quad & [v_1] & = & [u_1 \otimes a] & = & [u_1] + [a] \\
			& & & & & \\
			& v'_1 & = & u'_1 \otimes a' & & \\
			\implies \quad & [v'_1] & = & [u'_1 \otimes a'] & = & [u'_1] + [a'] \\
			& & & & & \\
			\implies \quad & [a] & = & [v_1] - [u_1] & & \\
			& & & [v'_1] - [u'_1] & = & [a']
		\end{array}
\end{eq*}
In other words, we have shown that the pair $\big( \, (u_1, u'_1) (a, a') \, \big)$ is an element of $D(v_1, v'_1)$. But by assumption $(v_1, v'_1), ..., (v_m', v'_{m'})$ was a decomposition of $(w, w')$, and hence the $D(v_i, v'_i)$ were supposed to be empty for each $i$, since that is when the decomposition finding process terminates. This is a contradiction, and hence our assumption that $(u_1, u'_1), ..., (u_m, u'_m)$ and $(v_1, v'_1), ..., (v_m', v'_{m'})$ were distinct decompositions of $(w, w')$ is false. Therefore, each $(w, w')$ in $\mathbb{N}^{\ast n} \times_{\mathbb{N}^n} \mathbb{N}^{\ast n}$ has a unique decomposition in terms of elements $(v_i, v'_i)$ for which $D(v_i, v'_i)$ is empty, and so $\mathbb{N}^{\ast n} \times_{\mathbb{N}^n} \mathbb{N}^{\ast n}$ is the free monoid whose generators are all such elements.
\end{proof}

It follows immediately from this that our earlier construction of an injective function $\mathbb{N}^{\ast n} \times_{\mathbb{N}^n} \mathbb{N}^{\ast n} \to G \times_{\mathbb{N}} \mathbb{N}^{\ast n}$ can always be extended to be an inclusion of monoids.

\begin{prop} \label{stGnsub} $(s \times t)(\mathbb{G}_n)$ is (isomorphic to) a submonoid of $\mathrm{Mor}(\mathbb{G}_n)$
\end{prop}
\begin{proof}
First, assume that the action operad $G$ is non-crossed. Then there exists an obvious injective monoid homomorphism
\begin{eq*} \begin{array}{rlrll}
			i & : & (s \times t)(\mathbb{G}_n) & \to & \mathrm{Mor}(\mathbb{G}_n) \\
			& : & \mathbb{N}^{\ast n} & \to & G \times_{\mathbb{N}} \mathbb{N}^{\ast n} \\
			& : & w & \mapsto & ( \, e_{|w|}, w \, )
		\end{array}
\end{eq*}
The homomorphism property follows from the fact that the length $|w|$ defined in \cref{lengthdef} is itself a homomorphism, so $|w \otimes w'| = |w|+|w'|$. Thus $(s \times t)(\mathbb{G}_n) \subseteq \mathrm{Mor}(\mathbb{G}_n)$ for non-crossed $G$.

Now assume that $G$ is crossed. For each generator $(z, z')$ of $\mathbb{N}^{\ast n} \times_{\mathbb{N}^n} \mathbb{N}^{\ast n}$, the words $z, z' \in \mathbb{N}^{\ast n}$ are permutations of each other, and the map $\pi : G(|z|) \to \mathrm{S}_{|z|}$ is surjective, and so there must be some $g \in G(|z|)$ with the property that $\pi(g)(z) = z'$. Choose from among these a representative element, which we'll call $\rho(z, z')$. Then because $\mathbb{N}^{\ast n} \times_{\mathbb{N}^n} \mathbb{N}^{\ast n}$ is a free monoid  by \cref{freemon}, these choices will extend to a well-defined, monoid homomorphism
\begin{eq*} \rho \, : \, \mathbb{N}^{\ast n} \times_{\mathbb{N}^n} \mathbb{N}^{\ast n} \longrightarrow G \end{eq*}
which preserves underlying permutation. This map will now possess the property that
\begin{eq*} \pi(\rho(w, w'))(w) \quad = \quad w' \end{eq*}
for any $(w, w') \in\mathbb{N}^{\ast n} \times_{\mathbb{N}^n} \mathbb{N}^{\ast n}$, not just the generators. Then from $\rho$ we'll define the homomorphism $i$ to be
\begin{eq*} \begin{array}{rlrll}
			i & : & (s \times t)(\mathbb{G}_n) & \to & \mathrm{Mor}(\mathbb{G}_n) \\
			& : & \mathbb{N}^{\ast n} \times_{\mathbb{N}^n} \mathbb{N}^{\ast n} & \to & G \times_{\mathbb{N}} \mathbb{N}^{\ast n} \\
			& : & (w, w') & \mapsto & ( \, \rho(w, w'), w \, )
		\end{array}
\end{eq*}
Moreover, for any two elements $(v, v')$, $(w, w')$ of $\mathbb{N}^{\ast n} \times_{\mathbb{N}^n} \mathbb{N}^{\ast n}$ we'll have
\begin{eq*} \begin{array}{rclcl}
		( \, \rho(v, v'), v \, ) & = & ( \, \rho(w, w'), w \, ) & \implies & \rho(v, v') \, = \, \rho(w, w'), \quad \quad v \, = \, w \\
		& & & \implies & v' \, = \, \pi(\rho(v, v'))(v) \, = \, \pi(\rho(w, w'))(w) \, = \, w'
		\end{array}
\end{eq*}
and thus $i$ is injective. Therefore the image of this $i$ is a submonoid of $G \times_{\mathbb{N}} \mathbb{N}^{\ast n}$ which is isomorphic to $\mathbb{N}^{\ast n} \times_{\mathbb{N}^n} \mathbb{N}^{\ast n}$, so again $(s \times t)(\mathbb{G}_n) \subseteq \mathrm{Mor}(\mathbb{G}_n)$ as required.
\end{proof}

In other words, this result says that the source and target data of $\mathbb{G}_n$ is isomorphic to the monoid made up of action morphisms
\begin{eq*} \alpha\big( \, \rho(x_1 \otimes ... \otimes x_m, x_{\sigma(1)} \otimes ... \otimes x_{\sigma(1)}) \, ; \, \mathrm{id}_{x_1}, ..., \mathrm{id}_{x_m} \, \big) \end{eq*}
when $G$ is crossed, and
\begin{eq*} \alpha(e_m; \mathrm{id}_{x_1}, ..., \mathrm{id}_{x_m}) \quad = \quad \mathrm{id}_{x_1 \otimes ... \otimes x_m} \end{eq*}
otherwise, for $\sigma \in S_m$, $x_1, ..., x_m \in \{z_1, ..., z_n\}$. Now, in theory the map $\rho : {\mathbb{N}^{\ast n} \times_{\mathbb{N}^n}} \mathbb{N}^{\ast n} \to G$ that we use to choose representatives can be any valid homomorphism between those monoids for which
\begin{eq*} \pi(\rho(w, w'))(w) \quad = \quad w' \end{eq*}
 but later on we will be able to make things easier on ourselves by making a more systematic choice.

So now we have shown that $(s \times t)(\mathbb{G}_n)$ is a submonoid of $\mathrm{Mor}(\mathbb{G}_n)$, but what we were really interested in is whether or not $(s \times t)(L\mathbb{G}_n)$ is a submonoid of $\mathrm{Mor}(L\mathbb{G}_n)$. To recover the latter result from the former, we will use our cokernel map $q: \mathbb{G}_{2n} \to L\mathbb{G}_n$. In particular, the surjectivity of $q$ combined with the case $(s \times t)(\mathbb{G}_{2n}) \subseteq \mathrm{Mor}(\mathbb{G}_{2n})$ from \cref{stGnsub} immediately gives us what we need.

\begin{cor} \label{stZsub} $(s \times t)(L\mathbb{G}_n)$ is (isomorphic to) a submonoid of $\mathrm{Mor}(L\mathbb{G}_n)$
\end{cor}
\begin{proof}
Let $i: (s \times t)(\mathbb{G}_{2n}) \hookrightarrow \mathrm{Mor}(\mathbb{G}_{2n})$ be an inclusion which allows us to view $(s \times t)(\mathbb{G}_{2n})$ as a submonoid of $\mathrm{Mor}(\mathbb{G}_{2n})$, as in \cref{stGnsub}. Also, let $\mathrm{Mor}(q): \mathrm{Mor}(\mathbb{G}_{2n}) \to \mathrm{Mor}(L\mathbb{G}_n)$ the restriction of the cokernel map $q: \mathbb{G}_{2n} \to L\mathbb{G}_n$ onto morphisms. Then the image of the composite of these two homomorphisms,
\begin{eq*} \mathrm{im}\big( \, \mathrm{Mor}(q) \circ i \, \big) \quad = \quad q\big( \, \mathrm{im}(i) \, \big) \quad \cong \quad q\big( \, (s \times t)(\mathbb{G}_{2n}) \, \big)\end{eq*}
is clearly a submonoid of $\mathrm{Mor}(L\mathbb{G}_n)$. 

But by \cref{qsurj} $q$ is a surjective functor. This means that there can exist a morphism $w \to v$ in $L\mathbb{G}_n$ if and only if there exists at least one morphism $w' \to v'$ in $\mathbb{G}_{2n}$, for some $w', v'$ which have $q(w') = w$ and $q(v') = v$. In other words,
\begin{eq*} q\big( \, (s \times t)(\mathbb{G}_{2n}) \, \big) \, = \, (s \times t)(L\mathbb{G}_n) \end{eq*}
and therefore the monoid $\mathrm{im}\big( \, \mathrm{Mor}(q) \circ i \, \big)$ that we saw above is really a submonoid of $\mathrm{Mor}(L\mathbb{G}_n)$ isomorphic to $(s \times t)(L\mathbb{G}_n)$, as required.
\end{proof} 

\section{Unit endomorphisms of $L\mathbb{G}_n$}

To help us understand $\mathrm{Mor}(L\mathbb{G}_n)$, we decided to break it down into two smaller pieces. The first of these was the source/target data $(s \times t)(L\mathbb{G}_n)$, which we explored in the previous section. The other piece that we now have to consider is the monoid of unit endomorphisms, $L\mathbb{G}_n(I,I)$. 

This is a particularly important submonoid of the morphisms $\mathrm{Mor}(L\mathbb{G}_n)$, since it is the only submonoid which is also a homset of the category $L\mathbb{G}_n$. Moreover, because the maps in $L\mathbb{G}_n(I,I)$ all share the same source and target, what we have is not just a monoid under tensor product but also under composition as well. This fact leads to a series of special properties for $L\mathbb{G}_n(I,I)$, the first of which is just another instance of the classic Eckmann-Hilton argument.

\begin{lem} \label{endcom} $L\mathbb{G}_n(I,I)$ is a commutative monoid under both tensor product and composition, with $f \otimes f' = f \circ f'$.
\end{lem}
\begin{proof}
Let $f, f'$ be arbitrary elements of the monoid $L\mathbb{G}_n(I,I)$. Since both of these are morphisms in the monoidal category $L\mathbb{G}_n$, we can use the law of interchange to show that
\begin{eq*} \begin{array}{rll}
			f \otimes f' & = & (f \circ \mathrm{id}_I) \otimes (\mathrm{id}_I \circ f') \\
			& = & (f \otimes \mathrm{id}_I) \circ (\mathrm{id}_I \otimes f') \\
			& = & f \circ f' \\
			& = & (\mathrm{id}_I \otimes f) \circ (f' \otimes \mathrm{id}_I) \\
			& = & (f' \circ \mathrm{id}_I) \otimes (\mathrm{id}_I \circ f) \\
			& = & f' \otimes f
		\end{array}
\end{eq*}
\end{proof}

In fact, since we already proved that the morphisms of $L\mathbb{G}_n$ are all actions morphisms, we can take this one step further.

\begin{prop} \label{endab} $L\mathbb{G}_n(I,I)$ is an abelian group.
\end{prop}
\begin{proof}
From \cref{allmapsaction} we know that every morphism $f$ in $L\mathbb{G}_n$ is of the form $\alpha(g; \mathrm{id}_{x_1}, ..., \mathrm{id}_{x_m})$, for some $g \in G(m)$ and $x_i \in \mathbb{Z}^{\ast n}$. It follows immediately that
\begin{eq*} \begin{array}{rl}
			& \alpha( \, g \, ; \, \mathrm{id}_{x_1}, ..., \mathrm{id}_{x_m} \, ) \circ \alpha( \, g^{-1} \, ; \, \mathrm{id}_{x_{\pi(g^{-1})(1)}}, ..., \mathrm{id}_{x_{\pi(g^{-1})(m)}} \, ) \\
			= & \alpha( \, gg^{-1} \, ; \, \mathrm{id}_{x_{\pi(g^{-1})(1)}}, ..., \mathrm{id}_{x_{\pi(g^{-1})(m)}} \, ) \\
			= & \alpha( \, e_m \, ; \, \mathrm{id}_{x_{\pi(g^{-1})(1)}}, ..., \mathrm{id}_{x_{\pi(g^{-1})(m)}} \, ) \\
			= & \mathrm{id}_{x_{\pi(g^{-1})(1)} \otimes ... \otimes x_{\pi(g^{-1})(m)}} \\
			& \\
			& \alpha( \, g^{-1} \, ; \, \mathrm{id}_{x_{\pi(g^{-1})(1)}}, ..., \mathrm{id}_{x_{\pi(g^{-1})(m)}} \, ) \circ \alpha( \, g \, ; \, \mathrm{id}_{x_1}, ..., \mathrm{id}_{x_m} \, ) \\
			= & \alpha( \, g^{-1}g \, ; \, \mathrm{id}_{x_1}, ..., \mathrm{id}_{x_m} \, ) \\
			= & \alpha( \, e_m \, ; \, \mathrm{id}_{x_1}, ..., \mathrm{id}_{x_m} \, ) \\
			= & \mathrm{id}_{x_1 \otimes ... \otimes x_m}
		\end{array}
\end{eq*}
In other words, every morphism $f: w \to v$ in $L\mathbb{G}_n$ has an inverse under composition, 
\begin{eq*} f^{-1} \quad := \quad \alpha(g^{-1}; \mathrm{id}_{x_{\pi(g^{-1})(1)}}, ..., \mathrm{id}_{x_{\pi(g^{-1})(m)}}) \end{eq*}
But we know from \cref{endcom} that tensor product and composition are the same for endomorphisms of the unit object of $L\mathbb{G}_n$. In particular this means that if some morphism $f: I \to I$ has a compositional inverse $f^{-1}$, then it will also be its monoidal inverse $f^*$. Thus every element of the commutative monoid $L\mathbb{G}_n(I,I)$ is invertible, or in other words $L\mathbb{G}_n(I,I)$ is an abelian group.
\end{proof}

Indeed, by using a slightly broader argument we can extend this result to every morphism of $L\mathbb{G}_n$.

\begin{prop} \label{tensinv} Every morphism $f: w \to v$ in $L\mathbb{G}_n$ has an inverse under tensor product, $f^*: w^* \to v^*$. That is, the monoid $\mathrm{Mor}(L\mathbb{G}_n)$ is actually a group.
\end{prop}
\begin{proof}
For any $f: w \to v$ in $L\mathbb{G}_n$, consider the map $\mathrm{id}_{w^*} \otimes f^{-1} \otimes \mathrm{id}_{v^*}$, where $f^{-1}$ is the compositional inverse of $f$, as in the proof of \cref{endab}. This morphism has source $w^* \otimes v \otimes v^* = w^*$ and target $w^* \otimes w \otimes v^* = v^*$, which allows us to apply the law of interchange to get
\begin{longtable}{RLL}
			f \otimes (\mathrm{id}_{w^*} \otimes f^{-1} \otimes \mathrm{id}_{v^*}) & = & \big( \, f \circ \mathrm{id}_w \, \big) \otimes \big( \, \mathrm{id}_{v^*} \circ  (\mathrm{id}_{w^*} \otimes f^{-1} \otimes \mathrm{id}_{v^*}) \, \big) \\
			& = & \big( \, f \otimes \mathrm{id}_{v^*} \, \big) \circ \big( \, \mathrm{id}_w \otimes (\mathrm{id}_{w^*} \otimes f^{-1} \otimes \mathrm{id}_{v^*}) \, \big) \\
			& = & ( f \otimes \mathrm{id}_{v^*} ) \circ ( f^{-1} \otimes \mathrm{id}_{v^*}) \\
			& = & (f \circ f^{-1}) \otimes (\mathrm{id}_{v^*} \circ \mathrm{id}_{v^*}) \\
			& = & \mathrm{id}_v \otimes \mathrm{id}_{v^*} \\
			& = & \mathrm{id}_I
\end{longtable}
and likewise
\begin{longtable}{RLL}
			(\mathrm{id}_{w^*} \otimes f^{-1} \otimes \mathrm{id}_{v^*}) \otimes f & = & \big( \, (\mathrm{id}_{w^*} \otimes f^{-1} \otimes \mathrm{id}_{v^*}) \circ \mathrm{id}_{w^*} \, \big) \otimes \big( \, \mathrm{id}_v \circ f \, \big) \\
			& = & \big( \, (\mathrm{id}_{w^*} \otimes f^{-1} \otimes \mathrm{id}_{v^*}) \otimes \mathrm{id}_v \, \big) \circ \big( \, \mathrm{id}_{w^*} \otimes f \, \big) \\
			& = & (\mathrm{id}_{w^*} \otimes f^{-1}) \circ (\mathrm{id}_{w^*} \otimes f) \\
			& = & (\mathrm{id}_{w^*} \circ \mathrm{id}_{w^*}) \otimes (f^{-1} \circ f)\\
			& = & \mathrm{id}_{w^*} \otimes \mathrm{id}_w \\
			& = & \mathrm{id}_I
\end{longtable}
In other words, $f^* := \mathrm{id}_{w^*} \otimes f^{-1} \otimes \mathrm{id}_{v^*}$ is the inverse of $f$ in the monoid $\mathrm{Mor}(L\mathbb{G}_n)$, as required.
\end{proof}

So $\mathrm{Mor}(L\mathbb{G}_n)$ and $L\mathbb{G}_n(I,I)$ both turn out to be groups under tensor product. Obviously it follows from this that $L\mathbb{G}_n(I,I)$ is a not just a submonoid of $\mathrm{Mor}(L\mathbb{G}_n)$ but a subgroup --- in particular an abelian subgroup, going by \cref{endab}. But $L\mathbb{G}_n(I,I)$ is actually an even more special subgroup than this.

\begin{prop} \label{endnorm} $L\mathbb{G}_n(I,I)$ is a normal subgroup of $\mathrm{Mor}(L\mathbb{G}_n)$. Moreover, if $G$ is a crossed action operad, then $L\mathbb{G}_n(I,I)$ is a subgroup of the center of $\mathrm{Mor}(L\mathbb{G}_n)$.
\end{prop}
\begin{proof}
From \cref{endab,tensinv}, we know that $L\mathbb{G}_n(I,I)$ is a subgroup of $\mathrm{Mor}(L\mathbb{G}_n)$. For normality, we need to again consider both crossed and non-crossed action operads separately. 

If $G$ is non-crossed, then by \cref{crossconcomp} we know that the map assigning objects of $L\mathbb{G}_n$ to their connected component is just the identity $\mathrm{id}_{\mathbb{Z}^{\ast n}}$. In other words, every object belongs to its own unique component, so that every morphism of $L\mathbb{G}_n$ is actually an endomorphism. It follows that the group $L\mathbb{G}_n(I,I)$ is the kernel of the source homomorphism $s$ from \cref{st} --- or equally the target homomorphism $t$.
\begin{eq*} \begin{tikzcd}
L\mathbb{G}_n(I,I) \ar[r] & \mathrm{Mor}(L\mathbb{G}_n) \ar[r, "s"] & \mathrm{Ob}(L\mathbb{G}_n)
\end{tikzcd} \end{eq*}
The kernel of a group homomorphism is always a normal subgroup of that homomorphism's source, and so in our case we have $L\mathbb{G}_n(I,I) \le \mathrm{Mor}(L\mathbb{G}_n)$. 

For crossed $G$, recall from \cref{spacial} that all crossed $\mathrm{E}G$-algebras are spacial, and so in particular $L\mathbb{G}_n$ is. This means that for any $h \in L\mathbb{G}_n(I,I)$ and $w \in \mathrm{Ob}(L\mathbb{G}_n)$ we will always have $h \otimes \mathrm{id}_w = \mathrm{id}_w \otimes h$. Thus for any $f:w \to v$ in $\mathrm{Mor}(L\mathbb{G}_n)$, we get
\begin{eq*} \begin{array}{rll}
		h \otimes f & = & (\mathrm{id}_I \circ h) \otimes (f \circ \mathrm{id}_w) \\
		& = & (\mathrm{id}_I \otimes f) \circ (h \otimes \mathrm{id}_w) \\
		& = & (f \otimes \mathrm{id}_I) \circ (\mathrm{id}_w \otimes h) \\
		& = & (f \circ \mathrm{id}_w) \otimes (\mathrm{id}_I \circ h) \\
		& = & f \otimes h
		\end{array}
\end{eq*}
That is, $L\mathbb{G}_n(I,I)$ is a subgroup of the centre of $\mathrm{Mor}(L\mathbb{G}_n)$. Then because
\begin{eq*} f \otimes h \otimes f^* \, = \, h \otimes f \otimes f^* \, = \, h \, \in L\mathbb{G}_n(I,I) \end{eq*}
it follows that $L\mathbb{G}_n(I,I)$ is a normal subgroup of $\mathrm{Mor}(L\mathbb{G}_n)$.
\end{proof}

\section{The morphisms of $L\mathbb{G}_n$} 

We have finally described all of the important properties of $(s \times t)(L\mathbb{G}_n)$ and $L\mathbb{G}_n(I,I)$ that we will need. Putting these results together will let us characterize the morphisms of $L\mathbb{G}_n$ as a product of groups, as promised at the beginning of the chapter. Before we do though, it will be worth going over a few well-known pieces of group theory (from for example \cite{grouptheory}) that we will be using in the proof of \cref{morprod}.

\begin{defn} Let $H$, $K$ and $N$ be groups. Then we say that $H$ is a \emph{group extension} of $K$ by $N$ if there exists a short exact sequence
\begin{eq*} \begin{tikzcd}
0 \ar[r] & N \ar[r, hookrightarrow, "i"] & H \ar[r, "p"] & K \ar[r] & 0
\end{tikzcd} \end{eq*}
In other words, $H$ is an extension of $K$ by $N$ whenever we have $K = H/N$. Moreover, if $N$ is a subgroup of the centre of $H$, we say that this is a \emph{central} extension, and if the map $p$ has a right-inverse, $r: K \to H$, $p \circ r = \mathrm{id}_K$, then we say that it is a \emph{split} extension.
\end{defn}

\begin{defn} Let $H$ be a group with subgroup $K$ and normal subgroup $N$. Then we say that $H$ is a \emph{semidirect product} $K \ltimes N$ if the underlying set of $H$ is the same as underlying set of $K \times N$, but multiplication is given by
\begin{eq*} (k,n) \cdot (k',n') \quad = \quad ( \, kk', \, nkn'k^{-1} \, ) \end{eq*}
\end{defn}

\begin{lem} \label{splitex} If $H$ is a split extension of $K$ by $N$ then $H = K \ltimes N$, with $r: K \to H$ acting as the subgroup inclusion. Further, if $H$ is split and central, then $H \cong K \times N$.
\end{lem}
\begin{proof}
Define a group homomorphism $f: H \to K \ltimes N$  by
\begin{eq*} f(h) \quad := \quad \big( \, p(h), h \cdot rp(h)^{-1} \, \big) \end{eq*}
This is a well-defined homomorphism, since
\begin{eq*} \begin{array}{rll}
			f(hh') & = & \big( \, p(hh'), hh' \cdot rp(hh')^{-1} \, \big) \\
			& = & \big( \, p(h) \cdot p(h'), h \cdot h' \cdot rp(h')^{-1} \cdot rp(h)^{-1} \, \big) \\
			& = & \big( \, p(h) \cdot p(h'), h \cdot rp(h)^{-1} \cdot rp(h) \cdot h' \cdot rp(h')^{-1} \cdot rp(h)^{-1} \, \big) \\
			& = & \big( \, p(h'), h \cdot rp(h)^{-1}, p(h) \, \big) \cdot \big( \, h' \cdot rp(h')^{-1} \, \big) \\
			& = & f(h) \cdot f(h')
		\end{array}
\end{eq*}
Next, define another map $f^{-1}: K \times N \to H$ by
\begin{eq*} f^{-1}(k, n) \quad := \quad n \cdot r(k) \end{eq*}
$f^{-1}$ is also well-defined, because
\begin{eq*} \begin{array}{rll}
			f^{-1}\big( \, (k, n) \cdot (k',n') \, \big) & = & f^{-1}\big( \,  \, kk', \, n \cdot r(k) \cdot n' \cdot r(k)^{-1} \, \big) \\
			& = & \big( \, n \cdot r(k) \cdot n' \cdot r(k)^{-1} \, \big) \cdot r(kk') \\
			& = & n \cdot r(k) \cdot n' \cdot r(k)^{-1} \cdot r(k) \cdot r(k') \\
			& = & n \cdot r(k) \cdot n' \cdot r(k') \\
			& = & f^{-1}(k,n) \cdot f^{-1}(k',n')
		\end{array} 
\end{eq*}
and due to the fact that $p: N \hookrightarrow H \to K$ is the zero map, $f$ and $f^{-1}$ are inverses:
\begin{longtable}{RLL}
	f^{-1}f(h) & = & f^{-1}\big( \, p(h), \, h \cdot rp(h)^{-1} \, \big) \\
	& = & \big( \, h \cdot rp(h)^{-1} \, \big) \cdot r\big( \, p(h) \, \big) \\
	& = & h \cdot rp(h)^{-1} \cdot rp(h) \\
	& = & h \\
	& & \\
	ff^{-1}(k,n) & = & f\big( \, n \cdot r(k) \, \big) \\
	& = & \Big( \, p\big( \, n \cdot r(k) \, \big), \, n \cdot r(k) \cdot rp\big( \, n \cdot r(k) \, \big)^{-1} \, \Big) \\
	& = & \big( \, p(n) \cdot pr(k), \, n \cdot r(k) \cdot rpr(k)^{-1} \cdot rp(n)^{-1} \, \big) \\
	& = & \big( \, e \cdot k, \, n \cdot r(k) \cdot r(k)^{-1} \cdot e \, \big) \\
	& = & (k, n)
\end{longtable}
Thus $f$ is an isomorphism of groups $H \cong K \ltimes N$. Also, if $N$ is in the center of $H$ then the multiplication in $K \ltimes N$ becomes
\begin{eq*} \begin{array}{rll}
			(k,n) \cdot (k',n') & = & ( \, kk', nkn'k^{-1} \, ) \\
			& = & ( \, kk', nn'kk^{-1} \, ) \\
			& = & (kk', nn')
		\end{array}
\end{eq*}
and so $H$ really is the direct product of groups $K \times N$.
\end{proof}

With that out of the way, we can now produce an expression for the morphisms of the algebra $L\mathbb{G}_n$.
 
\begin{prop} \label{morprod} For any action operad $G$,
\begin{eq*} \mathrm{Mor}(L\mathbb{G}_n) \quad \cong \quad (s \times t)(L\mathbb{G}_n) \ltimes L\mathbb{G}_n(I,I) \end{eq*}
Moreover, if $G$ is a crossed action operad, then
\begin{eq*} \mathrm{Mor}(L\mathbb{G}_n) \quad \cong \quad (s \times t)(L\mathbb{G}_n) \times L\mathbb{G}_n(I,I) \end{eq*}
\end{prop}
\begin{proof}
We just saw in \cref{endnorm} that $L\mathbb{G}_n(I,I)$ is a normal subgroup of $\mathrm{Mor}(L\mathbb{G}_n)$, so we can consider the quotient group
\begin{eq*} \begin{tikzcd}
L\mathbb{G}_n(I,I) \ar[r, hookrightarrow] & \mathrm{Mor}(L\mathbb{G}_n) \ar[r] & \bigquotient{\mathrm{Mor}(L\mathbb{G}_n)}{L\mathbb{G}_n(I,I)}
\end{tikzcd} \end{eq*}
By the universal property of quotients, the map $\mathrm{Mor}(L\mathbb{G}_n) \to \mathrm{Mor}(L\mathbb{G}_n) / L\mathbb{G}_n(I,I)$ will uniquely factor any homomorphism whose composite with the inclusion $L\mathbb{G}_n(I,I) \hookrightarrow \mathrm{Mor}(L\mathbb{G}_n)$ is the zero map. But our source/target map $s \times t : \mathrm{Mor}(L\mathbb{G}_n) \to (s \times t)(L\mathbb{G}_n)$ is one such homomorphism, since for any $h: I \to I$ clearly $(s \times t)(h) = (I, I)$, which is the identity element in $(s \times t)(L\mathbb{G}_n)$. Therefore there must exist a unique homomorphism $u$ making the triangle below commute:
\begin{eq*} \begin{tikzcd}
\mathrm{Mor}(L\mathbb{G}_n) \ar[dd] \ar[ddrr, "s \times t"] & & \\
& & \\
\bigquotient{\mathrm{Mor}(L\mathbb{G}_n)}{L\mathbb{G}_n(I,I)} \ar[rr, "u"] & & (s \times t)(L\mathbb{G}_n)
\end{tikzcd} \end{eq*}
This map $u$ will be surjective --- because $s \times t$ is --- but in fact it will also be injective. This is because if two morphisms $f, f'$ of $L\mathbb{G}_n$ have the same source and target, then the map $h = f^* \otimes f'$ is an element of $L\mathbb{G}_n(I,I)$ for which $f \otimes h = f'$, and so clearly $f$ and $f'$ are part of the same equivalence class in $\mathrm{Mor}(L\mathbb{G}_n)/L\mathbb{G}_n(I,I)$. More precisely, 
\begin{eq*} \begin{array}{rclcrcl}
		[f] & \neq & [f'] & \implies & [f]^* \otimes [f'] & \neq & [I] \\
		& & & \implies & [f^* \otimes f'] & \neq & [I] \\
		& & & \implies & f^* \otimes f' & \notin & L\mathbb{G}_n(I,I)
		\end{array}
\end{eq*}
\begin{eq*} \begin{array}{rrcl}
		\implies & (s \times t)(f^* \otimes f') & \neq & (I, I) \\
		\implies & (s \times t)(f)^* \otimes (s \times t)(f') & \neq & (I, I) \\
		\implies & (s \times t)(f) & \neq & (s \times t)(f')
		\end{array}
\end{eq*}
Thus $u$ is bijective, so that
\begin{eq*} \bigquotient{\mathrm{Mor}(L\mathbb{G}_n)}{L\mathbb{G}_n(I,I)} \quad \cong \quad (s \times t)(L\mathbb{G}_n) \end{eq*}
In other words, what have here is a group extension
\begin{eq*} \begin{tikzcd}
0 \ar[r] & L\mathbb{G}_n(I,I) \ar[r, hookrightarrow] & \mathrm{Mor}(L\mathbb{G}_n) \ar[r, "s \times t"] & (s \times t)(L\mathbb{G}_n) \ar[r] & 0
\end{tikzcd} \end{eq*}
But recall from \cref{stZsub} that $(s \times t)(L\mathbb{G}_n)$ is also a submonoid (and hence subgroup) of $\mathrm{Mor}(L\mathbb{G}_n)$, so that we have another map $i: (s \times t)(L\mathbb{G}_n) \to \mathrm{Mor}(L\mathbb{G}_n)$ for which $(s \times t) \circ i = \mathrm{id}_{(s \times t)(L\mathbb{G}_n)}$. That is, the above is a split extension of groups, or equivalently $\mathrm{Mor}(L\mathbb{G}_n)$ is a semi direct product $(s \times t)(L\mathbb{G}_n) \ltimes L\mathbb{G}_n(I,I)$. However, if $G$ is crossed then we also saw in \cref{endnorm} that $L\mathbb{G}_n(I,I)$ is a subgroup of the center of $\mathrm{Mor}(L\mathbb{G}_n)$, and so it will follow that $\mathrm{Mor}(L\mathbb{G}_n)$ is also a central extension of $(s \times t)(L\mathbb{G}_n)$. In that case $\mathrm{Mor}(L\mathbb{G}_n)$ is really just the direct product $(s \times t)(L\mathbb{G}_n) \times L\mathbb{G}_n(I,I)$, as required.
\end{proof}

In certain select cases, \cref{morprod} will actually be sufficient to fully determine $\mathrm{Mor}(L\mathbb{G}_n)$ --- specifically, whenever we know that the unit endomorphisms of $L\mathbb{G}_n$ are trivial. We already know of two examples like this, due to \cref{noscalarcross,noscalarnoncross}.

\begin{cor} \label{trivendo} If $G$ is a crossed action operad with $G(m) = G(0)$ for all $m \in \mathbb{N}$, then
\begin{eq*} \mathrm{Mor}(L\mathbb{G}_n) \quad = \quad (s \times t)(L\mathbb{G}_n) \quad = \quad \mathbb{Z}^{\ast n} \times_{\mathbb{Z}^n} \mathbb{Z}^{\ast n} \end{eq*}
Instead if $G$ is a $G(1)$-generated action operad, then
\begin{eq*} \mathrm{Mor}(L\mathbb{G}_n) \quad = \quad (s \times t)(L\mathbb{G}_n) \quad = \quad \mathrm{Ob}(L\mathbb{G}_n) \quad = \quad \mathbb{Z}^{\ast n} \end{eq*}
\end{cor} 

In the latter case, what this is saying is that every morphism in $L\mathbb{G}_n$ is just the identity element of some object.

But what about for more general $L\mathbb{G}_n$ with nontrivial unit endomorphisms? For crossed $G$, the key insight is that one half of the product in \cref{morprod}, $L\mathbb{G}_n(I, I)$, is always an abelian group. This means that it will remain untouched if we were to abelianise the entire product, thus providing a link between $\mathrm{Mor}(L\mathbb{G}_n)$ before and after abelianisation.

\begin{prop}\label{Zmor1} Let $G$ be a crossed action operad. Then the endomorphisms of the unit object of $L\mathbb{G}_n$ are
\begin{eq*} L\mathbb{G}_n(I, I) \quad = \quad \bigquotient{{\mathrm{Mor}(L\mathbb{G}_n)}^{\mathrm{ab}}}{{(s \times t)(L\mathbb{G}_n)}^{\mathrm{ab}}} \end{eq*}
and therefore
\begin{eq*} \mathrm{Mor}(L\mathbb{G}_n) \quad = \quad (s \times t)(L\mathbb{G}_n) \, \times \, \bigquotient{{\mathrm{Mor}(L\mathbb{G}_n)}^{\mathrm{ab}}}{{(s \times t)(L\mathbb{G}_n)}^{\mathrm{ab}}} \end{eq*}
\end{prop}
\begin{proof}
From \cref{morprod}, we know that
\begin{eq*} \mathrm{Mor}(L\mathbb{G}_n) \quad = \quad (s \times t)(L\mathbb{G}_n) \times L\mathbb{G}_n(I, I) \end{eq*}
Abelianising both sides of this equation, we get
\begin{eq*} \begin{array}{rll}
			{\mathrm{Mor}(L\mathbb{G}_n)}^{\mathrm{ab}} & = & \big( \, (s \times t)(L\mathbb{G}_n) \times L\mathbb{G}_n(I, I) \, \big)^{\mathrm{ab}} \\
			& = & {(s \times t)(L\mathbb{G}_n)}^{\mathrm{ab}} \times {L\mathbb{G}_n(I, I)}^{\mathrm{ab}} \\
			& = & {(s \times t)(L\mathbb{G}_n)}^{\mathrm{ab}} \times L\mathbb{G}_n(I, I) \\
		\end{array}
\end{eq*} 
since $L\mathbb{G}_n(I, I)$ is already abelian. Now, there is an obvious inclusion ${(s \times t)(L\mathbb{G}_n)}^{\mathrm{ab}} \hookrightarrow (s \times t)(L\mathbb{G}_n)^{\mathrm{ab}} \times L\mathbb{G}_n(I, I)$, and since everything here is abelian, all subgroups are normal subgroups. Thus we can take the quotient of the above equation by this map, to obtain 
\begin{eq*} L\mathbb{G}_n(I, I) \quad = \quad \bigquotient{{\mathrm{Mor}(L\mathbb{G}_n)}^{\mathrm{ab}}}{{(s \times t)(L\mathbb{G}_n)}^{\mathrm{ab}}} \end{eq*}
Finally, we can now substitute this expression back into our original equation, giving
\begin{eq*} \mathrm{Mor}(L\mathbb{G}_n) \quad = \quad (s \times t)(L\mathbb{G}_n) \, \times \, \bigquotient{{\mathrm{Mor}(L\mathbb{G}_n)}^{\mathrm{ab}}}{{(s \times t)(L\mathbb{G}_n)}^{\mathrm{ab}}} \end{eq*}
as required.
\end{proof}

Unfortunately, there is no general version of \cref{Zmor1} for when $G$ is not crossed. This is because if we try to abelianise the semidirect product from \cref{morprod}, we will arrive at a product of the relevant abelian group, but a new term will also appear indicating the degree to which $L\mathbb{G}_n(I, I)$ and $ (s \times t)(L\mathbb{G}_n)$ fail to commute.

\begin{lem} If $H$ is semidirect product $K \ltimes N$, then its abelianisation is
\begin{eq*} H^{\mathrm{ab}} \quad = \quad K^{\mathrm{ab}} \, \times \, \bigquotient{N^{\mathrm{ab}}}{[N,K]} \end{eq*}
where $[N,K]$ is the commutator of $N$ with $K$.
\end{lem}

We do not know what the unit endomorphism of $L\mathbb{G}_n$ are yet -- indeed, that's the one thing we are trying to figure out using this abelianisation tactic --- and so this new term $[L\mathbb{G}_n(I, I), (s \times t)(L\mathbb{G}_n)]$ is not something we can understand. The obvious exception to this is when our non-crossed $G$ is $G(1)$-generated, where we do know that $L\mathbb{G}_n(I, I)$ is trivial and so of course $[L\mathbb{G}_n(I, I), (s \times t)(L\mathbb{G}_n)] = (s \times t)(L\mathbb{G}_n)$.

If we stick to working with crossed action operads however, we are now only one step away from a complete expression for $\mathrm{Mor}(L\mathbb{G}_n)$. The last term whose value we do not know is $\mathrm{Mor}(L\mathbb{G}_n)^{\mathrm{gp}, \mathrm{ab}} = \mathrm{Mor}(L\mathbb{G}_n)^{\mathrm{ab}}$, and as one might expect this is related to the value that the algebra takes under the collapsed morphism left adjoint, $\mathrm{M}(L\mathbb{G}_n)^{\mathrm{gp}, \mathrm{ab}} = \mathrm{M}(L\mathbb{G}_n)^{\mathrm{ab}}$

\begin{prop} \label{colquot} Let $X$ be any monoidal category whose objects are morphisms are all invertible under tensor product. Then the group completion of the abelianisation of the collapsed morphisms of $X$ are
\begin{eq*} \mathrm{M}(X)^{\mathrm{ab}} \quad \cong \quad \bigquotient{\mathrm{Mor}(X)^{\mathrm{ab}}}{\mathrm{Ob}(X)^{\mathrm{ab}}} \end{eq*}
where we are viewing $\mathrm{Ob}(X)$ as a subgroup of $\mathrm{Mor}(X)$ under tensor product by using the inclusion
\begin{eq*} \begin{array}{rcl}
			\mathrm{Ob}(X) & \to & \mathrm{Mor}(X) \\
			x & \mapsto & \mathrm{id}_x
		\end{array}
\end{eq*}
\end{prop}
\begin{proof}
Recall \cref{tenscomp}, which says that in any monoidal category with invertible objects,
\begin{eq*} f' \circ f \quad = \quad f' \otimes \mathrm{id}_{y*} \otimes f \end{eq*}
We will proceed by checking what effect this relation in $\mathrm{Mor}(X)$ will have on the two quotients that we are comparing. 

First, consider the canonical homomorphism $\psi: \mathrm{Mor}(X) \to \mathrm{M}(X) \to \mathrm{M}(X)^{\mathrm{ab}}$, where $\mathrm{Mor}(X)$ is being considered as a group under $\otimes$. Also $\mathrm{M}(X)$ is a group rather than just a monoid, since if $f^*$ is the inverse of $f$ under tensor product in $\mathrm{Mor}(X)$, then the equivalence class $\mathrm{M}(f^*)$ is an inverse of $\mathrm{M}(f)$ under the collapsed product of $\mathrm{M}(X)$. Clearly this map obeys the relation $\psi(f' \circ f) = \psi(f' \otimes f)$ for any $f: x \to y$, $f': y \to z$ in $X$, because it passes through $\mathrm{M}(X)$, and so we also have
\begin{eq*} \begin{array}{rll}
			\psi(f' \otimes f) & = & \psi(f' \circ f) \\
			& = & \psi(f' \otimes \mathrm{id}_{y*} \otimes f) \\
			& = & \psi(f') \otimes \psi(\mathrm{id}_{y*}) \otimes \psi(f) \\
			& = & \psi(f') \otimes \psi(f) \otimes \psi(\mathrm{id}_{y*}) \\
			& = & \psi(f' \otimes f) \otimes \psi(\mathrm{id}_{y*}) \\
			& & \\
			\implies \quad \psi(\mathrm{id}_{y*}) & = & e
		\end{array}
\end{eq*}
But since $\psi$ is also a map from $\mathrm{Mor}(X)$ onto an abelian group, we know that it must factor uniquely though some other homomorphism $\mathrm{Mor}(X)^{\mathrm{ab}} \to \mathrm{M}(X)^{\mathrm{ab}}$, which we will call $\psi'$. This map will inherit from $\psi$ the property that
\begin{eq*} \psi'\big( \, \mathrm{ab}(\mathrm{id}_{x}) \, \big) \quad = \quad \psi(\mathrm{id}_{x}) \quad = \quad e \end{eq*}
for all  $x \in \mathrm{Ob}(X)$.

Now let $A$ be an abelian group and $\phi: \mathrm{Mor}(X)^{\mathrm{ab}} \to A$ any homomorphism of groups which satisfies the condition $\phi(\mathrm{ab}(\mathrm{id}_x)) = e$ for all objects $x$. Then
\begin{eq*} \begin{array}{rll}
			\phi\big( \, \mathrm{ab}(f' \circ f) \, \big)  & = & \phi\big( \, \mathrm{ab}(f' \otimes \mathrm{id}_{y*} \otimes f) \, \big) \\
			& = & \phi\big( \, \mathrm{ab}(f') \, \big) \otimes \phi\big( \, \mathrm{ab}(\mathrm{id}_{y*}) \, \big) \otimes \phi\big( \, \mathrm{ab}(f) \, \big) \\
			& = & \phi\big( \, \mathrm{ab}(f') \, \big) \otimes \phi\big( \, \mathrm{ab}(f) \, \big) \\
			& = & \phi\big( \, \mathrm{ab}(f' \otimes f) \, \big)
		\end{array}
\end{eq*}
By \cref{Morder} this is the defining relation for the group $\mathrm{M}(X)^{\mathrm{ab}}$. It follows that for any $\phi$ with $\phi(\mathrm{ab}(\mathrm{id}_x)) = e$, there must exist a unique homomorphism $\mathrm{M}(X)^{\mathrm{ab}} \to A$ which factors $\phi$ through $\psi'$. But this in turn is just the universal property of the quotient $\mathrm{Mor}(X)^{\mathrm{ab}}/\mathrm{Ob}(X)^{\mathrm{ab}}$ in $\mathrm{Ab}$. Since colimits like quotient groups are unique up to isomorphism, we can therefore conclude that
\begin{eq*} \mathrm{M}(X)^{\mathrm{ab}} \quad \cong \quad \bigquotient{\mathrm{Mor}(X)^{\mathrm{ab}}}{\mathrm{Ob}(X)^{\mathrm{ab}}} \end{eq*}
\end{proof}

Now at last we are finished. All that remains for us to do is simply chain together all of our previous results from this chapter into a single description of the group $\mathrm{Mor}(L\mathbb{G}_n)$.

\begin{prop} \label{Zmor} For crossed action operads $G$, the morphism monoid of $L\mathbb{G}_n$ is equal to
\begin{eq*} \mathrm{Mor}(L\mathbb{G}_n) \quad = \quad \mathbb{Z}^{\ast n} \times_{\mathbb{Z}^n} \mathbb{Z}^{\ast n}  \, \times \, \frac{\left(\quotient{{\mathrm{M}(\mathbb{G}_{2n})}^{\mathrm{gp},\mathrm{ab}}}{\Delta}\right)}{\left(\quotient{{(\mathbb{Z}^{\ast n} \times_{\mathbb{Z}^n} \mathbb{Z}^{\ast n})}^{\mathrm{ab}}}{\mathbb{Z}^n}\right)} \end{eq*}
\end{prop}
\begin{proof}
Consider the quotient group
\begin{eq*} L\mathbb{G}_n(I,I) \quad = \quad \bigquotient{{\mathrm{Mor}(L\mathbb{G}_n)}^{\mathrm{ab}}}{{(s \times t)(L\mathbb{G}_n)}^{\mathrm{ab}}} \end{eq*}
This quotient clearly depends on the way that have chosen to see $(s \times t)(L\mathbb{G}_n)$ as a subgroup of of the morphisms $L\mathbb{G}_n$. Recall that back in the proof of \cref{stGnsub}, we used the freeness of the monoid $\mathbb{N}^{\ast n} \times_{\mathbb{N}^n} \mathbb{N}^{\ast n}$ to define a subgroup by choosing values for some function $\rho$ on generators. Since these $\rho(z,z')$ can be whichever element of the appropriate $G(m)$ we want, we can retroactively pick them in a way that makes our current calculations easier. Specifically, if we let $\rho(z_i, z_i) = e_1$ for each generator $z_i$ of $\mathbb{N}^{\ast n}$, then the corresponding element of the subgroup $(s \times t)(L\mathbb{G}_n)$ will be
\begin{eq*} \alpha_{L\mathbb{G}_n}(e_1;z_i) \quad = \quad \mathrm{id}_{z_i}\end{eq*}
Given this choice, clearly the group
\begin{eq*} \mathrm{Ob}(L\mathbb{G}_n) \quad \cong \quad \{ \, \mathrm{id}_x \, ; \, x \in \, \mathrm{Ob}(L\mathbb{G}_n) \}\end{eq*}
will be a subgroup of $(s \times t)(L\mathbb{G}_n)$, and thus $\mathrm{Ob}(L\mathbb{G}_n)^{\mathrm{ab}}$ a normal subgroup of $(s \times t)(L\mathbb{G}_n)^{\mathrm{ab}}$. It follows that
\begin{eq*} \bigquotient{{\mathrm{Mor}(L\mathbb{G}_n)}^{\mathrm{ab}}}{{(s \times t)(L\mathbb{G}_n)}^{\mathrm{ab}}} \quad = \quad \frac{\left(\quotient{{\mathrm{Mor}(L\mathbb{G}_n)}^{\mathrm{ab}}}{\mathrm{Ob}(L\mathbb{G}_n)^{\mathrm{ab}}}\right)}{\left(\quotient{{(s \times t)(L\mathbb{G}_n)}^{\mathrm{ab}}}{\mathrm{Ob}(L\mathbb{G}_n)^{\mathrm{ab}}}\right)} \end{eq*}
Using \cref{colquot} to change the numerator and \cref{Zobj,stpullback} to simplify the denominator, this quotient becomes
\begin{eq*} \bigquotient{{\mathrm{Mor}(L\mathbb{G}_n)}^{\mathrm{ab}}}{{(s \times t)(L\mathbb{G}_n)}^{\mathrm{ab}}} \quad = \quad \frac{\left(\quotient{{\mathrm{M}(\mathbb{G}_{2n})}^{\mathrm{gp},\mathrm{ab}}}{\Delta}\right)}{\left(\quotient{{(\mathbb{Z}^{\ast n} \times_{\mathbb{Z}^n} \mathbb{Z}^{\ast n})}^{\mathrm{ab}}}{\mathbb{Z}^n}\right)} \end{eq*}
But from \cref{Zmor1} we know that
\begin{eq*} \mathrm{Mor}(L\mathbb{G}_n) \quad = \quad (s \times t)(L\mathbb{G}_n) \, \times \, \bigquotient{{\mathrm{Mor}(L\mathbb{G}_n)}^{\mathrm{ab}}}{{(s \times t)(L\mathbb{G}_n)}^{\mathrm{ab}}} \end{eq*}
and together these give the required description of the morphisms of $L\mathbb{G}_n$.
\end{proof}

\section{Abelianising sources and targets}
 
To say that the expression for $\mathrm{Mor}(L\mathbb{G}_n)$ we just found is `complicated' would probably be an understatement. If we are to have any hope of eventually being able to use \cref{Zmor}, we need to work out a more explicit presentation for its quotient part. We'll start by trying to find the value of $(s \times t)(L\mathbb{G}_n)^{\mathrm{ab}}$ for crossed $G$, the abelian group $(\mathbb{Z}^{\ast n} \times_{\mathbb{Z}^n} \mathbb{Z}^{\ast n})^{\mathrm{ab}}$. This will require some careful consideration, since in general limits such as the pullback do not interact with abelianisation in a simple way. What would help is a suitable presentation of $\mathbb{Z}^{\ast n} \times_{\mathbb{Z}^n} \mathbb{Z}^{\ast n}$ in terms of some generators and relations. 

\begin{prop} \label{pushpres} The pullback group $\mathbb{Z}^{\ast n} \times_{\mathbb{Z}^n} \mathbb{Z}^{\ast n}$ is generated by two families of elements,
\begin{eq*} \langle x \rangle \quad := \quad (x, x) \quad \quad \text{and} \quad \quad \langle xy \rangle \quad := \quad (xy, yx) \end{eq*}
where $x,y \in \{z_1, ..., z_n, z_1^*, ..., z_n^*\}$ are generators of the free group $\mathbb{Z}^{\ast n}$ or their inverses. These are subject to the relations
\begin{eq*} \begin{array}{c}
			\langle x \rangle^{-1} \quad = \quad \langle x^* \rangle, \quad \quad \quad \langle xy \rangle^{-1} \quad = \quad \langle y^*x^* \rangle \\
			\\
			\langle xx^* \rangle \quad = \quad e \quad = \quad \langle x^*x \rangle, \quad \quad \quad \langle xx \rangle \quad = \quad \langle x \rangle^2 \\
			\\
			\langle xy \rangle \langle x^* \rangle \langle xy^* \rangle \quad = \quad \langle x \rangle \\
			\\
			\langle xy \rangle \langle x^* \rangle \langle y^* \rangle \langle yx \rangle \quad = \quad \langle x \rangle \langle y \rangle  \quad = \quad \langle yx \rangle \langle x^* \rangle \langle y^* \rangle \langle xy \rangle \\
			\\
			\langle xy \rangle \langle x^* \rangle \langle xz \rangle \langle x^* \rangle \langle z^* \rangle \langle y^* \rangle \langle yz \rangle \langle y^* \rangle \langle yx \rangle \langle y \rangle \langle x^* \rangle \langle z^* \rangle^{-1} \langle zx \rangle \langle z^* \rangle \langle zy \rangle \quad = \quad \langle x \rangle\langle y \rangle\langle z \rangle 
		\end{array}
\end{eq*}
\end{prop}
\begin{proof}
We'll begin by constructing a certain monoidal category, which we'll call $Z$. 
\begin{itemize}
\item The objects of $Z$ are the elements of the group $\mathbb{Z}^{\ast n}$, with the usual multiplication as the tensor product.
\item There is a unique morphisms between any two objects $x$ and $y$ for which $\mathrm{ab}(x) = \mathrm{ab}(y)$, where $\mathrm{ab}: \mathbb{Z}^{\ast n} \to \mathbb{Z}^n$ is the quotient map of abelianisation. In other words, the morphisms of $Z$ are the elements of $\mathbb{Z}^{\ast n} \times_{\mathbb{Z}^n} \mathbb{Z}^{\ast n}$, with multiplication as the tensor product and composition given by
\begin{eq*} (x,y) \circ (y,z) \quad = \quad (x, z) \end{eq*}
\item The identity map on an object $x$ is then the unique map $(x,x) : x \to x$.
\end{itemize}
$Z$ is almost the subcategory of $L\mathbb{G}_n$ whose morphisms are the subgroup isomorphic to $(s \times t)(L\mathbb{G}_n)$ that we chose in \cref{stZsub}. However, we never required those representatives to be closed under composition, so $Z$ is a strictly formal version of the subcategory on $(s \times t)(L\mathbb{G}_n)$, one that doesn't involve any specific choice of the map $\rho$. It is a well-defined monoidal category; the only thing that might not be immediately clear is the law of interchange, which is just given by
\begin{eq*} \begin{array}{rll}
			\big( \, (x,y) \circ (y,z) \, \big) \otimes \big( \, (x',y') \circ (y',z') \, \big) & = & (x,z) \otimes (x',z') \\
			& = & (xx',zz') \\
			& = & (xx',yy') \circ (yy',zz') \\
			& = & \big( \, (x,y) \otimes (x',y') \, \big) \circ \big( \, (y,z) \otimes (y',z') \, \big) 
		\end{array}
\end{eq*}
But now recall from \cref{tenscomp} that in any monoidal category the composition of morphisms along an intertible object can be rewritten in terms of only the tensor product. In the case of $Z$, where all of the objects have inverses, we will have
\begin{eq*} (x,y) \circ (y, z) \quad = \quad (x, y) \otimes (y^*, y^*) \otimes (y, z) \end{eq*}
Using this composition operation will make it easier to understand the structure of the group $\mathbb{Z}^{\ast n} \times_{\mathbb{Z}^n} \mathbb{Z}^{\ast n}$.

Next, let $\mathbb{S}_{2n}$ be the free $\mathrm{E}S$-algebra on $2n$ objects, where $S$ is the symmetric action operad. Then there is an obvious monoidal functor $\psi : \mathbb{S}_{2n} \to Z$, given by
\begin{eq*} \begin{array}{rcrcl}
			\psi & : & \mathbb{S}_{2n} & \to & Z \\
			 & : & z_i & \mapsto & z_i \\
			 & : & z_{n+i} & \mapsto & z_i^* \\
			 & : & \alpha(\sigma; \mathrm{id}_{x_1}, ..., \mathrm{id}_{x_m}) & \mapsto & (x_1 \otimes ... \otimes x_m, x_{\sigma(1)} \otimes ... \otimes x_{\sigma(m)})
		\end{array}
\end{eq*}
A necessary condition for $(y, y')$ to be an element of $\mathbb{Z}^{\ast n} \times_{\mathbb{Z}^n} \mathbb{Z}^{\ast n}$ is that there exists some sequence of generators and their inverses $x_1, ..., x_m \in \{z_1, ..., z_n, z_1^*, ..., z_n^*\}$ and some permutation $\sigma \in S_m$ for which
\begin{eq*} y \, = \, x_1 \otimes ... \otimes x_m, \quad \quad \quad y' \, = \, x_{\sigma(1)} \otimes ... \otimes x_{\sigma(m)} \end{eq*}
Hence the functor $\psi$ is clearly surjective. It follows from this that if we can find a collection of morphisms which generate $\mathrm{Mor}(\mathbb{S}_{2n})$ under composition and tensor product, their images under $\psi$ will also generate $\mathrm{Mor}(Z) = \mathbb{Z}^{\ast n} \times_{\mathbb{Z}^n} \mathbb{Z}^{\ast n}$ under composition and tensor product, and hence under just tensor product. To begin, we know that any permutation $\sigma \in S_m$ can be written as a product $\sigma_{i_k} \cdot ... \cdot \sigma_{i_1}$ of elementary transpositions, giving
\begin{eq*} \begin{array}{rlll}
			\alpha( \, \sigma \, ; \, \mathrm{id}_{x_1}, ..., \mathrm{id}_{x_m} \, ) & = & \alpha( \, \sigma_{i_k} \cdot ... \cdot \sigma_{i_1} \, ; \,  \mathrm{id}_{x_1}, ..., \mathrm{id}_{x_m} \, ) \\
			& = &  \alpha( \, \sigma_{i_1} \, ; \,  \mathrm{id}_{x_1}, ..., \mathrm{id}_{x_m} \, )  \\
			& & \circ \, \alpha( \, \sigma_{i_2} \, ; \,  \mathrm{id}_{x_{\sigma_{i_1}(1)}}, ..., \mathrm{id}_{x_{\sigma_{i_1}(m)}} \, ) \, \circ ... \\
			& & \circ \, \alpha( \, \sigma_{i_k} \, ; \,  \mathrm{id}_{x_{\sigma_{i_{k-1}} \cdot ... \cdot \sigma_{i_1}(1)}}, ..., \mathrm{id}_{x_{\sigma_{i_{k-1}} \cdot ... \cdot \sigma_{i_1}(m)}} \, )
		\end{array}
\end{eq*}
Then if $\sigma_i = (i \, \, i+1) \in S_m$ is some elementary transposition we will have
\begin{eq*} \begin{array}{rll}
			\alpha( \, (i \, \, i+1) \, ; \, \mathrm{id}_{x_1}, ..., \mathrm{id}_{x_m} \, ) & = & \alpha( \, e_{i-1} \otimes (1 2) \otimes e_{m-i-1} \, ; \,  \mathrm{id}_{x_1}, ..., \mathrm{id}_{x_m} \, ) \\
			& = & \mathrm{id}_{x_1 \otimes ... \otimes x_{i-1}} \otimes \alpha( \, (1 2) \, ; \, \mathrm{id}_{x_i}, \mathrm{id}_{x_{i+1}} \, ) \otimes \mathrm{id}_{x_{i+2} \otimes ... \otimes x_m}
		\end{array}
\end{eq*}
Therefore all of the morphisms of $\mathbb{S}_{2n}$ are generated by just the identities and the action maps $\alpha( \, (1 2) \, ; \, \mathrm{id}_{x_1}, \mathrm{id}_{x_2} \, )$ for all pairs $x_1, x_2 \in \{z_1, ..., z_{2n} \}$. Passing through $\psi$, this means that elements of $\mathbb{Z}^{\ast n} \times_{\mathbb{Z}^n} \mathbb{Z}^{\ast n}$ can always be expressed as a tensor product of elements of the form
\begin{eq*} (x, x) \quad \quad \text{or} \quad \quad (x y, y x), \quad \quad \quad x, y \in \{z_1, ..., z_n, z_1^*, ..., z_n^* \} \end{eq*}
These are exactly the $\langle x \rangle$ and $\langle xy \rangle$ given in the statement of the proposition.

Now we need to consider what relations these generators will obey. Firstly, their definitions overlap in the following case:
\begin{eq*} \langle xx \rangle \quad = \quad (xx,xx) \quad = \quad (x,x) \otimes (x,x) \quad = \quad \langle x \rangle\langle x \rangle \end{eq*}
Next we have to account for the law of interchange we discussed earlier. Using \cref{tenscomp}, we see that this condition will induce the following relation:
\begin{eq*} \begin{array}{rll}
			\langle xy \rangle \langle x^* \rangle \langle y^* \rangle \langle yx \rangle & = & (xy, yx) \otimes (x^*, x^*) \otimes (y^*, y^*) \otimes (yx, xy) \\
			& = & (xy, yx) \otimes (yx, yx)^* \otimes (yx, xy) \\
			& = & (xy,yx) \circ (yx, xy) \\
			& = & (yx, xy) \otimes (yx, yx)^* \otimes (yx, xy) \\
			& = & (yx, xy) \otimes (x^*, x^*) \otimes (y^*, y^*) \otimes (xy, yx) \\
			& = & \langle yx \rangle \langle x^* \rangle \langle y^* \rangle \langle xy \rangle
		\end{array}
\end{eq*}
Also, by functoriality these generators will inherit any relations are obeyed to the corresponding morphisms of $\mathbb{S}_{2n}$, which in turn are just relations among different elementary transpositions. Each symmetric group $S_m$ is subject to three families of these, namely
\begin{eq*} \begin{array}{rrll}
			(\sigma_i)^2 & = & e & \\
			\sigma_i \sigma_j & = & \sigma_j \sigma_i, & \quad j \neq i \pm 1 \\
			(\sigma_i \sigma_{i+1})^3 & = & e &
		\end{array}
\end{eq*}
The first one, the symmetry condition, corresponds to the relation
\begin{eq*} \begin{array}{rrll}
			& (xy, yx) \circ (yx, xy) & = & (xy, xy) \\
			\implies & (xy, yx) \otimes (yx, yx)^* \otimes (yx, xy) & = & (x, x) \otimes (y,y) \\
			\implies & (xy, yx) \otimes (x^*, x^*) \otimes (y^*, y^*)  \otimes (yx, xy) & = & (x, x) \otimes (y,y) \\
			\implies & \langle xy \rangle\langle x^* \rangle\langle y^* \rangle\langle yx \rangle & = & \langle x \rangle\langle y \rangle \\
		\end{array}
\end{eq*}
The second relation is just an example of interchange, which we have already looked at. The third yields
\begin{eq*} \begin{array}{rll}
			(xy, yx)(x^*,x^*)(xz,zx)(x^*,x^*)(z^*,z^*)(y^*,y^*)(yz,zy) & & \\
			(y^*,y^*)(yx,xy)(y^*,y^*)(x^*,x^*)(z^*,z^*)(zx, xz)(z^*,z^*)(zy,yz) & = & (x,x)(y,y)(z,z) \\
		\end{array}
\end{eq*}
or more simply,
\begin{eq*} \langle xy \rangle \langle x^* \rangle \langle xz \rangle \langle x^* \rangle \langle z^* \rangle \langle y^* \rangle \langle yz \rangle \langle y^* \rangle \langle yx \rangle \langle y^* \rangle \langle x^* \rangle \langle z^* \rangle \langle zx \rangle \langle z^* \rangle \langle zy \rangle \quad = \quad \langle x \rangle\langle y \rangle\langle z \rangle \end{eq*}
Finally, we need to check how the invertibility of the objects of $Z$ interacts with these generators. Most obviously, we have
\begin{eq*} \begin{array}{rcccccl}
			\langle x \rangle^{-1} & = & (x, x)^* & = & (x^*, x^*) & = & \langle x^* \rangle \\
			\langle xy \rangle^{-1} & = & (xy, yx)^* & = & (y^*x^*, x^*y^*) & = & \langle y^*x^* \rangle \\
		\end{array}
\end{eq*}
\begin{eq*} \begin{array}{rcccccl}
			\langle xx^* \rangle & = & (xx^*, x^*x) & = & (I, I) & = & e \\
			\langle x^*x \rangle & = & (x^*x, xx^*) & = & (I,I) & = & e \\
		\end{array}
\end{eq*}
But we can also insert an element and its inverse into different points of the source and target:
\begin{eq*} \begin{array}{rll}
			\langle x \rangle & = & (x,x) \\
			& = & (xyy^*, yy^*x) \\
			& = & (xyy^*, yxy^*) \circ (yxy^*, yy^*x) \\
			& = & (xyy^*, yxy^*) \otimes (yxy^*, yxy^*)^* \otimes (yxy^*, yy^*x) \\
			& = & (xy, yx) \otimes (y^*,y^*) \otimes (y, y) \otimes (x,x)^*(y^*, y^*) \otimes (y, y) \otimes (xy^*, y^*x) \\
			& = & \langle xy \rangle \langle x^* \rangle \langle xy^* \rangle
		\end{array}
\end{eq*}
The relations $(xy, yx) = (zz^*xy, yzz^*x)$ and so forth are all composed of successive instance of the above, so these are all of the relations on our generators $\langle x \rangle$ and $\langle xy \rangle$.
\end{proof}

Of course, the collection of relations we just gave in \cref{pushpres} are nowhere near minimal. Many of them clearly interact with each other in ways that would let us simplify or cancel some relations, or even generators. However, we will not expend any effort trying to do this, because we do not need to. With this inefficient presentation of $\mathbb{Z}^{\ast n} \times_{\mathbb{Z}^n} \mathbb{Z}^{\ast n}$ in hand, we have in a sense already found its abelianisation. After all, for any presentation of some group $H$, the group $H^{\mathrm{ab}}$ possesses a presentation consisting of the exact same generators, subject to the same relations, plus a commutativity condition. This too will not normally be the most efficient description of the new group, but that remains true even if the presentation of $H$ we started with was minimal, and so any time spent finding one will just be wasted. Instead, we'll suppress the urge to simplify \cref{pushpres} and move straight on to tackling $(\mathbb{Z}^{\ast n} \times_{\mathbb{Z}^n} \mathbb{Z}^{\ast n})^{\mathrm{ab}}$.

\begin{prop} \label{abst}
\begin{eq*} (\mathbb{Z}^{\ast n} \times_{\mathbb{Z}^n} \mathbb{Z}^{\ast n})^{\mathrm{ab}} \quad = \quad \mathbb{Z}^n \times {\mathbb{Z}}^{{n}\choose{2}} \end{eq*}
\end{prop}
\begin{proof}
It follows immediately from \cref{pushpres} that the group $(\mathbb{Z}^{\ast n} \times_{\mathbb{Z}^n} \mathbb{Z}^{\ast n})^{\mathrm{ab}}$ has a presentation on generators
\begin{eq*} \langle x \rangle, \quad \langle xy \rangle, \quad x,y \in \{z_1, ..., z_n, z_1^*, ..., z_n^*\} \end{eq*}
subject to the relations
\begin{eq*} \begin{array}{c}
			\langle x \rangle^{-1} \quad = \quad \langle x^* \rangle, \quad \quad \quad \langle xy \rangle^{-1} \quad = \quad \langle y^*x^* \rangle \\
			\\
			\langle xx^* \rangle \quad = \quad e \quad = \quad \langle x^*x \rangle, \quad \quad \quad \langle xx \rangle \quad = \quad \langle x \rangle^2 \\
			\\
			\langle xy \rangle \langle x^* \rangle \langle xy^* \rangle \quad = \quad \langle x \rangle  \\
			\\
			\langle xy \rangle \langle x^* \rangle \langle y^* \rangle \langle yx \rangle \quad = \quad \langle x \rangle \langle y \rangle  \quad = \quad \langle yx \rangle \langle x^* \rangle \langle y^* \rangle \langle xy \rangle \\
			\\
			\langle xy \rangle \langle x^* \rangle \langle xz \rangle \langle x^* \rangle \langle z^* \rangle \langle y^* \rangle \langle yz \rangle \langle y^* \rangle \langle yx \rangle \langle y^* \rangle \langle x^* \rangle \langle z^* \rangle \langle zx \rangle \langle z^* \rangle \langle zy \rangle \quad = \quad \langle x \rangle\langle y \rangle\langle z \rangle 
		\end{array}
\end{eq*}
but then also the commutativity conditions
\begin{eq*} \langle x \rangle \langle y \rangle \, = \, \langle y \rangle \langle x \rangle, \quad \quad \langle x \rangle \langle yz \rangle \, = \, \langle z \rangle \langle xy \rangle, \quad \quad	\langle wx \rangle \langle yz \rangle \, = \, \langle yz \rangle \langle wx \rangle \end{eq*} 
Rearranging all of the former equations with the latter in mind, we get
\begin{eq*} \begin{array}{c}
			\langle x \rangle^{-1} \quad = \quad \langle x^* \rangle, \quad \quad \quad \langle xy \rangle^{-1} \quad = \quad \langle y^*x^* \rangle \\
			\\
			\langle xx^* \rangle \quad = \quad e \quad = \quad \langle x^*x \rangle, \quad \quad \quad \langle xx \rangle \quad = \quad \langle x \rangle^2  \quad = \quad \langle xy \rangle \langle xy^* \rangle \\
			\\
			\langle xy \rangle \langle yx \rangle \quad = \quad \langle x \rangle^2 \langle y \rangle^2 \\
			\\
			\langle xy \rangle \langle yx \rangle \langle xz \rangle \langle zx \rangle \langle yz \rangle \langle zy \rangle \quad = \quad \langle x \rangle^4 \langle y \rangle^4 \langle z \rangle^4 
		\end{array}
\end{eq*}
The last of these relations is just a consequence of the one above that,
\begin{eq*} \begin{array}{rll}
			\langle xy \rangle \langle yx \rangle \langle xz \rangle \langle zx \rangle \langle yz \rangle \langle zy \rangle & = & \big( \, \langle x \rangle^2 \langle y \rangle^2 \, \big)\big( \, \langle x \rangle^2 \langle z \rangle^2 \, \big)\big( \, \langle y \rangle^2 \langle y \rangle^2 \, \big) \\
			& = & \langle x \rangle^4 \langle y \rangle^4 \langle z \rangle^4 
		\end{array}
\end{eq*}
and in turn, the second-to-last follows from the relation above it,
\begin{eq*} \begin{array}{rll}
			\langle x \rangle^2 \langle y \rangle^2  & = & \big( \, \langle xy \rangle \langle xy^* \rangle \, \big)\big( \, \langle yx \rangle \langle yx^* \rangle \, \big) \\
			& = & \langle xy \rangle \langle yx \rangle \langle xy^* \rangle  \langle xy^* \rangle^{-1} \\
			& = & \langle xy \rangle \langle yx \rangle
		\end{array}
\end{eq*}
Without these, we are just left with equations in two or fewer variables. Then for any two $z_i, z_j \in \mathbb{Z}^{\ast n}$, $i<j$, the first three relations tell us that we only need to consider generators of the form
\begin{eq*} \langle z_i \rangle, \quad \langle z_j \rangle, \quad \langle z_i z_j \rangle, \quad \langle z_i^* z_j\rangle, \quad \langle z_i z_j^* \rangle, \quad \langle z_i^* z_j^* \rangle \end{eq*}
Finally, the remaining relation $\langle x \rangle^2  =  \langle xy \rangle \langle xy^* \rangle$ induces a system of four linear equations on these six generators, which can be solved to give
\begin{eq*} \begin{array}{rll}
			\langle z_i^* z_j \rangle & = & \langle z_j \rangle^2 \langle z_i z_j \rangle^{-1} \\
			\langle z_i z_j^* \rangle & = & \langle z_i \rangle^2 \langle z_i z_j \rangle^{-1} \\
			\langle z_i^* z_j^* \rangle & = & \langle z_i \rangle^{-2} \langle z_j \rangle^{-2} \langle z_i z_j \rangle \\
		\end{array}
\end{eq*}
and three independent variables, $\langle z_i \rangle$, $\langle z_j \rangle$, and $\langle z_i z_j \rangle$. In other words, $(\mathbb{Z}^{\ast n} \times_{\mathbb{Z}^n} \mathbb{Z}^{\ast n})^{\mathrm{ab}}$ is the free abelian group generated by elements of this form, for $1 \le i < j \le n$, which means that
\begin{eq*} (\mathbb{Z}^{\ast n} \times_{\mathbb{Z}^n} \mathbb{Z}^{\ast n})^{\mathrm{ab}} \quad = \quad \mathbb{Z}^n \times \mathbb{Z}^{{n}\choose{2}} \end{eq*}
\end{proof}

From this presentation, it should be immediately obvious how to calculate the denominator from \cref{Zmor}.

\begin{cor} \label{nchoose2}
\begin{eq*} \begin{array}{rll}
			 \bigquotient{{(\mathbb{Z}^{\ast n} \times_{\mathbb{Z}^n} \mathbb{Z}^{\ast n})}^{\mathrm{ab}}}{\mathbb{Z}^n} & = & \bigquotient{\mathbb{Z}^n \times \mathbb{Z}^{{n}\choose{2}}}{\mathbb{Z}^n} \\[\bigskipamount]
			& = & \mathbb{Z}^{{n}\choose{2}} 
		\end{array}
\end{eq*}
\end{cor}
\begin{proof}
The $\mathbb{Z}^n$ term in the product of \cref{abst} represents the free abelian group generated by the morphisms
\begin{eq*} \langle x \rangle \quad := \quad (x,x) \quad = \quad \mathrm{id}_{x} \end{eq*}
But this is exactly the same $\mathbb{Z}^n$ group that appears in the denominator of our quotient, $\mathrm{Ob}(L\mathbb{G}_n)^{\mathrm{ab}}$, so they cancel straightforwardly.
\end{proof}

Before moving on, we should be clear about exactly which $\mathbb{Z}^{{n}\choose{2}}$ subgroup of $\mathrm{M}(L\mathbb{G}_n)^{\mathrm{ab}}$ we have just identified --- after all, we will eventually need to perform a quotient involving it. In \cref{pushpres} we defined the generators $\langle z_i z_j \rangle$ to be the elements $(z_i \otimes z_j, z_j \otimes z_i)$ of the monoid $\mathbb{Z}^{\ast n} \times_{\mathbb{Z}^n} \mathbb{Z}^{\ast n}$, which are the source/target combinations of morphisms of $L\mathbb{G}_n$. Using \cref{stZsub} we can identify this with a particular submonoid of the morphisms of $L\mathbb{G}_n$, specifically the image under $q$ of the submonoid $\mathbb{N}^{\ast 2n} \times_{\mathbb{N}^{2n}} \mathbb{N}^{\ast 2n} = (s \times t)(\mathbb{G}_{2n}) \subseteq \mathrm{Mor}(\mathbb{G}_{2n})$ we chose in \cref{stGnsub}. In particular, since on objects we have $q(z_i) = z_i$ for all $1 \le i \le n$, the generators $(z_i \otimes z_j, z_j \otimes z_i)$ of $\mathbb{Z}^{\ast n} \times_{\mathbb{Z}^n} \mathbb{Z}^{\ast n}$ are clearly the image of the generators $(z_i \otimes z_j, z_j \otimes z_i)$ of $\mathbb{N}^{\ast 2n} \times_{\mathbb{N}^{2n}} \mathbb{N}^{\ast 2n}$. 

Thus, consider the following commutative diagram, whose top-left region comes from \cref{stZsub}, bottom-left from the naturality of the adjoint functor $\mathrm{M}( \, \_ \, )^{\mathrm{gp},\mathrm{ab}}$, and right-hand square from \cref{colquot}.
\begin{eq*} \begin{tikzcd}[column sep=tiny] 
& (s \times t)(\mathbb{G}_{2n}) \ar[dl, hookrightarrow] \ar[rr, "q"] & & (s \times t)(L\mathbb{G}_{n}) \ar[dl, hookrightarrow] \ar[dr] & \\
\mathrm{Mor}(\mathbb{G}_{2n}) \ar[rr, "q"] \ar[dr] & & \mathrm{Mor}(L\mathbb{G}_n) \ar[dr] & & \frac{\displaystyle (s \times t)(L\mathbb{G}_{n})^{\mathrm{ab}}}{\displaystyle \mathrm{Ob}(L\mathbb{G}_n)^{\mathrm{ab}}} \ar[dl, hookrightarrow] \\
& \mathrm{M}(\mathbb{G}_{2n})^{\mathrm{gp},\mathrm{ab}} \ar[rr, "\mathrm{M}(q)^{\mathrm{gp},\mathrm{ab}}"] & & \mathrm{M}(L\mathbb{G}_n)^{\mathrm{gp},\mathrm{ab}}
\end{tikzcd} \end{eq*}
What we've just said that if we start with the element $(z_i \otimes z_j, z_j \otimes z_i)$ of $(s \times t)(\mathbb{G}_{2n})$, moving clockwise around the diagram will send it to the generator $\langle z_i z_j \rangle$ in ${(s \times t)}(L\mathbb{G}_{n})^{\mathrm{ab}}/\mathrm{Ob}(L\mathbb{G}_n)^{\mathrm{ab}} = \mathbb{Z}^{{n}\choose{2}}$. If we instead move anticlockwise, then we will first pass to our chosen representative $\alpha_{\mathbb{G}_{2n}}(\rho(z_i \otimes z_j, z_j \otimes z_i); \mathrm{id}_{z_i}, \mathrm{id}_{z_j})$ in $\mathrm{Mor}(\mathbb{G}_{2n})$, then its equivalence class in $\mathrm{M}(\mathbb{G}_{2n})^{\mathrm{gp},\mathrm{ab}}$, then its equivalence class in $\mathrm{M}(L\mathbb{G}_n)^{\mathrm{gp},\mathrm{ab}}$, using the fact that $\mathrm{M}(q)^{\mathrm{gp},\mathrm{ab}}$ is the canonical map associated with the quotient
\begin{eq*} \mathrm{M}(L\mathbb{G}_n)^{\mathrm{gp, ab}} \quad = \quad \bigquotient{{\mathrm{M}(\mathbb{G}_{2n})}^{\mathrm{gp, ab}}}{\Delta} \end{eq*}
which we proved back in \cref{Zmor2}. Since the bottom-right inclusion completes this circuit, we see that the specific subgroup we are talking about in \cref{nchoose2} is
\begin{eq*} \mathbb{Z}^{{n}\choose{2}} \, = \, \big\{ \, \big[ \, \alpha_{\mathbb{G}_{2n}}\big( \, \rho(z_i \otimes z_j, z_j \otimes z_i) \, ; \,  \mathrm{id}_{z_i}, \mathrm{id}_{z_j} \, \big) \, \big] \, : \, 1 \le i < j \le n \, \big\} \, \subseteq \, \mathrm{M}(L\mathbb{G}_n)^{\mathrm{ab}}\end{eq*}

Of course, $\rho$ was an arbitrary permutation-preserving map $\mathbb{N}^{\ast n} \times_{\mathbb{N}} \mathbb{N}^{\ast n} \to G$, chosen using the freeness of its source monoid. Thus if we wanted to we could just pick the same element $\rho(2) \in \pi^{-1}((1 \, 2))$ to act as $\rho(z_i \otimes z_j, z_j \otimes z_i)$ for all $i, j$. For simplicity's sake, we will indeed be assuming this from now on.

\section{Freely generated action operads}

The next group we are interested in understanding a little better is $\mathrm{M}(\mathbb{G}_{2n})^{\mathrm{gp},\mathrm{ab}}$. Per \cref{Morder}, the operations needed to produce this group out of $\mathrm{Mor}(\mathbb{G}_{2n}) = G \times_{\mathbb{N}} \mathbb{N}^{\ast 2n}$ can be done in any order we choose, and so we will save the identification of $\otimes$ and $\circ$ until last. This will let us keep the tensor product as simple as possible whilst we are in the process of group completing and abelianising it.

So the obvious place to start is to ask how to simplify the expression $(G \times_{\mathbb{N}} \mathbb{N}^{\ast 2n})^{\mathrm{gp}}$. In principle we might not be able to, since for generic $G$ we lack any sort of a presentation by generators and relations. It would help if we at least knew that the group completion map $\mathrm{gp} : G \to G^{\mathrm{gp}}$ was injective --- or equivalently, that there exists any group $H$ and injective homomorphism $G \to H$ --- but proving this kind of statement is notoriously tricky. In 1935, a paper by Anton Sushkevich `proved' that a semigroup, and thus a monoid, could be embedded into a group if and only if it was cancellative.

\begin{defn} We say that a monoid $M$ is \emph{left-cancellative} if for any $a, b, c \in M$, we have
\begin{eq*} ab \, = \, ac \quad \implies \quad b \, = \, c \end{eq*}
That is, common factors may be cancelled out on the left. Similarly, we call $M$ \emph{right-cancellative} if common factors can be cancelled on the right:
\begin{eq*} ac \, = \, bc \quad \implies \quad a \, = \, b \end{eq*}
A monoid that is both left- and right-cancellative is simply referred to as \emph{cancellative}.
\end{defn}

However, just two years later Anatoly Malcev published a simple counterexample \cite{immer1} to Sushkevich's proposition. To make matters worse, in 1939 Malcev would go on to show that the actual set of necessary and sufficient conditions for a semigroup to be embeddable in a group consisted of an infinite collection of independent relations \cite{immer2}. Thus the requirement that the group completion of monoid be injective is a deceptively complicated one. 

Luckily for us though, there does exist a much simpler set of sufficient-but-not-necessary conditions for embeddability which all action operads $G$ happen to satisfy. These come from a 1948 paper by Raouf Doss \cite{semi}, and in addition to cancellativity they depend on the way that a monoid deals with multiples of different elements being equal.

\begin{defn} An element $a$ of a monoid $M$ is said to be \emph{regular on the left} if it shares a common left-multiple with every other element of $M$. That is,
\begin{eq*} \forall \, b \in M, \quad \exists \, c, d \in M \quad : \quad ca \, = \, db \end{eq*}
The monoid as a whole is said to be \emph{regular on the left} if all of its elements are, but we can also define a notion of $M$ being \emph{quasi-regular on the left}. This means that any two elements $a,b$ of $M$ will share a common left-multiple if and only if
\begin{eq*} \exists \, c, d \in M \quad : \quad ca \, = \, db, \quad \quad \text{$c$ or $d$ is regular in $M$} \end{eq*}
Again, we can define a similar condition for being quasi-regular on the right, and we say that a monoid is \emph{quasi-regular} when it is both.
\end{defn}

\begin{prop} If a monoid $M$ is cancellative and quasi-regular on the left, then it can be embedded into a group.
\end{prop}

For a given action operad, both of these conditions will follow from the way that operadic multiplication interacts with the elements of the abelian group $G(0)$.

\begin{prop} \label{cqr} Every action operad $G$ is both cancellative and quasi-regular as a monoid under tensor product.
\end{prop}
\begin{proof}
Let $g$ and $g'$ be any elements of $G$ which share a left-multiple, so that there exists at least one pair $h, h'$ in $G$ for which
\begin{eq*} h \otimes g \quad = \quad h' \otimes g' \end{eq*}
and without loss of generality assume that $|g| \ge |g'|$, so also $|h| \le |h'|$. The operadic product $\mu(h; e_0, ..., e_0)$ is clearly an element of the group $G(0)$, and we know from \cref{G0abel} that this is an abelian group under tensor product, so also let $\mu(h; e_0, ..., e_0)^*$ be its inverse. Then
\begin{eq*} \begin{array}{rll}
			g & = & \mu(h; e_0, ..., e_0)^* \otimes \mu(h; e_0, ..., e_0) \otimes \mu(g; e_1, ..., e_1) \\
			& = & \mu(h; e_0, ..., e_0)^* \otimes \mu\big( \, e_2 \, ; \, \mu(h; e_0, ..., e_0), \mu(g; e_1, ..., e_1) \, \big) \\
			& = & \mu(h; e_0, ..., e_0)^* \otimes \mu\big( \, \mu(e_2; h, g) \, ; \, e_0, ..., e_0, e_1, ..., e_1 \, \big) \\
			& = & \mu(h; e_0, ..., e_0)^* \otimes \mu\big( \, h \otimes g \, ; \, e_0, ..., e_0, e_1, ..., e_1 \, \big) \\
			& = & \mu(h; e_0, ..., e_0)^* \otimes \mu\big( \, h' \otimes g' \, ; \, e_0, ..., e_0, e_1, ..., e_1 \, \big) \\
			& = & \mu(h; e_0, ..., e_0)^* \otimes \mu\big( \, \mu(e_2; h', g') \, ; \, e_0, ..., e_0, e_1, ..., e_1 \, \big) \\
			& = & \mu(h; e_0, ..., e_0)^* \otimes \mu\big( \, e_2 \, ; \, \mu(h'; e_0, ..., e_0, e_1, ..., e_1), \mu(g'; e_1, ..., e_1) \, \big) \\
			& = & \mu(h; e_0, ..., e_0)^* \otimes \mu(h'; e_0, ..., e_0, e_1, ..., e_1) \otimes \mu(g'; e_1, ..., e_1) \\
			& = & \big( \, \mu(h; e_0, ..., e_0)^* \otimes \mu(h'; e_0, ..., e_0, e_1, ..., e_1) \, \big) \otimes g' \\
			& =: & k \otimes g'
		\end{array}
\end{eq*}
Put another way,
\begin{eq*} \exists \, e_0, k \in G \quad : \quad e_0 \otimes g \quad = \quad g \quad = \quad k \otimes g' \end{eq*}
and $e_0$ obviously regular, since it is the unit $I$ in $G$. Thus $G$ is quasi-regular on the left.  For quasi-regularity on the right, there is an argument  which is completely analogous to what we have done already, but which lets us rewrite $h'$ as $h \otimes k'$ for some $k' \in G$.

Moreover, if we set $h = h'$ then we see that
\begin{eq*} k \quad = \quad \mu(h; e_0, ..., e_0)^* \otimes \mu(h; e_0, ..., e_0) \quad = \quad I \end{eq*}
and so
\begin{eq*} h \otimes g \, = \, h \otimes g' \quad \implies \quad g \, = \, g'  \end{eq*}
which is left-cancellativity. Right-cancellativity follows from quasi-regularity on the right in the same way.
\end{proof}

\begin{cor} \label{gpcompin} The canonical map $\mathrm{gp} : G \to G^{\mathrm{gp}}$ associated with the group completion of $G$ is an inclusion.
\end{cor}

As a result of this, from now on we can just write $g$ for $\mathrm{gp}(g)$ and $g^*$ for $\mathrm{gp}(g)^*$ in order to save on space.

Knowing that the monoid $G \times_{\mathbb{N}} \mathbb{N}^{\ast n}$ always has a particularly well-behaved group completion is a good first step towards finding a description for said completion. However, it is worth noting that \cref{gpcompin} is true for all action operads $G$, which is more than we really need. After all, the only reason we care about $\mathrm{M}(\mathbb{G}_{2n})^{\mathrm{gp},\mathrm{ab}}$ is that we know from \cref{Zmor} that it is crucial to understanding the morphisms of \emph{crossed} action operads. Thus it would be nice if we could use some consequence of crossedness to tell us even more about the inclusion map $\mathrm{gp} : G \times_{\mathbb{N}} \mathbb{N}^{\ast n} \to {(G \times_{\mathbb{N}} \mathbb{N}^{\ast n})}^{\mathrm{gp}}$.

One such consequence was given back in \cref{noscalarcross}. If $G$ is a crossed action operad, then the action operad $G'$ defined by $G'(m) = G(m)/G(0)$ possesses the same free algebra on invertible algebra that $G$ does. In other words, we don't even need to worry about finding $\mathrm{M}(\mathbb{G}_{2n})^{\mathrm{gp},\mathrm{ab}}$ for all crossed $G$, merely those which have a trivial $G(0)$. As it turns out, this fact is hugely relevant to our search for group completions, since elements of $G(0)$ are the only ones in $G$ which might already have an inverse under tensor product. This follows from additivity of lengths:
\begin{eq*} \begin{array}{rclcrcccll}
			g \otimes h & = & e_0 & \quad \implies \quad & |g| + |h| & = & |e_0| & = & 0 & \\
			& & & \quad \implies \quad & & & |g| & = & -|h|, & \quad |g|, |h| \in \mathbb{N} \\
			& & & \quad \implies \quad & |g| & = & |h| & = & 0& 
		\end{array}
\end{eq*}
Cancellativity, quasi-regularity, and lack of invertible objects then combine to give something much stronger than mere injectivity of the group completion map.

\begin{prop} \label{Gfree} If $G$ is an action operad with trivial $G(0)$, then $G$ is a free monoid under tensor product.
\end{prop}
\begin{proof}
Let $\mathcal{G}$ be a subset of the monoid $G$, and $\mathcal{R}$ a collection of relations on the elements of $\mathcal{G}$, such that $(\mathcal{G},\mathcal{R})$ is a presentation of $G$. Notice that every relation in $\mathcal{R}$ can be written in the form $h \otimes g = h' \otimes g'$, where $g,g' \in \mathcal{G}$ are generators and $h,h' \in G$ some other elements. This is because the only other kind of relations are one like $h \otimes g = e_0$, and as we've seen this is not possible if $G(0)$ is trivial. We'll assume that in this case $|g| \ge |g'|$ and hence $|h| \le |h'|$. Using the reasoning from the proof of \cref{cqr}, we can then find $k, k' \in G$ for which
\begin{eq*} g \, = \, k \otimes g', \quad \quad \quad h' \, = \, h \otimes k' \end{eq*}
It follows that
\begin{eq*} h \otimes k \otimes g' \quad = \quad h \otimes g \quad = \quad h' \otimes g' \quad = \quad h \otimes k \otimes g' \end{eq*}
and thus by left- and right-cancellativity, $k = k'$.  In other words, the relation $h \otimes g = h' \otimes g'$ implies and is implied by a pair of relations $g = k \otimes g'$, $h' = h \otimes k$. 

There are a few scenarios to consider here. 
\begin{itemize}
\item $|k| = |g|$. This is actually not possible, as it would follow from additivity of length that $|g'|=0$, and thus by assumption $g' = e_0$, which is not a generator of $G$.
\item $|k|=0$. This would mean that $k=e_0$, and so we'd also get $g=g'$ and $h = h'$. Thus we could simplify the presentation of $G$ by replacing the relation $h \otimes g = h' \otimes g'$ in the set $\mathcal{R}$ with $h' = h$.
\item $0 < |k| < |g|$. In this case $|g| > |g'|$ and thus $g \neq g'$, and so we could change our presentation of $G$ by replacing $g$ with $k$ in the generator set $\mathcal{G}$, and also $h \otimes g = h' \otimes g'$ by $h' = h \otimes k$ in the relations $\mathcal{R}$.
\end{itemize}
Notice that in the latter two cases, we are always changing generators for ones that have strictly smaller lengths, and replacing relations with ones whose left- and right-hand side have strictly smaller total length. But lengths are natural numbers, and therefore if we choose any relation in $\mathcal{R}$ and repeatedly apply this process to it, after a finite number of steps we will find that we have replaced it with $e_0 = e_0$, the only relation whose sides have total length $0$. Proceeding like this will let us eliminated all of the relations in $\mathcal{R}$, leaving us with a set $\mathcal{G}$ that freely generates the action operad $G$ under tensor product.
\end{proof} 

Whenever we can be sure of that $G$ is a free monoid --- whether by using \cref{Gfree} or some other method --- this freeness will carry over directly to the algebras $\mathbb{G}_n$, giving us a new way to represent their morphisms.

\begin{prop} \label{freemor} Let $\mathcal{G}$ be a set that freely generates the action operad $G$ under tensor product, and for each $m \in \mathbb{N}$ define $\mathcal{G}_m := \mathcal{G} \, \cap \,  G(m)$, the subset of $\mathcal{G}$ containing all elements of length $m$. Then the monoid $\mathrm{Mor}(\mathbb{G}_n)$ is 
\begin{eq*} G \times_{\mathbb{N}} \mathbb{N}^{\ast n} \quad = \quad \mathbb{N}^{\ast ( \, |\mathcal{G}_0| + n|\mathcal{G}_1| + n^2 |\mathcal{G}_2| + ... \, )} \end{eq*}
\end{prop}
\begin{proof}
Let $(g, w)$ be an arbitrary element of $G \times_{\mathbb{N}} \mathbb{N}^{\ast n}$. The monoid $G$ is free of the generators $\mathcal{G}$, and $\mathbb{N}^{\ast n}$ is free on $\{z_1, ..., z_n\}$, so we can find unique expansions of $g$ and $w$ as tensor products
\begin{eq*} \begin{array}{rclcrcl}
			g & = & g_1 \otimes ... \otimes g_k, & \quad & g_1, ..., g_k & \in & \mathcal{G} \\
			w & = & x_1 \otimes ... \otimes x_m, & \quad & x_1, ..., x_m & \in & \{z_1, ..., z_n\}
		\end{array}
\end{eq*}
But each of the generators $z_1, ..., z_n$ has length 1, so the index $m$ here is really just the length $|w|$, which by the definition of $G \times_{\mathbb{N}} \mathbb{N}^{\ast n}$ is also the length $|g| = |g_1| + ... + |g_k|$. Therefore we may write
\begin{eq*} \begin{array}{rll}
			(g, w) & = & (g_1 \otimes ... \otimes g_k, x_1 \otimes ... \otimes x_{|w|}) \\
			& = & (g_1, x_1 \otimes ... \otimes x_{|g_1|}) \otimes (g_2, x_{|g_1|+1} \otimes ... \otimes x_{|g_1|+|g_2|}) \otimes ... \\
			& & \otimes (g_k, x_{|g_1| + ... + |g_{k-1}|+1} \otimes ... \otimes x_{|g_1| + ... + |g_k|})
		\end{array}
\end{eq*}
That is, every element in $G \times_{\mathbb{N}} \mathbb{N}^{\ast n}$ may be expressed as a product of elements from the subset $\mathcal{G} \times_{\mathbb{N}} \mathbb{N}^{\ast n}$. Furthermore, the freeness of $G$ and $\mathbb{N}^{\ast n}$ make sure that this expansion is unique, since
\begin{eq*} \begin{array}{rl}
			& (g_1, x_1 \otimes ... x_{|g_1|}) \otimes ... \otimes (g_k, x_{|g_1| + ... + |g_{k-1}|+1} \otimes ... \otimes x_{|g_1| + ... + |g_k|}) \\
			= & (g'_1, x'_1 \otimes ... \otimes x'_{|g'_1|}) \otimes ... \otimes (g'_{k'}, x'_{|g'_1| + ... + |g'_{k'-1}|+1} \otimes ... \otimes x'_{|g'_1| + ... + |g'_{k'}|})
		\end{array}
\end{eq*}
\begin{eq*} \begin{array}{rcclcccl}
			\implies \quad \quad & g_1 \otimes ... \otimes g_k & = & g'_1 \otimes ... \otimes g'_{k'}, & \quad \quad & x_1 \otimes ... \otimes x_{m} & = & x'_1 \otimes ... \otimes x'_{m'} \\
			& & & & & & & \\
			\implies \quad \quad & g_i \, = \, g'_i, & & 1 \le i \le k = k', & \quad \quad & x_j \, = \, x'_j, & & 1 \le j \le m = m' 
		\end{array}
\end{eq*}
Thus $G \times_{\mathbb{N}} \mathbb{N}^{\ast n}$ is the free monoid on the set 
\begin{eq*} \mathcal{G} \times_{\mathbb{N}} \mathbb{N}^{\ast n} \quad = \quad \mathcal{G}_0 \times \{ z_1, ..., z_n \}^0  \, \cup \, \mathcal{G}_1 \times \{ z_1, ..., z_n \}^1 \, \cup \, \mathcal{G}_2 \times \{ z_1, ..., z_n \}^2 \, \cup \, ...\end{eq*}
which is just the $m$-fold free product of $\mathbb{N}$ with itself, where $m$ is the number of generators,
\begin{eq*} \begin{array}{rll}
			|\mathcal{G} \times_{\mathbb{N}} \mathbb{N}^{\ast n}| & = & |\mathcal{G}_0| \cdot |\{ z_1, ..., z_n \}^0|  \, + \, |\mathcal{G}_1| \cdot |\{ z_1, ..., z_n \}^1| \, + \, |\mathcal{G}_2| \cdot |\{ z_1, ..., z_n \}^2| \, + \, ... \\
			& = & |\mathcal{G}_0| + n|\mathcal{G}_1| + n^2 |\mathcal{G}_2| + ... 
		\end{array}	
\end{eq*}
\end{proof}

This makes the group completion and abelianisation we want to do trivial. 

\begin{cor} \label{freemorgpab} If $\mathcal{G}$ is a set that freely generates $G$ under tensor product, and $\mathcal{G}_m := \mathcal{G} \, \cap \,  G(m)$, then the abelian group $\mathrm{Mor}(\mathbb{G}_n)^{\mathrm{gp}, \mathrm{ab}}$ is 
\begin{eq*} (G \times_{\mathbb{N}} \mathbb{N}^{\ast n})^{\mathrm{gp}, \mathrm{ab}} \quad = \quad \mathbb{Z}^{|\mathcal{G}_0| + n|\mathcal{G}_1| + n^2 |\mathcal{G}_2| + ...} \end{eq*}
\end{cor}

Now all that remains is to account for what happens when we collapse the morphisms of $\mathbb{G}_n$ --- that is, evaluate the quotient
\begin{eq*} \mathrm{M}(\mathbb{G}_n)^{\mathrm{gp}, \mathrm{ab}} \quad = \quad \bigquotient{\mathbb{Z}^{|\mathcal{G}_0| + n|\mathcal{G}_1| + n^2 |\mathcal{G}_2| + ...}}{\otimes \sim \circ} \end{eq*}
Unfortunately, because this will depend on the exact multiplicative structure of the operad groups $G(m)$, there is no way to carry out this computation in general. The best we can say is that as composition in $\mathrm{Mor}(\mathbb{G}_n)$ is partly determined by the group multiplication of the $G(m)$, then in place of $\mathcal{G}$ in the quotient in \cref{freemorgpab} it would suffice to have some collection of elements which generate $G$ when using multiplication as well as tensor product.

\begin{lem} Let $\mathcal{G}$ be a subset of the action operad $G$ that freely generates it under tensor product, and let $\mathcal{G'}$ be a subset of $\mathcal{G}$ which generates $G$ under a combination of tensor product and group multiplication. Also let $\mathcal{G}_m := \mathcal{G} \, \cap \,  G(m)$ and $\mathcal{G}'_m := \mathcal{G}' \, \cap \,  G(m)$. Then

\begin{eq*} \bigquotient{\mathbb{Z}^{|\mathcal{G}_0| + n|\mathcal{G}_1| + n^2 |\mathcal{G}_2| + ...}}{\otimes \sim \circ} \quad \quad = \quad \quad \bigquotient{\mathbb{Z}^{|\mathcal{G}'_0| + n|\mathcal{G}'_1| + n^2 |\mathcal{G}'_2| + ...}}{\otimes \sim \circ} \end{eq*}
\end{lem}
\begin{proof} 
Compostion in $\mathrm{Mor}(\mathbb{G}_n)$ is given by
\begin{eq*} \alpha(g'; \mathrm{id}_{x_{\pi(g^{-1})(1)}}, ..., \mathrm{id}_{x_{\pi(g^{-1})(m)}}) \, \circ \, \alpha(g; \mathrm{id}_{x_1}, ..., \mathrm{id}_{x_m}) \quad = \quad \alpha(g'g; \mathrm{id}_{x_1}, ..., \mathrm{id}_{x_m})\end{eq*}
which in $G \times_{\mathbb{N}} \mathbb{N}^{\ast n}$ terms is
\begin{eq*} \big( \, g', \pi(g^{-1})(w) \, \big) \, \circ \, (g, w) \quad = \quad (g'g, w) \end{eq*}
Thus any element $(g, w)$ of $\mathcal{G} \times_{\mathbb{N}} \mathbb{N}^{\ast n}$ can be expressed in terms of elements of $\mathcal{G}' \times_{\mathbb{N}} \mathbb{N}^{\ast n}$ by way of tensor product and compostion. All we need to do is find and expansion for $g$ using $\mathcal{G}'$, and then pull all of the multiplication and tensors outside of the brackets via the equation above and those we employed back in \cref{freemon}. This means that when we take the quotient by the relation $\otimes \sim \circ$, the equivalence class for $(g, w)$ will be a tensor product of equivalence classes of elements from $\mathcal{G}' \times_{\mathbb{N}} \mathbb{N}^{\ast n}$. In other words, every generator of $\mathbb{Z}^{|\mathcal{G}_0| + n|\mathcal{G}_1| + n^2 |\mathcal{G}_2| + ...}/\otimes \sim \circ$ is contained within the subgroup coming from $\mathcal{G}'$, and therefore so is the whole of the group. That is, 
\begin{eq*} \begin{array}{rcl}
			\bigquotient{\mathbb{Z}^{|\mathcal{G}_0| + n|\mathcal{G}_1| + n^2 |\mathcal{G}_2| + ...}}{\otimes \sim \circ} \quad & = & \quad \bigquotient{\mathbb{Z}^{|\mathcal{G}' \, \cap \, \mathcal{G}_0| + n|\mathcal{G}' \, \cap \, \mathcal{G}_1| + n^2 |\mathcal{G}' \, \cap \, \mathcal{G}_2| + ...}}{\otimes \sim \circ} \\
			& = & \quad \bigquotient{\mathbb{Z}^{|\mathcal{G}'_0| + n|\mathcal{G}'_1| + n^2 |\mathcal{G}'_2| + ...}}{\otimes \sim \circ}
		\end{array}
\end{eq*}
\end{proof}

Beyond this, the value of this quotient will have to be found separately for each individual action operad.  
 
\chapter{Complete descriptions of free invertible algebras}
\label{mainthm}

At last, we finally have an expression for the morphisms of $L\mathbb{G}_n$, one built out of smaller parts which we know how to calculate. This means that it is almost time to draw together everything we have done over the past three chapters into a single, complete description of free invertible $\mathrm{E}G$-algebras --- at least, in cases where $G$ is crossed or $G(1)$-generated. 

\section{The action of $L\mathbb{G}_n$}

At this stage, there is only one part of this $\mathrm{E}G$-algebra that we have yet to find --- its action, $\alpha_{L\mathbb{G}_n}$. When our action operad $G$ is $G(1)$-generated, everything is so simple that there is really only one thing the action could be.

\begin{lem} \label{G1act} Let $G$ be a $G(1)$-generated action operad, $g$ an element of $G(m)$ for some $m \in \mathbb{N}$, and $x_1, ..., x_m$ elements of $\mathbb{Z}^{\ast n}$. Then the action of $L\mathbb{G}_n$ is given by
\begin{eq*} \alpha_{L\mathbb{G}_n}( \, g \, ; \, \mathrm{id}_{x_1}, ..., \mathrm{id}_{x_m} \, ) \quad = \quad \mathrm{id}_{x_1 \otimes ... \otimes x_m} \end{eq*}
\end{lem}
\begin{proof}
In order for $\alpha_{L\mathbb{G}_n}$ to be a well-defined $\mathrm{E}G$-action, the map $\alpha_{L\mathbb{G}_n}(g; \mathrm{id}_{x_1}, ..., \mathrm{id}_{x_m})$ needs to have source $x_1 \otimes ... \otimes x_m$ and target $x_{\pi(g^{-1})(1)} \otimes ... \otimes x_{\pi(g^{-1})(m)}$, where by non-crossedness of $G$ the latter is also $x_1 \otimes ... \otimes x_m$. But we know from \cref{trivendo} that all morphisms in this $L\mathbb{G}_n$ are identities, and hence we get the result.
\end{proof}

For crossed $G$, things are more complicated. What we need to do is employ the trick that was previously mentioned in \cref{actmorLGn}, where we exploit the surjectivity of the algebra map $q: \mathbb{G}_{2n} \to L\mathbb{G}_n$. This will allow us to express $\alpha_{L\mathbb{G}_n}$ in terms of the action $\alpha_{\mathbb{G}_{2n}}$. 

\begin{prop} \label{crossact} Let $G$ be a crossed action operad, and for some $m \in \mathbb{N}$ choose an element $g \in G(m)$ and morphisms $(x_1, y_1, h_1), ..., (x_m, y_m, h_m)$ in $L\mathbb{G}_n$. That is, the $(x_i, y_i)$ are pairs of objects from $(s \times t)(L\mathbb{G}_n)$, and the $h_i$ are morphisms in $L\mathbb{G}_n(I,I)$. Then the action of $L\mathbb{G}_n$ is given by
\begin{eq*} \begin{array}{c}
			\alpha_{L\mathbb{G}_n}\big( \, g \, ; \, (x_1, y_1, h_1), ..., (x_m, y_m, h_m) \, \big) \\
			= \\
			\big( \, \, \bigotimes_i x_i, \quad \bigotimes_i y_{\pi(g^{-1})(i)}, \quad \Psi \alpha_{\mathbb{G}_{2n}}( \, g \, ; \, \mathrm{id}_{q^{-1}(y_1)}, ..., \mathrm{id}_{q^{-1}(y_m)} \, \, ) \, \otimes \, (\bigotimes_i h_i) \, \big) 
		\end{array}
\end{eq*}
Here $q^{-1}: \mathrm{Ob}(L\mathbb{G}_n) \to \mathrm{Ob}(\mathbb{G}_{2n})$ is the function 
\begin{eq*} \begin{array}{rcrcl}
			q^{-1} & : & \mathbb{Z}^{\ast n} & \to & \mathbb{N}^{\ast 2n} \\
			& : & z_i & \mapsto & z_i \\
			& : & z_i^* & \mapsto & z_{n+1} \\
			& : & w & \mapsto & \bigotimes_{i=1}^{|w|} \, q^{-1}\big( \, d(w, i) \, \big)
		\end{array}
\end{eq*}
with $\bigotimes_{i=1}^{|w|} d(w, i)$ the decomposition of $w$ given in \cref{decompdef}, and $\Psi: \mathrm{Mor}(\mathbb{G}_{2n}) \to L\mathbb{G}_n(I,I)$ is the canonical map associated with the repeated quotient
\begin{eq*} \begin{tikzcd}
\mathrm{Mor}(\mathbb{G}_{2n}) \ar[r] & \mathrm{M}(\mathbb{G}_{2n})^{\mathrm{gp}, \mathrm{ab}} \ar[r] & \bigquotient{\mathrm{M}(\mathbb{G}_{2n})^{\mathrm{gp}, \mathrm{ab}}}{\Delta} \ar[d, equals] & \\
& & \mathrm{M}(L\mathbb{G}_{n})^{\mathrm{gp}, \mathrm{ab}} \ar[r] & \bigquotient{\mathrm{M}(L\mathbb{G}_{n})^{\mathrm{gp}, \mathrm{ab}}}{\mathbb{Z}^{{n}\choose{2}}} \ar[d, equals] \\
& & & L\mathbb{G}_n(I,I)
\end{tikzcd} \end{eq*}
\end{prop} 
\begin{proof}
Firstly, by the rules governing $\mathrm{E}G$-actions and \cref{tenscomp}, we know that
\begin{eq*} \begin{array}{rl} 
			& \alpha_{L\mathbb{G}_n}\big( \, g \, ; \, (x_1, y_1, h_1), ..., (x_m, y_m, h_m) \, \big) \\
			= & \alpha_{L\mathbb{G}_n}( \, g \, ; \, \mathrm{id}_{y_1}, ..., \mathrm{id}_{y_m} \, ) \circ \big( \, (x_1, y_1, h_1) \otimes ... \otimes (x_m, y_m, h_m) \, \big) \\
			= & \alpha_{L\mathbb{G}_n}( \, g \, ; \, \mathrm{id}_{y_1}, ..., \mathrm{id}_{y_m} \, ) \circ ( \, x_1 \otimes ... \otimes x_m, \, y_1 \otimes ... \otimes y_m, \, h_1 \otimes ... \otimes h_m \, ) \\
			= & \alpha_{L\mathbb{G}_n}( \, g \, ; \, \mathrm{id}_{y_1}, ..., \mathrm{id}_{y_m} \, ) \otimes \mathrm{id}_{y_1 \otimes ... \otimes y_m}^* \otimes ( \, x_1 \otimes ... \otimes x_m, \, y_1 \otimes ... \otimes y_m, \, h_1 \otimes ... \otimes h_m \, ) \\
		\end{array}
\end{eq*}
Since we already understand tensor products of objects and unit endomorphisms, we now only need to find the action morphisms on identities. Moreover, we know that the source and target of $\alpha_{L\mathbb{G}_n}(g; \mathrm{id}_{y_1}, ..., \mathrm{id}_{y_m})$ have to be $y_1 \otimes ... \otimes y_m$ and $y_{\pi(g^{-1})(1)} \otimes ... \otimes y_{\pi(g^{-1})(m)}$ respectively, so to see this morphism as an element of the monoid
\begin{eq*} \mathrm{Mor}(L\mathbb{G}_n) \quad \cong \quad (s \times t)(L\mathbb{G}_n) \times L\mathbb{G}_n(I,I) \end{eq*}
all that is left to understand is its projection onto the unit endomorphisms.

Now, recall that $q: \mathbb{G}_{2n} \to L\mathbb{G}_n$ is a surjective map of $\mathrm{E}G$-algebras, so that for any $f_i \in \mathrm{Mor}(L\mathbb{G}_{n})$ there exist $f'_i \in \mathrm{Mor}(\mathbb{G}_{2n})$ with $q(f'_i) = f_i$, and hence
\begin{eq*} q\big( \, \alpha_{\mathbb{G}_{2n}}( \, g \, ; \, f'_1, ..., f'_m \, ) \, \big) \quad = \quad \alpha_{L\mathbb{G}_n}( \, g \, ; \, f_1, ..., f_m \, ) \end{eq*}
In particular, for the identities $\mathrm{id}_{y_i} \in \mathrm{Mor}(L\mathbb{G}_{n})$ we can choose $\mathrm{id}_{q^{-1}(y_i)} \in \mathrm{Mor}(\mathbb{G}_{2n})$, as by design $q(\mathrm{id}_{q^{-1}(y_i)}) = \mathrm{id}_{qq^{-1}(y_i)} = \mathrm{id}_{y_i}$. This means that if we denote by $p_I :  \mathrm{Mor}(L\mathbb{G}_{n}) \to L\mathbb{G}_{n}(I,I)$ the projection onto unit endomorphisms, we will have
\begin{eq*} p_I \big( \, \alpha_{L\mathbb{G}_n}( \, g \, ; \, \mathrm{id}_{y_1}, ..., \mathrm{id}_{y_m} \, ) \, \big) \quad = \quad  p_I q\big( \, \alpha_{\mathbb{G}_{2n}}( \, g \, ; \, \mathrm{id}_{q^{-1}(y_1)}, ..., \mathrm{id}_{q^{-1}(y_m)} \, ) \, \big) \end{eq*}
But $p_I \circ q$ is a map that can be described in a different way. Consider the commutative diagram
\begin{eq*} \begin{tikzcd}
\mathrm{Mor}(\mathbb{G}_{2n}) \ar[rr, "q"] \ar[dd] & & \mathrm{Mor}(L\mathbb{G}_n) \ar[d, "\mathrm{ab}"] \ar[rr, "p_I"] & &  L\mathbb{G}_{n}(I,I) \ar[d, equals] \\
& & \mathrm{Mor}(L\mathbb{G}_n)^{\mathrm{ab}} \ar[d] \ar[rr] & & \bigquotient{\mathrm{Mor}(L\mathbb{G}_{n})^{\mathrm{ab}}}{(s \times t)(L\mathbb{G}_n)^{\mathrm{ab}}} \ar[d, equals] \\
\mathrm{M}(\mathbb{G}_{2n})^{\mathrm{gp},\mathrm{ab}} \ar[rr, "\mathrm{M}(q)^{\mathrm{gp},\mathrm{ab}}"] & & \mathrm{M}(L\mathbb{G}_n)^{\mathrm{gp},\mathrm{ab}} \ar[rr] & & \bigquotient{\mathrm{M}(L\mathbb{G}_{n})^{\mathrm{gp}, \mathrm{ab}}}{\mathbb{Z}^{{n}\choose{2}}}
\end{tikzcd} \end{eq*}
where all unlabelled arrows are the appropriate quotient maps. The region on the left commutes by naturality of the adjoint functor $\mathrm{M}(\, \_ \,)^{\mathrm{gp},\mathrm{ab}}$, and the bottom-right square uses the fact that
\begin{eq*} \begin{array}{rllll}
			\bigquotient{\mathrm{Mor}(L\mathbb{G}_{n})^{\mathrm{ab}}}{(s \times t)(L\mathbb{G}_n)^{\mathrm{ab}}} & = & \frac{ \displaystyle  \left(\mathrm{Mor}(L\mathbb{G}_{n})^{\mathrm{ab}}/\mathrm{Ob}(L\mathbb{G}_{n})^{\mathrm{ab}} \right)}{ \displaystyle \left( (s \times t)(L\mathbb{G}_n)^{\mathrm{ab}}/\mathrm{Ob}(L\mathbb{G}_{n})^{\mathrm{ab}} \right)} & = & \bigquotient{\mathrm{M}(L\mathbb{G}_{n})^{\mathrm{gp}, \mathrm{ab}}}{\mathbb{Z}^{{n}\choose{2}}}
		\end{array}
\end{eq*}
As for the square on the top-right, remember that the split extension of groups
\begin{eq*} \begin{tikzcd}
L\mathbb{G}_n(I,I) \ar[r, hookrightarrow] & \mathrm{Mor}(L\mathbb{G}_n) \ar[r, shift left, "s \times t"] & (s \times t)(L\mathbb{G}_n) \ar[l, shift left, hookrightarrow, ""]
\end{tikzcd} \end{eq*}
was the source of our product description of morphisms of $L\mathbb{G}_n$. Thus by the proof of \cref{splitex}, the specific isomorphism we are using is
\begin{eq*} \begin{array}{rll}
			\mathrm{Mor}(L\mathbb{G}_n) & \cong & (s \times t)(L\mathbb{G}_n) \times L\mathbb{G}_n(I,I) \\
			f & \mapsto & \Big( \, s(f), \, t(f), \, f \otimes i\big( \, s(f), t(f) \, \big)^* \, \Big)
		\end{array}
\end{eq*}
and so the projection $p_I$ is given by tensoring a morphism with the inverse of the representative of its source and target under the inclusion $(s \times t)(L\mathbb{G}_n) \hookrightarrow \mathrm{Mor}(L\mathbb{G}_n)$. But the monoid $\mathrm{Mor}(L\mathbb{G}_{n})^{\mathrm{ab}}/(s \times t)(L\mathbb{G}_n)^{\mathrm{ab}}$ is exactly what we get when we quotient out by those representatives, so we see that
\begin{eq*} \begin{array}{rll}
			[\mathrm{ab}(f)] & = &  [\mathrm{ab}(f)] \otimes \Big[ \mathrm{ab}\Big( \, i\big( \, s(f), t(f) \, \big)^* \, \Big) \, \Big] \\
			& = & \Big[ \, \mathrm{ab}\Big( \, f \otimes i\big( \, s(f), t(f) \, \big)^* \, \Big) \, \Big] \\
			& = & \mathrm{ab}\big( \, p_I(f) \, \big) \\
			& = & p_I(f)
		\end{array}
\end{eq*}
Here we've used that fact that the equivalence class of a unit endomorphism under the quotient map $\mathrm{Mor}(L\mathbb{G}_n)^{\mathrm{gp},\mathrm{ab}} \to \mathrm{Mor}(L\mathbb{G}_{n})^{\mathrm{ab}}/(s \times t)(L\mathbb{G}_n)^{\mathrm{ab}} = L\mathbb{G}_n(I,I)$ is just the same endomorphism again, and also that $L\mathbb{G}_n(I,I)^{\mathrm{ab}} = L\mathbb{G}_n(I,I)$. 

Thus all of the regions within the diagram commute, and hence so will the outside. That is, $p_I \circ q$ is equal to the composite along the left and bottom edges, which is $\Psi$. This means that the projection onto $L\mathbb{G}_n(I,I)$ of our action on identities is
\begin{eq*} \begin{array}{rll}
			p_I \big( \, \alpha_{L\mathbb{G}_n}( \, g \, ; \, \mathrm{id}_{y_1}, ..., \mathrm{id}_{y_m} \, ) \, \big) & = &  p_I q\big( \, \alpha_{\mathbb{G}_{2n}}( \, g \, ; \, \mathrm{id}_{q^{-1}(y_1)}, ..., \mathrm{id}_{q^{-1}(y_m)} \, ) \, \big) \\
			& = & \Psi \big( \, \alpha_{\mathbb{G}_{2n}}( \, g \, ; \, \mathrm{id}_{q^{-1}(y_1)}, ..., \mathrm{id}_{q^{-1}(y_m)} \, ) \, \big)
		\end{array}
\end{eq*}
and therefore the action of $L\mathbb{G}_n$ is given by
\begin{eq*} \begin{array}{c}
			\alpha_{L\mathbb{G}_n}\big( \, g \, ; \, (x_1, y_1, h_1), ..., (x_m, y_m, h_m) \, \big) \\
			= \\
			\alpha_{L\mathbb{G}_n}( \, g \, ; \, \mathrm{id}_{y_1}, ..., \mathrm{id}_{y_m} \, ) \circ \, \bigotimes_i (x_i, y_i, h_i) \\
			= \\
			\big( \, \bigotimes_i y_i, \, \bigotimes_i y_{\pi(g^{-1})(i)}, \, \Psi \alpha_{\mathbb{G}_{2n}}( \, g \, ; \, \mathrm{id}_{q^{-1}(y_1)}, ..., \mathrm{id}_{q^{-1}(y_m)} \, ) \, \big) \otimes \, \mathrm{id}_{\otimes_i y_i}^* \otimes ( \, \bigotimes_i x_i, \, \bigotimes_i y_i, \, \bigotimes_i h_i \, ) \\
			= \\
			\big( \, \, \bigotimes_i x_i, \quad \bigotimes_i y_{\pi(g^{-1})(i)}, \quad \Psi \alpha_{\mathbb{G}_{2n}}( \, g \, ; \, \mathrm{id}_{q^{-1}(y_1)}, ..., \mathrm{id}_{q^{-1}(y_m)} \, \, ) \, \otimes \, (\bigotimes_i h_i) \, \big) 
		\end{array}
\end{eq*}
as required.
\end{proof}

\section{A full description of $L\mathbb{G}_n$}

With this last proposition proven, the results in this paper now collectively describe how to construct the free $\mathrm{E}G$-algebras on $n$ invertible objects for most values of $G$. However, since this characterization was discovered by us in such a piecemeal fashion, we will now restate everything in one place, for ease of reading. We'll begin with the non-crossed case, or as much of it as we were able to draw a complete conclusion about.

\begin{thm} \label{freeinvalgG1} Let $G$ be a $G(1)$-generated action operad. Then the free $\mathrm{E}G$-algebra on $n$ invertible objects is just the discrete category
\begin{eq*} L\mathbb{G}_n \quad = \quad \mathbb{Z}^{\ast n} \end{eq*}
equipped with a tensor product which is the usual monoid multiplication, and an $\mathrm{E}G$-action given by
\begin{eq*} \alpha_{L\mathbb{G}_n}( \, g \, ; \, \mathrm{id}_{x_1}, ..., \mathrm{id}_{x_m} \, ) \quad = \quad \mathrm{id}_{x_1 \otimes ... \otimes x_m} \end{eq*}
\end{thm}
\begin{proof}
The object monoid is from \cref{Zobj}, the fact that $L\mathbb{G}_n$ is discrete follows from \cref{trivendo}, and the action is given by \cref{G1act}.
\end{proof}

It is a shame that we were not able to find a formulation for uncrossed $L\mathbb{G}_n$ in full generality; this will have to be the subject of future research. For crossed action operads however, we were able to achieve this.

\begin{thm}\label{freeinvalgc} Let $G$ be a crossed action algebra, and let $G'$ be the action operad defined by $G'(m) := G(m)/G(0)$. Choose a subset $\mathcal{G}$ that generates $G'$ under a combination of tensor product and group multiplication, which itself has subsets $\mathcal{G}_m := \mathcal{G} \, \cap \, G'(m)$. Then denote by $A$ the abelian group obtained from the free abelian group 
\begin{eq*} \mathrm{F}( \mathcal{G} \times_{\mathbb{N}} \mathbb{N}^{\ast 2n}) \quad = \quad \mathbb{Z}^{2n|\mathcal{G}_1| + (2n)^2|\mathcal{G}_2| + ...} \end{eq*}
via the following steps:
\begin{enumerate}[leftmargin=*]
\item For all $g, g' \in G(m)$ and $w \in \mathbb{N}^{\ast 2n}$ with $|w| = m$, quotient out by the relation 
\begin{eq*} (g, w) \, \otimes \, \big( \, g', \pi(g^{-1})(w) \, \big) \quad \sim \quad (g \cdot g', w) \end{eq*}
\item Quotient out by the subgroup $\Delta$, which is generated by the equivalence classes of elements of the form
\begin{eq*} \begin{array}{c}
				\big( \, \mu( \, g \, ; \, e_{|\tilde{\delta}(x_1)|}, ..., e_{|\tilde{\delta}(x_m)|} \, ), \, \tilde{\delta}( \, x_1 \otimes ... \otimes x_m \, ) \, \big) \\
				\otimes \\
				\big( \, \mu( \, g \, ; \, e_{|\tilde{I}(x_1)|}, ..., e_{|\tilde{I}(x_m)|} \, ), \, \tilde{I}( \, x_1 \otimes ... \otimes x_m \, ) \, \big)^*
		\end{array} 
\end{eq*}
where $g \in G(m)$, the $x_i$ are generators of $\mathbb{N}^{\ast 4n}$, and for all $1 \le i \le n$,
\begin{eq*} \begin{array}{rclcrclcrclcrcl}
			\tilde{\delta}(z_i) & = & z_i, & & \tilde{\delta}(z_{2n+i}) & = & z_i \otimes z_{n+i}, & & \tilde{I}(z_i) & = & z_i, & &\tilde{I}(z_{2n+i}) & = & I,\\
			\tilde{\delta}(z_{n+i}) & = & z_{n+i}, &  & \tilde{\delta}(z_{3n+i}) & = & z_{n+i} \otimes z_i & & \tilde{I}(z_{n+i}) & = & z_{n+i}, & & \tilde{I}(z_{3n+i}) & = & I
			 \end{array}
\end{eq*} 
\item Choose any $\rho(2) \in \pi^{-1}((1 \, 2))$, and then quotient out by the $\mathbb{Z}^{{n}\choose{2}}$ subgroup generated by the equivalence classes of the elements 
\begin{eq*} \big( \, \rho(2) \, ; \, z_i, z_j \, \big), \quad \quad 1 \le i < j \le n \end{eq*}
\end{enumerate}
Also, denote by $\Psi: G \times_{\mathbb{N}} \mathbb{N}^{\ast 2n} \to A$ the corresponding quotient map. Then the free $\mathrm{E}G$-algebra on $n$ invertible objects is the category
\begin{eq*} L\mathbb{G}_n \quad = \quad \mathbb{Z}^{\ast n} \times_{\mathbb{Z}^n} \mathbb{Z}^{\ast n}  \, \times \, \mathrm{B}A \end{eq*}
equipped with a tensor product given by component-wise monoid multiplication,
\begin{eq*} (x', y' ,h') \otimes (x, y, h) \, = \, ( \, x' \otimes x, \, y' \otimes y, \, h'h \, ) \end{eq*}
and an $\mathrm{E}G$-action given by
\begin{eq*} \begin{array}{c}
			\alpha_{L\mathbb{G}_n}\big( \, g \, ; \, (x_1, y_1, h_1), ..., (x_m, y_m, h_m) \, \big) \\
			= \\
			\big( \, \, \bigotimes_i x_i, \quad \bigotimes_i y_{\pi(g^{-1})(i)}, \quad \Psi \alpha_{\mathbb{G}_{2n}}( \, g \, ; \, \mathrm{id}_{q^{-1}(y_1)}, ..., \mathrm{id}_{q^{-1}(y_m)} \, \, ) \, \otimes \, (\bigotimes_i h_i) \, \big) 
		\end{array}
\end{eq*}
where $q^{-1}$ is the function
\begin{eq*} \begin{array}{rcrcl}
			q^{-1} & : & \mathbb{Z}^{\ast n} & \to & \mathbb{N}^{\ast 2n} \\
			& : & z_i & \mapsto & z_i \\
			& : & z_i^* & \mapsto & z_{n+1} \\
			& : & w & \mapsto & \bigotimes_{i=1}^{|w|} \, q^{-1}\big( \, d(w, i) \, \big)
		\end{array}
\end{eq*}
\end{thm}
\begin{proof}
First, notice that we are allowed to quotient out by a factor of $G(0)$ because of \cref{noscalarcross}. Then the category $\mathbb{Z}^{\ast n} \times_{\mathbb{Z}^n} \mathbb{Z}^{\ast n} \times \mathrm{B}A$ is just the one which has objects $\mathbb{Z}^{\ast n} \times_{\mathbb{Z}^n} \mathbb{Z}^{\ast n}$, morphisms $\mathbb{Z}^{\ast n} \times_{\mathbb{Z}^n} \mathbb{Z}^{\ast n} \times A$, and composition
\begin{eq*} (y, z , h') \circ (x, y, h) \quad = \quad (x, z, h'h) \end{eq*}
We know that the objects and morphisms are correct by \cref{Zobj,Zmor,nchoose2,freemor}, and since these deal with their monoidal structure too we can also see that the given tensor product is correct. Then for composition, it follows from \cref{tenscomp} that
\begin{eq*} \begin{array}{rll} 
			(y, z , h') \circ (x, y, h) & = & (y, z , h') \otimes \mathrm{id}_{y^*} \otimes (x, y, h) \\
			& = & (y, z , h') \otimes (y^*, y^*, \mathrm{id}_I) \otimes (x, y, h) \\
			& = & (\, y \otimes y^* \otimes x, \,  z \otimes y^* \otimes y, \, h' \otimes \mathrm{id}_I \otimes h \, ) \\
			& = & (x, z, h'h)
		\end{array}
\end{eq*}
The action we just found in \cref{crossact} then completes this description of $L\mathbb{G}_n$.
\end{proof}

\section{Free symmetric monoidal categories on invertible objects}

Even collected all together, \cref{freeinvalgc} is still a fairly opaque result. In the next couple of sections we will work through some specific applications of the theorem, which will hopefully prove enlightening in this regard. A good place to start will be with the simplest of all the crossed action operads, the symmetric operad $\mathrm{S}$. As one might expect, the free invertible algebras $L\mathbb{S}_n$ have a particularly straightforward form when viewed as monoidal categories.

\begin{prop} \label{invsymcat} The underlying monoidal category of the free $\mathrm{ES}$-algebra on $n$ invertible objects is
\begin{eq*} L\mathbb{S}_n \quad = \quad \mathbb{Z}^{\ast n} \times_{\mathbb{Z}^n} \mathbb{Z}^{\ast n}  \, \times \, \mathrm{B}\mathbb{Z}_2^{n} \end{eq*}
with component-wise tensor product.
\end{prop}
\begin{proof}
The symmetric operad has only one nullary operation, $e_0$, the identity permutation on 0 objects, and so the quotient operad $\mathrm{S}/\mathrm{S}_0$ is still just $\mathrm{S}$. Moreover, we saw back in \cref{operad} that the symmetric groups $\mathrm{S}_m$ are generated by the elementary transpositions $(i \, \, \, i+1)$, which in turn are tensor products
\begin{eq*} \begin{array}{rll}
			(i \, \, \,  i+1) & = & e_{i-1} \otimes (1 \, 2) \otimes e_{m-i-1} \\
			& = & (e_1)^{\otimes (i-1)} \otimes (1 \, 2) \otimes (e_1)^{\otimes (m-i-1)}
		\end{array}
\end{eq*}
in the operad $\mathrm{S}$. Therefore the set $\mathcal{S} = \{ e_1, (1 \, 2) \}$ will suffice to generate $\mathrm{S}$ under multiplication and tensor product, and so our search for the unit endomorphisms of $L\mathbb{S}_n$ can begin with the group
\begin{eq*} \mathbb{Z}^{2n|\mathcal{S}_1| + (2n)^2|\mathcal{S}_2| + ...}  \quad = \quad \mathbb{Z}^{2n + (2n)^2} \end{eq*}

First of all, we need to collapse the composition and tensor product inherited from $\mathbb{S}_{2n}$ into the same operation. For the generators with permutation part $e_1$, we have
\begin{eq*} \begin{array}{rllll}
			(e_1; z_i) \, \otimes \, (e_1; z_i) & \sim & (e_1 \cdot e_1; z_i) & = & (e_1; z_i) \\
			\implies \quad (e_1; z_i) & \sim & I
		\end{array}
\end{eq*}
and this will allow us to immediately eliminate those elements, leaving the group $\mathbb{Z}^{(2n)^2}$ coming from the $(1 \, 2)$ generators. The effect that collapsing composition has on these elements will depend on how elementary transpositions interact under group multiplication. This comes down two three conditions from \cref{sympres},
\begin{eq*} \begin{array}{rclll}
			(i \, \, \, i+1)^2 & = & e & & \\
			(i-1 \, \, \, i)(i \, \, \, i+1)(i-1 \, \, \, i) & = & (i \, \, \, i+1)(i-1 \, \, \, i)(i \, \, \, i+1) & & \\
			(i \, \, \, i+1)(j \, \, \, j+1) & = & (j \, \, \, j+1)(i \, \, \, i+1), & & i+1 < j
		\end{array}
\end{eq*}
The last of these will not induce any new relation on our generators, since they all already commute. Likewise, we know that 
\begin{eq*} (i \, \, \,  i+1) \, = \, e_{i-1} \otimes (1 \, 2) \otimes e_{n-i-1}, \quad \quad (e_1; z_1) \, \sim \, I  \end{eq*}
for any $i$, and so the second condition is implied by the specific case
\begin{eq*} (1 \, 2)(2 \, 3)(1 \, 2) \quad = \quad (2 \, 3)(1 \, 2)(2 \, 3) \end{eq*}
which only produces a commutativity condition on our generators:
\begin{longtable}{RL}
			& \big( \, (1 \, 2) \, ; \, z_i, z_j \, \big) \, \otimes \, \big( \, (1 \, 2) \, ; \, z_i, z_k \, \big) \, \otimes \, \big( \, (1 \, 2) \, ; \, z_j, z_k \, \big) \\
			\sim & (e_1; z_k) \otimes \big( \, (1 \, 2) \, ; \, z_i, z_j \, \big) \otimes \big( \, (1 \, 2) \, ; \, z_i, z_k \, \big) \otimes (e_1; z_j) \otimes (e_1; z_i) \otimes \big( \, (1 \, 2) \, ; \, z_j, z_k \, \big)\\
			\sim & \big( \, (e_1 \otimes (1 \, 2)) \cdot ((1 \, 2) \otimes e_1) \cdot (e_1 \otimes (1 \, 2)) \, ; \, z_i, z_j, z_k \, \big) \\
			= & \big( \, (2 \, 3)(1 \, 2)(2 \, 3) \, ; \, z_i, z_j, z_k \, \big) \\
			= & \big( \, (1 \, 2)(2 \, 3)(1 \, 2)  \, ; \, z_i, z_j, z_k \, \big) \\
			= & \big( \, ((1 \, 2) \otimes e_1) \cdot (e_1 \otimes (1 \, 2)) \cdot ((1 \, 2) \otimes e_1) \, ; \, z_i, z_j, z_k \, \big) \\
			\sim & \big( \, (1 \, 2) \, ; \, z_j, z_k \, \big) \otimes (e_1; z_i) \otimes (e_1; z_j) \otimes \big( \, (1 \, 2) \, ; \, z_i, z_k \, \big) \otimes \big( \, (1 \, 2) \, ; \, z_i, z_j \, \big) \otimes (e_1; z_k) \\
			\sim & \big( \, (1 \, 2) \, ; \, z_j, z_k \, \big) \otimes \big( \, (1 \, 2) \, ; \, z_i, z_k \, \big) \otimes \big( \, (1 \, 2) \, ; \, z_i, z_j \, \big)
\end{longtable}
Thus the only restraint we need to impose on our remaining generators is the one that comes from the symmetry condition,
\begin{eq*} \begin{array}{rll}
			\big( \, (1 \, 2) \, ; \, z_i, z_j \, \big) \, \otimes \, \big( \, (1 \, 2) \, ; \, z_j, z_i \, \big) & \sim & \big( \, (1 \, 2) \cdot (1 \, 2) \, ; \, z_i, z_j \, \big) \\
			& = & (e_2; z_i, z_j) \\
			& = & (e_1; z_i) \otimes (e_1; z_j) \\
			& = & I
		\end{array}
\end{eq*}
which can be treated as two different cases depending on the values of the indices. From $i \neq j$ we will get ${2n}\choose{2}$ pairs of distinct generators $((1 \, 2);z_i, z_j)$, $((1 \, 2);z_j, z_i)$ whose equivalence classes are inverses of one other, and from $i = j$ we see that the classes of the $2n$ generators $((1 \, 2);z_i, z_i)$ are all self-inverse. In other words,
\begin{eq*} \bigquotient{\mathbb{Z}^{2n + (2n)^2}}{\otimes \sim \circ} \quad = \quad \mathbb{Z}_2^{2n} \times \mathbb{Z}^{{2n}\choose{2}} \end{eq*}
where ${Z}_2$ is the cyclic group of order 2.

Next, we need to consider the subgroup $\Delta$, which comes from the equivalence classes of elements of the form
\begin{eq*} \begin{array}{c}
				\big( \, \mu( \, g \, ; \, e_{|\tilde{\delta}(x_1)|}, ..., e_{|\tilde{\delta}(x_m)|} \, ), \, \tilde{\delta}( \, x_1 \otimes ... \otimes x_m \, ) \, \big) \\
				\otimes \\
				\big( \, \mu( \, g \, ; \, e_{|\tilde{I}(x_1)|}, ..., e_{|\tilde{I}(x_m)|} \, ), \, \tilde{I}( \, x_1 \otimes ... \otimes x_m \, ) \, \big)^*
		\end{array} 
\end{eq*}
for $x_i \in \{z_1, ..., z_{4n} \}$. At this point we are only interested in cases where $g$ is $(1 \, 2)$, and thus $m=2$, so pick any $1 \le i,j \le n$ and then suppose that $x_1 = z_i$ and $x_2 = z_j$. The corresponding element will just be
\begin{eq*} \big( \, \mu( \, (1 \, 2) \, ; \, e_1, e_1 \, ) \, ; \, z_i, z_j \, \big) \, \otimes \, \big( \, \mu( \, (1 \, 2) \, ; \, e_1, e_1 \, ) \, ; \, z_i, z_j \, \big)^* \quad = \quad I \end{eq*}
which contributes nothing to the group $\Delta$; the same is also true when instead either $x_1 = z_{n+i}$ or $x_2 = z_{n+j}$, or both. A more interesting result is what happens when $x_1 = z_i$ and $x_2 = z_{2n+j}$: 
\begin{eq*} \begin{array}{rl}
			& \big( \, \mu( \, (1 \, 2) \, ; \, e_1, e_2 \, ) \, ; \, z_i , z_j, z_{n+j} \, \big) \, \otimes \, \big( \,  \mu( \, (1 \, 2) \, ; \, e_1, e_0 \, ) \, ; \, z_i \, \big)^* \\
			= & \big( \, ( \, e_1 \otimes (1 \, 2) \, ) \cdot ( \, (1 \, 2) \otimes e_1 \,) \, ; \, z_i , z_j, z_{n+j} \, \big) \, \otimes \, \big( \, e_1 \, ; \, z_i \, \big)^* \\
			\sim & \big( \, ( \, e_1 \otimes (1 \, 2) \, ) \cdot ( \, (1 \, 2) \otimes e_1 \,) \, ; \, z_i , z_j, z_{n+j} \, \big) \\
			= &  (e_1 ; z_j) \otimes ( \, (1 \, 2) \, ;  z_i, z_{n+j} \, ) \otimes ( \, (1 \, 2) \, ;  z_i , z_{j} \, ) \otimes (e_1 ;z_{n+j}) \\
			\sim & ( \, (1 \, 2) \, ;  z_i, z_{n+j} \, ) \otimes ( \, (1 \, 2) \, ;  z_i , z_j \, )
		\end{array}
\end{eq*}
The presence of elements like the above will mean that when we quotient out by $\Delta$, we will force equivalence classes of the generators $((1 \, 2);  z_i , z_j )$ and $((1 \, 2) ; z_i, z_{n+j})$ to become inverses of one another. In an analogous way, setting $x_1 = z_{2n+j}$ and $x_2 = z_j$ shows that $((1 \, 2) ; z_{n+i}, z_j)$ will also become an inverse of $((1 \, 2); z_i , z_j)$, which means that $((1 \, 2) ; z_{n+i}, z_j) \sim ((1 \, 2) ; z_i, z_{n+j})$, whilst the choices $x_1 = z_{n+i}$ and $x_2 = z_{2n+j}$ will yield $((1 \, 2) ; z_{n+i}, z_j)^* \sim ((1 \, 2) ; z_{n+i}, z_{n+j})$, and hence $((1 \, 2) ; z_{n+i}, z_{n+j}) \sim ((1 \, 2) ; z_i, z_j)$. All other combinations of $x_1, x_2$ will end up repeating one of these relations, and so when we are done all that is left are the $n^2 = n + {{n}\choose{2}}$ generators of the form $((1 \, 2) ; z_i, z_j)$. That is,
\begin{eq*} \bigquotient{\mathbb{Z}_2^{2n} \times \mathbb{Z}^{{2n}\choose{2}}}{\Delta} \quad = \quad \mathbb{Z}_2^{n} \times \mathbb{Z}^{{n}\choose{2}} \end{eq*}

The last step needed in order to find the group $A$ is to quotient out by a $\mathbb{Z}^{{n}\choose{2}}$ subgroup, the one generated by equivalence classes of elements $(\rho(2) ; z_i, z_j )$ for given $\rho(2) \in \pi^{-1}((1 \, 2))$ and $1 \le i < j \le n$. Of course, the underlying permutation map of permutations $\pi^{\mathrm{S}}$ is the identity, so $\rho(2)$ must be $(1 \, 2)$ itself. This gives a nice easy final quotient,
\begin{eq*} \bigquotient{\mathbb{Z}_2^{n} \times \mathbb{Z}^{{n}\choose{2}}}{\mathbb{Z}^{{n}\choose{2}}} \quad = \quad \mathbb{Z}_2^{n}\end{eq*}
and so the underlying monoidal category we are looking for is 
\begin{eq*} L\mathbb{S}_n \quad = \quad \mathbb{Z}^{\ast n} \times_{\mathbb{Z}^n} \mathbb{Z}^{\ast n}  \, \times \, \mathrm{B}\mathbb{Z}_2^{n} \end{eq*}
\end{proof} 

If we are to understand $L\mathbb{S}_n$'s role as a \emph{symmetric} monoidal category, we now just need to use the rest of \cref{freeinvalgc} to find its $\mathrm{ES}$-action, which will dictate which morphisms act as the various symmetries $\beta_{x, y}$. However, this operation too is incredibly simple.

\begin{prop} The action of $L\mathbb{S}_n$ is fully determined by the values
\begin{eq*} \alpha\big( \, (1 \, 2) \, ; \, \mathrm{id}_{z_i}, \mathrm{id}_{z_j} \, \big) \quad = \, 
		\begin{cases}
			\quad \big( \, z_i \otimes z_j, \, z_j \otimes z_i, \, (0, ..., 0) \, \big) & \text{if} \quad i \neq j \\
			\quad \big( \, z_i \otimes z_i, \, z_i \otimes z_i, \, (0,...,0, 1, 0,...,0) \, \big) & \text{if} \quad i = j
		\end{cases} 
\end{eq*}
where the $1$ appears in the $i$th coordinate of $\mathbb{Z}_2^{n}$, along with the identities
\begin{eq*} \begin{array}{rll} 
			\alpha\big( \, (1 \, 2) \, ; \, \mathrm{id}_{z_i}, \mathrm{id}_{z_j} \, \big) & = & \alpha\big( \, (1 \, 2) \, ; \, \mathrm{id}_{z_i^*}, \mathrm{id}_{z_j} \, \big) \\[\medskipamount]
			& = & \alpha\big( \, (1 \, 2) \, ; \, \mathrm{id}_{z_i}, \mathrm{id}_{z_j^*} \, \big) \\[\medskipamount]
			& = & \alpha\big( \, (1 \, 2) \, ; \, \mathrm{id}_{z_i^*}, \mathrm{id}_{z_j^*} \, \big)
		\end{array}
\end{eq*}
\end{prop}
\begin{proof}
We know that all $\mathrm{E}G$-actions obey the conditions
\begin{eq*} \alpha(g; f_1, ..., f_m) \quad = \quad \alpha(g; \mathrm{id}_{y_1}, ..., \mathrm{id}_{y_m}) \circ (f_1 \otimes ... \otimes f_m) \end{eq*}
for all morphisms $f_i: x_i \to y_i$, and
\begin{eq*} \begin{array}{rl}
			& \alpha( \, g \, ; \, \mathrm{id}_{x_1}, ..., \mathrm{id}_{x_{i-1}}, \mathrm{id}_{x_i \otimes x'_{i}}, \mathrm{id}_{x_{i+1}}, ... \mathrm{id}_{x_m} \, ) \\
			= & \alpha\big( \, g \, ; \, \alpha(e_1;\mathrm{id}_{x_1}), ..., \alpha(e_1;\mathrm{id}_{x_{i-1}}), \alpha(e_2;\mathrm{id}_{x_i}, \mathrm{id}_{x'_i}), \alpha(e_1;\mathrm{id}_{x_{i+1}}), ...,  \alpha(e_1;\mathrm{id}_{x_m}) \, \big) \\
			= & \alpha\big( \, \mu(g; e_1, ..., e_1, e_2, e_1, ..., e_1) \, ; \, \mathrm{id}_{x_1}, ..., \mathrm{id}_{x_{i-1}}, \mathrm{id}_{x_i}, \mathrm{id}_{x'_{i}}, \mathrm{id}_{x_{i+1}}, ..., \mathrm{id}_{x_m} \, \big)
		\end{array}
\end{eq*}
for all elements $g \in G$ and objects $x_1, ..., x_m, x'_i$. Hence we can recover all values of $\alpha_{\mathbb{S}_{2n}}$ from those on identities morphisms, and more specifically identities of generators and their inverses. Further, the fact that we can express any $\sigma \in \mathrm{S}$ in terms of $e_1$ and $(1 \, 2)$ via tensor product and group multiplication tells us that the action will also be determined solely by its values on $(1 \, 2)$. Thus the equations in the statement of the proposition really would suffice to fix $\alpha_{L\mathbb{S}_n}$; all we need now is prove that they hold. The sources and targets are easy enough, so we'll focus on the $\mathbb{Z}_2^{n}$ coordinate.

Per \cref{freeinvalgc}, we will start by forming the action morphisms
\begin{eq*} \alpha_{\mathbb{S}_{2n}}\big( \, (1 \, 2) \, ; \, \mathrm{id}_{q^{-1}(z_i)}, \mathrm{id}_{q^{-1}(z_j)} \, \big) \quad = \quad \alpha_{\mathbb{S}_{2n}}\big( \, (1 \, 2) \, ; \, \mathrm{id}_{z_i}, \mathrm{id}_{z_j} \, \big) \end{eq*}
and then find their images under the map $\Psi: \mathrm{S} \times_{\mathbb{N}} \mathbb{N}^{\ast 2n} \to \mathbb{Z}_2^{n}$. However, we just saw in \cref{invsymcat} how this homomorphism is built up as a composite
\begin{eq*} \begin{tikzcd}
 \mathrm{S} \times_{\mathbb{N}} \mathbb{N}^{\ast 2n} \ar[r] & \mathbb{Z}^{2n + (2n)^2} \ar[r] & \mathbb{Z}_2^{2n} \times \mathbb{Z}^{{2n}\choose{2}} \ar[r] & \mathbb{Z}_2^{n} \times \mathbb{Z}^{{n}\choose{2}}  \ar[r] & \mathbb{Z}_2^{n}
\end{tikzcd} \end{eq*}
When $i \neq j$, the equivalence classes of the morphisms $\alpha((1 \, 2);\mathrm{id}_{z_i}, \mathrm{id}_{z_j})$ get sent to zero by the rightmost arrow, whereas the $\alpha((1 \, 2);\mathrm{id}_{z_i}, \mathrm{id}_{z_i})$ are each sent to a different generator of $\mathbb{Z}_2^{n}$, which is denoted by the appropriate $n$-tuple $(0,...,0, 1, 0,...,0)$. 

So now we just need to check the morphisms involving the inverses of generators as well. The $\mathbb{S}_{2n}$ versions of these are
\begin{eq*} \begin{array}{rll}
			\alpha_{\mathbb{S}_{2n}}\big( \, (1 \, 2) \, ; \, \mathrm{id}_{q^{-1}(z_i^*)}, \mathrm{id}_{q^{-1}(z_j)} \, \big) & = & \alpha_{\mathbb{S}_{2n}}\big( \, (1 \, 2) \, ; \, \mathrm{id}_{z_{n+i}}, \mathrm{id}_{z_j} \, \big) \\
			\alpha_{\mathbb{S}_{2n}}\big( \, (1 \, 2) \, ; \, \mathrm{id}_{q^{-1}(z_i)}, \mathrm{id}_{q^{-1}(z_j^*)} \, \big) & = & \alpha_{\mathbb{S}_{2n}}\big( \, (1 \, 2) \, ; \, \mathrm{id}_{z_i}, \mathrm{id}_{z_{n+j}} \, \big) \\
			\alpha_{\mathbb{S}_{2n}}\big( \, (1 \, 2) \, ; \, \mathrm{id}_{q^{-1}(z_i^*)}, \mathrm{id}_{q^{-1}(z_j^*)} \, \big) & = & \alpha_{\mathbb{S}_{2n}}\big( \, (1 \, 2) \, ; \, \mathrm{id}_{z_{n+i}}, \mathrm{id}_{z_{n+j}} \, \big)
		\end{array}
\end{eq*}
But again, we saw in the proof of \cref{invsymcat} that the second-to-last arrow in the above diagram --- the one representing the quotient by $\Delta$ --- will make the equivalence class of $\alpha((1 \, 2);\mathrm{id}_{z_i}, \mathrm{id}_{z_j})$ equal to that of $\alpha((1 \, 2); \mathrm{id}_{z_{n+i}}, \mathrm{id}_{z_{n+j}})$, and inverse to the class containing both $\alpha((1 \, 2); \mathrm{id}_{z_{n+i}}, \mathrm{id}_{z_j})$ and $\alpha((1 \, 2); \mathrm{id}_{z_{n+i}}, \mathrm{id}_{z_j})$. Since every element of the group $\mathbb{Z}_2^{n}$ is self-inverse, this amounts to saying that all of these morphisms are equivalent under $\Psi$, which completes the proof. 
\end{proof}

Thus we see that in the free symmetric monoidal category on $n$ invertible objects, every morphism can be expressed as a composite of tensor products of identities and symmetries maps
\begin{eq*} \beta_{z_i, z_j} \quad = \quad \alpha\big( \, (1 \, 2) \, ; \, \mathrm{id}_{z_i}, \mathrm{id}_{z_j} \, \big) \end{eq*}
Moreover, two parallel morphisms in $L\mathbb{S}_n$ are equal if and only if the number of symmetries from
\begin{eq*} \big\{ \, \beta_{z_i, z_i}, \, \beta_{z_i^*, z_i}, \, \beta_{z_i, z_i^*}, \, \beta_{z_i^*, z_i^*} \, \big\} \end{eq*}
appearing in these two expressions has the same parity, for each $1 \le i \le n$.

\section{Free braided monoidal categories on invertible objects} 

Having successfully understood the symmetric monoidal case, we should now be ready to tackle the very similar world of braided monoidal categories. Indeed, since the only difference between the braid groups $B_n$ and the symmetry groups $\mathrm{S}_n$ is the presence or absence of a self-invertibility condition, the abelian group $L\mathbb{B}_n(I,I)$ is simply the value we would gotten for $L\mathbb{S}_n(I,I)$ if we had never set $((1 \, 2); z_i, z_j) \otimes ((1 \, 2); z_i, z_j) \sim I$.

\begin{prop} \label{invbraidcat} The underlying monoidal category of the free $\mathrm{E}B$-algebra on $n$ invertible objects is
\begin{eq*} L\mathbb{B}_n \quad = \quad \mathbb{Z}^{\ast n} \times_{\mathbb{Z}^n} \mathbb{Z}^{\ast n}  \, \times \, \mathrm{B}(\mathbb{Z}^{n} \times \mathbb{Z}^{{n}\choose{2}} ) \end{eq*}
with component-wise tensor product.
\end{prop}
\begin{proof}
The beginning of this proof is identical to that of \cref{invsymcat}. First, the braid operad $B$ has $B_0 = \{e_0\}$, so we don't need to take a quotient of our action operad. Next, we know from \cref{braidop} that the braid groups $B_m$ are generated by the elementary braids $b_i$, and these are just tensor products
\begin{eq*} b_i \quad = \quad (e_1)^{\otimes (i-1)} \otimes b \otimes (e_1)^{\otimes (m-i-1)} \end{eq*}
where $b$ is the elementary braid of $B_2$. Thus we can generate $\mathrm{B}$ under multiplication and tensor product from the set $\mathcal{B} = \{ e_1, b \}$, and so as before we get
\begin{eq*} \mathbb{Z}^{2n|\mathcal{B}_1| + (2n)^2|\mathcal{B}_2| + ...}  \quad = \quad \mathbb{Z}^{2n + (2n)^2} \end{eq*}
Collapsing the composition of $\mathbb{B}_{2n}$ will then let us eliminate any generators involving $e_1$, since
\begin{eq*} \begin{array}{rllll}
			(e_1; z_i) \, \otimes \, (e_1; z_i) & \sim & (e_1 \cdot e_1; z_i) & = & (e_1; z_i) \\
			\implies \quad (e_1; z_i) & \sim & I
		\end{array}
\end{eq*}
Moreover, the rules governing the elementary braids only state that
\begin{eq*} b_i b_{i+1} b_i \, = \, b_{i+1} b_i b_{i+1}, \quad \quad \quad \quad \quad b_i b_j \, = \, b_j b_i, \quad i+1 < j \end{eq*}
both of which just produce commutativity conditions on the remaining generators. In the latter case this should be obvious, and in the former it follows from the fact that
\begin{longtable}{RL}
			& ( \, b \, ; \, z_i, z_j \, ) \, \otimes \, ( \, b \, ; \, z_i, z_k \, ) \, \otimes \, ( \, b \, ; \, z_j, z_k \, ) \\
			\sim & (e_1; z_k) \otimes ( \, b \, ; \, z_i, z_j \, ) \otimes ( \, b \, ; \, z_i, z_k \, ) \otimes (e_1; z_j) \otimes (e_1; z_i) \otimes ( \, b \, ; \, z_j, z_k \, )\\
			\sim & \big( \, (e_1 \otimes b) \cdot (b \otimes e_1) \cdot (e_1 \otimes b) \, ; \, z_i, z_j, z_k \, \big) \\	
			= & ( \, b_2 b_1 b_2 \, ; \, z_i, z_j, z_k \, ) \\
			= & ( \, b_1 b_2 b_1 \, ; \, z_i, z_j, z_k \, ) \\
			= & \big( \, (b \otimes e_1) \cdot (e_1 \otimes b) \cdot (b \otimes e_1) \, ; \, z_i, z_j, z_k \, \big) \\
			\sim & ( \, b\, ; \, z_j, z_k \, ) \otimes (e_1; z_i) \otimes (e_1; z_j) \otimes ( \, b \, ; \, z_i, z_k \, ) \otimes ( \, b \, ; \, z_i, z_j \, ) \otimes (e_1; z_k) \\
			\sim & ( \, b \, ; \, z_j, z_k \, ) \otimes ( \, b \, ; \, z_i, z_k \, ) \otimes ( \, b \, ; \, z_i, z_j \, )
\end{longtable}
Thus we again arrive at a group $\mathbb{Z}^{(2n)^2}$, whose generators all have the form $(b; z_i, z_j)$. But without the self-invertibility that we had in the symmetric case we are already done with step 1 of \cref{freeinvalgc}, so that
\begin{eq*} \bigquotient{\mathbb{Z}^{2n + (2n)^2}}{\otimes \sim \circ} \quad = \quad \mathbb{Z}^{(2n)^2} \end{eq*}

For step 2, we need quotient out by the subgroup $\Delta$. For exactly the same reasons as in \cref{invsymcat}, we see that it contains the equivalence classes of the elements
\begin{eq*} \begin{array}{rl}
			& \big( \, \mu( \, b \, ; \, e_1, e_2 \, ) \, ; \, z_i , z_j, z_{n+j} \, \big) \, \otimes \, \big( \,  \mu( \, b \, ; \, e_1, e_0 \, ) \, ; \, z_i \, \big)^* \\
			= & \big( \, ( \, e_1 \otimes b \, ) \cdot ( \, b \otimes e_1 \,) \, ; \, z_i , z_j, z_{n+j} \, \big) \, \otimes \, ( \, e_1 \, ; \, z_i \, )^* \\
			\sim & \big( \, ( \, e_1 \otimes b \, ) \cdot ( \, b \otimes e_1 \,) \, ; \, z_i , z_j, z_{n+j} \, \big) \\
			\sim &  (e_1 ; z_j) \otimes ( b ;  z_i, z_{n+j}) \otimes ( b ;  z_i , z_{j} ) \otimes (e_1 ;z_{n+j}) \\
			\sim & (b ;  z_i, z_{n+j}) \otimes (b;  z_i , z_j)
		\end{array}
\end{eq*}
for $1 \le i,j \le n$, as well as ones like
\begin{eq*} (b ;  z_{n+i}, z_j) \otimes (b;  z_i , z_j), \quad \quad \quad (b; z_{n+i}, z_{n+j}) \otimes (b;  z_{n+i} , z_j) \end{eq*}
and so forth. This means that our quotient group will be
\begin{eq*} \bigquotient{\mathbb{Z}^{(2n)^2}}{\Delta} \quad = \quad \mathbb{Z}^{n^2} \end{eq*}
whose generators are the classes $[(b; z_i, z_j)] = [(b; z_{n+i}, z_{n+j})]$, with inverses $[(b; z_{n+i}, z_j)] = [(b; z_i, z_{n+j})]$. Moreover, this group clearly has a $\mathbb{Z}^{{n}\choose{2}}$ subgroup coming from those classes $[(b; z_i, z_j)]$ which have $1 \le i < j \le n$. Thus if we choose $\rho(2) \in \pi^{-1}((1 \, 2))$ to be the elementary braid $b$, the third and final quotient will give
\begin{eq*} \bigquotient{\mathbb{Z}^{n^2}}{\mathbb{Z}^{{n}\choose{2}}} \quad = \quad \mathbb{Z}^{n^2 - {{n}\choose{2}}} \quad = \quad \mathbb{Z}^{n} \times \mathbb{Z}^{{n}\choose{2}} \end{eq*}
and therefore
\begin{eq*} L\mathbb{B}_n \quad = \quad \mathbb{Z}^{\ast n} \times_{\mathbb{Z}^n} \mathbb{Z}^{\ast n}  \, \times \, \mathrm{B}(\mathbb{Z}^{n} \times \mathbb{Z}^{{n}\choose{2}}) \end{eq*}
as a monoidal category.
\end{proof} 

Just to be clear, the first $n$ generators of this group $\mathbb{Z}^{n} \times \mathbb{Z}^{{n}\choose{2}}$ are the images under $q: \mathbb{B}_{2n} \to L\mathbb{B}_n$ of the action morphisms $\alpha_{\mathbb{B}_{2n}}(b;\mathrm{id}_{z_i},\mathrm{id}_{z_i})$, and the other ${n}\choose{2}$ come from the $\alpha_{\mathbb{B}_{2n}}(b;\mathrm{id}_{z_i},\mathrm{id}_{z_j})$ for $i > j$. This seems a little strange at first --- why would $L\mathbb{B}_n$ have this kind of directionality to it, where the $i<j$ generators have been cancelled out but the $i > j$ remain? The important thing to realise is this group is representing the unit endomorphisms $L\mathbb{B}_n(I,I)$, which have the same source and target. By contrast, if $i \neq j$ then $\alpha_{\mathbb{B}_{2n}}(b;\mathrm{id}_{z_i},\mathrm{id}_{z_j})$ will have distinct source and target $z_i \otimes z_j \neq z_j \otimes z_i$, and thus the only way we can add it onto a composite without changing the source and target is to also add in the corresponding $\alpha_{\mathbb{B}_{2n}}(b;\mathrm{id}_{z_j},\mathrm{id}_{z_i})$ somewhere. Therefore we really only need to keep track of one of these two kinds of morphisms, such as all of the ones where $i > j$. This is also reflected in the action of this algebra.

\begin{prop} \label{invbraidact} The action of $L\mathbb{B}_n$ is fully determined by the values
\begin{eq*} \alpha( \, b \, ; \, \mathrm{id}_{z_i}, \mathrm{id}_{z_j} \, ) \quad = \, 
		\begin{cases}
			\quad \big( \, z_i \otimes z_j, \, z_j \otimes z_i, \, (0, ..., 0) \, \big) & \text{if} \quad i < j \\
			\quad \big( \, z_i \otimes z_j, \, z_j \otimes z_i, \, (0,...,0, 1, 0,...,0) \, \big) & \text{if} \quad i \ge j
		\end{cases} 
\end{eq*}
where the $1$ appears in the $i$th coordinate of $\mathbb{Z}^{n}$ when $i=j$, and the $(i,j)$th coordinate of $\mathbb{Z}^{{n}\choose{2}}$ when $i>j$, and also
\begin{eq*} \begin{array}{rll} 
			\alpha( \, b \, ; \, \mathrm{id}_{z_i}, \mathrm{id}_{z_j} \, ) & = & \alpha( \, b \, ; \, \mathrm{id}_{z_i^*}, \mathrm{id}_{z_j} \, )^* \\[\medskipamount]
			& = & \alpha( \, b \, ; \, \mathrm{id}_{z_i}, \mathrm{id}_{z_j^*} \, )^* \\[\medskipamount]
			& = & \alpha( \, b \, ; \, \mathrm{id}_{z_i^*}, \mathrm{id}_{z_j^*} \, )
		\end{array}
\end{eq*}
\end{prop}
\begin{proof}
Similarly to the symmetric case, the fact that any braid $x \in B_m$ can be written as tensor product and group multiple of $e_1$ and $b$ will let us recover all of the values of $\alpha_{L\mathbb{S}_n}$ from just those four families of action morphisms which appear in the proposition. Their sources and targets are clearly correct, so all we need to do examine their $\mathbb{Z}^{n} \times \mathbb{Z}^{{n}\choose{2}}$ coordinates.

We saw in the the proof of \cref{invbraidcat} that under the map
\begin{eq*} \begin{tikzcd}
B \times_{\mathbb{N}} \mathbb{N}^{\ast 2n} \ar[r] & \mathbb{Z}^{2n + (2n)^2} \ar[r] & \mathbb{Z}^{(2n)^2} \ar[r] & \mathbb{Z}^{n^2} \ar[r] & \mathbb{Z}^{n} \times \mathbb{Z}^{{n}\choose{2}} 
\end{tikzcd} \end{eq*}
the action morphisms
\begin{eq*} \alpha_{\mathbb{S}_{2n}}( \, b \, ; \, \mathrm{id}_{q^{-1}(z_i)}, \mathrm{id}_{q^{-1}(z_j)} \, ) \quad = \quad \alpha_{\mathbb{S}_{2n}}( \, b \, ; \, \mathrm{id}_{z_i}, \mathrm{id}_{z_j} \, ) \end{eq*}
are sent to one of the generators of $\mathbb{Z}^{n} \times \mathbb{Z}^{{n}\choose{2}}$ when $i \ge j$, and are sent to zero otherwise. Moreover, we also proved that the morphisms
\begin{eq*} \begin{array}{rll} 
			\alpha_{\mathbb{S}_{2n}}( \, b \, ; \, \mathrm{id}_{q^{-1}(z_i^*)}, \mathrm{id}_{q^{-1}(z_j^*)} \, \big) & = & \alpha_{\mathbb{S}_{2n}}( \, b \, ; \, \mathrm{id}_{z_{n+i}}, \mathrm{id}_{z_{n+j}} \, )
		\end{array}
\end{eq*}
are sent to the exact same generators as the $\alpha_{\mathbb{S}_{2n}}(b;\mathrm{id}_{z_i}, \mathrm{id}_{z_j})$, whilst the corresponding
\begin{eq*} \begin{array}{rll}
			\alpha_{\mathbb{S}_{2n}}( \, b\, ; \, \mathrm{id}_{q^{-1}(z_i^*)}, \mathrm{id}_{q^{-1}(z_j)} \, ) & = & \alpha_{\mathbb{S}_{2n}}( \, b \, ; \, \mathrm{id}_{z_{n+i}}, \mathrm{id}_{z_j} \, ) \\
			\alpha_{\mathbb{S}_{2n}}( \, b \, ; \, \mathrm{id}_{q^{-1}(z_i)}, \mathrm{id}_{q^{-1}(z_j^*)} \, ) & = & \alpha_{\mathbb{S}_{2n}}( \, b \, ; \, \mathrm{id}_{z_i}, \mathrm{id}_{z_{n+j}} \, ) 
		\end{array}
\end{eq*}
are sent to that generator's inverse. Thus by \cref{freeinvalgc}, we obtain the required relations for the action $\alpha_{L\mathbb{S}_n}$.
\end{proof}

To put this in a more categorical perspective, suppose that we decide to call the following kinds of braiding isomorphisms `positive',
\begin{eq*} \begin{array}{rllcrll}
			\beta_{z_i, z_j} & = & \alpha( \, b \, ; \, \mathrm{id}_{z_i}, \mathrm{id}_{z_j} \, ), & \quad \quad \quad & \beta_{z_i^*, z_j}^{-1} & = & \alpha( \, b \, ; \, \mathrm{id}_{z_i^*}, \mathrm{id}_{z_j} \, )^{-1}, \\
			\beta_{z_i, z_j^*}^{-1} & = & \alpha( \, b \, ; \, \mathrm{id}_{z_i}, \mathrm{id}_{z_j^*} \, )^{-1}, & \quad \quad \quad & \beta_{z_i^*, z_j^*} & = & \alpha( \, b \, ; \, \mathrm{id}_{z_i^*}, \mathrm{id}_{z_j^*} \, ) \\
		\end{array}
\end{eq*}
and likewise call their inverses `negative'. Then what \cref{invbraidact} is saying is that in the free braided monoidal category on $n$ invertible objects, parallel morphisms coincide only when the number of positive braidings minus the number of negative braidings they contain is the same.

Something else to notice about $L\mathbb{B}_n$ is that we've actually seen its unit endomorphism group before. Back in \cref{abst} we proved that for any crossed action operad $G$,
\begin{eq*} (s \times t)(L\mathbb{G}_n)^{\mathrm{ab}} \quad = \quad (\mathbb{Z}^{\ast n} \times_{\mathbb{Z}^n} \mathbb{Z}^{\ast n})^{\mathrm{ab}} \quad = \quad \mathbb{Z}^n \times {\mathbb{Z}}^{{n}\choose{2}} \end{eq*}
This means that in the case of the braid operad, we have the unusual identity
\begin{eq*} (s \times t)(L\mathbb{B}_n)^{\mathrm{ab}} \quad \cong \quad L\mathbb{B}_n(I,I) \end{eq*}
What is the significance of this fact? It is not entirely clear, though certainly the isomorphism involved is highly nontrivial. For example the $\mathbb{Z}^n$ subgroup of $(s \times t)(L\mathbb{B}_n)^{\mathrm{ab}}$ has generators representing maps with source and target $z_i \to z_i$, $1 \le i \le n$, while the same generators of $\mathbb{Z}^n \subseteq L\mathbb{B}_n(I,I)$ represent the braidings $\beta_{z_i, z_i} = \alpha( b;\mathrm{id}_{z_i}, \mathrm{id}_{z_i})$. Of course, it is possible that this connection between the groups that make up $\mathrm{Mor}(L\mathbb{B}_n)$ could simply be a conincidence. It would help if we could compare $B$ to another action operad which shares this property --- either another crossed $G$ whose algebra has the same underlying category as the $L\mathbb{B}_n$, or an uncrossed $G$ whose algebra has $L\mathbb{G}_n(I,I) = (\mathbb{Z}^{\ast n})^{\mathrm{ab}} = \mathbb{Z}^{n}$ --- but none of these are currently known to the author.

\section{Free ribbon braided monoidal categories on invertible objects}

The last action operad whose invertible algebras we will calculate explicitly is the ribbon braid operad, $RB$. The details will prove largely similar to those we saw for the braided case in \cref{invbraidcat}, much as the braided case itself was built upon the symmetric case with a few small changes. 

\begin{prop} \label{invribboncat} The underlying monoidal category of the free $\mathrm{E}RB$-algebra on $n$ invertible objects is
\begin{eq*} L\mathbb{RB}_n \quad = \quad \mathbb{Z}^{\ast n} \times_{\mathbb{Z}^n} \mathbb{Z}^{\ast n}  \, \times \, \mathrm{B}(\mathbb{Z}^{n} \times \mathbb{Z}^{n} \times \mathbb{Z}^{{n}\choose{2}}) \end{eq*}
with componentwise tensor product. Moreover, the action of $L\mathbb{RB}_n$ is determined by its restriction to the subcategory $L\mathbb{B}_n \subseteq L\mathbb{RB}_n$, plus the values
\begin{eq*} \alpha( \, t \, ; \, \mathrm{id}_{z_i} \, ) \quad = \quad \big( \, z_i, \, z_i, \, (0,...,0, 1, 0,...,0) \, \big) \end{eq*}
where the $1$ appears in the $i$th coordinate of the copy of $\mathbb{Z}^{n}$ which is not shared with $L\mathbb{B}_n$, and
\begin{eq*} \alpha( \, t \, ; \, \mathrm{id}_{z_i^*} \, ) \quad = \quad \alpha( \, t \, ; \, \mathrm{id}_{z_i} \, )^* \otimes \alpha( \, b \, ; \, \mathrm{id}_{z_i}, \mathrm{id}_{z_i} \, )^{\otimes 2} \end{eq*}
\end{prop}
\begin{proof}
The ribbon braid operad has $RB_0 = \{e_0\}$ and is generated under $\otimes$ and $\cdot$ by the set $\mathcal{RB} = \{ e_1, b, t \}$. Thus our starting point will be the group
\begin{eq*} \mathbb{Z}^{2n|\mathcal{RB}_1| + (2n)^2|\mathcal{RB}_2| + ...} \quad = \quad \mathbb{Z}^{4n + (2n)^2} \end{eq*}
Since the free $\mathrm{E}B$-algebra $\mathbb{B}_{2n}$ is clearly a subcategory of $\mathbb{RB}_{2n}$, when we collapse its composition we will at the least have to quotient out by all of the same relations we did in \cref{invbraidcat}. This will amount to eliminating all of the $e_1$ generators, which will get us down to $\mathbb{Z}^{2n + (2n)^2}$. We also have to collapse our morphisms according to the rules which govern multiplication by twists, but just as with the braids it turns out that these are already implicit in commutativity. For example, in $RB_2$ we have
\begin{eq*} \begin{array}{rrl}
			 ( \, b \, ; \, z_i, z_j \, ) \otimes ( \, t \, ; \, z_i \,) & \sim & ( \, b \, ; \, z_i, z_j \, ) \otimes ( \, t \, ; \, z_i \,) \otimes (e_1; z_j) \\
			& \sim & \big( \, b \cdot (t \otimes e_1) \, ; \, z_i, z_j \, \big) \\	
			& = & ( \, b_1 t_1 \, ; \, z_i, z_j \,) \\
			& = & ( \, t_2 b_1 \, ; \, z_i, z_j \,) \\
			& = & \big( \, (e_1 \otimes t) \cdot b \, ; \, z_i, z_j \, \big) \\
			& \sim & (e_1; z_j) \otimes ( t ; z_i ) \otimes ( b; z_i, z_j ) \\
			& \sim & ( t ;z_i ) \otimes ( b ; z_i, z_j)
		\end{array}
\end{eq*}
Therefore,
\begin{eq*} \bigquotient{\mathbb{Z}^{4n + (2n)^2}}{\otimes \sim \circ} \quad = \quad \mathbb{Z}^{2n + (2n)^2} \end{eq*}
The next step is to quotient out by $\Delta$, and again this will at the very least end up imposing all of the same constraints that we had in the braided case, namely
\begin{eq*} [ \, ( \, b \, ; z_i, z_j \, ) \, ] \quad = \quad [ \, ( \, b \, ; \, z_{n+i}, z_j \, ) \, ]^* \quad = \quad [ \, ( \, b \, ; \, z_i, z_{n+j} \, ) \, ]^* \quad = \quad [ \, ( \, b \, ; \, z_{n+i}, z_{n+j} \, ) \, ]  \end{eq*}
But we also have those elements of $\Delta$ which come from the twist $t$:
\begin{eq*} \begin{array}{rl}
			& \big( \, \mu( \, t \, ; \, e_2 \, ) \, ; \, z_i, z_{n+i} \, \big) \, \otimes \, \big( \,  \mu( \, t \, ; \, e_0 \, ) \, ; \, - \, \big)^* \\
			= & \big( \, ( \, t \otimes t \, ) \cdot b \cdot b \, ; \, z_i, z_{n+i} \, \big) \, \otimes \, (e_0 ; - )^* \\
			= & \big( \, ( \, t \otimes t \, ) \cdot b \cdot b \, ; \, z_i, z_{n+i} \, \big) \\
			\sim &  (t ; z_i) \otimes (t ; z_{n+i}) \otimes (b ;  z_{n+i}, z_i) \otimes (b ;  z_i, z_{n+i}) \\
			\sim & (t ; z_i) \otimes (t ; z_{n+i}) \otimes (b ;  z_i, z_i)^* \otimes (b;  z_i, z_i)^*
		\end{array}
\end{eq*}
Quotienting out by these will allow us to express twists on objects with index greater than $n$ in terms of the other generators,
\begin{eq*} [ \, ( \, t \, ; \, z_{n+i} \, ) \, ] \quad = \quad [ \, ( \, t \, ; \, z_i \, ) \, ]^* \otimes [ \, ( \, b \, ; \, z_i, z_i \, ) \, ]^{\otimes 2} \end{eq*}
and so overall we will get
\begin{eq*} \bigquotient{\mathbb{Z}^{2n + (2n)^2}}{\Delta} \quad = \quad \mathbb{Z}^{n + n^2} \end{eq*}
Then the $\mathbb{Z}^{{n}\choose{2}}$ coming from $\rho(2)$ will be the same as in the braided case, so that
\begin{eq*} \bigquotient{\mathbb{Z}^{n + n^2}}{\mathbb{Z}^{{n}\choose{2}}} \quad = \quad \mathbb{Z}^{n + n^2 - {{n}\choose{2}}} \quad = \quad \mathbb{Z}^{n} \times \mathbb{Z}^{n} \times \mathbb{Z}^{{n}\choose{2}} \end{eq*}
and therefore
\begin{eq*} L\mathbb{RB}_n \quad = \quad \mathbb{Z}^{\ast n} \times_{\mathbb{Z}^n} \mathbb{Z}^{\ast n}  \, \times \, \mathrm{B}(\mathbb{Z}^{n} \times \mathbb{Z}^{n} \times \mathbb{Z}^{{n}\choose{2}}) \end{eq*}

Finally, the same reasoning we have used previously tells us that we can recover the whole action of $L\mathbb{RB}_n$ from just the values
\begin{eq*} \begin{array}{ccccccc}
			\alpha( \, b \, ; \, \mathrm{id}_{z_i}, \mathrm{id}_{z_j} \, ), & \quad \quad & \alpha( \, b \, ; \, \mathrm{id}_{z_i^*}, \mathrm{id}_{z_j} \, ) & \quad \quad & \alpha( \, b \, ; \, \mathrm{id}_{z_i}, \mathrm{id}_{z_j^*} \, ), & \quad \quad & \alpha( \, b \, ; \, \mathrm{id}_{z_i^*}, \mathrm{id}_{z_j^*} \, ) \\
			& & \alpha( \, t \, ; \, \mathrm{id}_{z_i} \, ) & \quad \quad & \alpha( \, t \, ; \, \mathrm{id}_{z_i^*} \, ) & &
		\end{array}
\end{eq*}
The process for working out the first four is no different than before, which means that $\alpha_{L\mathbb{RB}_n}$ acts on the braids in the exact same ways that $\alpha_{L\mathbb{B}_n}$ does. Furthermore, it is not hard to see that
\begin{eq*} \alpha( \, t \, ; \, \mathrm{id}_{z_i} \, ) \quad = \quad \big( \, z_i, \, z_i, \, (0,...,0, 1, 0,...,0) \, \big) \end{eq*}
where the $1$ corresponds to the $(t ; z_i)$ generator of $\mathbb{Z}^{n} \times \mathbb{Z}^{n} \times \mathbb{Z}^{{n}\choose{2}}$, and also that the process of quotienting by $\Delta$ will translate to 
\begin{eq*} \alpha( \, t \, ; \, \mathrm{id}_{z_i^*} \, ) \quad = \quad \alpha( \, t \, ; \, \mathrm{id}_{z_i} \, )^* \otimes \alpha( \, b \, ; \, \mathrm{id}_{z_i}, \mathrm{id}_{z_i} \, )^{\otimes 2} \end{eq*}
as required.
\end{proof}



%\begin{prop} Cactus group case
%\end{prop}  
    

  
 \begin{thebibliography}{9}

\bibitem{lpac}
J. Adámek and J. Rosicky.
\textit{ Locally presentable and accessible categories},
London Mathematical Society Lecture Note Series 189 (1994), Cambridge University Press, Cambridge  

\bibitem{hda5}
John C. Baez; Aaron D. Lauda.
\textit{Higher-Dimensional Algebra V: 2-Groups},
Theory and Applications of Categories 12 (2004), 423-491.

\bibitem{2monad}
R.Blackwell; G.M.Kelly; A.J.Power.
\textit{Two-dimensional monad theory},
Journal of Pure and Applied Algebra 59, Issue 1 (1989), Pages 1-41

\bibitem{ptncld1}
Eugenia Cheng; Nick Gurski.
\textit{The periodic table of n-categories for low dimensions I: degenerate categories and degenerate bicategories},
Categories in Algebra, Geometry and Mathematical Physics, Contemp. Math. 431, 2005, 143-164

\bibitem{ptncld2}
Eugenia Cheng; Nick Gurski.
\textit{The periodic table of n-categories for low dimensions II: degenerate tricategories},
Cahiers de Topologie et Géométrie Différentielle Catégoriques, Volume 52 (2011) no. 2 , p. 82-125

\bibitem{ogge}
Alexander S. Corner; Nick Gurski.
\textit{Operads with general groups of equivariance, and some 2-categorical aspects of operads in Cat},
Preprint available at http://arxiv.org/abs/1312.5910.

\bibitem{semi}
Raouf Doss.
\textit{Sur l'immersion d'un semi-groupe dans un groupe},
Bulletin des Sciences Mathématiques, (2) 72, (1948). 139–150. 

\bibitem{eckhil}
Eckmann, B.; Hilton, P. J. 
\textit{Group-like structures in general categories. I. Multiplications and comultiplications},
Mathematische Annalen, 145 (3), pp. 227–255

\bibitem{braidedop}
Z. Fiedorowicz; M. Stelzer; R.M. Vogt.
\textit{Homotopy Colimits of Algebras Over
Cat-Operads and Iterated Loop Spaces},
Advances in Mathematics 248, 2013, 1089-1155
  
\bibitem{aft}
Peter Freyd.
\textit{Abelian Categories}, 
Harper \& Row, New York (1964) 

\bibitem{operadborel} 
Nick Gurski. 
\textit{Operads, tensor products, and the categorical Borel construction},
Preprint available at http://arxiv.org/abs/1508.04050.

\bibitem{adjoint}
Daniel Kan.
\textit{Adjoint functors},
Transactions of the American Mathematical Society 87, No. 2 (1958), 294-329 

\bibitem{monad2}
Steve Lack; Ross Street.
\textit{The formal theory of monads II},
Journal of Pure and Applied Algebra 175 (2002), No. 1-3, 243–265

\bibitem{aif}
F. William Lawvere.
\textit{Adjointness in foundations},
Dialectica 23 (1969)

\bibitem{hohc}
Tom Leinster.
\textit{Higher Operads, Higher Categories}
London Mathematical Society Lecture Note Series 298, Cambridge University Press, Cambridge, 2004.

\bibitem{bct}
Tom Leinster.
\textit{Basic Category Theory},
Cambridge Studies in Advanced Mathematics, Vol. 143, Cambridge University Press, Cambridge, 2014.

\bibitem{cwm}
Saunders Mac Lane.
\textit{Categories for the Working Mathematician},
Springer Science+Business Media, 1971

\bibitem{immer1}
Malcev, A.
\textit{On the Immersion of an Algebraic Ring into a Field},
Mathematische Annalen 113 (1937): DCLXXXVI-DCXCI. 

\bibitem{immer2}
Malcev, A.
\textit{On the immersion of associative systems into groups},
Matematicheskii Sbornik 6 (1939), no.2, 331-336

\bibitem{gils}
J. P. May.
\textit{The Geometry of Iterated Loop Spaces},
Springer-Verlag Berlin Heidelberg 1972

\bibitem{grouptheory}
Derek J. S. Robinson.
\textit{A Course in the Theory of Groups},
Graduate Texts in Mathematics 80 (1996), Springer

\bibitem{ribbon2}
Paolo Salvatore; Nathalie Wahl. 
\textit{Framed discs operads and Batalin-Vilkovisky algebras},
Q. J. Math., 54(2):213–231, 2003.

\bibitem{graphicalmon}
Peter Selinger.
\textit{A survey of graphical languages for monoidal categories},
Springer Lecture Notes in Physics 813, pp. 289-355, 2011

\bibitem{monad1}
Ross Street.
\textit{The formal theory of monads},
Journal of Pure and Applied Algebra 2 (1972), 149–168 

\bibitem{picard}
K. H. Ulbrich.
\textit{Group cohomology for Picard categories},
Journal of Algebra, 91 (1984), pp. 464-498 

\bibitem{ribbon1}
Nathalie Wahl.
\textit{Ribbon braids and related operads},
Thesis (Ph.D.)–University of Oxford (2001) 
 
\bibitem{groupop}
Wenbin Zhang.
\textit{Group Operads and Homotopy Theory},
Thesis (Ph.D.)–National University of Singapore (2012)

\end{thebibliography}



\end{document}