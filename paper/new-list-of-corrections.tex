These are the things we don't like very much:

Section 1:
1. Obviously we still need a real intro, one that takes into account later work explicitly
2. I think the Conventions 1.1 could probably use with some more entries, maybe a better location

Section 2:
1. I think some of sec 2.1 is in an odd order; also if we are thinking that this is on the expository side, it doesn't explain much
2. Example 2.12 makes it look like we won't talk about ribbon braids, but then we do in 2.13; should just reorganize these examples
3. Is the proof of 2.21 too long/janky?
4. 2.22 has a notational issue since we have both $T$ (operad) and $\mathbf{T}$ (action operad) notation
5. I think 2.23 is overcomplicating things, since pure = in the kernel.
6. Should 2.24 be made into a proposition and expanded a little?
7. Seems like 2.25 should have come earlier
8. Check 2.32
9. Should 2.35 come earlier?
10. Is the last part of the proof of 2.36 right? seems to contradict 2.35
11. Should maybe give an example of a presentation of an action operad 
12. Theorem 2.27 - $\delta_{1;n}$ in axiom 5 should go $\Lambda(1) \rightarrow \Lambda(n)$? So wouldn't be the identity? What do we actually mean here? Should it be $\delta_{1;1} \colon \Lambda(1) \rightarrow \Lambda(1)$ being the identity?


Notes from our chat/logic workshop:

intro rewrite, thank the LMS 
notational/terminological conventions: does this cover everything we need? move operad/beta/delta discussion to later; suggestion to use bullet for coequalizer notation
each section needs a solid intro

2.1: is symmetric to non-sym to braided the order we want? how about non-sym, then triangle picture, then symmetric, then examples, then braided

2.2: redo examples; some minor text fixes plus refs; remark 2.11; explain non-examples (with non-example environment); put all the examples together, with uniform presentation; explain pi notation somewhere

2.3: title is bad; proof of prop 2.21 is too long, maybe use 2.27; more remark 2.24 to some intro; proof of 2.25 - explain each line on the right rather than after calculation; 

2.4: give a presentation for \Sigma

3: better intro so it doesn't just feel like the same thing again

3.1: extra ( in def 3.1; fix example env's in the same way as before; too long and incoherent; introduce ``cocomplete symmetric monoidal category'' terminology; clean up the equivariance discussion in 3.6/3.7; endo operad should be a definition; rem 3.18 #3 isn't worded well; explain how 3.22/3.23/3.24 get used, forward refs, maybe an example

3.2: where should this section live? maybe with clubs?

4: better/more intro

4.1: a few coequalizers; reword conv 4.2 slightly to emphasize what is new (2-d); refs for weak maps of 2-monads or maybe not, depending on if we can easily find one

4.2: three topics with unequal length, maybe split into finitary+coherence (with more explanation) as one and then cartesian as its own section

5: once again needs an intro; maybe this should be sections 5.1, 3.2, 5.3, 6.4 (in some order)

5.1: need refs for 5.2, 5.3; put with section 3.2??

5.2: short, not completely satisfying - do we use the \Lambda_{\infty} stuff? maybe go in the intro??

5.3: this has more presentations, how to incorporate with other parts? make a presentation, eg the presentation of symmetric monoidal categories via the presentation of the action operad \Sigma; put the club part of section 6.2 in here as another example; check that the cartesian stuff in 4.2 and the clubs stuff match, and give the same answers for the composition product; mysterious note: one composition product is a pullback, one is a coend, but they agree, like for spans/profunctors

6: needs an intro

6.1: remove??? make a real section

6.2: really just coboundary categories; where does the beginning part of this go? move most of this to the end of section 2 for another example; ``Alex makes balloon cacti''

6.3: maybe should go in section 4??

6.4: should this be combined with 3.2?


