%!TEX root = ../operads_paper.tex
\section{Introduction}
QQQ Needs an intro. Might be something salvageable from the original papers.

Original paper intro:

Operads were defined by May \cite{maygeom} in the early 70's to provide a convenient tool to approach problems in algebraic topology, notably the question of when a space $X$ admits an $n$-fold delooping $Y$ so that $X \simeq \Omega^{n}Y$.  An operad, like an algebraic theory \cite{lawvere-thesis}, is something like a presentation for a monad or algebraic structure.  The theory of operads has seen great success, and we would like to highlight two reasons.  First, operads can be defined in any suitable symmetric monoidal category, so that there are operads of topological spaces, of chain complexes, of simplicial sets, and of categories, to name a few examples.  Moreover, symmetric (lax) monoidal functors carry operads to operads, so we can use operads in one category to understand objects in another via transport by such a functor.  Second, operads in a fixed category are highly flexible tools.  In particular, the categories listed above all have some inherent notion of ``homotopy equivalence'' which is weaker than that of isomorphism, so we can study operads which are equivalent but not isomorphic.  These tend to have algebras which have similar features in an ``up-to-homotopy'' sense but very different combinatorial or geometric properties arising from the fact that different objects make up these equivalent but not isomorphic operads.

Operads in the category $\mb{Cat}$ of small categories have a unique flavor arising from the fact that $\mb{Cat}$ is not just a category but a 2-category.  These 2-categorical aspects have not been widely treated in the literature, although a few examples can be found.  Lack \cite{lack-cod} mentions braided $\mb{Cat}$-operads (the reader new to braided operads should refer to the work of Fiedorowicz \cite{fie-br}) in his work on coherence for 2-monads, and Batanin \cite{bat-eh} uses lax morphisms of operads in $\mb{Cat}$ in order to define the notion of an internal operad.  But aside from a few appearances, the basic theory of operads in $\mb{Cat}$ and their 2-categorical properties seems missing.  This paper was partly motivated by the need for such a theory to be explained from the ground up.

There were two additional motivations for the work in this paper.  In thinking about coherence for monoidal functors, the first author was led to a general study of algebras for multicategories internal to $\mb{Cat}$.  These give rise to 2-monads (or perhaps pseudomonads, depending on how the theory is set up), and checking abstract properties of these 2-monads prompts one to consider the simpler case of operads in $\mb{Cat}$ instead of multicategories.  The other motivation was from the second author's attempt to understand the interplay between operads in $\mb{Cat}$, operads in $\mb{Top}$, and the passage from (bi)permutative categories to $E_{\infty}$ (ring) spaces.  The first of these motivations raised the issue of when operads in $\mb{Cat}$ are cartesian, while the second led us to consider when an operad in $\mb{Cat}$ possesses a pseudo-commutative structure.

While considering how to best tackle a general discussion of operads in $\mb{Cat}$, it became clear that restricting attention to the two most commonly used types of operads, symmetric and non-symmetric operads, was both short-sighted and unnecessary.  Many theorems apply to both kinds of operads at once, with the difference in proofs being negligible; in fact, most of the arguments which applied to the symmetric case seemed to apply to the case of braided operads as well.   This led us to the notion of an action operad $\mb{G}$, and then to a definition of $\mb{G}$-operads.  In essence, this is merely the general notion of what it means for an operad $P = \{ P(n) \}_{n \in \N}$ to have groups of equivariance $\mb{G} = \{ G(n) \}_{n \in \N}$ such that $G(n)$ acts on $P(n)$.  Choosing different natural families of groups $\mb{G}$, we recover known variants of the definition of operad. \\ \begin{center}
\begin{tabular}{c|c}
Groups $\mb{G}$ & Type of operad  \\ \hline
Terminal groups & Non-symmetric operad \\
Symmetric groups & Symmetric operad \\
Braid groups & Braided operad \\
\end{tabular} \\ \end{center}
These definitions have appeared, with minor variations, in two sources of which we are aware.  In Wahl's thesis \cite{wahl-thesis}, the essential definitions appear but not in complete generality as she requires a surjectivity condition.  Zhang \cite{zhang-grp} also studies these notions\footnote{Zhang calls our action operad a \textit{group operad}.  We dislike this terminology as it seems to imply that we are dealing with an operad in the category of groups, which is not the case unless all of the maps $\pi_{n}:G(n) \rightarrow \Sigma_{n}$ are zero maps.}, once again in the context of homotopy theory, but requires the  superfluous condition that $e_{1} = \textrm{id}$ (see Lemma \ref{calclem}).

This paper consists of the following.  In Section 1, we give the definition of an action operad $\mb{G}$ and a $\mb{G}$-operad.  We develop this definition abstractly so as to apply it in any suitable symmetric monoidal category.   It is standard to express operads as monoids in a particular functor category using a composition tensor product.  In order to show that our $\mb{G}$-operads fit into this philosophy, we must work abstractly and use the calculus of coends together with the Day convolution product \cite{day-thesis}.  The reader uninterested in these details can happily skip them, although we find the route taken here to be quite satisfactory in justifying the axioms for an action operad $\mb{G}$ and the accompanying notion of $\mb{G}$-operad.  Many of our calculations are generalizations of those appearing in work of Kelly \cite{kelly-op}, although there are slight differences in flavor between the two treatments.
%Kelly:  On the operads of JP May

Section 2 works through the basic 2-categorical aspects of operads in $\mb{Cat}$.  We explain how every operad gives rise to a 2-monad, and show that all of the various 1-cells between algebras of the associated 2-monad correspond to the obvious sorts of 1-cells one might define between algebras over an operad in $\mb{Cat}$.  Similarly, we show that the algebra 2-cells, using the 2-monadic approach, correspond to the obvious notion of transformation one would define using the operad.

Section 3 studies three basic 2-categorical properties of an operad, namely the property of being finitary, the property of being 2-cartesian, and the coherence property.  The first of these always holds, as a simple calculation shows.  The second of these turns out to be equivalent to the action of $G(n)$ on $P(n)$ being free for all $n$, at least up to a certain kernel.  In particular, our characterization clearly shows that every non-symmetric operad is 2-cartesian, and that a symmetric operad is 2-cartesian if and only if the symmetric group actions are all free.  (It is useful to note that a 2-monad on $\mb{Cat}$ is 2-cartesian if and only if the underlying monad on the category of small categories is cartesian in the usual sense as the (strict) 2-pullback of a diagram is the same as its pullback.)  The third property is also easily shown to hold for any $\mb{G}$-operad on $\mb{Cat}$ using a factorization system argument due to Power \cite{power-gen}.

Section 4 then goes on to study the question of when a $\mb{G}$-operad $P$ admits a pseudo-commutative structure.  Such a structure provides the 2-category of algebras with a richer structure that includes well-behaved notions of tensor product, internal hom, and multilinear map that fit together much as the analogous notions do in the category of vector spaces.  When $P$ is contractible (i.e., each $P(n)$ is equivalent to the terminal category), this structure can be obtained from a collection of elements $t_{m,n} \in G(mn)$ satisfying certain properties.  In particular, we show that every contractible symmetric operad is pseudo-commutative, and we prove that there exist such elements $t_{m,n} \in Br_{mn}$ so that every contractible braided operad is pseudo-commutative as well (in fact in two canonical ways).  Thus Section 4 can be seen as a continuation, in the operadic context, of the work in \cite{HP}, and in particular the ``geometric'' proof of the existence of a pseudo-commutative structure for braided strict monoidal categories demonstrates the power of being able to change the groups of equivariance.

The authors would like to thank John Bourke, Martin Hyland, Tom Leinster, and Peter May for various conversations which led to this paper.  While conducting this research, the second author was supported by an EPSRC Early Career Fellowship. 

Original Borel intro:


Categories of interest are often monoidal: sets, topological spaces, and vector spaces are all symmetric monoidal, while the category of finite ordinals (under ordinal sum) is merely monoidal.  But other categories have more exotic monoidal structures.  The first such type of structure discovered was that of a braided monoidal category.  These arise in categories whose morphisms have a geometric flavor like cobordisms embedded in some ambient space \cite{js}, in  categories produced from double loop spaces \cite{fie-br}, and categories of representations over objects like quasitriangular (or braided) bialgebras \cite{street-quantum} .  Another such exotic monoidal structure is that of a coboundary category, arising in examples from the representation theory of quantum groups \cite{drin-quasihopf}.

Going back to the original work of May on iterated loop spaces \cite{maygeom}, operads were defined in both symmetric and nonsymmetric varieties.  But Fiedorowicz's work on double loop spaces \cite{fie-br} showed that there was utility in considering another kind of operad, this time with braid group actions instead of symmetric group actions.  There is a clear parallel between these definitions of different types of operads and the definitions of different kinds of monoidal category, with each given by some general schema in which varying an $\mathbb{N}$-indexed collection of groups produced the types of operads or monoidal categories seen in nature.  Building on the work in \cite{cg}, the goal of this paper is to show that this parallel can be upgraded from an intuition to precise mathematics using the notion of action operad.

An action operad $\mb{\Lambda}$ is an operad which incorporates all of the essential features of the operad of symmetric groups.  Thus $\Lambda(n)$ is no longer just a set, but instead also has a group structure together with a map $\pi_{n}:\Lambda(n) \to \Sigma_{n}$.  Operadic composition then satisfies an additional equivariance condition using the maps $\pi_n$ and the group structures.  Each action operad $\mb{\Lambda}$ produces a notion of $\mb{\Lambda}$-operad which encodes equivariance conditions using both the groups $\Lambda(n)$ and the maps $\pi_n$.  Examples include the symmetric groups, the terminal groups (giving nonsymmetric operads), the braid groups (giving braided operads), and the $n$-fruit cactus groups \cite{hk-cobound} (giving a new notion of operad one might call cactus operads).  Using a formula resembling the classical Borel construction for spaces with a group action, we can produce from any action operad $\mb{\Lambda}$ a notion of $\mb{\Lambda}$-monoidal category, in which the group $\Lambda(n)$ acts naturally on $n$-fold tensor powers of any object.  Thus the categorical Borel construction embeds action operads into a category of monads on $\mb{Cat}$, and we characterize the image of this embedding as those monads describing monoidal structures of a precise kind.

The paper is organized into the following sections.  Section 1 reviews the definition of an action operad, and defines the categorical Borel construction on them.  The key result, which reappears in proofs throughout the paper, is \cref{thm:charAOp}, characterizing action operads in terms of two new operations mimicking the block sum of permutations and the operation which takes a permutation of $n$ letters and produces a new permutation on $k_1 + k_2 + \cdots + k_n$ letters by permuting the blocks of $k_i$ letters.  In Section 2, we use this characterization and Kelly's theory of clubs \cite{kelly_club1, kelly_club0, kelly_club2} to embed action operads into monads on $\mb{Cat}$ and determine the essential image of this embedding.  Section 3 gives a construction of the free action operad from a suitable collection of data, and relates this to how clubs can be described using generators and relations.  The results of Sections 2 and 3 show that the definitions of symmetric monoidal category or coboundary category, for example, correspond to the action operad constructed from the corresponding free symmetric monoidal or coboundary category on one object; these and other examples appear in detail in Section 4.  Section 5 then extends the definition of $\mb{\Lambda}$-operad to that of $\mb{\Lambda}$-multicategory and shows that these arise abstractly via a Kleisli construction.

The author would like to thank Alex Corner and Ed Prior for conversations contributing to this research.

This research was supported by EPSRC 134023.

Further acknowledgements:
Alex needs to thank the LMS for a Research Reboot grant. Anybody else we've talked to about these things since their inception? Angelica? Niles? Dan Graves. Nathaniel Arkor.






\begin{conv}\label{conv1}
We adopt the following conventions throughout.
\begin{enumerate}
\item\label{conv:symm_sigma} $\Sigma_{n}$ is the symmetric group on $n$ letters, and $B_{n}$ is the braid group on $n$ strands.
\item\label{conv:g-action} For a group $G$, a right $G$-action on a set $X$ will be denoted $(x,g) \mapsto x \cdot g$. We will use both $\cdot$ and concatenation to represent multiplication in a group.
\item\label{conv:e_identity} The symbol $e$ will generically represent an identity element in a group. If we are considering a set of groups $\{ \Lambda(n) \}_{n \in \N}$ indexed by the natural numbers, then $e_{n}$ is the identity element in $\Lambda(n)$. We will often drop the subscripts and just write $e$ when the index $n$ in $\Lambda(n)$ is either clear from context or unimportant to the argument at hand. Occasionally we will write $\Lambda_n$ in place of $\Lambda(n)$, especially in diagrams.
\item\label{conv:coeq} We will often be interested in elements of a product of the form
\[
A \times B_{1} \times \cdots \times B_{n} \times C
\]
(or similar, for example without $C$). We will write elements of this set as $(a; b_{1}, \ldots, b_{n}; c)$, and in the case that we need equivalence classes of such elements they will be written as $[a; b_{1}, \ldots, b_{n}; c]$. This will often be the case when we are interested in the coequalizer of left and right group actions in the following sense. A coequalizer of maps
    \[
        \xy
            (0,0)*+{A \times G \times B}="00";
            (30,0)*+{A \times B}="10";
            (60,0)*+{\coeqb{A}{B}{G}}="20";
            {\ar@<1ex>^{\lambda} "00" ; "10"};
            {\ar@<-1ex>_{\rho} "00" ; "10"};
            {\ar^{\varepsilon} "10" ; "20"};
        \endxy
    \]
will be written as $\coeqb{A}{B}{G}$, where $\rho$ represents a right action of $G$ on $A$, and $\lambda$ a left action of $G$ on $B$. This is similar to the notation often used to denote pullbacks, however we find in this work that no confusion arises from using notation in this way.
\item\label{conv:beta_delta} In the following definitions of operads, we define operad multiplication as a function
  \[
    \mu \colon O(n) \times O(k_1) \times \ldots O(k_n) \rightarrow O(k_1 + \ldots + k_n)
  \]
for each $n$, $k_1$, $\ldots$, $k_n$, and we use the following two functions as a shorthand for two commonly occuring instances of such. First we define a function
  \[
    \beta \colon O(k_1) \times \ldots \times O(k_n) \rightarrow O(k_1 + \ldots + k_n)
  \]
for each $n$, $k_1$, $\ldots$, $k_n$, which takes elements $\tau_1 \in O(k_1)$, $\ldots$, $\tau_n \in O(k_n)$ and produces the element
  \[
    \beta(\tau_1, \ldots, \tau_n) = \mu(e_n; \tau_1, \ldots, \tau_n).
  \]
We think of this element as the block sum of the elements $\tau_1$, $\ldots$, $\tau_n$. We also define a function
  \[
    \delta_{n;k_1,\ldots,k_n} \colon O(n) \rightarrow O(k_1 + \ldots + k_n)
  \]
for each $n$, $k_1$, $\ldots$, $k_n$, which takes an element $\sigma \in O(n)$ and produces the element
  \[
    \delta(\sigma) = \mu(\sigma;e_{k_1}, \ldots, e_{k_n}).
  \]

In the particular case of the symmetric groups, these are functions
  \[
    \beta \colon \Sigma_{k_1} \times \Sigma_{k_n} \rightarrow \Sigma_{k_1 + \ldots + k_n}
  \]
and
  \[
    \delta_{n;k_1,\ldots,k_n} \colon \Sigma_{n} \rightarrow \Sigma_{k_1 + \ldots + k_n}.
  \]
In the case of $\beta$ we form the block sum permutation $\beta(\tau_1,\ldots,\tau_n)$ which permutes the first $k_{1}$ elements according to $\tau_{1}$, the next $k_{2}$ elements according to $\tau_{2}$ and so on; this is an element of $\Sigma_{k_{1} + \cdots + k_{n}}$. For $\delta$ we take the permutation $\delta(\sigma) \in \Sigma_{k_{1} + \cdots + k_{n}}$ to be that which permutes the $n$ different blocks $1$ through $k_{1}$, $k_{1}+1$ through $k_{1} + k_{2}$, and so on, according to the permutation $\sigma \in \Sigma_{n}$. We expand on this at various points, including in \cref{Rem:perm_matrices}.
\item\label{conv:perm_shorthand} Throughout we will be using maps $\pi_n \colon O(n) \rightarrow \Sigma_n$, where $O(n)$ is the object of $n$-ary operations of an operad $O$ and $\Sigma_n$ is the symmetric group on $n$ elements. The map $\pi_n$ in each case will represent a form of `underlying permutation' of each element, which we will then use to act on operad multiplication. As the notation starts to become cumbersome, we will often write $\sigma^{-1}(i)$ which should be read as $\pi_n(\sigma)^{-1}(i)$, where $\sigma \in O(n)$.
\item\label{conv:op_superscript} We adopt the convention that if an equation requires using operadic composition in more than one operad, we will indicate this by a superscript on each instance of $\mu$ unless it is entirely clear from context. This can be seen, for example, in \cref{Defi:op_map}.
\end{enumerate}
\end{conv}