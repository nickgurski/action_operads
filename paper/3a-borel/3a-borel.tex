%!TEX root = ../operads_paper.tex
\section{The Borel construction for action operads}
\subsection{The Borel construction for action operads}


The classical Borel construction is a functor from $G$-spaces to spaces, sending a $G$-space $X$ to $\coeqb{EG}{X}{G}$. Our goal in this section is to use the formal description of the Borel construction to construct some special operads in $\mb{Cat}$.  We start by reviewing the analogues of the functors $E, B \colon \mb{Grp} \rightarrow \mb{Top}$ now taking values in $\mb{Cat}$.

\begin{Defi}\label{Defi:e_b}
  \begin{enumerate}
    \item Let $X$ be a set. We define the \textit{translation category} $EX$ to have objects the elements of $X$ and morphisms consisting of a unique isomorphism between any two objects.
    \item Let $G$ be a group. The category $BG$ has a single object $*$, and hom-set $BG(*,*) = G$ with composition and identity given by multiplication and the unit element in the group, respectively.
  \end{enumerate}
\end{Defi}

\begin{Defi}
A functor $F \colon X \rightarrow Y$ is an \emph{isofibration} if given $x \in X$ and an isomorphism $f\colon y \xrightarrow{\cong} F(x)$ in $Y$, then there exists an isomorphism $g \colon y \cong x$ in $X$ such that $F(g) = f$.
\end{Defi}

\begin{prop}
There exists a natural transformation $p \colon EU \Rightarrow B$, where $U$ is the underlying set of a group, which is pointwise an isofibration. Applying the classifying space functor to the component $p_{G}$ gives a universal principal $G$-bundle.
\end{prop}
\begin{proof}
Given a group $G$, $p_{G} \colon EUG \rightarrow BG$ sends every object of $EUG$ to the unique object of $BG$. The unique isomorphism $g \rightarrow  h$ in $EUG$ is mapped to $hg^{-1} \colon * \rightarrow *$. It is easy to directly check that this is an isofibration, as well as to see that the classifying spaces of $EUG$ and $BG$ are the spaces classically known as $EG,BG$, with $|p_{G}|$ being the standard universal principal $G$-bundle.
\end{proof}

We will also need the functors $E, B$ defined for more than just a single set or group, in particular for the sets or groups which make up an operad and are indexed by the natural numbers.

\begin{nota}\label{nota:e_b}
Let $S$ be a set which we view as a discrete category.
  \begin{enumerate}
    \item For any functor $F \colon S \rightarrow \mb{Sets}$, let $EF$ denote the composite $E \circ F \colon S \rightarrow \mb{Cat}$; we often view $F$ as an indexed set $\{ F(s) \}$, in which case $EF$ is the indexed category $\{ EF(s) \}$.
    \item For any functor $F \colon S \rightarrow \mb{Grp}$, let $BF$ denote the composite $B \circ F \colon S \rightarrow \mb{Cat}$; we often view $F$ as an indexed group $\{ F(s) \}$, in which case $BF$ is the indexed category $\{ BF(s) \}$.
  \end{enumerate}
\end{nota}

The following lemma is a straightforward verification.

\begin{lem}\label{symmoncor}
The functor $E \colon \mb{Sets} \rightarrow \mb{Cat}$ is right adjoint to the set of objects functor. Therefore $E$ preserves all limits, and in particular is a symmetric monoidal functor when both categories are equipped with their cartesian monoidal structures.
\end{lem}

We are additionally interested in $\Lambda$-operads in $\mb{Cat}$ (or other cocomplete symmetric monoidal categories in which the tensor product preserves colimits in each variable). While the definition above gives the correct notion of a $\Lambda$-operad in $\mb{Cat}$ if we interpret the two equivariance axioms to hold for both objects and morphisms, it is useful to give a purely diagrammatic expression of these axioms. In the diagrams below, expressions of the form $G \times C$ for a group $G$ and category $C$ mean that the group $G$ is to be treated as a discrete category. This follows the standard method of how one expresses group actions in categories other than $\mb{Sets}$ using a copower. Thus the diagrams below are the two equivariance axioms given in \cref{Defi:lamop} expressed diagrammatically, where $K = k_1 + \ldots + k_n$.
    %% Expanded diagram
  % \[
  %   \xy
  %     (0,0)*+{\scriptstyle P(n) \times P(k_{1}) \times \cdots \times P(k_{n}) \times \Lambda(k_{1}) \times \cdots \times \Lambda(k_{n}) } ="00";
  %     (0,-15)*+{\scriptstyle P(\underline{k}) \times \Lambda(\underline{k}) } ="01";
  %     (60,0)*+{\scriptstyle P(n) \times P(k_{1}) \times \Lambda(k_{1}) \times \cdots \times P(k_{n}) \times  \Lambda(k_{n}) } ="20";
  %     (60,-15)*+{\scriptstyle P(n) \times P(k_{1}) \times \cdots \times P(k_{n}) } ="21";
  %     (30, -25)*+{\scriptstyle P(\underline{k}) } ="12";% diagram
  %     {\ar^{\cong} "00" ; "20"};
  %     {\ar^{1 \times \alpha_{k_1} \times \cdots \times \alpha_{k_n}} "20" ; "21"};
  %     {\ar^{\mu^P} "21" ; "12"};
  %     {\ar_{\mu^P \times \mu^\Lambda(e;-)} "00" ; "01"};
  %     {\ar_{\alpha_{\underline{k}}} "01" ; "12"};
  %   \endxy
  % \]
  \[
    \xy
      (0,0)*+{\scriptstyle P(n) \times \left(\prod_{i=1}^n P(k_i)\right) \times \left(\prod_{i=1}^n \Lambda(k_i)\right)} ="00";
      (0,-15)*+{\scriptstyle P(K) \times \Lambda(K) } ="01";
      (60,0)*+{\scriptstyle P(n) \times \left(\prod_{i=1}^n \left(P(k_{i}) \times \Lambda(k_{i})\right)\right)} ="20";
      (60,-15)*+{\scriptstyle P(n) \times \left(\prod_{i=1}^n P(k_i)\right) } ="21";
      (30, -25)*+{\scriptstyle P(K) } ="12";% diagram
      {\ar^{\cong} "00" ; "20"};
      {\ar^{1 \times \prod_{i=1}^n \alpha_i} "20" ; "21"};
      {\ar^{\mu^P} "21" ; "12"};
      {\ar_{\mu^P \times \mu^\Lambda(e;-)} "00" ; "01"};
      {\ar_{\alpha_{K}} "01" ; "12"};
    \endxy
  \]
    %% Expanded diagram
  % \[
  %   \xy
  %     (0,0)*+{\scriptstyle P(n) \times \Lambda(n) \times P(k_{1}) \times \cdots \times P(k_{n}) } ="00";
  %     (0,-10)*+{\scriptstyle P(n) \times \Lambda(n) \times \Lambda(n) \times P(k_{1}) \times \cdots \times P(k_{n}) } ="01";
  %     (0,-20)*+{\scriptstyle P(n) \times \Lambda(n) \times P(k_{1}) \times \cdots \times P(k_{n}) \times \Lambda(n) } ="02";
  %     (0,-30)*+{\scriptstyle P(n) \times \Sigma_{n} \times P(k_{1}) \times \cdots \times P(k_{n}) \times \Lambda(n) } ="03";
  %     (55,-30)*+{\scriptstyle P(\underline{k}) \times \Lambda(\underline{k}) } ="13";
  %     (70,0)*+{\scriptstyle P(n) \times P(k_{1}) \times \cdots \times P(k_{n}) } ="20";
  %     (70,-18)*+{\scriptstyle P(\underline{k}) } ="21";
  %     {\ar_{1 \times \Delta \times 1} "00" ; "01"};
  %     {\ar^{\cong} "01" ; "02"};
  %     {\ar_{1 \times \pi_{n} \times 1} "02" ; "03"};
  %     {\ar^{} "03" ; "13"};
  %     (35,-33)*{\scriptstyle \tilde{\mu}^P \times \mu^\Lambda(-;\underline{e})};
  %     {\ar_{\alpha_{\underline{k}}} "13" ; "21"};
  %     {\ar^{\alpha_{n} \times 1} "00" ; "20"};
  %     {\ar^{\mu^P} "20" ; "21"};
  %   \endxy
  % \]
  \[
    \xy
      (0,0)*+{\scriptstyle P(n) \times \Lambda(n) \times \prod_{i=1}^n P(k_i) } ="00";
      (0,-10)*+{\scriptstyle P(n) \times \Lambda(n) \times \Lambda(n) \times \prod_{i=1}^n P(k_i) } ="01";
      (0,-20)*+{\scriptstyle P(n) \times \Lambda(n) \times \prod_{i=1}^n P(k_i) \times \Lambda(n) } ="02";
      (0,-30)*+{\scriptstyle P(n) \times \Sigma_{n} \times \prod_{i=1}^n P(k_i) \times \Lambda(n) } ="03";
      (55,-30)*+{\scriptstyle P(K) \times \Lambda(K) } ="13";
      (70,0)*+{\scriptstyle P(n) \times \prod_{i=1}^n P(k_i) } ="20";
      (70,-18)*+{\scriptstyle P(K) } ="21";
      {\ar_{1 \times \Delta \times 1} "00" ; "01"};
      {\ar_{\cong} "01" ; "02"};
      {\ar_{1 \times \pi_{n} \times 1} "02" ; "03"};
      {\ar^{} "03" ; "13"};
      (35,-33)*{\scriptstyle \tilde{\mu}^P \times \mu^\Lambda(-;\underline{e})};
      {\ar_{\alpha_{K}} "13" ; "21"};
      {\ar^{\alpha_{n} \times 1} "00" ; "20"};
      {\ar^{\mu^P} "20" ; "21"};
    \endxy
  \]
In the second diagram, the morphism
    \[
        \tilde{\mu}^P \colon P(n) \times \Sigma_n \times \prod_{i=1}^n P(k_i) \rightarrow P(K) \times \Lambda(K)
    \]
is first the left action of $\Sigma_n$ on the product followed by the operad multiplication, and $\underline{e}$ is $e_{k_{1}}, \ldots, e_{k_{n}}$.

\begin{Defi}\label{Defi:actop_to_cat}
Let $\Lambda$ be an action operad. Then $B\Lambda$ (see \cref{nota:e_b}) is the category with objects the natural numbers and
  \[
    B\Lambda(m,n) = \left\{ \begin{array}{lc}
    \Lambda(n), & m = n \\
    \emptyset, & m \neq n,
    \end{array} \right.
  \]
where composition is given by group multiplication and the identity morphism is the unit element $e_n \in \Lambda(n)$.
\end{Defi}

\begin{thm}\label{preserveGop}
Let $M,N$ be cocomplete symmetric monoidal categories in which the tensor product preserves colimits in each variable, and let $F \colon M \rightarrow N$ be a symmetric lax monoidal functor with unit constraint $\varphi_{0}$ and tensor constraint $\varphi_{2}$. Let $\Lambda$ be an action operad, and $P$ a $\Lambda$-operad in $M$. Then $FP = \{ F(P(n)) \}$ has a canonical $\Lambda$-operad structure, giving a functor
  \[
    F_{*} \colon \Lambda\mbox{-}Op(M) \rightarrow \Lambda\mbox{-}Op(N)
  \]
from the category of $\Lambda$-operads in $M$ to the category of $\Lambda$-operads in $N$.
\end{thm}
\begin{proof}
The category of $\Lambda$-operads in $M$ is the category of monoids for the composition product $\circ_{M}$ on $[B\Lambda^{\textrm{op}}, M]$ constructed in \cref{sec:opasmon}. Composition with $F$ gives a functor
  \[
    F_{*} \colon  [B\Lambda^{\textrm{op}}, M] \rightarrow [B\Lambda^{\textrm{op}}, N],
  \]
  and to show that it gives a functor between the categories of monoids we need only prove that $F_{*}$ is lax monoidal with respect to $\circ_{M}$ and $\circ_{N}$. In other words, we must construct natural transformations with components $F_{*}X \circ_{N} F_{*}Y \rightarrow F_{*}(X \circ_{M} Y)$ and $I_{Op(N)} \rightarrow F_{*}(I_{Op(M)})$ and then verify the lax monoidal functor axioms. We note that in the calculations below, we often write $F$ instead of $F_{*}$, but it should be clear from context when we are applying $F$ to objects and morphisms in $M$ and when we are applying $F_{*}$ to a functor $ B\Lambda^{\textrm{op}} \rightarrow M$.

We first remind the reader about copowers in cocomplete categories. For an object $X$ and set $S$, the copower $X \odot S$ is the coproduct $\coprod_{s \in S} X$. We have natural isomorphisms $(X \odot S) \odot T \cong X \odot (S \times T)$ and $X \odot 1 \cong X$, and using these we can define an action of a group $G$ on an object $X$ using a map $X \odot G \rightarrow X$. Any functor $F$ between categories with coproducts is lax monoidal with respect to those coproducts:  the natural map $FA \coprod FB \rightarrow F(A \coprod B)$ is just the map induced by the universal property of the coproduct using $F$ applied to the coproduct inclusions $A \hookrightarrow A \coprod B, B \hookrightarrow A \coprod B$. In particular, for any functor $F$ there exists an induced map $FX \odot S \rightarrow F(X \odot S)$.

The unit object in $[B\Lambda^{\textrm{op}}, M]$ for $\circ_{M}$ is the copower $I_{M} \odot B\Lambda(-,1)$. Thus the unit constraint for $F_{*}$ is the composite
  \[
    I_{N} \odot B\Lambda(-,1) \stackrel{\varphi_{0} \odot 1}{\longrightarrow} FI_{M} \odot B\Lambda(-,1) \rightarrow F(I_{M} \odot B\Lambda(-,1) ).
  \]

For the tensor constraint, we will require a map
  \[
    t \colon (FY)^{\star n}(k) \rightarrow F\left(Y^{\star n}(k)\right)
  \]
 where $\star$ is the Day convolution product; having constructed one, the tensor constraint is then the following composite.
% \[
% \begin{array}{rcl}
% (FX \circ FY)(k) & \cong & \int^{n} FX(n) \otimes (FY)^{\star n}(k) \\
% & \stackrel{ \int 1 \otimes t}{\longrightarrow}  & \int^{n} FX(n) \otimes F(Y^{\star n}(k)) \\
% & \stackrel{\int \varphi_{2}}{\longrightarrow}  & \int^{n} F(X(n) \otimes Y^{\star n}(k)) \\
% & \longrightarrow & F (\int^{n} X(n) \otimes Y^{\star n}(k)) \\
% & \cong & F(X \circ Y)(k),
% \end{array}
% \]
  \begin{align*}
    (FX \circ FY)(k)  &\xrightarrow{\cong} \int^{n} FX(n) \otimes (FY)^{\star n}(k) \\
    &\xrightarrow{\int 1 \otimes t} \int^{n} FX(n) \otimes F(Y^{\star n}(k)) \\
    &\xrightarrow{\int \varphi_2} \int^{n} F(X(n) \otimes Y^{\star n}(k)) \\
    &\longrightarrow F \left(\int^{n} X(n) \otimes Y^{\star n}(k)\right) \\
    &\xrightarrow{\cong} F(X \circ Y)(k)
  \end{align*}

Both isomorphisms in the composite above are induced by universal properties (see \cref{section:operads_in_Cat} for more details) and the unlabeled arrow is induced by the same argument as that for coproducts above but this time using coends. The arrow $t$ is constructed in a similar fashion, and is the composite below.
% \[
% \begin{array}{rcl}
% (FY)^{\star n}(k) & = & \int^{k_{1}, \ldots, k_{n}} FY(k_{1}) \otimes \cdots \otimes FY(k_{n}) \odot B\Lambda(k, \sum k_{i}) \\
% & \rightarrow &  \int^{k_{1}, \ldots, k_{n}} F(Y(k_{1}) \otimes \cdots \otimes Y(k_{n})) \odot B\Lambda(k, \sum k_{i}) \\
% & \rightarrow & \int^{k_{1}, \ldots, k_{n}} F(Y(k_{1}) \otimes \cdots \otimes Y(k_{n}) \odot B\Lambda(k, \sum k_{i}) ) \\
% & \rightarrow & F\int^{k_{1}, \ldots, k_{n}} Y(k_{1}) \otimes \cdots \otimes Y(k_{n}) \odot B\Lambda(k, \sum k_{i})  \\
% & = & F(Y^{\star n}(k))
% \end{array}
% \]
  \begin{align*}
    (FY)^{\star n}(k) & =  \int^{k_{1}, \ldots, k_{n}} FY(k_{1}) \otimes \cdots \otimes FY(k_{n}) \odot B\Lambda(k, \Sigma k_{i}) \\
    & \rightarrow   \int^{k_{1}, \ldots, k_{n}} F(Y(k_{1}) \otimes \cdots \otimes Y(k_{n})) \odot B\Lambda(k, \Sigma k_{i}) \\
    & \rightarrow  \int^{k_{1}, \ldots, k_{n}} F(Y(k_{1}) \otimes \cdots \otimes Y(k_{n}) \odot B\Lambda(k, \Sigma k_{i}) ) \\
    & \rightarrow  F\int^{k_{1}, \ldots, k_{n}} Y(k_{1}) \otimes \cdots \otimes Y(k_{n}) \odot B\Lambda(k, \Sigma k_{i})  \\
    & =  F(Y^{\star n}(k))
  \end{align*}
Checking the lax monoidal functor axioms is tedious but entirely routine using the lax monoidal functor axioms for $F$ together with various universal properties of colimits, and we leave the details to the reader.
\end{proof}

Combining \cref{preserveGop} and \cref{gisgop} with \cref{symmoncor}, we immediately obtain the following.

\begin{cor}\label{cor:elambda_lambdaop}
Let $\Lambda$ be an action operad. Then $E\Lambda = \{ E\left(\Lambda(n)\right) \}$ (see \cref{nota:e_b}) is a $\Lambda$-operad in $\mb{Cat}$.
\end{cor}

Any $\Lambda$-operad $P$ in $\mb{Cat}$ gives rise to a $2$-monad on $\mb{Cat}$ which we will often also denote by $P$ or, as in \cref{section:operads_in_Cat}, by $\underline{P}$. In the context of \cref{cor:elambda_lambdaop}, that $2$-monad $\underline{E\Lambda}$ is given by
  \[
    X \mapsto \coprod_{n \geq 0} \coeq{E\Lambda}{X}{\Lambda}{n}
  \]
where the action of $\Lambda(n)$ on $E\Lambda(n)$ is given by the obvious multiplication action on the right, and the action of $\Lambda(n)$ on $X^{n}$ is given using $\pi_{n} \colon \Lambda(n) \rightarrow \Sigma_{n}$ together with the standard left action of $\Sigma_{n}$ on $X^{n}$ in any symmetric monoidal category. The $2$-monad $\underline{E\Sigma}$ is that for symmetric strict monoidal categories (see \cref{sec:examples} for this and further examples).

\begin{Defi}\label{lmc}
A \emph{$\Lambda$-monoidal category} is a strict algebra for the $2$-monad $\EL$. A \emph{$\Lambda$-monoidal functor} is a strict morphism for this $2$-monad $\EL$. The $2$-category $\lmc$ is the $2$-category $\EL\mbox{-}\mb{Alg}_{s}$ of strict algebras, strict morphisms, and algebra $2$-cells for $\EL$.
\end{Defi}

% QQQ Here is probably a natural place to split into another chapter? Previous stuff "Abstract categorical properties of action operads", later stuff "Monoidal structures and multicategories". Or after these results/the next section.
% QQQ We came to the conclusion just to cite Yau's results in Chapter 19. `A strict $G$-monoidal category is..., etc.'

Strict $\Lambda$-monoidal categories, in the sense of being strict algebras for the monad $\underline{E\Lambda}$, can be characterised in more familiar terms by specifying a monoidal structure with appropriate equivariant interaction with the $\Lambda(n)$-actions. Explicit proofs of such can be found in Chapter 19 of \cite{yau_infinity_2021}, along with similar characterisations for strict $\Lambda$-monoidal functors and $\Lambda$-monoidal categories whose underlying monoidal structure is weak.

% \begin{prop}\label{el_via_moncats}
% A strict $\Lambda$-monoidal category structure on $X$ determines and is determined by
% \begin{itemize}
% \item a strict monoidal category structure on $X$: $(X, \otimes, I)$,
% \item for each $\sigma \in \Lambda(n)$ and $x_1$, $\ldots$, $x_n \in X$, an isomorphism
%   \[
%     \sigma_{x_1,\ldots,x_n} \colon x_1 \otimes \ldots x_n \rightarrow x_{\pi(\sigma)^{-1}(1)} \otimes \ldots \otimes x_{\pi(\sigma)^{-1}(n)}
%   \]
% which is natural in each $x_i$,
% \item such that for any $\tau_i \in \Lambda(k_i)$ and $\sigma \in \Lambda(n)$, where $1 \leq i \leq n$, the following diagram commutes
% \[
%   \xy
%     (0,0)*+{\bigotimes_{i=1}^n \bigotimes_{j=1}^{k_i} x_{i,j}}="a";
%     (50,0)*+{\bigotimes_{i=1}^n \bigotimes_{j=1}^{k_i} x_{i,\pi(\tau_i)^{-1}(j)}}="b";
%     (25,-20)*+{\bigotimes_{i=1}^n \bigotimes_{j=1}^{k_i} x_{\pi(\sigma)^{-1}(i),\pi\left(\tau_{\pi(\sigma)^{-1}(i)}\right)^{-1}(j)}}="c";
%     %
%     {\ar^(.4){\tau_1 \otimes \cdots \otimes \tau_n} "a" ; "b"};
%     {\ar^{\sigma^{+}} "b" ; "c"};
%     {\ar_{\mu^{\Lambda}(\sigma;\tau_1,\ldots,\tau_n)} "a" ; "c"};
%   \endxy
% \]
% \item QQQ needs to be some equivariance condition in here, otherwise appropriate axioms (e.g., hexagon identities) don't come out of this (can't remember why I came to this conclusion...)
% \end{itemize}
% \end{prop}
% \begin{proof}


% First we shall show an $\EL$-algebra structure on $X$ can be used to specify the monoidal structure described above. If $X$ is an $\EL$-algebra then we also have the family of morphisms $\alpha_n \colon E\Lambda(n) \times X^n \rightarrow X$ satisfying the usual axioms. We define the monoidal product of two objects $x$, $y \in X$ to be
%   \[
%     x \otimes y = \alpha_2(e_2;x,y)
%   \]
% and the unit object to be
%   \[
%     I = \alpha_0(e_0;).
%   \]
% Strict associativity follows from the the $\EL$-algebra axioms and \cref{calclem} as follows:
%   \begin{align*}
%     (x_1 \otimes x_2) \otimes x_3 &= \alpha_2(e_2;\alpha_2(e_2;x_1,x_2),x_3)\\
%     &= \alpha_2(e_2;\alpha_2(e_2;x_1,x_2),\alpha_1(e_1;x_3))\\
%     &= \alpha_3(\mu(e_2;e_2,e_1);x_1,x_2,x_3)\\
%     &= \alpha_3(e_3;x_1,x_2,x_3)\\
%     &= \alpha_3(\mu(e_2;e_1,e_2);x_1,x_2,x_3)\\
%     &= \alpha_2(e_2;\alpha(e_1;x_1),\alpha(e_2;x_2,x_3))\\
%     &= \alpha_2(e_2;x_1,\alpha_2(e_2;x_2,x_3))\\
%     &= x_1 \otimes (x_2 \otimes x_3).
%   \end{align*}
% Due to this we can then write $n$-fold monoidal products as
%   \[
%     x_1 \otimes \ldots \otimes x_n = \alpha_n(e_n;x_1,\ldots,x_n)
%   \]
% and for any $\sigma \in \Lambda(n)$ we now define
%   \[
%     x_{\pi(\sigma)^{-1}(1)} \otimes \ldots \otimes x_{\pi(\sigma)^{-1}(n)} = \alpha_n(\sigma;x_1,\ldots,x_n).
%   \]
% For the unit object we have
%   \begin{align*}
%     I \otimes x &= \alpha_2(e_2;I,x)\\
%     &= \alpha_2(e_2;\alpha_0(e_0;),\alpha_1(e_1;x))\\
%     &= \alpha_1(\mu^\Lambda(e_2;e_0,e_1);x)\\
%     &= \alpha_1(e_1;x)\\
%     &= x,
%   \end{align*}
% with similar working to show that $x \otimes I = x$.

% Next we specify the permutation isomorphisms. Given $\sigma \in \Lambda(n)$, there exists a unique isomorphism
%   \[
%     ! \colon e_n \rightarrow \sigma
%   \]
% in $E\Lambda(n)$, which we use to define the isomorphism
%   \[
%     \tilde{\sigma} = \alpha_n(!;\underline{\id}) \colon \alpha_n(e_n;x_1,\ldots,x_n) \rightarrow \alpha_n(\sigma;x_1,\ldots,x_n).
%   \]
% That these are natural follows simply from the functoriality of each $\alpha_n$, giving
%   \[
%   \alpha_n(!;\underline{\id})\alpha(\id;f_1,\ldots,f_n) = \alpha_n(!;f_1,\ldots,f_n) = \alpha_n(\id;f_1,\ldots,f_n)\alpha_n(!;\underline{\id})
%   \]
% for $f_i \colon x_i \rightarrow x_i'$. The isomorphisms also satisfy the axioms specified in the diagram above as a direct result of the $\EL$-algebra axioms.

% Conversely, we begin with a strict monoidal category $(X,\otimes,I)$, along with isomorphisms
%   \[
%     \tilde{\sigma} \colon x_1 \otimes \ldots \otimes x_n \rightarrow x_{\pi(\sigma)^{-1}(1)} \otimes \ldots \otimes \pi(\sigma)^{-1}(n)
%   \]
% satisfying the axioms specified above. We need to describe morphisms
%   \[
%     \alpha_n \colon E\Lambda(n) \times X^n \rightarrow X
%   \]
% which satisfy the axioms of an $\EL$-algebra. To begin with we first define
%   \[
%     \alpha_0(e_0;) = I.
%   \]
% To define each $\alpha_n$ on objects, we then put
%   \[
%     \alpha_n(\sigma;x_1,\ldots,x_n) = x_{\pi(\sigma)^{-1}(1)} \otimes \ldots \otimes x_{\pi(\sigma)^{-1}(n)}.
%   \]
% This is easily checked to be well-defined.

% A morphism in $E\Lambda(n) \times_{\Lambda(n)} X^n$ from $[\sigma;x_1,\ldots,x_n]$ to $[\tau;y_1,\ldots,y_n]$ consists of a unique isomorphism $! \colon \sigma \cong \tau$ along with morphisms $f_i \colon x_i \rightarrow y_i$ in $X$. We define $\alpha_n(!;f_1,\ldots,f_n)$ to be the following composite.
%     \[
%         \alpha_n(\sigma;x_1,\ldots,x_n) \xrightarrow{\tilde{\sigma}^{-1}} \alpha_n(e_n;x_1,\ldots,x_n) \xrightarrow{\otimes f_i} \alpha_n(e_n;y_1,\ldots,y_n) \xrightarrow{\tilde{\tau}} \alpha_n(\tau;y_1,\ldots,y_n)
%     \]
% Again, well-definedness and functoriality conditions are easily checked.

% The unit axiom is satisfied immediately since we then have $\alpha_1(e_1;x) = x$. The other $\EL$-algebra axiom then follows, with some careful consideration of indices, from two applications of the diagram described in the statement of the proposition.
% \end{proof}

% \begin{prop}\label{el_strictmap}
% Let $X,Y$ be strict $\Lambda$-monoidal categories, and $F \colon X \rightarrow Y$ be a functor. $F$ is a strict $\Lambda$-monoidal functor if and only if
% \end{prop}
% \begin{proof}

% \end{proof}

% \begin{prop}\label{el_weakmap}
% Let $X,Y$ be strict $\Lambda$-monoidal categories, and $F \colon X \rightarrow Y$ be a functor. The structure of a weak $\Lambda$-monoidal functor determines and is determined by some stuff.
% \end{prop}
% \begin{proof}

% \end{proof}

% \begin{prop}\label{el_2cells}
% Let $X,Y$ be strict $\Lambda$-monoidal categories, and $F, G \colon X \rightarrow Y$ be strict $\Lambda$-monoidal functors. The structure of an $\EL$-algebra $2$-cell $\alpha \colon F \Rightarrow G$ determines and is determined by some stuff.
% \end{prop}
% \begin{proof}

% \end{proof}

\subsection{Free $\Lambda$-monoidal categories}


It will be useful for our calculations later to give an explicit description of the categories $\coeq{E\Lambda}{X}{\Lambda}{n}$. Objects are equivalence classes of tuples $(g; x_1, \ldots, x_n)$ where $g \in \Lambda(n)$ and the $x_{i}$ are objects of $X$, with the equivalence relation given by
  \[
    (gh; x_1, \ldots, x_n) \sim \left(g; x_{h^{-1}(1)}, \ldots, x_{h^{-1}(n)}\right);
  \]
  we write these classes as $[g; x_1, \ldots, x_n]$. Morphisms are then equivalence classes of morphisms
  \[
    (!; f_1, \ldots, f_n) \colon  (g; x_1, \ldots, x_n) \rightarrow \left(h; x_1', \ldots, x_n'\right).
  \]
We have two distinguished classes of morphisms, one for which the map $! \colon  g \rightarrow h$ is the identity and one for which all the $f_{i}$'s are the identity. Every morphism in $E\Lambda(n) \times X^{n}$ is uniquely a composite of a  morphism of the first type followed by one of the second type. Now $\coeq{E\Lambda}{X}{\Lambda}{n}$ is a quotient of $E\Lambda(n) \times X^{n}$ by a free group action, so every morphism of $\coeq{E\Lambda}{X}{\Lambda}{n}$ is in the image of the quotient map. Using this fact, we can prove the following useful lemma.

\begin{lem}\label{hom-set-lemma}
For an action operad $\Lambda$ and any category $X$, the set of morphisms from $[e; x_1, \ldots, x_n]$ to $[e; y_1, \ldots, y_n]$ in $\coeq{E\Lambda}{X}{\Lambda}{n}$ is
  \[
    \coprod_{g \in \Lambda(n)} \prod_{i=1}^{n} X\left(x_i, y_{g(i)}\right).
  \]
\end{lem}
\begin{proof}
A morphism with source $(e; x_1, \ldots, x_n)$ in $E\Lambda(n) \times X^{n}$ is uniquely a composite
  \[
    (e; x_1, \ldots, x_n) \stackrel{(\id; f_{1}, \ldots, f_{n})}{\longrightarrow} \left(e; x_1', \ldots, x_n'\right) \stackrel{(!; \id, \ldots, \id)}{\longrightarrow} \left(g; x_1', \ldots, x_n'\right).
  \]
Descending to the quotient, this becomes a morphism
  \[
    [e; x_1, \ldots, x_n] \rightarrow \left[g; x_1', \ldots, x_n'\right] = \left[e; x_{g^{-1}(1)}', \ldots, x_{g^{-1}(n)}'\right],
  \]
and therefore is a morphism $[e; x_1, \ldots, x_n] \rightarrow [e; y_1, \ldots, y_n]$ precisely when $y_i = x_{g^{-1}(i)}'$, and so $f_i \in   X(x_i, y_{g(i)})$.
\end{proof}

% QQQ -------------------------------
% Not needed any more.
% \begin{nota}\label{tensor_notation}
% For $g \in \Lambda(n)$ and objects $x_1, \ldots, x_n$ of a $\Lambda$-monoidal category $X$, we write 
%   \[
%     g^{\otimes} \colon x_1 \otimes \cdots x_n \rightarrow x_{g^{-1}(1)} \otimes \cdots \otimes x_{g^{-1}(n)}
%   \]
% for the image of the map
%   \[
%     (!; \id, \ldots, \id) \colon  (e; x_1, \ldots, x_n) \rightarrow (g; x_1, \ldots, x_n)
%   \]
% in $E\Lambda(n) \times_{\Lambda(n)} X^{n}$.
% \end{nota}

% \begin{Defi}\label{action_map}
% We call the map 
%   \[
%     g^{\otimes} \colon x_1 \otimes \cdots x_n \rightarrow \otimes x_{g^{-1}(1)} \otimes \cdots \otimes x_{g^{-1}(n)}
%   \]
% the \emph{action} of $g$ on $x_1 \otimes \cdots \otimes x_n$.
% \end{Defi}
% 
% \begin{rem}
% It is obvious that $g^{\otimes} \otimes h^{\otimes} = \mu(e_2; g, h)^{\otimes}$. The latter can also be written $\beta(g, h)^{\otimes}$ (using \cref{thm:charAOp}).
% % or $(g \oplus h)^{\otimes}$  (using \cref{beta_to_oplus}).
% \end{rem}
% ------------------------

The $2$-monad $\underline{E\Lambda}$ is both finitary and cartesian (see \cref{sec:propofopsincat}). In fact we can characterize this operad uniquely (up to equivalence) using a standard argument.

\begin{Defi}
Let $\Lambda$ be an action operad. A \textit{$\Lambda_{\infty}$ operad} $P$ is a $\Lambda$-operad in which each action $P(n) \times \Lambda(n) \rightarrow P(n)$ is free and each $P(n)$ is contractible.
\end{Defi}

\begin{rem}
One should also note that by \cref{cor:elambda_lambdaop} there exists a canonical $\Lambda_{\infty}$ operad in $\mb{Cat}$, namely $E\Lambda$ itself, and thus also in the category of simplicial sets by taking the nerve (the nerve functor is represented by a cosimplicial category, namely $\Delta \subseteq \mb{Cat}$, so preserves products) and in suitable categories of topological spaces by taking the geometric realization (once again, product-preserving with the correct category of spaces). Thus we have something like a Barratt-Eccles $\Lambda_{\infty}$ operad for any action operad $\Lambda$.
\end{rem}

\begin{rem}
The above definition makes sense in a wide context, but needs interpretation. We can interpret the freeness condition in any complete category, as completeness allows one to compute fixed points using equalizers. Contractibility then requires a notion of equivalence or weak equivalence, such as in an $(\infty, 1)$-category or Quillen model category, and a terminal object. Our interest is in the above definition interpreted in $\mb{Cat}$, in which case both conditions (free action and contractible $P(n)$'s) mean the obvious thing.
\end{rem}

\begin{prop}
For any two $\Lambda_{\infty}$ operads $P,Q$ in $\mb{Cat}$, there exists a span $P \leftarrow R \rightarrow Q$ of pointwise equivalences of $\Lambda$-operads.
\end{prop}
\begin{proof}
Given $\Lambda_{\infty}$ operads $P$, $Q$ in $\mb{Cat}$, the product $P \times Q$ with the diagonal action is also $\Lambda_{\infty}$. Each of the projection maps is a pointwise equivalence of $\Lambda$-operads.
\end{proof}
\begin{rem}
Once again, this proof holds in a wide context. We required that the product of free actions is again free, true in any complete category. We also required that the product of contractible objects is contractible; this condition will hold, for example, in any Quillen model category in which all objects are fibrant or in which the product of weak equivalences is again a weak equivalence.
\end{rem}


\subsection{Abstract properties of the Borel construction}

Kelly's theory of clubs \cite{kelly_club1, kelly_club0, kelly_club2} was designed to simplify and explain certain aspects of coherence results, namely the fact that many coherence results rely on extrapolating information about general free objects for a $2$-monad $T$ from information about the specific free object $T1$ where $1$ denotes the terminal category. This occurs, for example, in the study of the many different flavors of monoidal category:  plain monoidal category, braided monoidal category, symmetric monoidal category, and so on. This section will explain how every action operad gives rise to a club, as well as compute the clubs which arise as the image of this procedure.

We begin by reminding the reader of the notion of a club, or more specifically what Kelly \cite{kelly_club1,kelly_club2} calls a club over $\mb{P}$. We will only be interested in clubs over $\mb{P}$, and thusly shorten the terminology to club from this point onward. Defining clubs is accomplished most succinctly using Leinster's terminology of generalized operads \cite{leinster}.

\begin{Defi}
Let $C$ be a category with finite limits.
\begin{enumerate}
\item A monad $T \colon C \rightarrow C$ is \textit{cartesian} if the functor $T$ preserves pullbacks, and the naturality squares for the unit $\eta$ and the multiplication $\mu$ for $T$ are all pullbacks.
\item The category of \textit{$T$-collections}, $T\mbox{-}\mb{Coll}$, is the slice category $C/T1$, where $1$ denotes the terminal object.
\item Given a pair of $T$-collections $X \stackrel{x}{\rightarrow} T1, Y \stackrel{y}{\rightarrow} T1$, their \textit{composition product} $X \circ Y$ is given by the pullback below together with the morphism along the top.
  \[
    \xy
      (0,0)*+{X \circ Y} ="00";
      (15,0)*+{TY} ="10";
      (30,0)*+{T^{2}1} ="20";
      (45,0)*+{T1} ="30";
      (0,-10)*+{X} ="01";
      (15,-10)*+{T1} ="11";
      {\ar^{} "00" ; "10"};
      {\ar^{Ty} "10" ; "20"};
      {\ar^{\mu} "20" ; "30"};
      {\ar^{T!} "10" ; "11"};
      {\ar_{x} "01" ; "11"};
      {\ar^{} "00" ; "01"};
      (3,-3)*{\lrcorner};
    \endxy
  \]
\item The composition product, along with the unit of the adjunction $\eta \colon 1 \rightarrow T1$, give $T\mbox{-}\mb{Coll}$ a monoidal structure. A \textit{$T$-operad} is a monoid in $T\mbox{-}\mb{Coll}$.
\end{enumerate}
\end{Defi}

\begin{rem}
Everything in the above definition can be $\mb{Cat}$-enriched without any substantial modifications. Thus we require our ground $2$-category to have finite limits in the enriched sense, and the slice and pullbacks are the $2$-categorical (and not bicategorical) versions. If we take this $2$-category to be $\mb{Cat}$, then in each case the underlying category of the $2$-categorical construction is given by the corresponding $1$-categorical version. From this point, we will not distinguish between the $1$-dimensional and $2$-dimensional theory. Our interest, of course, is in the $2$-dimensional version.
\end{rem}

Let $\Sigma$ be the operad of symmetric groups. This is the terminal object of the category of action operads, with each $\pi_{n}$ the identity map. Then $\underline{E\Sigma}$ is a $2$-monad on $\mb{Cat}$, and by results in \cref{sec:propofopsincat} it is cartesian.

\begin{Defi}
A \textit{club} is a $T$-operad in $\mb{Cat}$ for $T = \underline{E\Sigma}$.
\end{Defi}

\begin{rem}
The category $\mb{P}$ in Kelly's terminology is the result of applying $\underline{E\Sigma}$ to $1$, and can be identified with $B\Sigma = \coprod B\Sigma_{n}$.
\end{rem}

It is useful to break down the definition of a club. A club consists of
\begin{enumerate}
\item a category $K$ together with a functor $K \rightarrow B \Sigma$,
\item a multiplication map $K \circ K \rightarrow K$, and
\item a unit map $1 \rightarrow K$
\end{enumerate}
satisfying the axioms to be a monoid in the monoidal category of $E\Sigma$-collections. By the definition of $K \circ K$ as a pullback, objects are tuples of objects of $K$ $(x; y_{1}, \ldots, y_{n})$ where $\pi(x) = n$. A morphism
  \[
    (x; y_{1}, \ldots, y_{n}) \rightarrow (z; w_{1}, \ldots, w_{m})
  \]
exists only when $n=m$ (since $B\Sigma$ only has endomorphisms) and then consists of a morphism $f \colon x \rightarrow z$ in $K$ together with morphisms $g_{i} \colon y_{i} \rightarrow z_{\pi(x)(i)}$ in $K$.

\begin{nota}\label{nota:clubmult}
For a club $K$ and a morphism $(f; g_{1}, \ldots, g_{n})$ in $K \circ K$, we write $f(g_{1}, \ldots, g_{n})$ for the image of the morphism under the functor $K \circ K \rightarrow K$.
\end{nota}

We will usually just refer to a club by its underlying category $K$.


\begin{thm}
Let $\Lambda$ be an action operad. Then the map of operads $\pi \colon \Lambda \rightarrow \Sigma$ gives the category $B\Lambda = \coprod B\Lambda(n)$ the structure of a club.
\end{thm}
\begin{proof}
To give the functor $B\pi \colon B\Lambda \rightarrow B \Sigma$ the structure of a club it suffices (see \cite{leinster}) to show that
\begin{itemize}
\item the induced monad, which we will show to be $\underline{E\Lambda}$, is a cartesian monad on $\mb{Cat}$,
\item the transformation $\tilde{\pi} \colon \underline{E\Lambda} \Rightarrow \underline{E\Sigma}$ induced by the functor $E\pi$ is cartesian, and
\item $\tilde{\pi}$ commutes with the monad structures.
\end{itemize}
The monad $\underline{E\Lambda}$ is always cartesian by results of \cref{sec:propofopsincat}. The transformation $\tilde{\pi}$ is the coproduct of the maps $\tilde{\pi}_{n}$ which are induced by the universal property of the coequalizer as shown below.
  \[
    \xy
      (0,0)*+{\scriptstyle E\Lambda(n) \times \Lambda(n) \times X^n} ="00";
      (0,-15)*+{\scriptstyle E\Sigma_{n} \times \Sigma_{n} \times X^n} ="01";
      (30,0)*+{\scriptstyle E\Lambda(n) \times X^n} ="10";
      (30,-15)*+{\scriptstyle E\Sigma_{n} \times X^n} ="11";
      (60,0)*+{\scriptstyle E\Lambda(n) \times_{\Lambda(n)} X^n} ="20";
      (60,-15)*+{\scriptstyle E\Sigma_{n} \times_{\Sigma_{n}}  X^n} ="21";
      {\ar (11,1)*{}; (22,1)*{} };
      {\ar (11,-1)*{}; (22,-1)*{} };
      {\ar_{E\pi \times \pi \times 1} "00" ; "01"};
      {\ar (10,-14)*{}; (23,-14)*{} };
      {\ar (10,-16)*{}; (23,-16)*{} };
      {\ar_{E\pi \times 1} "10" ; "11"};
      {\ar@{.>}^{\tilde{\pi}_{n}} "20" ; "21"};
      {\ar "10" ; "20"};
      {\ar "11" ; "21"};
    \endxy
  \]
Naturality is immediate, and since $\pi$ is a map of operads $\tilde{\pi}$ also commutes with the monad structures.

It only remains to show that $\tilde{\pi}$ is cartesian and that the induced monad is actually $\underline{E\Lambda}$. Since the monads $\underline{E\Lambda}$ and $\underline{E\Sigma}$ both decompose into a disjoint union of functors, we only have to show that, for any $n$, the square below is a pullback.
  \[
    \xy
      (0,0)*+{\coeq{E\Lambda}{X}{\Lambda}{n}} ="00";
      (0,-10)*+{B\Lambda(n)} ="01";
      (35,0)*+{E\Sigma_{n} \times_{\Sigma_{n}} X^n} ="10";
      (35,-10)*+{B\Sigma_{n}} ="11";
      {\ar^{} "00" ; "10"};
      {\ar^{} "10" ; "11"};
      {\ar^{} "00" ; "01"};
      {\ar^{} "01" ; "11"};
    \endxy
  \]
By \cref{coeq-lem}
%(QQQ happy with this reference? QQQ Seems like the right thing?)
, this amounts to showing that the square below is a pullback.
  \[
    \xy
      (0,0)*+{\left(E\Lambda(n) \times X^n\right)/\Lambda(n)} ="00";
      (0,-10)*+{B\Lambda(n)} ="01";
      (35,0)*+{\left(E\Sigma_{n} \times X^n\right)/\Sigma_{n}} ="10";
      (35,-10)*+{B\Sigma_{n}} ="11";
      {\ar^{} "00" ; "10"};
      {\ar^{} "10" ; "11"};
      {\ar^{} "00" ; "01"};
      {\ar^{} "01" ; "11"};
    \endxy
  \]
Here, $(A \times B)/G$ is the category whose objects are equivalence classes of pairs $(a,b)$ where $(a,b) \sim (ag, g^{-1}b)$, and similarly for morphisms. Now the bottom map is clearly bijective on objects since these categories only have one object. An object in the top right is an equivalence class
  \[
    [\sigma; x_{1}, \ldots, x_{n}] = \left[e; x_{\sigma^{-1}(1)}, \ldots, x_{\sigma^{-1}(n)}\right].
  \]
A similar description holds for objects in the top left, with $g \in \Lambda(n)$ replacing $\sigma$ and $\pi(g)^{-1}$ replacing $\sigma^{-1}$ in the subscripts. The map along the top sends $[g; x_{1}, \ldots, x_{n}]$ to $[\pi(g); x_{1}, \ldots, x_{n}]$, and thus sends $[e; x_{1}, \ldots, x_{n}]$ to $[e; x_{1}, \ldots, x_{n}]$, giving a bijection on objects.

Now a morphism in $(E\Lambda(n) \times X^{n})/\Lambda(n)$ can be given as
  \[
    [e; x_{1}, \ldots, x_{n}] \stackrel{[!; f_{i}]}{\longrightarrow} [g; y_{1}, \ldots, y_{n}].
  \]
Mapping down to $B\Lambda(n)$ gives $ge^{-1} = g$, while mapping over to $(E\Sigma_{n} \times X^{n})/\Sigma_{n}$ gives $[!; f_{i}]$ where $! \colon e \rightarrow \pi(g)$ is now a morphism in $E\Sigma_{n}$. In other words, a morphism in the upper left corner of our putative pullback square is determined completely by its images along the top and lefthand functors. Furthermore, given $g \in \Lambda(n)$, $\tau = \pi(g)$, and morphisms $f_{i} \colon x_{i} \rightarrow y_{i}$ in $X$, the morphism $[! \colon e \rightarrow g; f_{i}]$ maps to the pair $(g, [! \colon e \rightarrow \tau; f_{i}])$, completing the proof that this square is indeed a pullback.
\end{proof}

The club, which we now denote $K_{\Lambda}$, associated to $E\Lambda$ has the following properties. First, the functor $K_{\Lambda} \rightarrow B\Sigma$ is a functor between groupoids. Second, the functor $K_{\Lambda} \rightarrow B\Sigma$ is  bijective-on-objects. We claim that these properties characterize those clubs which arise from action operads. Thus the clubs arising from action operads are very similar to PROPs \cite{mac_prop, markl_prop}.

\begin{thm}\label{thm:club=operad}
Let $K$ be a club such that
\begin{itemize}
\item the map $K \rightarrow B\Sigma$ is bijective on objects and
\item $K$ is a groupoid.
\end{itemize}
Then $K \cong K_{\Lambda}$ for some action operad $\Lambda$. The assignment $\Lambda \mapsto K_{\Lambda}$ is a full and faithful embedding of the category of action operads $\mb{AOp}$ into the category of clubs.
\end{thm}
\begin{proof}
Let $K$ be such a club. Our hypotheses immediately imply that $K$ is a groupoid with objects in bijection with the natural numbers; we will now assume the functor $K \rightarrow B\Sigma$ is the identity on objects. Let $\Lambda(n) = K(n,n)$. Now $K$ comes equipped with a functor to $B\Sigma$, in other words group homomorphisms $\pi_{n} \colon \Lambda(n) \rightarrow \Sigma_{n}$. We claim that the club structure on $K$ makes the collection of groups $\{ \Lambda(n) \}$ an action operad. In order to do so, we will employ \cref{thm:charAOp}.

First, we give the group homomorphism $\beta$ using \cref{nota:clubmult}. Define
  \[
    \beta(g_{1}, \ldots, g_{n}) = e_{n}(g_{1}, \ldots, g_{n})
  \]
 (see \ref{nota:clubmult}) where $e_{n}$ is the identity morphism $n \rightarrow n$ in $K(n,n)$. Functoriality of the club multiplication map immediately implies that this is a group homomorphism. Second, we define the function $\delta$ in a similar fashion:
  \[
    \delta_{n; k_{1}, \ldots, k_{n}}(f) = f(e_{1}, \ldots, e_{n}),
  \]
where here $e_{i}$ is the identity morphism of $k_{i}$ in $K$.

There are now nine axioms to verify in \cref{thm:charAOp}. The club multiplication functor is a map of collections, so a map over $B\Sigma$; this fact immediately implies that Axioms \eqref{eq1} (using morphisms in $K \circ K$ with only $g_{i}$ parts) and \eqref{eq4} (using morphisms in $K \circ K$ with only $f$ parts) hold. The mere fact that multiplication is a functor also implies Axioms \eqref{eq6} (once again using morphisms with only $f$ parts) and \eqref{eq8} (by considering the composite of a morphism with only an $f$ with a morphism with only $g_{i}$'s). Axiom \eqref{eq2} is the equation $e_{1}(g) = g$ which is a direct consequence of the unit axiom for the club $K$; the same is true of Axiom \eqref{eq5}. Axioms \eqref{eq3}, \eqref{eq7},  and \eqref{eq9} all follow from the associativity of the club multiplication.

Finally, we would like to show that this gives a full and faithful embedding
    \[
        K_{-} \colon \mb{AOp} \rightarrow \mb{Club}
    \]
of the category of action operads into the category of clubs. Let $f, f' \colon \Lambda \rightarrow \Lambda'$ be maps between action operads. Then if $K_{f} = K_{f'}$ as maps between clubs, then they must be equal as functors $K_{\Lambda} \rightarrow K_{\Lambda'}$. But these functors are nothing more than the coproducts of the functors
  \[
    B(f_{n}), B(f_{n}') \colon B\Lambda(n) \rightarrow B\Lambda'(n),
  \]
and the functor $B$ from groups to categories is faithful, so $K_{-}$ is also faithful. Now let $f \colon K_{\Lambda} \rightarrow K_{\Lambda'}$ be a maps of clubs. We clearly get group homomorphisms $f_{n} \colon \Lambda(n) \rightarrow \Lambda'(n)$ such that $\pi^{\Lambda}_{n} = \pi^{\Lambda'}_{n} f_{n}$, so we must only show that the $f_{n}$ also constitute an operad map. Using the description of the club structure above in terms of the maps $\beta, \delta$, we are able to see that commuting with the club multiplication implies commuting with both of these, which in turn is equivalent to commuting with operad multiplication. Thus $K_{-}$ is full as well.
\end{proof}

\begin{rem}
First, one should note that being a club over $B\Sigma$ means that every $K$-algebra has an underlying strict monoidal structure. Second, requiring that $K \rightarrow B\Sigma$ be bijective on objects ensures that $K$ does not have  operations other than $\otimes$, such as duals or internal hom-objects, from which to build new types of objects. Finally, $K$ being a groupoid ensures that all of the ``constraint morphisms'' that exist in algebras for $K$ are invertible.

These hypotheses could be relaxed somewhat. Instead of having a club over $B\Sigma$, we could have a club over the free symmetric monoidal category on one object (note that the free symmetric monoidal category monad on $\mb{Cat}$ is still cartesian). This would produce $K$-algebras with underlying monoidal structures which are not necessarily strict. This change should have relatively little impact on how the theory is developed. Changing $K$ to be a category instead of a groupoid would likely have a larger impact, as the resulting action operads would have monoids instead of groups at each level. We have made repeated use of inverses throughout the proofs in the basic theory of action operads, and these would have to be revisited if groups were replaced by monoids in the definition of action operads.
\end{rem}

In \cite{kelly_club1}, Kelly discusses clubs given by generators and relations. His generators include functorial operations more general than what we are interested in here, and the natural transformations are not required to be invertible. In our case, the only generating operations we require are those of a unit and tensor product, as the algebras for $E\Lambda$ are always strict monoidal categories with additional structure. Tracing through his discussion of generators and relations for a club gives the following theorem.

\begin{thm}\label{pres1}
Let $\Lambda$ be an action operad with presentation given by $(\mathbf{g},\mathbf{r}, s_{i}, p)$. Then the club $E\Lambda$ is generated by
\begin{itemize}
  \item functors giving the unit object and tensor product, and
  \item natural transformations given by the collection $\mathbf{g}$:  each element $x$ of $\mathbf{g}$ with $\pi(x) = \sigma_{x} \in \Sigma_{|x|}$ gives a natural transformation from the $n$th tensor power functor to itself,
\end{itemize}
subject to relations such that the following axioms hold.
\begin{itemize}
  \item The monoidal structure given by the unit and tensor product is strict.
  \item The transformations given by the elements of $\mathbf{g}$ are all natural isomorphisms.
  \item For each element $y \in \mathbf{r}$, the equation $s_{1}(y) = s_{2}(y)$ holds.
\end{itemize}
\end{thm}

Bringing this down to a concrete level we have the following corollary.

\begin{cor}\label{pres2}
Assume we have a notion $\mathcal{M}$ of strict monoidal category which is given by  a set natural isomorphisms
  \[
    \mathcal{G} = \left\{ (f, \pi_{f}) \, | \,  x_{1} \otimes \cdots \otimes x_{n} \stackrel{f}{\longrightarrow} x_{\pi_{f}^{-1}(1)} \otimes \cdots \otimes x_{\pi_{f}^{-1}(n)} \right\}
  \]
subject to a set $\mathcal{R}$ of axioms. Each such axiom is given by the data
\begin{itemize}
  \item two finite sets $f_{1}, \ldots, f_{n}$ and $f_{1}', \ldots, f_{m}'$ of elements of $\mathcal{G}$; and
  \item two formal composites $F,F'$ using only composition and tensor product operations and the $f_{i}$, respectively $f_{i}'$, 
\end{itemize}
such that the underlying permutation of $F$ equals the underlying permutation of $F'$ (we compute the underlying permutations using the functions $\beta, \delta$ of \cref{thm:charAOp}). The element $\left(\underline{f}, \underline{f}', F, F'\right)$ of the set $\mathcal{R}$ of axioms corresponds to the requirement that the composite of the morphisms $f_{i}$ using $F$ equals the composite of the morphisms $f_{j}'$ using $F'$ in any strict monoidal category of type $\mathcal{M}$. Then strict monoidal categories of type $\mathcal{M}$ are given as the algebras for the club $E\Lambda$ where $\Lambda$ is the action operad with
\begin{itemize}
  \item $\mathbf{g} = \mathcal{G}$,
  \item $\mathbf{r} = \mathcal{R}$,
  \item $s_{1}$ given by mapping the generator $\left(\underline{f}, \underline{f}', F, F'\right)$ to the operadic composition of the $f_{i}$ using $F$ via $\beta, \delta$, and
  \item $s_{2}$ given by mapping the generator $\left(\underline{f}, \underline{f}', F, F'\right)$ to the operadic composition of the $f_{i}'$ using $F'$ via $\beta, \delta$.
\end{itemize}
\end{cor}
