\documentclass{amsart}

\usepackage{amssymb}
\usepackage{amsmath}
\usepackage{amscd}
\usepackage{eucal}
\usepackage{amsthm}
\usepackage{hyperref}
\usepackage{xcolor}
%\usepackage{geometry}
\newcommand{\bs}{\boldsymbol}
\newcommand{\mb}{\mathbf}
\renewcommand{\dot}{\centerdot}
\newcommand{\D}{\textrm{-}}
\newcommand{\ul}{\underline}
\addtolength{\hoffset}{-2cm}
\addtolength{\textwidth}{4cm}
%\pdfshift

\begin{document}

This is a section-by-section list of all the major results, with their dependencies.
\textcolor{blue}{Throughout: track down specific refs, ie Theorem 2.7 in [3], not just [3]}

\section{Introduction}

Nothing to say here, I think.

\section{Action operads}

\textbf{Definition.} symmetric operad
\\ \textbf{Definition.} non-symmetric operad
\\ \textbf{Definition.} braided operad
\\ \textbf{Definition.} operad map
\\ \textbf{Definition.} action operad
\\ \textbf{Definition.} map of action operads
\\ \textbf{Definition/Example.} ribbon braids, and their (action) operad
\\ \textbf{Result (1).} $\pi$ is a map of operads. \textcolor{magenta}{Dependency: defs}
\\ \textbf{Result (2).} Operads internal to groups are action operads. \textcolor{magenta}{Dependency: (1)}
\\ \textbf{Result (3).} The kernel of an action operad is an action operad. \textcolor{magenta}{Dependency: (1,2)}
\\ \textbf{Result (4).} The image, in $\Sigma$ of an action operad is an action operad. \textcolor{magenta}{Dependency: (1)}
\\ \textbf{Result (5).} A kernel/image short exact sequence. \textcolor{magenta}{Dependency: (3,4)}
\\ \textbf{Result (6).} Some calculations with $e_i's$. \textcolor{magenta}{Dependency: defs}
\\ \textbf{Result (7).} Some calculations with $\Lambda(0)$. \textcolor{magenta}{Dependency: (6)}
\\ \textbf{Result (8).} The big $\beta, \delta$ theorem. \textcolor{magenta}{Dependency: (1)}
\\ \textbf{Result (9).} $\pi$ is zero or surjective. \textcolor{magenta}{Dependency: (8)}
\\ \textbf{Examples.} Cyclic, reflexive, hyperoctahedral, alternating. \textcolor{magenta}{Dependency: (8)}
\\ \textbf{Definition.} lfp stuff
\\ \textbf{Result (10).} The category of action operads is lfp. \textcolor{magenta}{Dependency: defs, external}
\\ \textbf{Result (11).} $U: \mathbf{AOp} \to \mathbf{Sets}/\mathcal{S}$ preserves limits and filtered colimits. \textcolor{magenta}{Dependency: defs} \textcolor{blue}{Note: seriously check proof}
\\ \textbf{Result (12).} $F: \mathbf{Sets}/\mathcal{S} \to \mathbf{AOp}$ left adjoint to $U$. \textcolor{magenta}{Dependency: external}
\\ \textbf{Definition.} presentations for action operads \textcolor{magenta}{Dependency: (12)}

\section{Operads with equivariance}

\textbf{Definition.} $\Lambda$-operad
\\ \textbf{Definition.} map of $\Lambda$-operads
\\ \textbf{Definition.} category of $\Lambda$-operads
\\ \textbf{Result (13).} $\Lambda$ is a $\Lambda$-operad. \textcolor{magenta}{Dependency: defs}
\\ \textbf{Definition.} algebra over a non-symmetric operad \textcolor{blue}{Note: delete?}
\\ \textbf{Definition.} algebra over a $\Lambda$-operad
\\ \textbf{Definition.} category of algebras over a $\Lambda$-operad
\\ \textbf{Result (14).} Endomorphism operad is a $\Lambda$-operad. \textcolor{magenta}{Dependency: defs} \textcolor{blue}{Note: should have independent endomorphism operad def beforehand, maybe rework all this stuff}
\\ \textbf{Result (15).} Change-of-operad functor. \textcolor{magenta}{Dependency: defs}
\\ \textbf{Result (16).} Algebras are operad maps into endomorphisms operad. \textcolor{magenta}{Dependency: (14)}
\\ \textbf{Definition.} monad associated to a $\Lambda$-operad
\\ \textbf{Result (17).} Monad algebra category is operad algebra category. \textcolor{magenta}{Dependency: defs}
\\ \textbf{Result (18).} $\Lambda$-algebras, as a $\Lambda$-operad, are monoids. \textcolor{magenta}{Dependency: defs, maybe (16)} \textcolor{blue}{Note: unclear hypotheses, should say in sets I think}
\\ \textbf{Result (19).} Three-part theorem about the adjunction between $\Lambda$- and $\Sigma$-operads and their categories of algebras. \textcolor{magenta}{Dependency: defs} \textcolor{blue}{Note: check proof}
\\ \textbf{Definition.} monad map\textcolor{blue}{Note: some text after that needs to be in an environment}
\\ \textbf{Definition.} cocomplete SMC \textcolor{blue}{Note: no emph in def, is wrong}
\\ \textbf{Result (19).} Lax symmetric monoidal functors transport operads, with a comparison monad map. \textcolor{magenta}{Dependency: FUTURE!} \textcolor{blue}{Note: eep in general! where did we define the tensor product over a group notation?}
\\ \textbf{Result (20).} Operad maps induce monad maps. \textcolor{magenta}{Dependency: stuff that isn't in an environment above} \textcolor{blue}{Note: continued eep}
\\ \textbf{Result (21).} Combining to get an adjunction. \textcolor{magenta}{Dependency: (19, 20)} \textcolor{blue}{Note: continued eep}
\\ \textbf{Definition.} collections, maps, the category thereof
\\ \textbf{Definition.} substitution product of collections
\\ \textbf{Result (22).} Substitution product gives monoidal structure, and monoids are operads. \textcolor{magenta}{Dependency: (19, 20)}
\\ \textbf{Result (23).} $B\Lambda$ is a strict monoidal category. \textcolor{magenta}{Dependency: FUTURE! also (6)}
\\ \textbf{Result (24).} $n$-fold Day convolution is a functor $B\Lambda \to \mathbf{Sets}$. \textcolor{magenta}{Dependency: (23)}
\\ \textbf{Result (25).} Substitution product as coend using Day convolution. \textcolor{magenta}{Dependency: ??} \textcolor{blue}{Note: seriously check proof}
\\ \textbf{Proof of (22).} \textcolor{magenta}{Dependency: (23,24,25)} \textcolor{blue}{Note: seriously check proof}


\section{Operads in the category of categories}

\textcolor{blue}{Note: worth revisiting introductory material, maybe some of it needs environments}
\\ \textbf{Definition.} pseudoalgebras
\\ \textbf{Definition.} strict algebras \textcolor{magenta}{Dependency: previous defn}
\\ \textbf{Definition.} pseudomorphisms
\\ \textbf{Definition.} strict morphisms \textcolor{magenta}{Dependency: previous defn}
\\ \textbf{Definition.} algebra transformations
\\ \textbf{Definition.} $P$-alg, strict and strong
\\ \textbf{Definition.} 2-monads versions of the above
\\ \textbf{Result (26).} 2-monad and operad algebra 2-categories agree, strict and strong. \textcolor{magenta}{Dependency: definitions here}
\\ \textbf{Result (27).} 2-monad from an operad is finitary. \textcolor{magenta}{Dependency: definitions here} \textcolor{blue}{Note: check proof}
\\ \textbf{Result (28).} 2-monad from an operad preserves bijective-on-objects funtors. \textcolor{magenta}{Dependency: definitions here}
\\ \textbf{Result (29).} Pseudoalgebras equivalent to strict ones. \textcolor{magenta}{Dependency: definitions here} \textcolor{blue}{Note: worth explaining how this strictifies unbiased monoidal categories to strict ones, but not biased ones}
\\ \textbf{Definition.} 2-cartesian 2-monad \textcolor{blue}{Note: we seem to need some definitions here}
\\ \textbf{Result (30).} Coequalizer of actions is sometimes the quotient. \textcolor{magenta}{Dependency: none}\textcolor{blue}{Note: this looks like it could be improved, many aspects unclear}
\\ \textbf{Result (31).} Unit for $\underline{P}$ is cartesian for any symmetric operad $P$. \textcolor{magenta}{Dependency: definitions here}
\\ \textbf{Result (32).} The 2-monad $\underline{P}$ preserves pullbacks iff group action is free. \textcolor{magenta}{Dependency: (30)}
\\ \textbf{Result (33).} Multiplication for $\underline{P}$ is cartesian if all group actions are free. \textcolor{magenta}{Dependency: (30)} \textcolor{blue}{Note: has some suspect proof-by-example looking text}
\\ \textbf{Result (34).} $\underline{P}$ is 2-cartesian if and only if all group actions are free (symmetric case). \textcolor{magenta}{Dependency: (31, 32, 33; 30)} 
\\ \textbf{Result (35).} $\underline{P}$ is 2-cartesian if and only if all group actions are free (symmetric case). \textcolor{magenta}{Dependency: (31, 32, 33; 30)} 
\\ \textbf{Result (36).} Lemma about free $\Sigma$-actions on categories with $\Lambda$-action. \textcolor{magenta}{Dependency: defns} \textcolor{blue}{Note: needs to be fixed up a bit in the whole groups actions on categories rework}
\\ \textbf{Result (37).} $\underline{P}$ is 2-cartesian if and only if all group actions are free ($\Lambda$ case). \textcolor{magenta}{Dependency: (35, 36; 30, 31, 32, 33, 34)} 


\section{The Borel construction for action operads}

\textbf{Definition.} $EG, BG$
\\ \textbf{Definition.} isofibration
\\ \textbf{Result (38).} $p:EU \Rightarrow B$ pointwise isofibration. \textcolor{magenta}{Dependency: definitions here} \textcolor{blue}{Note: needs ref}
\\ \textbf{Result (39).} $E$ right adjoint to set of objects functor, symmetric monoidal wrt cartesian products. \textcolor{magenta}{Dependency: definitions here}
\\ \textbf{Definition.} the category $B\Lambda$ \textcolor{blue}{Note: this was used back in (23)}
\\ \textbf{Result (40).} Lax symmetric monoidal functors induce functors between categories of $\Lambda$-operads \textcolor{magenta}{Dependency: (22; 23, 24, 25)}
\\ \textbf{Result (41).} $E\Lambda$ is an action operad. \textcolor{magenta}{Dependency: (13, 39, 40; 22, 23, 24, 25)}
\\ \textbf{Definition.} $\Lambda$-monoidal categories
\\ \textbf{Result (42).} Formula for morphisms in $E\Lambda(n) \times_{\Lambda(n)} X^n$. \textcolor{magenta}{Dependency: (30?)}
\\ \textbf{Result (43).} $\ul{E\Lambda}$ is finitary and 2-cartesian. \textcolor{magenta}{Dependency: (30?)} \textcolor{blue}{Note: currently just in free text}
\\ \textcolor{blue}{Delete $\Lambda_{\infty}$ stuff??}
\\ \textbf{Definition.} cartesian monad, collections for those, and operads for those. \textcolor{blue}{Note: should unify with the 2-cartesian definitions??}
\\ \textbf{Definition.} clubs
\\ \textcolor{blue}{I think the free text explicitly breaking down the definition of a club has some errors, double check}
\\ \textbf{Result (44).} $B\Lambda$ is a club. \textcolor{magenta}{Dependency: defs here, (30, 42, 43)}
\\ \textbf{Result (45).} Characterization of which clubs are action operads. \textcolor{magenta}{Dependency: defs here, (8)}
\\ \textbf{Result (46).} Presentations of clubs, for action operads. \textcolor{magenta}{Dependency: defs here, external?}
\\ \textbf{Result (47).} Presentations of strict monoidal structures arising from action operads, via clubs. \textcolor{magenta}{Dependency: (46)}
\\ \textcolor{blue}{I think this is where NG puts the presentation for strict symmetric monoidal cats}


\section{Monoidal structures and multicategories}
\textcolor{blue}{Lots of examples here, unify them; should go in a section with presentation stuff}
\\ \textbf{Definition.} coboundary category
\\ \textbf{Definition.} coboundary functors
\\ \textbf{Definition.} 2-category of coboundary stuff
\\ \textbf{Result (48).} Strictification of coboundary cats. \textcolor{magenta}{Dependency: defs here}
\\ \textbf{Definition.} stuff to define operad: disjoint, contains, $s_{p,q}$'s, $J_n$'s
\\ \textbf{Result (49).} Action operad structure on $J$'s. \textcolor{magenta}{Dependency: (8), defs here} \textcolor{blue}{Note: issue in proof}
\\ \textcolor{blue}{Conflict: $C$ vs $J$ notation; I understand now - $C$ is the monad, $J$'s give the club}
\\ \textbf{Result (50).} The 2-monad $C$ for strict coboundary cats is a club. \textcolor{magenta}{Dependency: (47)}
\\ \textbf{Result (51).} $C1 \cong BJ$. \textcolor{magenta}{Dependency: (49)}
\\ \textbf{Result (52).} $C \cong \ul{EJ}$. \textcolor{magenta}{Dependency: (45, 50,51)}
\\ \textbf{Definition.} strength stuff \textcolor{blue}{Note: change to left/right}
\\ \textbf{Definition.} pseudo-commutative 2-monad
\\ \textcolor{blue}{AC should redo all our straight line string diagrams to look pretty like his curvy, colored one}
\\ \textbf{Definition.} pseudo-commutative operad
\\ \textbf{Result (53).} Pseudo-commutative operads give pseudo-commutative 2-monads. \textcolor{magenta}{Dependency: defs here}
\\ \textbf{Result (54).} Non-symmetric operads never give pseudo-commutative 2-monads. \textcolor{magenta}{Dependency: defs here} \textcolor{blue}{Note: last sentence of proof should probably be tightened up}
\\ \textbf{Definition.} symmetric pseudo-commutative 2-monad
\\ \textbf{Result (55).} From operad to symmetric p-c 2-monad. \textcolor{magenta}{Dependency: (53)} 
\\ \textbf{Definition.} contractible operad
\\ \textbf{Result (56).} $P$ contractible and has $t$'s, then p-c. \textcolor{magenta}{Dependency: (53, 55)}
\\ \textbf{Result (57).} Contractible symmetric operads have symm p-c. \textcolor{magenta}{Dependency: (56; 53, 55)}
\\ \textcolor{blue}{Remark about symmetrization not preserving contractibility doesn't read well}
\\ \textbf{Result (58).} The 2-monad for strict braided monoidal cats has two non-symmetric p-c structures.
\\ \textbf{Result (59).} The 2-monad for strict braided monoidal cats is $\ul{EB}$.   \textcolor{magenta}{Dependency: external}
\\ \textbf{Definition.} positive braids
\\ \textbf{Definition.} minimal braids \textcolor{blue}{Note: combine defs?}
\\ \textbf{Result (60).} Bijection between positive minimal braids and permutations.   \textcolor{magenta}{Dependency: external}
\\ \textbf{Proof of (58).} \textcolor{magenta}{Dependency: defs here, (53 ,60)} \textcolor{blue}{Note: seriously check proof}


\end{document} 