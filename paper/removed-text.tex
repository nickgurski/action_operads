Removed from operads paper:

Intro:

Original paper intro:

Operads were defined by May \cite{maygeom} in the early 70's to provide a convenient tool to approach problems in algebraic topology, notably the question of when a space $X$ admits an $n$-fold delooping $Y$ so that $X \simeq \Omega^{n}Y$.  An operad, like an algebraic theory \cite{lawvere-thesis}, is something like a presentation for a monad or algebraic structure.  The theory of operads has seen great success, and we would like to highlight two reasons.  First, operads can be defined in any suitable symmetric monoidal category, so that there are operads of topological spaces, of chain complexes, of simplicial sets, and of categories, to name a few examples.  Moreover, symmetric (lax) monoidal functors carry operads to operads, so we can use operads in one category to understand objects in another via transport by such a functor.  Second, operads in a fixed category are highly flexible tools.  In particular, the categories listed above all have some inherent notion of ``homotopy equivalence'' which is weaker than that of isomorphism, so we can study operads which are equivalent but not isomorphic.  These tend to have algebras which have similar features in an ``up-to-homotopy'' sense but very different combinatorial or geometric properties arising from the fact that different objects make up these equivalent but not isomorphic operads.

Operads in the category $\mb{Cat}$ of small categories have a unique flavor arising from the fact that $\mb{Cat}$ is not just a category but a 2-category.  These 2-categorical aspects have not been widely treated in the literature, although a few examples can be found.  Lack \cite{lack-cod} mentions braided $\mb{Cat}$-operads (the reader new to braided operads should refer to the work of Fiedorowicz \cite{fie-br}) in his work on coherence for $2$-monads, and Batanin \cite{bat-eh} uses lax morphisms of operads in $\mb{Cat}$ in order to define the notion of an internal operad.  But aside from a few appearances, the basic theory of operads in $\mb{Cat}$ and their 2-categorical properties seems missing.  This paper was partly motivated by the need for such a theory to be explained from the ground up.

There were two additional motivations for the work in this paper.  In thinking about coherence for monoidal functors, the first author was led to a general study of algebras for multicategories internal to $\mb{Cat}$.  These give rise to $2$-monads (or perhaps pseudomonads, depending on how the theory is set up), and checking abstract properties of these $2$-monads prompts one to consider the simpler case of operads in $\mb{Cat}$ instead of multicategories.  The other motivation was from the second author's attempt to understand the interplay between operads in $\mb{Cat}$, operads in $\mb{Top}$, and the passage from (bi)permutative categories to $E_{\infty}$ (ring) spaces.  The first of these motivations raised the issue of when operads in $\mb{Cat}$ are cartesian, while the second led us to consider when an operad in $\mb{Cat}$ possesses a pseudo-commutative structure.

While considering how to best tackle a general discussion of operads in $\mb{Cat}$, it became clear that restricting attention to the two most commonly used types of operads, symmetric and non-symmetric operads, was both short-sighted and unnecessary.  Many theorems apply to both kinds of operads at once, with the difference in proofs being negligible; in fact, most of the arguments which applied to the symmetric case seemed to apply to the case of braided operads as well.   This led us to the notion of an action operad $\mb{G}$, and then to a definition of $\mb{G}$-operads.  In essence, this is merely the general notion of what it means for an operad $P = \{ P(n) \}_{n \in \N}$ to have groups of equivariance $\mb{G} = \{ G(n) \}_{n \in \N}$ such that $G(n)$ acts on $P(n)$.  Choosing different natural families of groups $\mb{G}$, we recover known variants of the definition of operad. \\ \begin{center}
\begin{tabular}{c|c}
Groups $\mb{G}$ & Type of operad  \\ \hline
Terminal groups & Non-symmetric operad \\
Symmetric groups & Symmetric operad \\
Braid groups & Braided operad \\
\end{tabular} \\ \end{center}
These definitions have appeared, with minor variations, in two sources of which we are aware.  In Wahl's thesis \cite{wahl-thesis}, the essential definitions appear but not in complete generality as she requires a surjectivity condition.  Zhang \cite{zhang-grp} also studies these notions\footnote{Zhang calls our action operad a \textit{group operad}.  We dislike this terminology as it seems to imply that we are dealing with an operad in the category of groups, which is not the case unless all of the maps $\pi_{n}:G(n) \rightarrow \Sigma_{n}$ are zero maps.}, once again in the context of homotopy theory, but requires the  superfluous condition that $e_{1} = \textrm{id}$ (see Lemma \ref{lem:calclem}).

This paper consists of the following.  In Section 1, we give the definition of an action operad $\mb{G}$ and a $\mb{G}$-operad.  We develop this definition abstractly so as to apply it in any suitable symmetric monoidal category.   It is standard to express operads as monoids in a particular functor category using a composition tensor product.  In order to show that our $\mb{G}$-operads fit into this philosophy, we must work abstractly and use the calculus of coends together with the Day convolution product \cite{day-thesis}.  The reader uninterested in these details can happily skip them, although we find the route taken here to be quite satisfactory in justifying the axioms for an action operad $\mb{G}$ and the accompanying notion of $\mb{G}$-operad.  Many of our calculations are generalizations of those appearing in work of Kelly \cite{kelly-op}, although there are slight differences in flavor between the two treatments.
%Kelly:  On the operads of JP May

Section 2 works through the basic 2-categorical aspects of operads in $\mb{Cat}$.  We explain how every operad gives rise to a $2$-monad, and show that all of the various 1-cells between algebras of the associated $2$-monad correspond to the obvious sorts of 1-cells one might define between algebras over an operad in $\mb{Cat}$.  Similarly, we show that the algebra 2-cells, using the $2$-monadic approach, correspond to the obvious notion of transformation one would define using the operad.

Section 3 studies three basic 2-categorical properties of an operad, namely the property of being finitary, the property of being 2-cartesian, and the coherence property.  The first of these always holds, as a simple calculation shows.  The second of these turns out to be equivalent to the action of $G(n)$ on $P(n)$ being free for all $n$, at least up to a certain kernel.  In particular, our characterization clearly shows that every non-symmetric operad is 2-cartesian, and that a symmetric operad is 2-cartesian if and only if the symmetric group actions are all free.  (It is useful to note that a $2$-monad on $\mb{Cat}$ is 2-cartesian if and only if the underlying monad on the category of small categories is cartesian in the usual sense as the (strict) 2-pullback of a diagram is the same as its pullback.)  The third property is also easily shown to hold for any $\mb{G}$-operad on $\mb{Cat}$ using a factorization system argument due to Power \cite{power-gen}.

Section 4 then goes on to study the question of when a $\mb{G}$-operad $P$ admits a pseudo-commutative structure.  Such a structure provides the 2-category of algebras with a richer structure that includes well-behaved notions of tensor product, internal hom, and multilinear map that fit together much as the analogous notions do in the category of vector spaces.  When $P$ is contractible (i.e., each $P(n)$ is equivalent to the terminal category), this structure can be obtained from a collection of elements $t_{m,n} \in G(mn)$ satisfying certain properties.  In particular, we show that every contractible symmetric operad is pseudo-commutative, and we prove that there exist such elements $t_{m,n} \in Br_{mn}$ so that every contractible braided operad is pseudo-commutative as well (in fact in two canonical ways).  Thus Section 4 can be seen as a continuation, in the operadic context, of the work in \cite{HP}, and in particular the ``geometric'' proof of the existence of a pseudo-commutative structure for braided strict monoidal categories demonstrates the power of being able to change the groups of equivariance.

The authors would like to thank John Bourke, Martin Hyland, Tom Leinster, and Peter May for various conversations which led to this paper.  While conducting this research, the second author was supported by an EPSRC Early Career Fellowship. 

Original Borel intro:


Categories of interest are often monoidal: sets, topological spaces, and vector spaces are all symmetric monoidal, while the category of finite ordinals (under ordinal sum) is merely monoidal.  But other categories have more exotic monoidal structures.  The first such type of structure discovered was that of a braided monoidal category.  These arise in categories whose morphisms have a geometric flavor like cobordisms embedded in some ambient space \cite{js}, in  categories produced from double loop spaces \cite{fie-br}, and categories of representations over objects like quasitriangular (or braided) bialgebras \cite{street-quantum} .  Another such exotic monoidal structure is that of a coboundary category, arising in examples from the representation theory of quantum groups \cite{drin-quasihopf}.

Going back to the original work of May on iterated loop spaces \cite{maygeom}, operads were defined in both symmetric and nonsymmetric varieties.  But Fiedorowicz's work on double loop spaces \cite{fie-br} showed that there was utility in considering another kind of operad, this time with braid group actions instead of symmetric group actions.  There is a clear parallel between these definitions of different types of operads and the definitions of different kinds of monoidal category, with each given by some general schema in which varying an $\mathbb{N}$-indexed collection of groups produced the types of operads or monoidal categories seen in nature.  Building on the work in \cite{cg}, the goal of this paper is to show that this parallel can be upgraded from an intuition to precise mathematics using the notion of action operad.

An action operad $\mb{\Lambda}$ is an operad which incorporates all of the essential features of the operad of symmetric groups.  Thus $\Lambda(n)$ is no longer just a set, but instead also has a group structure together with a map $\pi_{n}:\Lambda(n) \to \Sigma_{n}$.  Operadic composition then satisfies an additional equivariance condition using the maps $\pi_n$ and the group structures.  Each action operad $\mb{\Lambda}$ produces a notion of $\mb{\Lambda}$-operad which encodes equivariance conditions using both the groups $\Lambda(n)$ and the maps $\pi_n$.  Examples include the symmetric groups, the terminal groups (giving nonsymmetric operads), the braid groups (giving braided operads), and the $n$-fruit cactus groups \cite{hk-cobound} (giving a new notion of operad one might call cactus operads).  Using a formula resembling the classical Borel construction for spaces with a group action, we can produce from any action operad $\mb{\Lambda}$ a notion of $\mb{\Lambda}$-monoidal category, in which the group $\Lambda(n)$ acts naturally on $n$-fold tensor powers of any object.  Thus the categorical Borel construction embeds action operads into a category of monads on $\mb{Cat}$, and we characterize the image of this embedding as those monads describing monoidal structures of a precise kind.

The paper is organized into the following sections.  Section 1 reviews the definition of an action operad, and defines the categorical Borel construction on them.  The key result, which reappears in proofs throughout the paper, is \cref{thm:charAOp}, characterizing action operads in terms of two new operations mimicking the block sum of permutations and the operation which takes a permutation of $n$ letters and produces a new permutation on $k_1 + k_2 + \cdots + k_n$ letters by permuting the blocks of $k_i$ letters.  In Section 2, we use this characterization and Kelly's theory of clubs \cite{kelly_club1, kelly_club0, kelly_club2} to embed action operads into monads on $\mb{Cat}$ and determine the essential image of this embedding.  Section 3 gives a construction of the free action operad from a suitable collection of data, and relates this to how clubs can be described using generators and relations.  The results of Sections 2 and 3 show that the definitions of symmetric monoidal category or coboundary category, for example, correspond to the action operad constructed from the corresponding free symmetric monoidal or coboundary category on one object; these and other examples appear in detail in Section 4.  Section 5 then extends the definition of $\mb{\Lambda}$-operad to that of $\mb{\Lambda}$-multicategory and shows that these arise abstractly via a Kleisli construction.

Copied from text:
Yau \cite{yau_infinity_2021} collects together a large number of results on the topic of action operads while also investigating the setting of infinity group operads. 


Further acknowledgements:
Alex needs to thank the LMS for a Research Reboot grant. Anybody else we've talked to about these things since their inception? Angelica? Niles? Dan Graves. Nathaniel Arkor.

Conventions:

%\begin{conv}\label{conv1}
%We adopt the following conventions throughout.
%\begin{enumerate}
%\item\label{conv:symm_sigma} $\Sigma_{n}$ is the symmetric group on $n$ letters, and $B_{n}$ is the braid group on $n$ strands.
%\item\label{conv:g-action} For a group $G$, a right $G$-action on a set $X$ will be denoted $(x,g) \mapsto x \cdot g$. We will use both $\cdot$ and concatenation to represent multiplication in a group.
%\item\label{conv:e_identity} The symbol $e$ will generically represent an identity element in a group. If we are considering a set of groups $\{ \Lambda(n) \}_{n \in \N}$ indexed by the natural numbers, then $e_{n}$ is the identity element in $\Lambda(n)$. We will often drop the subscripts and just write $e$ when the index $n$ in $\Lambda(n)$ is either clear from context or unimportant to the argument at hand. Occasionally we will write $\Lambda_n$ in place of $\Lambda(n)$, especially in diagrams.
%\item\label{conv:coeq} We will often be interested in elements of a product of the form
%\[
%A \times B_{1} \times \cdots \times B_{n} \times C
%\]
%(or similar, for example without $C$). We will write elements of this set as $(a; b_{1}, \ldots, b_{n}; c)$, and in the case that we need equivalence classes of such elements they will be written as $[a; b_{1}, \ldots, b_{n}; c]$. This will often be the case when we are interested in the coequalizer of left and right group actions in the following sense. A coequalizer of maps
%    \[
%        \xy
%            (0,0)*+{A \times G \times B}="00";
%            (30,0)*+{A \times B}="10";
%            (60,0)*+{\coeqb{A}{B}{G}}="20";
%            {\ar@<1ex>^{\lambda} "00" ; "10"};
%            {\ar@<-1ex>_{\rho} "00" ; "10"};
%            {\ar^{\varepsilon} "10" ; "20"};
%        \endxy
%    \]
%will be written as $\coeqb{A}{B}{G}$, where $\rho$ represents a right action of $G$ on $A$, and $\lambda$ a left action of $G$ on $B$. This is similar to the notation often used to denote pullbacks, however we find in this work that no confusion arises from using notation in this way.
%\item\label{conv:beta_delta} In the following definitions of operads, we define operad multiplication as a function
%  \[
%    \mu \colon O(n) \times O(k_1) \times \ldots O(k_n) \rightarrow O(k_1 + \cdots + k_n)
%  \]
%for each $n$, $k_1$, $\ldots$, $k_n$, and we use the following two functions as a shorthand for two commonly occuring instances of such. First we define a function
%  \[
%    \beta \colon O(k_1) \times \ldots \times O(k_n) \rightarrow O(k_1 + \cdots + k_n)
%  \]
%for each $n$, $k_1$, $\ldots$, $k_n$, which takes elements $\tau_1 \in O(k_1)$, $\ldots$, $\tau_n \in O(k_n)$ and produces the element
%  \[
%    \beta(\tau_1, \ldots, \tau_n) = \mu(e_n; \tau_1, \ldots, \tau_n).
%  \]
%We think of this element as the block sum of the elements $\tau_1$, $\ldots$, $\tau_n$. We also define a function
%  \[
%    \delta_{n;k_1,\ldots,k_n} \colon O(n) \rightarrow O(k_1 + \cdots + k_n)
%  \]
%for each $n$, $k_1$, $\ldots$, $k_n$, which takes an element $\sigma \in O(n)$ and produces the element
%  \[
%    \delta(\sigma) = \mu(\sigma;e_{k_1}, \ldots, e_{k_n}).
%  \]
%
%In the particular case of the symmetric groups, these are functions
%  \[
%    \beta \colon \Sigma_{k_1} \times \Sigma_{k_n} \rightarrow \Sigma_{k_1 + \cdots + k_n}
%  \]
%and
%  \[
%    \delta_{n;k_1,\ldots,k_n} \colon \Sigma_{n} \rightarrow \Sigma_{k_1 + \cdots + k_n}.
%  \]
%In the case of $\beta$ we form the block sum permutation $\beta(\tau_1,\ldots,\tau_n)$ which permutes the first $k_{1}$ elements according to $\tau_{1}$, the next $k_{2}$ elements according to $\tau_{2}$ and so on; this is an element of $\Sigma_{k_{1} + \cdots + k_{n}}$. For $\delta$ we take the permutation $\delta(\sigma) \in \Sigma_{k_{1} + \cdots + k_{n}}$ to be that which permutes the $n$ different blocks $1$ through $k_{1}$, $k_{1}+1$ through $k_{1} + k_{2}$, and so on, according to the permutation $\sigma \in \Sigma_{n}$. We expand on this at various points, including in \cref{rem:perm_matrices}.
%\item\label{conv:perm_shorthand} Throughout we will be using maps $\pi_n \colon O(n) \rightarrow \Sigma_n$, where $O(n)$ is the object of $n$-ary operations of an operad $O$ and $\Sigma_n$ is the symmetric group on $n$ elements. The map $\pi_n$ in each case will represent a form of `underlying permutation' of each element, which we will then use to act on operad multiplication. As the notation starts to become cumbersome, we will often write $\sigma^{-1}(i)$ which should be read as $\pi_n(\sigma)^{-1}(i)$, where $\sigma \in O(n)$.
%\item\label{conv:op_superscript} We adopt the convention that if an equation requires using operadic composition in more than one operad, we will indicate this by a superscript on each instance of $\mu$ unless it is entirely clear from context. This can be seen, for example, in \cref{Defi:op_map}.
%\end{enumerate}
%\end{conv}

Operads:
Some old axiom diagrams
% \[
% \xy
% (0,0)*+{\scriptstyle O(n) \times O(k_{1}) \times \cdots \times O(k_{n}) \times O(l_{1,1}) \times \cdots \times O(l_{{1},k_{1}}) \times \cdots \times O(l_{n,1}) \times \cdots \times O(l_{{n},k_{n}})} ="00";
% (0,-50)*+{\scriptstyle O(k_{1} + \cdots + k_{n}) \times O(l_{1,1}) \times \cdots \times O(l_{{1},k_{1}}) \times \cdots \times O(l_{n,1})\times \cdots \times O(l_{{n},k_{n}})} ="02";
% (55,-10)*+{\scriptstyle O(n) \times \prod_{i=1}^n O(k_{i}) \times O(l_{i,1}) \times \cdots \times O(l_{{i}, k_{i}}) } ="20";
% (55,-25)*+{\scriptstyle O(n) \times O(\sum l_{1,-}) \times \cdots \times O(\sum l_{n,-})} ="21";
% (55, -40)*+{\scriptstyle  O(\sum l_{-,-})} ="22";
% {\ar_{\scriptstyle \mu \times 1} "00" ; "02"};
% {\ar_{\mu} "02" ; "22"};
% {\ar^{\cong} "00" ; "20"};
% {\ar^{1 \times \prod \mu} "20" ; "21"};
% {\ar^{\mu} "21" ; "22"};
% \endxy
% \]
%
%the following two equations hold.
%  \begin{align*}
%    \mu(x;y_1 \cdot \tau_1,\ldots,y_n \cdot \tau_n) &= \mu(x;y_1,\ldots,y_n)\cdot \beta(\tau_1,\ldots,\tau_n)\\
%    \mu(x \cdot \sigma; y_1, \ldots, y_n) &= \mu\left(x;y_{\sigma^{-1}(1)},\ldots,y_{\sigma^{-1}(n)}\right)\cdot \delta(\sigma)
%  \end{align*}
%  % \begin{align*}
%  %   \mu(x;y_1 \cdot \tau_1,\ldots,y_n \cdot \tau_n) &= \mu(x;y_1,\ldots,y_n)\cdot(\tau_1 \oplus \ldots \oplus \tau_n)\\
%  %   \mu(x \cdot \sigma; y_1, \ldots, y_n) &= \mu\left(x;y_{\sigma^{-1}(1)},\ldots,y_{\sigma_{-1}(n)}\right)\cdot \sigma^+
%  % \end{align*}
%  For the first equation above, we 
 % \[
  %   \xy
  %     {\ar@{-} (0,0)*{}; (25,-10)*{} };
  %     {\ar@{-} (0,-20)*{}; (25,-10)*{} };
  %     {\ar@{-} (0,0)*{}; (0,-20)*{} };
  %     {\ar@{-} (25,-10)*{}; (35,-10)*{} };
  %     {\ar@{-} (0,0)*{}; (-10,0)*{} };
  %     {\ar@{-} (0,-3)*{}; (-10,-3)*{} };
  %     {\ar@{-} (0,-17)*{}; (-10,-17)*{} };
  %     {\ar@{-} (0,-20)*{}; (-10,-20)*{} };
  %     (11,-10)*{x}; (-5,-10)*{\vdots}
  %   \endxy
  % \]
    % \[
  %   \xy
  %     {\ar@{-} (0,0)*{}; (25,-10)*{} };
  %     {\ar@{-} (0,-20)*{}; (25,-10)*{} };
  %     {\ar@{-} (0,0)*{}; (0,-20)*{} };
  %     {\ar@{-} (25,-10)*{}; (35,-10)*{} };
  %     {\ar@{-} (0,-3)*{}; (-10,-3)*{} };
  %     {\ar@{-} (0,-17)*{}; (-10,-17)*{} };
  %     (11,-10)*{x};
  %     {\ar@{-} (-25,2)*{}; (-10,-3)*{} };
  %     {\ar@{-} (-25,-8)*{}; (-10,-3)*{} };
  %     {\ar@{-} (-25,2)*{}; (-25,-8)*{} };
  %     {\ar@{-} (-25,1)*{}; (-30,1)*{} };
  %     {\ar@{-} (-30,-7)*{}; (-25,-7)*{} };
  %     (-19,-3)*{y_{1}};
  %     {\ar@{-} (-25,-12)*{}; (-10,-17)*{} };
  %     {\ar@{-} (-25,-22)*{}; (-10,-17)*{} };
  %     {\ar@{-} (-25,-12)*{}; (-25,-22)*{} };
  %     {\ar@{-} (-25,-13)*{}; (-30,-13)*{} };
  %     {\ar@{-} (-25,-17)*{}; (-30,-17)*{} };
  %     {\ar@{-} (-25,-21)*{}; (-30,-21)*{} };
  %     (-19,-17)*{y_{2}};
  %   \endxy
  % \]
  %\begin{rem}\label{Rem:sigma_conditions}
%It is useful to write out in full what the sets in the diagram of the second axiom above mean. The use of numerous products and indices is to save space but the full picture becomes much clearer when these are expanded. For the equations in the third axiom above to make sense, we must have
%\begin{itemize}
%\item $x \in O(n)$,
%\item $y_{i} \in O(k_{i})$ for $i=1, \ldots, n$,
%\item $\tau_{i} \in \Sigma_{k_{i}}$,
%\item $\sigma \in \Sigma_{n}$, and
%\item $\beta(\tau_1,\ldots,\tau_n), \delta(\sigma) \in \Sigma_{k_1 + \cdots + k_n}$ as described in \cref{conv1} \eqref{conv:beta_delta}.
%\end{itemize}
%
%\end{rem}

 % \[
  %   \xy
  %     {\ar@{-} (0,0)*{}; (5,-5)*{} };
  %     {\ar@{-} (5,0)*{}; (0,-5)*{} };
  %     {\ar@{-} (12,0)*{}; (17,-5)*{} };
  %     {\ar@{-} (17,0)*{}; (12,-5)*{} };
  %     {\ar@{-} (22,0)*{}; (27,-5)*{} };
  %     {\ar@{-} (27,0)*{}; (22,-5)*{} };
  %     {\ar@{-} (34,0)*{}; (44,-5)*{} };
  %     {\ar@{-} (39,0)*{}; (39,-5)*{} };
  %     {\ar@{-} (44,0)*{}; (34,-5)*{} };
  %     {\ar@{-} (0,-5)*{}; (17,-13)*{} };
  %     {\ar@{-} (5,-5)*{}; (22,-13)*{} };
  %     {\ar@{-} (12,-5)*{}; (29,-13)*{} };
  %     {\ar@{-} (17,-5)*{}; (34,-13)*{} };
  %     {\ar@{-} (22,-5)*{}; (39,-13)*{} };
  %     {\ar@{-} (27,-5)*{}; (44,-13)*{} };
  %     {\ar@{-} (34,-5)*{}; (0,-13)*{} };
  %     {\ar@{-} (39,-5)*{}; (5,-13)*{} };
  %     {\ar@{-} (44,-5)*{}; (10,-13)*{} };
  %   \endxy
  % \]
  
 
%Note that $\trans{1}{2}\trans{3}{4} \in \Sigma(4)$ is actually $\mu(e_{2}; \trans{1}{2}, \trans{1}{2})$, where $e_{2} \in \Sigma_{2}$ is the identity permutation. Using this and operad associativity, one can easily check that
%  \[
%    \mu \left( (1 \, 2 \, 3); \trans{1}{2}, \trans{1}{2}\trans{3}{4}, \trans{1}{3} \right) = \mu \left( (1 \, 2 \, 3 \, 4); \trans{1}{2}, \trans{1}{2}, \trans{1}{2}, \trans{1}{3} \right),
%  \]
%where now the composition on the right side uses the function
%  \[
%    \mu \colon \Sigma(4) \times \Sigma(2) \times \Sigma(2) \times \Sigma(2) \times \Sigma(3) \rightarrow \Sigma(9).
%  \]
%This equality is obvious using the picture above, but verifiable directly using only the algebra of the symmetric operad.

% \acnote{Another example of non-symmetric operad is the operad of \emph{pure} braid groups: see James Griffin's comment below the old \href{https://golem.ph.utexas.edu/category/2014/03/operads_of_finite_groups.html}{blog post}.}

%satisfying the following axioms.
%  \begin{align*}
%    \mu(x;y_1 \cdot \tau_1,\ldots,y_n \cdot \tau_n) &= \mu(x;y_1,\ldots,y_n)\cdot\beta(\tau_1,\ldots,\tau_n)\\
%    \mu(x \cdot \sigma; y_1, \ldots, y_n) &= \mu\left(x;y_{\sigma^{-1}(1)},\ldots,y_{\sigma^{-1}(n)}\right)\cdot \delta_{n; k_1, \ldots, k_n}(\sigma)
%  \end{align*}
%  

%For the above equations to make sense, we require similar conditions on the elements to those in \cref{rem:sigma_conditions}.
% For the above equations to make sense, we must have
%   \begin{itemize}
%       \item $x \in O(n)$,
%       \item $y_{i} \in O(k_{i})$ for $i=1, \ldots, n$,
%       \item $\tau_{i} \in B_{k_{i}}$, and
%       \item $\sigma \in B_{n}$.
%   \end{itemize}

 % \[
  %   \xy
  %     {\ar@{-} (0,0)*{}; (5,-5)*{} };
  %     {\ar@{-} (5,-5)*{}; (29,-10)*{} };
  %     {\ar@{-} (5,0)*{}; (0,-5)*{} };
  %     {\ar@{-} (0,-5)*{}; (24,-10)*{} };
  %     {\ar@{-} (12,0)*{}; (12,-5)*{} };
  %     {\ar@{-} (12,-5)*{}; (0,-10)*{} };
  %     {\ar@{-} (17,0)*{}; (17,-5)*{} };
  %     {\ar@{-} (17,-5)*{}; (5,-10)*{} };
  %     {\ar@{-} (24,0)*{}; (29,-5)*{} };
  %     {\ar@{-} (29,-5)*{}; (17,-10)*{} };
  %     {\ar@{-} (29,0)*{}; (24,-5)*{} };
  %     {\ar@{-} (24,-5)*{}; (12,-10)*{} };
  %     {\ar@{-} (0,-10)*{}; (5,-15)*{} };
  %     {\ar@{-} (5,-10)*{}; (0,-15)*{} };
  %     {\ar@{-} (12,-10)*{}; (17,-15)*{} };
  %     {\ar@{-} (17,-10)*{}; (12,-15)*{} };
  %     {\ar@{-} (24,-10)*{}; (24,-15)*{} };
  %     {\ar@{-} (29,-10)*{}; (29,-15)*{} };
  %     {\ar@{-} (0,-15)*{}; (0,-20)*{} };
  %     {\ar@{-} (5,-15)*{}; (5,-20)*{} };
  %     {\ar@{-} (12,-15)*{}; (24,-20)*{} };
  %     {\ar@{-} (17,-15)*{}; (29,-20)*{} };
  %     {\ar@{-} (24,-15)*{}; (12,-20)*{} };
  %     {\ar@{-} (29,-15)*{}; (17,-20)*{} };
  %     {\ar@{-} (40,0)*{}; (45,-5)*{} };
  %     {\ar@{-} (45,0)*{}; (40,-5)*{} };
  %     {\ar@{-} (52,0)*{}; (52,-5)*{} };
  %     {\ar@{-} (57,0)*{}; (57,-5)*{} };
  %     {\ar@{-} (64,0)*{}; (69,-5)*{} };
  %     {\ar@{-} (69,0)*{}; (64,-5)*{} };
  %     {\ar@{-} (40,-5)*{}; (40,-10)*{} };
  %     {\ar@{-} (45,-5)*{}; (45,-10)*{} };
  %     {\ar@{-} (52,-5)*{}; (57,-10)*{} };
  %     {\ar@{-} (57,-5)*{}; (52,-10)*{} };
  %     {\ar@{-} (64,-5)*{}; (69,-10)*{} };
  %     {\ar@{-} (69,-5)*{}; (64,-10)*{} };
  %     {\ar@{-} (40,-10)*{}; (64,-15)*{} };
  %     {\ar@{-} (45,-10)*{}; (69,-15)*{} };
  %     {\ar@{-} (52,-10)*{}; (40,-15)*{} };
  %     {\ar@{-} (57,-10)*{}; (45,-15)*{} };
  %     {\ar@{-} (64,-10)*{}; (52,-15)*{} };
  %     {\ar@{-} (69,-10)*{}; (57,-15)*{} };
  %     {\ar@{-} (40,-15)*{}; (40,-20)*{} };
  %     {\ar@{-} (45,-15)*{}; (45,-20)*{} };
  %     {\ar@{-} (52,-15)*{}; (64,-20)*{} };
  %     {\ar@{-} (57,-15)*{}; (69,-20)*{} };
  %     {\ar@{-} (64,-15)*{}; (52,-20)*{} };
  %     {\ar@{-} (69,-15)*{}; (57,-20)*{} };
  %     (34.5,-10)*+{=};
  %     (4.5,-25)*{\scriptstyle \mu\left((23);(12),(12), e_2\right) \cdot \mu\left((132); (12), e_2, (12)\right) };
  %     (64.5,-25)*{\scriptstyle \mu\left((23)\cdot (132); e_2 \cdot (12), (12) \cdot e_2, (12) \cdot (12)\right)} ;
  %   \endxy
  % \]
  
  %For the first claim, let $g \in \Lambda(1)$. Then
%  \begin{align*}
%    g & = g \cdot e_{1} \\
%    &= \mu(g; \id) \cdot \mu(\id; e_{1}) \\
%    &= \mu(g \cdot \id; \id \cdot e_{1}) \\
%    &= \mu(g \cdot \id; \id) \\
%    &= g \cdot \id
%  \end{align*}
%using that $e_{1}$ is the unit element for the group structure, that $\id$ is a two-sided unit for operad multiplication, and the final axiom for an action operad together with the fact that the only element of the symmetric group $\Sigma_{1}$ is the identity permutation. Thus $g = g \cdot \id$, so $\id = e_{1}$.

%  Note first that
%    \begin{align*}
%      \mu(e_2; g, h) &= \mu(e_2 \cdot e_2; g \cdot e_0, e_0 \cdot h) \\
%      &= \mu(e_2; g, e_0) \cdot \mu(e_2; e_0, h),
%    \end{align*}
%  so for the first claim that $\mu(e_2; g, h) = g \cdot h$ it will suffice to show that $\mu(e_2; g, e_0) = g$ and $\mu(e_2; e_0, h) = h$ for all $g, h \in \Lambda(0)$. We will use the fact that $\mu(e_0; ) = e_0$, which follows below from a similar argument found in the second part of \cref{calclem}.
%    \begin{align*}
%      \mu(e_0;)\mu(e_0;) &= \mu\left(e_0^2;\right) \\
%      &= \mu(e_0;).
%    \end{align*}
%  Since $\mu(e_0;)$ is an idempotent element of $\Lambda(0)$, then it must be the identity $e_0$. Now we have this, the following sequence of calculations shows that $\mu(e_2; g, e_0) = g$, while a similar calculation would show that $\mu(e_2; e_0, h) = h$. Using \cref{calclem},
%    \begin{align*}
%      g &= \mu(e_1; g) \\
%      &= \mu(\mu(e_2; e_0, e_1); g) \\
%      &= \mu(e_2; \mu(e_0;), \mu(e_1; g)) \\
%      &= \mu(e_2; e_0, g).
%    \end{align*}
%  It remains to show that $\Lambda(0)$ is abelian, which relies on the coincidence of the operad multiplication and the group operation in $\Lambda(0)$, in an Eckmann-Hilton style argument.
%    \begin{align*}
%      g \cdot h &= \mu(e_2; g, h) \\
%      &= \mu(e_2 \cdot e_2; e_0 \cdot g, h \cdot e_0) \\
%      &= \mu(e_2; e_0, h) \cdot \mu(e_2; g, e_0) \\
%      &= h \cdot g.
%    \end{align*}

%Note that the operad of symmetric groups $\Sigma$ has its action operad structure determined by two auxiliary operations, which we have previously used to describe particular types of operadic composition as described in \cref{conv1} \eqref{conv:beta_delta} and \cref{exSigma}. Rather than simply being convenient notation for common occurences of operadic composition, we will use these ideas to give a characterisation of action operads. First we recall these notions in more detail in the case of the symmetric operad $\Sigma$. The first operation is the block sum of permutations which we denote by
%  \[
%    \beta \colon \Sigma_{k_{1}} \times \cdots \times \Sigma_{k_{n}} \rightarrow \Sigma_{K},
%  \]
%where $K = \sum k_{i}$. The second is a kind of diagonal map which is defined for any natural number $n$ together with natural numbers $k_{1}, \ldots, k_{n}$. Then
%  \[
%    \delta = \delta_{n; k_{1}, \ldots, k_{n}} \colon \Sigma_{n} \rightarrow \Sigma_{K},
%  \]
%is defined on $\sigma \in \Sigma_{n}$ by permuting the elements $1, 2, \ldots, k_{1}$ together in a block according to the action of $\sigma \in \Sigma_{n}$ on $1$, then $k_{1}+1, \ldots, k_{1}+k_{2}$ together in a block according to the action of $\sigma$ on $2$, and so on. The first of these, $\beta$, is a group homomorphism, while $\delta$ is a sort of twisted homomorphism, and taken together they define operadic multiplication in $\Sigma$.

% \begin{Defi}\label{Defi:aop_bl}
% Let $\Lambda$ be an action operad. For $h_i \in \Lambda(k_i)$ and $g \in \Lambda(n)$, define
%   \begin{align*}
%     \beta(h_{1}, \ldots, h_{n}) &= \mu(e_n; h_{1}, \ldots, h_{n}), \\
%     \delta_{n; k_{1}, \ldots, k_{n}}(g) &= \mu(g; e_{k_1}, \ldots, e_{k_n}).
%   \end{align*}
% \end{Defi}

%
%
%Assume that $\pi_n$ is zero for some $n \geq 2$. Now let $\sigma \in \operatorname{Im} \pi_{n+1}$, so there exists $g \in \Lambda(n+1)$ such that $\pi_{n+1}(g) = \sigma$. Now $\delta_{n+1;\underline{1},0,\underline{1}}(\sigma)$ is an element of $\Sigma_n$ and
%  \begin{align*}
%    \delta_{n+1;\underline{1},0,\underline{1}}(\sigma) &= \delta_{n+1;\underline{1},0,\underline{1}}(\pi_{n+1})(g) \\
%    &= \pi_n(\delta_{n+1;\underline{1},0,\underline{1}}(g)) \\
%    &= e_n,
%  \end{align*}
%using Axiom \eqref{eq4} of \ref{thm:charAOp}. Hence $\sigma = e_{n+1}$ or $\sigma = \trans{a}{a+1}$.
%
%If $\sigma = e_{n+1}$ then $\pi_{n+1}$ is zero. But if $a \neq 1$, then using Axiom \eqref{eq4} again
%  \begin{align*}
%    \trans{a-1}{a} &= \delta_{n+1;0,\underline{1}}(\trans{a}{a+1}) \\
%    &= \delta_{n+1;0,\underline{1}}(\pi_{n+1}(g)) \\
%    &= \pi_n(\delta_{n+1;0,\underline{1}}(g)) \\
%    &= e_n,
%  \end{align*}
%a contradiction. If $a = 1$, then $\trans{a}{a+1} = \trans{1}{2}$ and again using Axiom \eqref{eq4}
%  \begin{align*}
%    \trans{1}{2} &= \delta_{n+1;\underline{1},0}(\trans{a}{a+1}) \\
%    &= \delta_{n+1;\underline{1},0}(\pi_{n+1}(g)) \\
%    &= \pi_n(\delta_{n+1;\underline{1}}(g),0) \\
%    &= e_n,
%  \end{align*}
%again a contradiction. Hence $\sigma = e_{n+1}$ and so the image of $\pi_{n+1}$ is trivial.

% QQQ Old long proof:
% We will prove each case separately. The two cases coincide for $n = 0$ and $n = 1$ as $\pi_0$ and $\pi_1$ are both the zero map and since $\Sigma_0$ and $\Sigma_1$ are the trivial group then these maps are also surjective.

% Any homomorphism $G \rightarrow \Sigma_2$ must necessarily be surjective or the zero map. To begin we will show that if $\pi_2 \colon \Lambda(2) \rightarrow \Sigma_2$ is surjective then, by induction, the rest of the maps $\pi_n$ are also surjective. If $\pi_2$ is surjective then there exists an element $g_{1,1} \in \Lambda(2)$ such that $\pi_2(g_{1,1}) = \trans{1}{2}$. Now assume that for some $j \geq 2$ that $\pi_j$ is surjective. In particular, this means that for each transposition $\trans{a}{a+k} \in \Sigma_j$, where $1 \leq a \leq j-1$ and $a + k \leq j$, there exists $g_{a,k} \in \Lambda(j)$ such that $\pi_j(g_{a,k}) = \trans{a}{a+k}$. We will use these assumptions to show that each transposition in $\Sigma_{j+1}$ can be written as the image under $\pi_{j+1}$ of some element in $\Lambda(j+1)$, hence any permutation in $\Sigma_{j+1}$ can similarly be written.


% Let $\trans{a}{a+k} \in \Sigma_j$, where $1 \leq a \leq j$ and $a + k \leq j+1$. If $a + k \leq j$, then $\trans{a}{a+k} = \trans{a}{a+k}(j + 1) \in \Sigma_{j+1}$. This transposition can be written as
%   \begin{align*}
%     \trans{a}{a+k} &= \mu^{\Sigma}(e_2 ; \trans{a}{a+k}, e_1) \\
%               &= \mu^{\Sigma}(\pi_2(e_2); \pi_j(g_{a,k}), \pi_1(e_1)) \\
%               &= \pi_{j+1}\left(\mu^{\Lambda}(e_2; g_{a,k},e_1)\right).
%   \end{align*}
% Similarly, if $a > 1$ then this transposition can be written as
%   \begin{align*}
%    \trans{a}{a+k} &= \mu^{\Sigma}(e_2 ; e_1, \trans{a-1}{a+k-1}) \\
%               &= \mu^{\Sigma}(\pi_2(e_2); \pi_1(e_1), \pi_j(g_{a-1,k})) \\
%               &= \pi_{j+1}\left(\mu^{\Lambda}(e_2; e_1, g_{a-1,k})\right).
%   \end{align*}
% Finally, if $a = j$ and $k = 1$, then $\trans{a}{a+k} = \trans{j}{j+1}$. This can be written as
%   \begin{align*}
%     (j \,\,\, j + 1) &= \mu^{\Sigma}(e_2 ; e_{j-1}, (1 \,\,\, 2)) \\
%               &= \mu^{\Sigma}(\pi_2(e_2); \pi_{j-1}(e_{j-1}), \pi_2(g_{1,1})) \\
%               &= \pi_{j+1}\left(\mu^{\Lambda}(e_2; e_{j-1}, g_{1,1})\right).
%   \end{align*}
% Hence $\pi_{j+1}$ is surjective and so all $\pi_n$ are surjective.

% Now instead suppose that $\pi_2 \colon \Lambda(2) \rightarrow \Sigma_2$ is the zero map. Assume that $\pi_j \colon \Lambda(j) \rightarrow \Sigma_j$ is the zero map for some $j \geq 2$. Letting $\sigma \in \textrm{Im}\,\pi_{j+1}$, there exists $g \in \Lambda(j+1)$ such that $\pi_{j+1}(g) = \sigma$. We can then consider the elements $\mu^\Sigma(\sigma; \underline{e_1}, e_o, \underline{e_1})$ where $\underline{e_1}$ means a sequence of $e_1$'s and with $e_0$ in the $k$th position, with $1 \leq k \leq j + 1$. Now
%   \begin{align*}
%     \mu^\Sigma(\sigma; \underline{e_1}, e_o, \underline{e_1}) &= \mu^\Sigma(\pi_{j+1}(g); \underline{\pi_1(e_1)}, \pi_0(e_0), \underline{\pi_1(e_1)}) \\
%     &= \pi_j(\mu^\Lambda(g; \underline{e_1}, e_0, \underline{e_1})) \\
%     &= e_j.
%   \end{align*}
% We can think of this permutation as being $\sigma$ with the $k$th string removed - in \cref{rem:crossed} we comment on such `face' and `degeneracy' maps as used here, and see \cite{ber-simplicial} for a more careful treatment of this idea. Now each of these is the identity $e_j$, as shown above. This means that $\sigma$ must either have been the identity $e_{j+1}$ or a transposition of the form $\trans{a}{a+1}$, where $1 \leq a \leq j$.

% If $\sigma$ is the identity $e_{j+1}$, then we are done since this would give $\textrm{Im}\,\pi_{j+1} = \{e_{j+1}\}$. Instead suppose that $\sigma = \trans{a}{a+1} \in \Sigma_{j + 1}$. Then if $1 < a \leq j$ we can use this to give
%   \begin{align*}
%     \trans{a-1}{a} &= \mu^\Sigma(\trans{a}{a+1}; e_0, \underline{e_1}) \\
%     &= \mu^\Sigma(\pi_{j+1}(g);\pi_0(e_0), \underline{\pi_1(e_1)}) \\
%     &= \pi_j(\mu^\Lambda(g; e_0, \underline{e_1})) \\
%     &= e_j.
%   \end{align*}
% This gives a contradiction, hence $\sigma \neq \trans{a}{a+1}$ and must be the identity $e_{j+1}$.

% Similarly, if $\sigma = \trans{1}{2} \in \Sigma_{j+1}$, then in $\Sigma_j$ we find that
%   \begin{align*}
%     \trans{1}{2} &= \mu^\Sigma(\trans{1}{2} ; \underline{e_1}, e_0) \\
%     &= \mu^\Sigma(\pi_{j+1}(g); \underline{\pi_1(e_1)}, \pi_0(e_0)) \\
%     &= \pi_j(\mu^\Lambda(g ; \underline{e_1}, e_0)) \\
%     &= e_j.
%   \end{align*}
% Again, a contradiction, hence $\sigma = e_{j+1}$.

Examples of action operads:
%
%
%Copied:
%
%In order to make sense of this definition, we must define $\beta(\tau_1,\ldots,\tau_n)$ and $\delta(\sigma)$ in the context of braids. The first is the block sum in the obvious sense:  given $n$ different braids on $k_{1}, \ldots, k_{n}$ strands, respectively, we form a new braid on $k_{1} + \cdots + k_{n}$ strands by taking a disjoint union where the braid $\tau_{i}$ is to the left of $\tau_{j}$ if $i < j$. The braid $\delta(\sigma)$ is obtained by replacing the $i$th strand with $k_{i}$ consecutive strands, all of which are braided together according to $\sigma$. These operations are sometimes referred to as `cabling' operations for braids, as described in, for example, \cite{doucot_local_2025}.
%
%Copied 2:
%
%One can form an operad $B$ where $B(n)$ is the underlying set of the $n$th braid group, $B_{n}$. This is done in much the same way as we did for the symmetric operad using the `cabling' operations for braids described after \cref{broperad}. The collections of maps $\pi_{n} \colon B_{n} \rightarrow \Sigma_{n}$ giving the underlying permutation of each braid constitutes an operad map (of non-symmetric or braided operads) operads $Br \rightarrow \Sigma$.

% \acnoteil{Feeling out how much detail we need for ribbon braid groups to be officially `checked' as an action operad. I'm quite happy with the details, but we might get away with a diagram and some comments to hold off all the checking.}
% For a ribbon braid $t_1^{m_1}\cdots t_n^{m_n}\sigma$ and the Garside twist
%     \[
%         \gamma_n = (\sigma_1 \sigma_2 \cdots \sigma_n)(\sigma_1 \sigma_2 \cdots \sigma_{n-1})\cdots(\sigma_1 \sigma_2)(\sigma_1),
%     \]
% it's something like this:
%     \[
%         \delta^{RB}_{n;k_1,\ldots,k_n}(t_1^{m_1}\cdots t_n^{m_n}\sigma) = \beta(\beta(\underline{t})^{m_1}\gamma_{k_1}^{2m_1},\ldots,\beta(\underline{t})^{m_n}\gamma_{k_n}^{2m_n})\delta^B_{n;k_1,\ldots,k_n}(\sigma)
%     \]
% A lot of the axioms rely on general braids commuting with the full Garside twist on the same number of strands and the fact that $\pi_n(\gamma_n^2) = e_n$ and $\pi(t) = e_1$.
% E.g., for the ribbon braid $\alpha = t_1^2t_2t_3\sigma_1\sigma_2\sigma_2$ and $\delta_{3;3,1,2}$ we get
%     \begin{align*}
%         \delta(\alpha) &= \beta(\beta(t,t,t)^2\gamma_3^4,\beta(t),\beta(t,t)\gamma_2^2)\delta(\sigma_1\sigma_2\sigma_2)\\
%         &= \beta(t^2,t^2,t^2,t,t,t)\beta(\gamma_3^4,e_1,\gamma_2^2)\delta(\sigma_1\sigma_2\sigma_2).
%     \end{align*}
% Actually this might give a better simplification of the formula for $\delta$:
%     \[
%         \delta^{RB}_{n;k_1,\ldots,k_n}(t_1^{m_1},\ldots,t_n^{m_n}\sigma) = \beta(\underline{t^{m_1}},\ldots,\underline{t^{m_n}})\beta(\gamma_{k_1}^{2m_1},\ldots,\gamma_{k_n}^{2m_n})\delta^B_{n;k_1,\ldots,k_n}(\sigma)
%     \]
% where $\underline{t^{m_i}}$ is $t_{m_i}$ repeated $k_i$ times.
% \acnoteil{Something where $\delta_{1;n}(t) = $ `full Garside twist', e.g., $\delta_{1;3}(t) = \beta(t,t,t)(\sigma_1 \sigma_2 \sigma_3)(\sigma_1 \sigma_2)(\sigma_1)(\sigma_1 \sigma_2 \sigma_3)(\sigma_1 \sigma_2)(\sigma_1)$, whichever the right way round is. I don't like the lack of details in Wahl's thesis to deal with the twists for our purposes, so I'm trying to fill them in - there it's mostly just `similar to the symmetric and braid case' apart from some hidden details in proofs. I've mostly written it out and checked Theorem 4.15. but it depends how much detail we want to go into. It's on the level of the cactus operad proof.}

% \ngnoteil{old:}
% For each $n \in \mathbb{N}$, the \emph{ribbon braid group} $RB_{n}$ is the group whose presentation is the same as that of the braid group $B_{n}$, except with the addition of $n$ new generators $t_1, \ldots, t_n$, known as the \emph{twists}. These twists all commute with one other, and also commute with all braids except in the following cases:
%   \begin{align*}
%     b_i \cdot t_i &= t_{i+1} \cdot b_i,\\
%     b_i \cdot t_{i+1} &= t_i \cdot b_i.
%   \end{align*}
% The \emph{ribbon braid operad} $RB$ is then the operad made up of these groups in a way that extends the definition of the braid operad. In other words, the identity is still $e_1 \in RB_1$, and the operadic multiplication is built up in stages in exactly the same ways as in \cref{rem:br-op-needed}, but with some additional rules for dealing with twists. With regards to the tensor product\acnote{tensor product? does this need rewriting with $\beta$ and $\delta$?}, we have that for any twist $t_i \in RB_{n}$,
%   \[
%     t_i = e_{i-1} \otimes t \otimes e_{n-i}
%   \]
% where $t$ is the sole twist in $RB_1$, and for the `block twists' $t_{(m)}$ we again work recursively:
%   \[
%     t_{(0)} = e_n, \quad \quad \quad t_{(m+m')} = \left(t_{(m)} \otimes t_{(m')}\right) \cdot b_{(m', m)} \cdot b_{(m, m')}
%   \]\acnote{not sure what this notation $b_{(m,m')}$ is either?}

% \ngnoteil{as much as I like the picture, I am commenting out the twist}

%Much as the symmetric groups can be represented by crossings of a collection of strings, and the braid groups by braidings of strings, the ribbon braid groups deal with the ways that one can braid together several flat ribbons, including the ability to twist a ribbon about its own axis by 360 degrees. The actual definition of the ribbon braid groups is as the fundamental group of a configuration space in which points have labels in the circle, $S^{1}$; see \cite{sal-wahl}.
%\begin{center} \begin{tabular}{ccc}
%			\begin{tikzpicture}[baseline]
%				\node(xl1) at (-0.7,1){};
%				\node(xr1) at (-0.3,1){};
%				\node(yl1) at (0.3,1){};
%				\node(yr1) at (0.7,1){};
%				\node(yl2) at (-0.7, -1){};
%				\node(yr2) at (-0.3, -1){};
%				\node(xl2) at (0.3, -1){};
%				\node(xr2) at (0.7, -1){};
%				\node(b) at (0,0)[circle,fill=white, minimum size=0.5cm]{};
%       				\draw[rounded corners](xl1.north) to (-0.7,0.5) to (0.3,-0.5) to (xl2.south);
%       				\draw[rounded corners](xr1.north) to (-0.3,0.5) to (0.7,-0.5) to (xr2.south);
%				\begin{pgfonlayer}{bg}
%				\draw[rounded corners](yl1.north) to (0.3, 0.5) to (-0.7, -0.5) to (yl2.south);
%				\draw[rounded corners](yr1.north) to (0.7, 0.5) to (-0.3, -0.5) to (yr2.south);
%    				\end{pgfonlayer}
%				\draw(xl1.north) to (xr1.north);
%				\draw(xl2.south) to (xr2.south);
%				\draw(yl1.north) to (yr1.north);
%				\draw(yl2.south) to (yr2.south);
%			\end{tikzpicture} & \quad \quad \quad \quad \quad \quad \quad &
%			\begin{tikzpicture}[baseline]
%				\node(xl1) at (-0.2,1){};
%				\node(xr1) at (0.2,1){};	
%				\node(xl2) at (-0.2, -1){};
%				\node(xr2) at (0.2, -1){};
%				\draw[rounded corners](xl1.north) to (-0.2,0.4) to (0.2, 0.3) to (0.2, -0.3) to (-0.2, -0.4) to (xl2.south);	
%       				\draw[rounded corners](xr1.north) to (0.2,0.4) to (-0.2, 0.3) to (-0.2, -0.3) to (0.2, -0.4) to (xr2.south);
%				\draw(xl1.north) to (xr1.north);
%				\draw(xl2.south) to (xr2.south);	
%			\end{tikzpicture} \\
%			$b$ & & $t$ 
%\end{tabular} \end{center}
% This operad $RB$ is also clearly an action operad, since we can just define $\pi^{RB}_n \colon RB_{n} \rightarrow \Sigma_n$ to act like $\pi^B_n$ on any braids, at which point the fact that $\pi(t) \in S_1 = \{e_1\}$ will automatically take care of the twists.

%\begin{enumerate}
%\item We now describe two less trivial action operads, those given by the braid groups (\cref{ex:braid_operad_B}), $\Lambda = B$, and the ribbon braid groups, $\Lambda = RB$. In each case, the homomorphism $\pi$ is given by taking underlying permutations, and the operad structure is given geometrically by using the procedure explained after \cref{broperad}. For $RB$, the fact that $\pi(t) \in \Sigma_1 = \{e_1\}$ automatically takes care of the twists. We refer the reader to \cite{fie-br} for more information about braided operads, and to \cite{sal-wahl, wahl-thesis} for information about the ribbon case.

 % as permutations $\pi$ of the set $\{-n, 1-n, \ldots, -1, 1, \ldots, n-1, n\}$ such that $\pi(i) = -\pi(-i)$ for all $i$, as the subgroup of $O(n)$ consisting of those matrices with all integer coefficients, or as invertible $n \times n$-matrices whose entries consist of $-1$, $0$, or $1$ and in which each row and column has exactly one non-zero entry. 
%E.g., we can consider an element of $H_3$ as a permutation matrix and a $3$-tuple of elements of $C_2 = \{-1,1\}$, or simply as a signed permutation matrix:
%  \[
%    \left(\trans{2}{3}
%    ;
%    -1,1,-1\right)
%    =
%    \left(\begin{bmatrix}
%    1 & 0 & 0 \\
%    0 & 0 & 1 \\
%    0 & 1 & 0
%    \end{bmatrix}
%    ;
%    -1,1,-1\right)
%    =
%    \begin{bmatrix}
%    -1 & 0 & 0 \\
%    0 & 0 & -1 \\
%    0 & 1 & 0
%    \end{bmatrix}
%    \in H_3.
%  \]
%E.g.,
%  \[
%    \beta\left(
%      \begin{bmatrix}
%      -1 & 0 & 0 \\
%      0 & 0 & -1 \\
%      0 & 1 & 0
%      \end{bmatrix},
%      \begin{bmatrix}
%      0 & 1 \\
%      -1 & 0
%      \end{bmatrix}
%    \right)
%    =
%    \begin{bmatrix}
%      -1 &  0 &  0 &  0 & 0 \\
%      0  &  0 & -1 &  0 & 0 \\
%      0  &  1 &  0 &  0 & 0 \\
%      0  &  0 &  0 &  0 & 1 \\
%      0  &  0 &  0 & -1 & 0
%    \end{bmatrix}.
%  \]
        % \[
        %   \delta_{1;3}([-1])\beta(\sigma) = r_n \cdot \sigma =
        %   \begin{bmatrix}
        %   0 & -1 & 0 \\
        %   0 & 0 & 1 \\
        %   1 & 0 & 0
        %   \end{bmatrix}
        % \]
  % \[
  %   \beta(\sigma)\delta_{1;3}([-1]) = \sigma \cdot r_n =
  %   \begin{bmatrix}
  %   0 & 0 & 1 \\
  %   1 & 0 & 0 \\
  %   0 & -1 & 0
  %   \end{bmatrix}.
  % \]

Extensions:
%We have the following corollary.
%\begin{cor}\label{corZ}
%For an action operad $\Lambda$, the sets
%  \[
%    \mathrm{Ker}\,\pi_n = \{g \in \Lambda(n)~\colon~\pi_{n}(g) = e_{n} \}
%  \]
%form an action operad for which the inclusion $\mathrm{Ker}\,\pi \hookrightarrow \Lambda$ is a map of action operads.
%\end{cor}
%\begin{proof}
%For $\textrm{Ker}\,\pi \hookrightarrow \Lambda$ to be a map of action operads, we must define the map $\textrm{Ker}\,\pi \rightarrow \Sigma$ to be zero,  and we must check that the operadic multiplication of elements in the kernel is also in the kernel. This last fact is a trivial consequence of $\pi$ being an operad map.
%\end{proof}
%
%\begin{cor}\label{image}
%For an action operad $\Lambda$, the sets
%  \[
%    \mathrm{Im}\,\pi_n = \{\pi_n(g)~\colon~g \in \Lambda(n)\}
%  \]
%form an action operad for which the inclusion $\mathrm{Im}\,\pi \hookrightarrow \Sigma$ is a map of action operads.
%\end{cor}
%\begin{proof}
%The operad multiplication of elements in the image is also in the image, as is the unit element in $\mathrm{Im}\,\pi_1$, following again as a consequence of $\pi$ being an operad map. The map $\mathrm{Im}\,\pi \hookrightarrow \Sigma$ is an inclusion and so is immediately seen to be a map of action operads.
%\end{proof}

Presentations:
%A useful method for constructing new examples of some given algebraic structure is through the use of presentations. A presentation consists of generating data together with relations between generators using the operations of the algebra involved. In categorical terms, the generators and relations are both given as free gadgets on some underlying data, and the presentation itself is a coequalizer. This section will establish the categorical structure necessary to give presentations for action operads, and then explain how such a presentation is reflected in the associated club and $2$-monad. The most direct route to the desired results uses the theory of locally finitely presentable categories. We recall the main definitions briefly, but refer the reader to \cite{ar} for additional details.
%
%\begin{Defi}\label{def:filtered}
%  A \textit{filtered category} is a nonempty category $C$ such that
%    \begin{itemize}
%      \item if $a,b$ are objects of $C$, then there exists another object $c \in C$ and morphisms $a \rightarrow c, b \rightarrow c$; and
%      \item if $f,g \colon a \rightarrow b$ are parallel morphisms in $C$, then there exists a morphism $h \colon b \rightarrow c$ such that $hf = hg$.
%    \end{itemize}
%\end{Defi}
%
%\begin{Defi}
%  A \emph{filtered colimit} is a colimit over a filtered category.
%\end{Defi}
%
%\begin{Defi}
%  Let $C$ be a category with all filtered colimits. An object $x \in C$ is \textit{finitely presentable} if the representable functor $C(x, -) \colon C \rightarrow \mb{Sets}$ preserves filtered colimits.
%\end{Defi}
%
%\begin{Defi}
%  A \textit{locally finitely presentable category} is a category $C$ such that
%  \begin{itemize}
%    \item $C$ is cocomplete and
%    \item there exists a small subcategory $C_{fp} \subseteq C$ of finitely presentable objects such that any object $x \in C$ is the filtered colimit of some diagram in $C_{fp}$.
%  \end{itemize}
%\end{Defi}
%
%The definition of a locally finitely presentable category has many equivalent variants, but we find this one most practicable to work with in this setting.
%
%\begin{thm}
%The category $\mb{AOp}$ is locally finitely presentable.
%\end{thm}
%\begin{proof}
%First note that we can define a category $\mb{Op}^{g}$, whose objects are operads $P$ in which each $P(n)$ also carries a group structure. This is an equational theory using equations with only finitely many elements, so $\mb{Op}^{g}$ is locally finitely presentable \cite[Corollary 3.7]{ar}. A slice category of a locally finitely presentable category is itself locally finite presentable \cite[Proposition 1.57]{ar} and since the symmetric operad is an object of $\mb{Op}^{g}$, the slice category $\mb{Op}^{g}/\Sigma$ is locally finitely presentable.
%
%There is an obvious inclusion functor $\mb{AOp} \hookrightarrow \mb{Op}^{g}/\Sigma$. Now $\mb{AOp}$ is a full subcategory of $\mb{Op}^{g}/\Sigma$ which is closed under products, subobjects. Since any object of $\mb{Op}^{g}/\Sigma$ isomorphic to an action operad is in fact an action operad, the inclusion   $\mb{AOp} \hookrightarrow \mb{Op}^{g}/\Sigma$ is actually the inclusion of a reflective subcategory. One can easily check that $\mb{AOp}$ is in fact closed under all limits and filtered colimits in $\mb{Op}^{g}/\Sigma$, so by the Reflection Theorem (2.48 in \cite{ar}), $\mb{AOp}$ is locally finitely presentable.
%\end{proof}
%
%
%
%\begin{rem}
%In standard presentations of the theory of operads (see, for example, \cite{mss-op}), a nonsymmetric operad will have an underlying collection (or $\mathbb{N}$-indexed collection of sets) while a symmetric operad will have an underlying symmetric collection (or $\mathbb{N}$-indexed collection of sets in which the $n$th set has an action of $\Sigma_{n}$). Our collections over $\SS$ more closely resemble the former as there is no group action present.
%\end{rem}
%
%\begin{example}
%One can easily form new action operads from old ones by taking limits. To take a limit of a diagram in $\mb{AOp}$, one forgets down to the category of operads over $\Sigma$ and takes the limit there. Concretely, products in $\mb{AOp}$ are computed as products in $\mb{Op}/\Sigma$ which themselves are (possibly wide) pullbacks in the category of operads. This pullback will then be computed levelwise, showing that at each dimension there is a group structure with a group homomorphism to the appropriate $\Sigma_{n}$ and that the final action operad axiom holds since it does in each component. The equalizer of a pair of maps will then just be the levelwise equalizer. This shows that the pointwise product of an action operad $P$ with an action operad of the form $Z(Q)$ (as in \cref{Z}) is again an action operad, but the pointwise product of two arbitrary action operads might not be.
%\end{example}
%\begin{thm}\label{underlyingSS}
%There exists a forgetful functor $U \colon \mb{AOp} \rightarrow \mb{Sets}/\SS$ which preserves all limits and filtered colimits.
%\end{thm}
%
%\begin{proof}
%For a given action operad $\Lambda$, we put $U(\Lambda) = \left(\coprod_{\mathbb{N}} \Lambda(n), \coprod_{\mathbb{N}} \pi_n \right)$ and this easily extends to a mapping on morphisms using the universal property of the coproduct. The preservation of filtered colimits follows from the fact that these are computed pointwise, together with the fact that every map between action operads preserves underlying permutations.
%As equalizers are computed levelwise, and the product $\Lambda \times \Lambda'$ has underlying operad the pullback $\Lambda \times_{\Sigma} \Lambda'$; this pullback is itself computed levelwise. Together, these imply that $U$ also preserves all limits.
%\end{proof}
%
%\begin{proof}
%The category $\mb{Sets}/\SS$ is locally finitely presentable as it is equivalent to the functor category $[\SS, \mb{Sets}]$ (here $\SS$ is treated as a discrete category) and any presheaf category is locally finitely presentable. The functor $U$ preserves limits and filtered colimits between locally finitely presentable categories, so has a left adjoint (see Theorem 1.66 in \cite{ar}).
%\end{proof}


