Removed from operads paper:

Intro:

Original paper intro:

Operads were defined by May \cite{maygeom} in the early 70's to provide a convenient tool to approach problems in algebraic topology, notably the question of when a space $X$ admits an $n$-fold delooping $Y$ so that $X \simeq \Omega^{n}Y$.  An operad, like an algebraic theory \cite{lawvere-thesis}, is something like a presentation for a monad or algebraic structure.  The theory of operads has seen great success, and we would like to highlight two reasons.  First, operads can be defined in any suitable symmetric monoidal category, so that there are operads of topological spaces, of chain complexes, of simplicial sets, and of categories, to name a few examples.  Moreover, symmetric (lax) monoidal functors carry operads to operads, so we can use operads in one category to understand objects in another via transport by such a functor.  Second, operads in a fixed category are highly flexible tools.  In particular, the categories listed above all have some inherent notion of ``homotopy equivalence'' which is weaker than that of isomorphism, so we can study operads which are equivalent but not isomorphic.  These tend to have algebras which have similar features in an ``up-to-homotopy'' sense but very different combinatorial or geometric properties arising from the fact that different objects make up these equivalent but not isomorphic operads.

Operads in the category $\mb{Cat}$ of small categories have a unique flavor arising from the fact that $\mb{Cat}$ is not just a category but a 2-category.  These 2-categorical aspects have not been widely treated in the literature, although a few examples can be found.  Lack \cite{lack-cod} mentions braided $\mb{Cat}$-operads (the reader new to braided operads should refer to the work of Fiedorowicz \cite{fie-br}) in his work on coherence for $2$-monads, and Batanin \cite{bat-eh} uses lax morphisms of operads in $\mb{Cat}$ in order to define the notion of an internal operad.  But aside from a few appearances, the basic theory of operads in $\mb{Cat}$ and their 2-categorical properties seems missing.  This paper was partly motivated by the need for such a theory to be explained from the ground up.

There were two additional motivations for the work in this paper.  In thinking about coherence for monoidal functors, the first author was led to a general study of algebras for multicategories internal to $\mb{Cat}$.  These give rise to $2$-monads (or perhaps pseudomonads, depending on how the theory is set up), and checking abstract properties of these $2$-monads prompts one to consider the simpler case of operads in $\mb{Cat}$ instead of multicategories.  The other motivation was from the second author's attempt to understand the interplay between operads in $\mb{Cat}$, operads in $\mb{Top}$, and the passage from (bi)permutative categories to $E_{\infty}$ (ring) spaces.  The first of these motivations raised the issue of when operads in $\mb{Cat}$ are cartesian, while the second led us to consider when an operad in $\mb{Cat}$ possesses a pseudo-commutative structure.

While considering how to best tackle a general discussion of operads in $\mb{Cat}$, it became clear that restricting attention to the two most commonly used types of operads, symmetric and non-symmetric operads, was both short-sighted and unnecessary.  Many theorems apply to both kinds of operads at once, with the difference in proofs being negligible; in fact, most of the arguments which applied to the symmetric case seemed to apply to the case of braided operads as well.   This led us to the notion of an action operad $\mb{G}$, and then to a definition of $\mb{G}$-operads.  In essence, this is merely the general notion of what it means for an operad $P = \{ P(n) \}_{n \in \N}$ to have groups of equivariance $\mb{G} = \{ G(n) \}_{n \in \N}$ such that $G(n)$ acts on $P(n)$.  Choosing different natural families of groups $\mb{G}$, we recover known variants of the definition of operad. \\ \begin{center}
\begin{tabular}{c|c}
Groups $\mb{G}$ & Type of operad  \\ \hline
Terminal groups & Non-symmetric operad \\
Symmetric groups & Symmetric operad \\
Braid groups & Braided operad \\
\end{tabular} \\ \end{center}
These definitions have appeared, with minor variations, in two sources of which we are aware.  In Wahl's thesis \cite{wahl-thesis}, the essential definitions appear but not in complete generality as she requires a surjectivity condition.  Zhang \cite{zhang-grp} also studies these notions\footnote{Zhang calls our action operad a \textit{group operad}.  We dislike this terminology as it seems to imply that we are dealing with an operad in the category of groups, which is not the case unless all of the maps $\pi_{n}:G(n) \rightarrow \Sigma_{n}$ are zero maps.}, once again in the context of homotopy theory, but requires the  superfluous condition that $e_{1} = \textrm{id}$ (see Lemma \ref{lem:calclem}).

This paper consists of the following.  In Section 1, we give the definition of an action operad $\mb{G}$ and a $\mb{G}$-operad.  We develop this definition abstractly so as to apply it in any suitable symmetric monoidal category.   It is standard to express operads as monoids in a particular functor category using a composition tensor product.  In order to show that our $\mb{G}$-operads fit into this philosophy, we must work abstractly and use the calculus of coends together with the Day convolution product \cite{day-thesis}.  The reader uninterested in these details can happily skip them, although we find the route taken here to be quite satisfactory in justifying the axioms for an action operad $\mb{G}$ and the accompanying notion of $\mb{G}$-operad.  Many of our calculations are generalizations of those appearing in work of Kelly \cite{kelly-op}, although there are slight differences in flavor between the two treatments.
%Kelly:  On the operads of JP May

Section 2 works through the basic 2-categorical aspects of operads in $\mb{Cat}$.  We explain how every operad gives rise to a $2$-monad, and show that all of the various 1-cells between algebras of the associated $2$-monad correspond to the obvious sorts of 1-cells one might define between algebras over an operad in $\mb{Cat}$.  Similarly, we show that the algebra 2-cells, using the $2$-monadic approach, correspond to the obvious notion of transformation one would define using the operad.

Section 3 studies three basic 2-categorical properties of an operad, namely the property of being finitary, the property of being 2-cartesian, and the coherence property.  The first of these always holds, as a simple calculation shows.  The second of these turns out to be equivalent to the action of $G(n)$ on $P(n)$ being free for all $n$, at least up to a certain kernel.  In particular, our characterization clearly shows that every non-symmetric operad is 2-cartesian, and that a symmetric operad is 2-cartesian if and only if the symmetric group actions are all free.  (It is useful to note that a $2$-monad on $\mb{Cat}$ is 2-cartesian if and only if the underlying monad on the category of small categories is cartesian in the usual sense as the (strict) 2-pullback of a diagram is the same as its pullback.)  The third property is also easily shown to hold for any $\mb{G}$-operad on $\mb{Cat}$ using a factorization system argument due to Power \cite{power-gen}.

Section 4 then goes on to study the question of when a $\mb{G}$-operad $P$ admits a pseudo-commutative structure.  Such a structure provides the 2-category of algebras with a richer structure that includes well-behaved notions of tensor product, internal hom, and multilinear map that fit together much as the analogous notions do in the category of vector spaces.  When $P$ is contractible (i.e., each $P(n)$ is equivalent to the terminal category), this structure can be obtained from a collection of elements $t_{m,n} \in G(mn)$ satisfying certain properties.  In particular, we show that every contractible symmetric operad is pseudo-commutative, and we prove that there exist such elements $t_{m,n} \in Br_{mn}$ so that every contractible braided operad is pseudo-commutative as well (in fact in two canonical ways).  Thus Section 4 can be seen as a continuation, in the operadic context, of the work in \cite{HP}, and in particular the ``geometric'' proof of the existence of a pseudo-commutative structure for braided strict monoidal categories demonstrates the power of being able to change the groups of equivariance.

The authors would like to thank John Bourke, Martin Hyland, Tom Leinster, and Peter May for various conversations which led to this paper.  While conducting this research, the second author was supported by an EPSRC Early Career Fellowship. 

Original Borel intro:


Categories of interest are often monoidal: sets, topological spaces, and vector spaces are all symmetric monoidal, while the category of finite ordinals (under ordinal sum) is merely monoidal.  But other categories have more exotic monoidal structures.  The first such type of structure discovered was that of a braided monoidal category.  These arise in categories whose morphisms have a geometric flavor like cobordisms embedded in some ambient space \cite{js}, in  categories produced from double loop spaces \cite{fie-br}, and categories of representations over objects like quasitriangular (or braided) bialgebras \cite{street-quantum} .  Another such exotic monoidal structure is that of a coboundary category, arising in examples from the representation theory of quantum groups \cite{drin-quasihopf}.

Going back to the original work of May on iterated loop spaces \cite{maygeom}, operads were defined in both symmetric and nonsymmetric varieties.  But Fiedorowicz's work on double loop spaces \cite{fie-br} showed that there was utility in considering another kind of operad, this time with braid group actions instead of symmetric group actions.  There is a clear parallel between these definitions of different types of operads and the definitions of different kinds of monoidal category, with each given by some general schema in which varying an $\mathbb{N}$-indexed collection of groups produced the types of operads or monoidal categories seen in nature.  Building on the work in \cite{cg}, the goal of this paper is to show that this parallel can be upgraded from an intuition to precise mathematics using the notion of action operad.

An action operad $\mb{\Lambda}$ is an operad which incorporates all of the essential features of the operad of symmetric groups.  Thus $\Lambda(n)$ is no longer just a set, but instead also has a group structure together with a map $\pi_{n}:\Lambda(n) \to \Sigma_{n}$.  Operadic composition then satisfies an additional equivariance condition using the maps $\pi_n$ and the group structures.  Each action operad $\mb{\Lambda}$ produces a notion of $\mb{\Lambda}$-operad which encodes equivariance conditions using both the groups $\Lambda(n)$ and the maps $\pi_n$.  Examples include the symmetric groups, the terminal groups (giving nonsymmetric operads), the braid groups (giving braided operads), and the $n$-fruit cactus groups \cite{hk-cobound} (giving a new notion of operad one might call cactus operads).  Using a formula resembling the classical Borel construction for spaces with a group action, we can produce from any action operad $\mb{\Lambda}$ a notion of $\mb{\Lambda}$-monoidal category, in which the group $\Lambda(n)$ acts naturally on $n$-fold tensor powers of any object.  Thus the categorical Borel construction embeds action operads into a category of monads on $\mb{Cat}$, and we characterize the image of this embedding as those monads describing monoidal structures of a precise kind.

The paper is organized into the following sections.  Section 1 reviews the definition of an action operad, and defines the categorical Borel construction on them.  The key result, which reappears in proofs throughout the paper, is \cref{thm:charAOp}, characterizing action operads in terms of two new operations mimicking the block sum of permutations and the operation which takes a permutation of $n$ letters and produces a new permutation on $k_1 + k_2 + \cdots + k_n$ letters by permuting the blocks of $k_i$ letters.  In Section 2, we use this characterization and Kelly's theory of clubs \cite{kelly_club1, kelly_club0, kelly_club2} to embed action operads into monads on $\mb{Cat}$ and determine the essential image of this embedding.  Section 3 gives a construction of the free action operad from a suitable collection of data, and relates this to how clubs can be described using generators and relations.  The results of Sections 2 and 3 show that the definitions of symmetric monoidal category or coboundary category, for example, correspond to the action operad constructed from the corresponding free symmetric monoidal or coboundary category on one object; these and other examples appear in detail in Section 4.  Section 5 then extends the definition of $\mb{\Lambda}$-operad to that of $\mb{\Lambda}$-multicategory and shows that these arise abstractly via a Kleisli construction.

Copied from text:
Yau \cite{yau_infinity_2021} collects together a large number of results on the topic of action operads while also investigating the setting of infinity group operads. 


Further acknowledgements:
Alex needs to thank the LMS for a Research Reboot grant. Anybody else we've talked to about these things since their inception? Angelica? Niles? Dan Graves. Nathaniel Arkor.

Conventions:

%\begin{conv}\label{conv1}
%We adopt the following conventions throughout.
%\begin{enumerate}
%\item\label{conv:symm_sigma} $\Sigma_{n}$ is the symmetric group on $n$ letters, and $B_{n}$ is the braid group on $n$ strands.
%\item\label{conv:g-action} For a group $G$, a right $G$-action on a set $X$ will be denoted $(x,g) \mapsto x \cdot g$. We will use both $\cdot$ and concatenation to represent multiplication in a group.
%\item\label{conv:e_identity} The symbol $e$ will generically represent an identity element in a group. If we are considering a set of groups $\{ \Lambda(n) \}_{n \in \N}$ indexed by the natural numbers, then $e_{n}$ is the identity element in $\Lambda(n)$. We will often drop the subscripts and just write $e$ when the index $n$ in $\Lambda(n)$ is either clear from context or unimportant to the argument at hand. Occasionally we will write $\Lambda_n$ in place of $\Lambda(n)$, especially in diagrams.
%\item\label{conv:coeq} We will often be interested in elements of a product of the form
%\[
%A \times B_{1} \times \cdots \times B_{n} \times C
%\]
%(or similar, for example without $C$). We will write elements of this set as $(a; b_{1}, \ldots, b_{n}; c)$, and in the case that we need equivalence classes of such elements they will be written as $[a; b_{1}, \ldots, b_{n}; c]$. This will often be the case when we are interested in the coequalizer of left and right group actions in the following sense. A coequalizer of maps
%    \[
%        \xy
%            (0,0)*+{A \times G \times B}="00";
%            (30,0)*+{A \times B}="10";
%            (60,0)*+{\coeqb{A}{B}{G}}="20";
%            {\ar@<1ex>^{\lambda} "00" ; "10"};
%            {\ar@<-1ex>_{\rho} "00" ; "10"};
%            {\ar^{\varepsilon} "10" ; "20"};
%        \endxy
%    \]
%will be written as $\coeqb{A}{B}{G}$, where $\rho$ represents a right action of $G$ on $A$, and $\lambda$ a left action of $G$ on $B$. This is similar to the notation often used to denote pullbacks, however we find in this work that no confusion arises from using notation in this way.
%\item\label{conv:beta_delta} In the following definitions of operads, we define operad multiplication as a function
%  \[
%    \mu \colon O(n) \times O(k_1) \times \ldots O(k_n) \rightarrow O(k_1 + \cdots + k_n)
%  \]
%for each $n$, $k_1$, $\ldots$, $k_n$, and we use the following two functions as a shorthand for two commonly occuring instances of such. First we define a function
%  \[
%    \beta \colon O(k_1) \times \ldots \times O(k_n) \rightarrow O(k_1 + \cdots + k_n)
%  \]
%for each $n$, $k_1$, $\ldots$, $k_n$, which takes elements $\tau_1 \in O(k_1)$, $\ldots$, $\tau_n \in O(k_n)$ and produces the element
%  \[
%    \beta(\tau_1, \ldots, \tau_n) = \mu(e_n; \tau_1, \ldots, \tau_n).
%  \]
%We think of this element as the block sum of the elements $\tau_1$, $\ldots$, $\tau_n$. We also define a function
%  \[
%    \delta_{n;k_1,\ldots,k_n} \colon O(n) \rightarrow O(k_1 + \cdots + k_n)
%  \]
%for each $n$, $k_1$, $\ldots$, $k_n$, which takes an element $\sigma \in O(n)$ and produces the element
%  \[
%    \delta(\sigma) = \mu(\sigma;e_{k_1}, \ldots, e_{k_n}).
%  \]
%
%In the particular case of the symmetric groups, these are functions
%  \[
%    \beta \colon \Sigma_{k_1} \times \Sigma_{k_n} \rightarrow \Sigma_{k_1 + \cdots + k_n}
%  \]
%and
%  \[
%    \delta_{n;k_1,\ldots,k_n} \colon \Sigma_{n} \rightarrow \Sigma_{k_1 + \cdots + k_n}.
%  \]
%In the case of $\beta$ we form the block sum permutation $\beta(\tau_1,\ldots,\tau_n)$ which permutes the first $k_{1}$ elements according to $\tau_{1}$, the next $k_{2}$ elements according to $\tau_{2}$ and so on; this is an element of $\Sigma_{k_{1} + \cdots + k_{n}}$. For $\delta$ we take the permutation $\delta(\sigma) \in \Sigma_{k_{1} + \cdots + k_{n}}$ to be that which permutes the $n$ different blocks $1$ through $k_{1}$, $k_{1}+1$ through $k_{1} + k_{2}$, and so on, according to the permutation $\sigma \in \Sigma_{n}$. We expand on this at various points, including in \cref{rem:perm_matrices}.
%\item\label{conv:perm_shorthand} Throughout we will be using maps $\pi_n \colon O(n) \rightarrow \Sigma_n$, where $O(n)$ is the object of $n$-ary operations of an operad $O$ and $\Sigma_n$ is the symmetric group on $n$ elements. The map $\pi_n$ in each case will represent a form of `underlying permutation' of each element, which we will then use to act on operad multiplication. As the notation starts to become cumbersome, we will often write $\sigma^{-1}(i)$ which should be read as $\pi_n(\sigma)^{-1}(i)$, where $\sigma \in O(n)$.
%\item\label{conv:op_superscript} We adopt the convention that if an equation requires using operadic composition in more than one operad, we will indicate this by a superscript on each instance of $\mu$ unless it is entirely clear from context. This can be seen, for example, in \cref{Defi:op_map}.
%\end{enumerate}
%\end{conv}

Operads:
Some old axiom diagrams
% \[
% \xy
% (0,0)*+{\scriptstyle O(n) \times O(k_{1}) \times \cdots \times O(k_{n}) \times O(l_{1,1}) \times \cdots \times O(l_{{1},k_{1}}) \times \cdots \times O(l_{n,1}) \times \cdots \times O(l_{{n},k_{n}})} ="00";
% (0,-50)*+{\scriptstyle O(k_{1} + \cdots + k_{n}) \times O(l_{1,1}) \times \cdots \times O(l_{{1},k_{1}}) \times \cdots \times O(l_{n,1})\times \cdots \times O(l_{{n},k_{n}})} ="02";
% (55,-10)*+{\scriptstyle O(n) \times \prod_{i=1}^n O(k_{i}) \times O(l_{i,1}) \times \cdots \times O(l_{{i}, k_{i}}) } ="20";
% (55,-25)*+{\scriptstyle O(n) \times O(\sum l_{1,-}) \times \cdots \times O(\sum l_{n,-})} ="21";
% (55, -40)*+{\scriptstyle  O(\sum l_{-,-})} ="22";
% {\ar_{\scriptstyle \mu \times 1} "00" ; "02"};
% {\ar_{\mu} "02" ; "22"};
% {\ar^{\cong} "00" ; "20"};
% {\ar^{1 \times \prod \mu} "20" ; "21"};
% {\ar^{\mu} "21" ; "22"};
% \endxy
% \]
%
%the following two equations hold.
%  \begin{align*}
%    \mu(x;y_1 \cdot \tau_1,\ldots,y_n \cdot \tau_n) &= \mu(x;y_1,\ldots,y_n)\cdot \beta(\tau_1,\ldots,\tau_n)\\
%    \mu(x \cdot \sigma; y_1, \ldots, y_n) &= \mu\left(x;y_{\sigma^{-1}(1)},\ldots,y_{\sigma^{-1}(n)}\right)\cdot \delta(\sigma)
%  \end{align*}
%  % \begin{align*}
%  %   \mu(x;y_1 \cdot \tau_1,\ldots,y_n \cdot \tau_n) &= \mu(x;y_1,\ldots,y_n)\cdot(\tau_1 \oplus \ldots \oplus \tau_n)\\
%  %   \mu(x \cdot \sigma; y_1, \ldots, y_n) &= \mu\left(x;y_{\sigma^{-1}(1)},\ldots,y_{\sigma_{-1}(n)}\right)\cdot \sigma^+
%  % \end{align*}
%  For the first equation above, we 
 % \[
  %   \xy
  %     {\ar@{-} (0,0)*{}; (25,-10)*{} };
  %     {\ar@{-} (0,-20)*{}; (25,-10)*{} };
  %     {\ar@{-} (0,0)*{}; (0,-20)*{} };
  %     {\ar@{-} (25,-10)*{}; (35,-10)*{} };
  %     {\ar@{-} (0,0)*{}; (-10,0)*{} };
  %     {\ar@{-} (0,-3)*{}; (-10,-3)*{} };
  %     {\ar@{-} (0,-17)*{}; (-10,-17)*{} };
  %     {\ar@{-} (0,-20)*{}; (-10,-20)*{} };
  %     (11,-10)*{x}; (-5,-10)*{\vdots}
  %   \endxy
  % \]
    % \[
  %   \xy
  %     {\ar@{-} (0,0)*{}; (25,-10)*{} };
  %     {\ar@{-} (0,-20)*{}; (25,-10)*{} };
  %     {\ar@{-} (0,0)*{}; (0,-20)*{} };
  %     {\ar@{-} (25,-10)*{}; (35,-10)*{} };
  %     {\ar@{-} (0,-3)*{}; (-10,-3)*{} };
  %     {\ar@{-} (0,-17)*{}; (-10,-17)*{} };
  %     (11,-10)*{x};
  %     {\ar@{-} (-25,2)*{}; (-10,-3)*{} };
  %     {\ar@{-} (-25,-8)*{}; (-10,-3)*{} };
  %     {\ar@{-} (-25,2)*{}; (-25,-8)*{} };
  %     {\ar@{-} (-25,1)*{}; (-30,1)*{} };
  %     {\ar@{-} (-30,-7)*{}; (-25,-7)*{} };
  %     (-19,-3)*{y_{1}};
  %     {\ar@{-} (-25,-12)*{}; (-10,-17)*{} };
  %     {\ar@{-} (-25,-22)*{}; (-10,-17)*{} };
  %     {\ar@{-} (-25,-12)*{}; (-25,-22)*{} };
  %     {\ar@{-} (-25,-13)*{}; (-30,-13)*{} };
  %     {\ar@{-} (-25,-17)*{}; (-30,-17)*{} };
  %     {\ar@{-} (-25,-21)*{}; (-30,-21)*{} };
  %     (-19,-17)*{y_{2}};
  %   \endxy
  % \]
  %\begin{rem}\label{Rem:sigma_conditions}
%It is useful to write out in full what the sets in the diagram of the second axiom above mean. The use of numerous products and indices is to save space but the full picture becomes much clearer when these are expanded. For the equations in the third axiom above to make sense, we must have
%\begin{itemize}
%\item $x \in O(n)$,
%\item $y_{i} \in O(k_{i})$ for $i=1, \ldots, n$,
%\item $\tau_{i} \in \Sigma_{k_{i}}$,
%\item $\sigma \in \Sigma_{n}$, and
%\item $\beta(\tau_1,\ldots,\tau_n), \delta(\sigma) \in \Sigma_{k_1 + \cdots + k_n}$ as described in \cref{conv1} \eqref{conv:beta_delta}.
%\end{itemize}
%
%\end{rem}

 % \[
  %   \xy
  %     {\ar@{-} (0,0)*{}; (5,-5)*{} };
  %     {\ar@{-} (5,0)*{}; (0,-5)*{} };
  %     {\ar@{-} (12,0)*{}; (17,-5)*{} };
  %     {\ar@{-} (17,0)*{}; (12,-5)*{} };
  %     {\ar@{-} (22,0)*{}; (27,-5)*{} };
  %     {\ar@{-} (27,0)*{}; (22,-5)*{} };
  %     {\ar@{-} (34,0)*{}; (44,-5)*{} };
  %     {\ar@{-} (39,0)*{}; (39,-5)*{} };
  %     {\ar@{-} (44,0)*{}; (34,-5)*{} };
  %     {\ar@{-} (0,-5)*{}; (17,-13)*{} };
  %     {\ar@{-} (5,-5)*{}; (22,-13)*{} };
  %     {\ar@{-} (12,-5)*{}; (29,-13)*{} };
  %     {\ar@{-} (17,-5)*{}; (34,-13)*{} };
  %     {\ar@{-} (22,-5)*{}; (39,-13)*{} };
  %     {\ar@{-} (27,-5)*{}; (44,-13)*{} };
  %     {\ar@{-} (34,-5)*{}; (0,-13)*{} };
  %     {\ar@{-} (39,-5)*{}; (5,-13)*{} };
  %     {\ar@{-} (44,-5)*{}; (10,-13)*{} };
  %   \endxy
  % \]
  
 
%Note that $\trans{1}{2}\trans{3}{4} \in \Sigma(4)$ is actually $\mu(e_{2}; \trans{1}{2}, \trans{1}{2})$, where $e_{2} \in \Sigma_{2}$ is the identity permutation. Using this and operad associativity, one can easily check that
%  \[
%    \mu \left( (1 \, 2 \, 3); \trans{1}{2}, \trans{1}{2}\trans{3}{4}, \trans{1}{3} \right) = \mu \left( (1 \, 2 \, 3 \, 4); \trans{1}{2}, \trans{1}{2}, \trans{1}{2}, \trans{1}{3} \right),
%  \]
%where now the composition on the right side uses the function
%  \[
%    \mu \colon \Sigma(4) \times \Sigma(2) \times \Sigma(2) \times \Sigma(2) \times \Sigma(3) \rightarrow \Sigma(9).
%  \]
%This equality is obvious using the picture above, but verifiable directly using only the algebra of the symmetric operad.

% \acnote{Another example of non-symmetric operad is the operad of \emph{pure} braid groups: see James Griffin's comment below the old \href{https://golem.ph.utexas.edu/category/2014/03/operads_of_finite_groups.html}{blog post}.}

%satisfying the following axioms.
%  \begin{align*}
%    \mu(x;y_1 \cdot \tau_1,\ldots,y_n \cdot \tau_n) &= \mu(x;y_1,\ldots,y_n)\cdot\beta(\tau_1,\ldots,\tau_n)\\
%    \mu(x \cdot \sigma; y_1, \ldots, y_n) &= \mu\left(x;y_{\sigma^{-1}(1)},\ldots,y_{\sigma^{-1}(n)}\right)\cdot \delta_{n; k_1, \ldots, k_n}(\sigma)
%  \end{align*}
%  

%For the above equations to make sense, we require similar conditions on the elements to those in \cref{rem:sigma_conditions}.
% For the above equations to make sense, we must have
%   \begin{itemize}
%       \item $x \in O(n)$,
%       \item $y_{i} \in O(k_{i})$ for $i=1, \ldots, n$,
%       \item $\tau_{i} \in B_{k_{i}}$, and
%       \item $\sigma \in B_{n}$.
%   \end{itemize}

 % \[
  %   \xy
  %     {\ar@{-} (0,0)*{}; (5,-5)*{} };
  %     {\ar@{-} (5,-5)*{}; (29,-10)*{} };
  %     {\ar@{-} (5,0)*{}; (0,-5)*{} };
  %     {\ar@{-} (0,-5)*{}; (24,-10)*{} };
  %     {\ar@{-} (12,0)*{}; (12,-5)*{} };
  %     {\ar@{-} (12,-5)*{}; (0,-10)*{} };
  %     {\ar@{-} (17,0)*{}; (17,-5)*{} };
  %     {\ar@{-} (17,-5)*{}; (5,-10)*{} };
  %     {\ar@{-} (24,0)*{}; (29,-5)*{} };
  %     {\ar@{-} (29,-5)*{}; (17,-10)*{} };
  %     {\ar@{-} (29,0)*{}; (24,-5)*{} };
  %     {\ar@{-} (24,-5)*{}; (12,-10)*{} };
  %     {\ar@{-} (0,-10)*{}; (5,-15)*{} };
  %     {\ar@{-} (5,-10)*{}; (0,-15)*{} };
  %     {\ar@{-} (12,-10)*{}; (17,-15)*{} };
  %     {\ar@{-} (17,-10)*{}; (12,-15)*{} };
  %     {\ar@{-} (24,-10)*{}; (24,-15)*{} };
  %     {\ar@{-} (29,-10)*{}; (29,-15)*{} };
  %     {\ar@{-} (0,-15)*{}; (0,-20)*{} };
  %     {\ar@{-} (5,-15)*{}; (5,-20)*{} };
  %     {\ar@{-} (12,-15)*{}; (24,-20)*{} };
  %     {\ar@{-} (17,-15)*{}; (29,-20)*{} };
  %     {\ar@{-} (24,-15)*{}; (12,-20)*{} };
  %     {\ar@{-} (29,-15)*{}; (17,-20)*{} };
  %     {\ar@{-} (40,0)*{}; (45,-5)*{} };
  %     {\ar@{-} (45,0)*{}; (40,-5)*{} };
  %     {\ar@{-} (52,0)*{}; (52,-5)*{} };
  %     {\ar@{-} (57,0)*{}; (57,-5)*{} };
  %     {\ar@{-} (64,0)*{}; (69,-5)*{} };
  %     {\ar@{-} (69,0)*{}; (64,-5)*{} };
  %     {\ar@{-} (40,-5)*{}; (40,-10)*{} };
  %     {\ar@{-} (45,-5)*{}; (45,-10)*{} };
  %     {\ar@{-} (52,-5)*{}; (57,-10)*{} };
  %     {\ar@{-} (57,-5)*{}; (52,-10)*{} };
  %     {\ar@{-} (64,-5)*{}; (69,-10)*{} };
  %     {\ar@{-} (69,-5)*{}; (64,-10)*{} };
  %     {\ar@{-} (40,-10)*{}; (64,-15)*{} };
  %     {\ar@{-} (45,-10)*{}; (69,-15)*{} };
  %     {\ar@{-} (52,-10)*{}; (40,-15)*{} };
  %     {\ar@{-} (57,-10)*{}; (45,-15)*{} };
  %     {\ar@{-} (64,-10)*{}; (52,-15)*{} };
  %     {\ar@{-} (69,-10)*{}; (57,-15)*{} };
  %     {\ar@{-} (40,-15)*{}; (40,-20)*{} };
  %     {\ar@{-} (45,-15)*{}; (45,-20)*{} };
  %     {\ar@{-} (52,-15)*{}; (64,-20)*{} };
  %     {\ar@{-} (57,-15)*{}; (69,-20)*{} };
  %     {\ar@{-} (64,-15)*{}; (52,-20)*{} };
  %     {\ar@{-} (69,-15)*{}; (57,-20)*{} };
  %     (34.5,-10)*+{=};
  %     (4.5,-25)*{\scriptstyle \mu\left((23);(12),(12), e_2\right) \cdot \mu\left((132); (12), e_2, (12)\right) };
  %     (64.5,-25)*{\scriptstyle \mu\left((23)\cdot (132); e_2 \cdot (12), (12) \cdot e_2, (12) \cdot (12)\right)} ;
  %   \endxy
  % \]
  
  %For the first claim, let $g \in \Lambda(1)$. Then
%  \begin{align*}
%    g & = g \cdot e_{1} \\
%    &= \mu(g; \id) \cdot \mu(\id; e_{1}) \\
%    &= \mu(g \cdot \id; \id \cdot e_{1}) \\
%    &= \mu(g \cdot \id; \id) \\
%    &= g \cdot \id
%  \end{align*}
%using that $e_{1}$ is the unit element for the group structure, that $\id$ is a two-sided unit for operad multiplication, and the final axiom for an action operad together with the fact that the only element of the symmetric group $\Sigma_{1}$ is the identity permutation. Thus $g = g \cdot \id$, so $\id = e_{1}$.

%  Note first that
%    \begin{align*}
%      \mu(e_2; g, h) &= \mu(e_2 \cdot e_2; g \cdot e_0, e_0 \cdot h) \\
%      &= \mu(e_2; g, e_0) \cdot \mu(e_2; e_0, h),
%    \end{align*}
%  so for the first claim that $\mu(e_2; g, h) = g \cdot h$ it will suffice to show that $\mu(e_2; g, e_0) = g$ and $\mu(e_2; e_0, h) = h$ for all $g, h \in \Lambda(0)$. We will use the fact that $\mu(e_0; ) = e_0$, which follows below from a similar argument found in the second part of \cref{calclem}.
%    \begin{align*}
%      \mu(e_0;)\mu(e_0;) &= \mu\left(e_0^2;\right) \\
%      &= \mu(e_0;).
%    \end{align*}
%  Since $\mu(e_0;)$ is an idempotent element of $\Lambda(0)$, then it must be the identity $e_0$. Now we have this, the following sequence of calculations shows that $\mu(e_2; g, e_0) = g$, while a similar calculation would show that $\mu(e_2; e_0, h) = h$. Using \cref{calclem},
%    \begin{align*}
%      g &= \mu(e_1; g) \\
%      &= \mu(\mu(e_2; e_0, e_1); g) \\
%      &= \mu(e_2; \mu(e_0;), \mu(e_1; g)) \\
%      &= \mu(e_2; e_0, g).
%    \end{align*}
%  It remains to show that $\Lambda(0)$ is abelian, which relies on the coincidence of the operad multiplication and the group operation in $\Lambda(0)$, in an Eckmann-Hilton style argument.
%    \begin{align*}
%      g \cdot h &= \mu(e_2; g, h) \\
%      &= \mu(e_2 \cdot e_2; e_0 \cdot g, h \cdot e_0) \\
%      &= \mu(e_2; e_0, h) \cdot \mu(e_2; g, e_0) \\
%      &= h \cdot g.
%    \end{align*}

%Note that the operad of symmetric groups $\Sigma$ has its action operad structure determined by two auxiliary operations, which we have previously used to describe particular types of operadic composition as described in \cref{conv1} \eqref{conv:beta_delta} and \cref{exSigma}. Rather than simply being convenient notation for common occurences of operadic composition, we will use these ideas to give a characterisation of action operads. First we recall these notions in more detail in the case of the symmetric operad $\Sigma$. The first operation is the block sum of permutations which we denote by
%  \[
%    \beta \colon \Sigma_{k_{1}} \times \cdots \times \Sigma_{k_{n}} \rightarrow \Sigma_{K},
%  \]
%where $K = \sum k_{i}$. The second is a kind of diagonal map which is defined for any natural number $n$ together with natural numbers $k_{1}, \ldots, k_{n}$. Then
%  \[
%    \delta = \delta_{n; k_{1}, \ldots, k_{n}} \colon \Sigma_{n} \rightarrow \Sigma_{K},
%  \]
%is defined on $\sigma \in \Sigma_{n}$ by permuting the elements $1, 2, \ldots, k_{1}$ together in a block according to the action of $\sigma \in \Sigma_{n}$ on $1$, then $k_{1}+1, \ldots, k_{1}+k_{2}$ together in a block according to the action of $\sigma$ on $2$, and so on. The first of these, $\beta$, is a group homomorphism, while $\delta$ is a sort of twisted homomorphism, and taken together they define operadic multiplication in $\Sigma$.

% \begin{Defi}\label{Defi:aop_bl}
% Let $\Lambda$ be an action operad. For $h_i \in \Lambda(k_i)$ and $g \in \Lambda(n)$, define
%   \begin{align*}
%     \beta(h_{1}, \ldots, h_{n}) &= \mu(e_n; h_{1}, \ldots, h_{n}), \\
%     \delta_{n; k_{1}, \ldots, k_{n}}(g) &= \mu(g; e_{k_1}, \ldots, e_{k_n}).
%   \end{align*}
% \end{Defi}

%
%
%Assume that $\pi_n$ is zero for some $n \geq 2$. Now let $\sigma \in \operatorname{Im} \pi_{n+1}$, so there exists $g \in \Lambda(n+1)$ such that $\pi_{n+1}(g) = \sigma$. Now $\delta_{n+1;\underline{1},0,\underline{1}}(\sigma)$ is an element of $\Sigma_n$ and
%  \begin{align*}
%    \delta_{n+1;\underline{1},0,\underline{1}}(\sigma) &= \delta_{n+1;\underline{1},0,\underline{1}}(\pi_{n+1})(g) \\
%    &= \pi_n(\delta_{n+1;\underline{1},0,\underline{1}}(g)) \\
%    &= e_n,
%  \end{align*}
%using Axiom \eqref{eq4} of \ref{thm:charAOp}. Hence $\sigma = e_{n+1}$ or $\sigma = \trans{a}{a+1}$.
%
%If $\sigma = e_{n+1}$ then $\pi_{n+1}$ is zero. But if $a \neq 1$, then using Axiom \eqref{eq4} again
%  \begin{align*}
%    \trans{a-1}{a} &= \delta_{n+1;0,\underline{1}}(\trans{a}{a+1}) \\
%    &= \delta_{n+1;0,\underline{1}}(\pi_{n+1}(g)) \\
%    &= \pi_n(\delta_{n+1;0,\underline{1}}(g)) \\
%    &= e_n,
%  \end{align*}
%a contradiction. If $a = 1$, then $\trans{a}{a+1} = \trans{1}{2}$ and again using Axiom \eqref{eq4}
%  \begin{align*}
%    \trans{1}{2} &= \delta_{n+1;\underline{1},0}(\trans{a}{a+1}) \\
%    &= \delta_{n+1;\underline{1},0}(\pi_{n+1}(g)) \\
%    &= \pi_n(\delta_{n+1;\underline{1}}(g),0) \\
%    &= e_n,
%  \end{align*}
%again a contradiction. Hence $\sigma = e_{n+1}$ and so the image of $\pi_{n+1}$ is trivial.

% QQQ Old long proof:
% We will prove each case separately. The two cases coincide for $n = 0$ and $n = 1$ as $\pi_0$ and $\pi_1$ are both the zero map and since $\Sigma_0$ and $\Sigma_1$ are the trivial group then these maps are also surjective.

% Any homomorphism $G \rightarrow \Sigma_2$ must necessarily be surjective or the zero map. To begin we will show that if $\pi_2 \colon \Lambda(2) \rightarrow \Sigma_2$ is surjective then, by induction, the rest of the maps $\pi_n$ are also surjective. If $\pi_2$ is surjective then there exists an element $g_{1,1} \in \Lambda(2)$ such that $\pi_2(g_{1,1}) = \trans{1}{2}$. Now assume that for some $j \geq 2$ that $\pi_j$ is surjective. In particular, this means that for each transposition $\trans{a}{a+k} \in \Sigma_j$, where $1 \leq a \leq j-1$ and $a + k \leq j$, there exists $g_{a,k} \in \Lambda(j)$ such that $\pi_j(g_{a,k}) = \trans{a}{a+k}$. We will use these assumptions to show that each transposition in $\Sigma_{j+1}$ can be written as the image under $\pi_{j+1}$ of some element in $\Lambda(j+1)$, hence any permutation in $\Sigma_{j+1}$ can similarly be written.


% Let $\trans{a}{a+k} \in \Sigma_j$, where $1 \leq a \leq j$ and $a + k \leq j+1$. If $a + k \leq j$, then $\trans{a}{a+k} = \trans{a}{a+k}(j + 1) \in \Sigma_{j+1}$. This transposition can be written as
%   \begin{align*}
%     \trans{a}{a+k} &= \mu^{\Sigma}(e_2 ; \trans{a}{a+k}, e_1) \\
%               &= \mu^{\Sigma}(\pi_2(e_2); \pi_j(g_{a,k}), \pi_1(e_1)) \\
%               &= \pi_{j+1}\left(\mu^{\Lambda}(e_2; g_{a,k},e_1)\right).
%   \end{align*}
% Similarly, if $a > 1$ then this transposition can be written as
%   \begin{align*}
%    \trans{a}{a+k} &= \mu^{\Sigma}(e_2 ; e_1, \trans{a-1}{a+k-1}) \\
%               &= \mu^{\Sigma}(\pi_2(e_2); \pi_1(e_1), \pi_j(g_{a-1,k})) \\
%               &= \pi_{j+1}\left(\mu^{\Lambda}(e_2; e_1, g_{a-1,k})\right).
%   \end{align*}
% Finally, if $a = j$ and $k = 1$, then $\trans{a}{a+k} = \trans{j}{j+1}$. This can be written as
%   \begin{align*}
%     (j \,\,\, j + 1) &= \mu^{\Sigma}(e_2 ; e_{j-1}, (1 \,\,\, 2)) \\
%               &= \mu^{\Sigma}(\pi_2(e_2); \pi_{j-1}(e_{j-1}), \pi_2(g_{1,1})) \\
%               &= \pi_{j+1}\left(\mu^{\Lambda}(e_2; e_{j-1}, g_{1,1})\right).
%   \end{align*}
% Hence $\pi_{j+1}$ is surjective and so all $\pi_n$ are surjective.

% Now instead suppose that $\pi_2 \colon \Lambda(2) \rightarrow \Sigma_2$ is the zero map. Assume that $\pi_j \colon \Lambda(j) \rightarrow \Sigma_j$ is the zero map for some $j \geq 2$. Letting $\sigma \in \textrm{Im}\,\pi_{j+1}$, there exists $g \in \Lambda(j+1)$ such that $\pi_{j+1}(g) = \sigma$. We can then consider the elements $\mu^\Sigma(\sigma; \underline{e_1}, e_o, \underline{e_1})$ where $\underline{e_1}$ means a sequence of $e_1$'s and with $e_0$ in the $k$th position, with $1 \leq k \leq j + 1$. Now
%   \begin{align*}
%     \mu^\Sigma(\sigma; \underline{e_1}, e_o, \underline{e_1}) &= \mu^\Sigma(\pi_{j+1}(g); \underline{\pi_1(e_1)}, \pi_0(e_0), \underline{\pi_1(e_1)}) \\
%     &= \pi_j(\mu^\Lambda(g; \underline{e_1}, e_0, \underline{e_1})) \\
%     &= e_j.
%   \end{align*}
% We can think of this permutation as being $\sigma$ with the $k$th string removed - in \cref{rem:crossed} we comment on such `face' and `degeneracy' maps as used here, and see \cite{ber-simplicial} for a more careful treatment of this idea. Now each of these is the identity $e_j$, as shown above. This means that $\sigma$ must either have been the identity $e_{j+1}$ or a transposition of the form $\trans{a}{a+1}$, where $1 \leq a \leq j$.

% If $\sigma$ is the identity $e_{j+1}$, then we are done since this would give $\textrm{Im}\,\pi_{j+1} = \{e_{j+1}\}$. Instead suppose that $\sigma = \trans{a}{a+1} \in \Sigma_{j + 1}$. Then if $1 < a \leq j$ we can use this to give
%   \begin{align*}
%     \trans{a-1}{a} &= \mu^\Sigma(\trans{a}{a+1}; e_0, \underline{e_1}) \\
%     &= \mu^\Sigma(\pi_{j+1}(g);\pi_0(e_0), \underline{\pi_1(e_1)}) \\
%     &= \pi_j(\mu^\Lambda(g; e_0, \underline{e_1})) \\
%     &= e_j.
%   \end{align*}
% This gives a contradiction, hence $\sigma \neq \trans{a}{a+1}$ and must be the identity $e_{j+1}$.

% Similarly, if $\sigma = \trans{1}{2} \in \Sigma_{j+1}$, then in $\Sigma_j$ we find that
%   \begin{align*}
%     \trans{1}{2} &= \mu^\Sigma(\trans{1}{2} ; \underline{e_1}, e_0) \\
%     &= \mu^\Sigma(\pi_{j+1}(g); \underline{\pi_1(e_1)}, \pi_0(e_0)) \\
%     &= \pi_j(\mu^\Lambda(g ; \underline{e_1}, e_0)) \\
%     &= e_j.
%   \end{align*}
% Again, a contradiction, hence $\sigma = e_{j+1}$.

Examples of action operads:
%
%
%Copied:
%
%In order to make sense of this definition, we must define $\beta(\tau_1,\ldots,\tau_n)$ and $\delta(\sigma)$ in the context of braids. The first is the block sum in the obvious sense:  given $n$ different braids on $k_{1}, \ldots, k_{n}$ strands, respectively, we form a new braid on $k_{1} + \cdots + k_{n}$ strands by taking a disjoint union where the braid $\tau_{i}$ is to the left of $\tau_{j}$ if $i < j$. The braid $\delta(\sigma)$ is obtained by replacing the $i$th strand with $k_{i}$ consecutive strands, all of which are braided together according to $\sigma$. These operations are sometimes referred to as `cabling' operations for braids, as described in, for example, \cite{doucot_local_2025}.
%
%Copied 2:
%
%One can form an operad $B$ where $B(n)$ is the underlying set of the $n$th braid group, $B_{n}$. This is done in much the same way as we did for the symmetric operad using the `cabling' operations for braids described after \cref{broperad}. The collections of maps $\pi_{n} \colon B_{n} \rightarrow \Sigma_{n}$ giving the underlying permutation of each braid constitutes an operad map (of non-symmetric or braided operads) operads $Br \rightarrow \Sigma$.

% \acnoteil{Feeling out how much detail we need for ribbon braid groups to be officially `checked' as an action operad. I'm quite happy with the details, but we might get away with a diagram and some comments to hold off all the checking.}
% For a ribbon braid $t_1^{m_1}\cdots t_n^{m_n}\sigma$ and the Garside twist
%     \[
%         \gamma_n = (\sigma_1 \sigma_2 \cdots \sigma_n)(\sigma_1 \sigma_2 \cdots \sigma_{n-1})\cdots(\sigma_1 \sigma_2)(\sigma_1),
%     \]
% it's something like this:
%     \[
%         \delta^{RB}_{n;k_1,\ldots,k_n}(t_1^{m_1}\cdots t_n^{m_n}\sigma) = \beta(\beta(\underline{t})^{m_1}\gamma_{k_1}^{2m_1},\ldots,\beta(\underline{t})^{m_n}\gamma_{k_n}^{2m_n})\delta^B_{n;k_1,\ldots,k_n}(\sigma)
%     \]
% A lot of the axioms rely on general braids commuting with the full Garside twist on the same number of strands and the fact that $\pi_n(\gamma_n^2) = e_n$ and $\pi(t) = e_1$.
% E.g., for the ribbon braid $\alpha = t_1^2t_2t_3\sigma_1\sigma_2\sigma_2$ and $\delta_{3;3,1,2}$ we get
%     \begin{align*}
%         \delta(\alpha) &= \beta(\beta(t,t,t)^2\gamma_3^4,\beta(t),\beta(t,t)\gamma_2^2)\delta(\sigma_1\sigma_2\sigma_2)\\
%         &= \beta(t^2,t^2,t^2,t,t,t)\beta(\gamma_3^4,e_1,\gamma_2^2)\delta(\sigma_1\sigma_2\sigma_2).
%     \end{align*}
% Actually this might give a better simplification of the formula for $\delta$:
%     \[
%         \delta^{RB}_{n;k_1,\ldots,k_n}(t_1^{m_1},\ldots,t_n^{m_n}\sigma) = \beta(\underline{t^{m_1}},\ldots,\underline{t^{m_n}})\beta(\gamma_{k_1}^{2m_1},\ldots,\gamma_{k_n}^{2m_n})\delta^B_{n;k_1,\ldots,k_n}(\sigma)
%     \]
% where $\underline{t^{m_i}}$ is $t_{m_i}$ repeated $k_i$ times.
% \acnoteil{Something where $\delta_{1;n}(t) = $ `full Garside twist', e.g., $\delta_{1;3}(t) = \beta(t,t,t)(\sigma_1 \sigma_2 \sigma_3)(\sigma_1 \sigma_2)(\sigma_1)(\sigma_1 \sigma_2 \sigma_3)(\sigma_1 \sigma_2)(\sigma_1)$, whichever the right way round is. I don't like the lack of details in Wahl's thesis to deal with the twists for our purposes, so I'm trying to fill them in - there it's mostly just `similar to the symmetric and braid case' apart from some hidden details in proofs. I've mostly written it out and checked Theorem 4.15. but it depends how much detail we want to go into. It's on the level of the cactus operad proof.}

% \ngnoteil{old:}
% For each $n \in \mathbb{N}$, the \emph{ribbon braid group} $RB_{n}$ is the group whose presentation is the same as that of the braid group $B_{n}$, except with the addition of $n$ new generators $t_1, \ldots, t_n$, known as the \emph{twists}. These twists all commute with one other, and also commute with all braids except in the following cases:
%   \begin{align*}
%     b_i \cdot t_i &= t_{i+1} \cdot b_i,\\
%     b_i \cdot t_{i+1} &= t_i \cdot b_i.
%   \end{align*}
% The \emph{ribbon braid operad} $RB$ is then the operad made up of these groups in a way that extends the definition of the braid operad. In other words, the identity is still $e_1 \in RB_1$, and the operadic multiplication is built up in stages in exactly the same ways as in \cref{rem:br-op-needed}, but with some additional rules for dealing with twists. With regards to the tensor product\acnote{tensor product? does this need rewriting with $\beta$ and $\delta$?}, we have that for any twist $t_i \in RB_{n}$,
%   \[
%     t_i = e_{i-1} \otimes t \otimes e_{n-i}
%   \]
% where $t$ is the sole twist in $RB_1$, and for the `block twists' $t_{(m)}$ we again work recursively:
%   \[
%     t_{(0)} = e_n, \quad \quad \quad t_{(m+m')} = \left(t_{(m)} \otimes t_{(m')}\right) \cdot b_{(m', m)} \cdot b_{(m, m')}
%   \]\acnote{not sure what this notation $b_{(m,m')}$ is either?}

% \ngnoteil{as much as I like the picture, I am commenting out the twist}

%Much as the symmetric groups can be represented by crossings of a collection of strings, and the braid groups by braidings of strings, the ribbon braid groups deal with the ways that one can braid together several flat ribbons, including the ability to twist a ribbon about its own axis by 360 degrees. The actual definition of the ribbon braid groups is as the fundamental group of a configuration space in which points have labels in the circle, $S^{1}$; see \cite{sal-wahl}.
%\begin{center} \begin{tabular}{ccc}
%			\begin{tikzpicture}[baseline]
%				\node(xl1) at (-0.7,1){};
%				\node(xr1) at (-0.3,1){};
%				\node(yl1) at (0.3,1){};
%				\node(yr1) at (0.7,1){};
%				\node(yl2) at (-0.7, -1){};
%				\node(yr2) at (-0.3, -1){};
%				\node(xl2) at (0.3, -1){};
%				\node(xr2) at (0.7, -1){};
%				\node(b) at (0,0)[circle,fill=white, minimum size=0.5cm]{};
%       				\draw[rounded corners](xl1.north) to (-0.7,0.5) to (0.3,-0.5) to (xl2.south);
%       				\draw[rounded corners](xr1.north) to (-0.3,0.5) to (0.7,-0.5) to (xr2.south);
%				\begin{pgfonlayer}{bg}
%				\draw[rounded corners](yl1.north) to (0.3, 0.5) to (-0.7, -0.5) to (yl2.south);
%				\draw[rounded corners](yr1.north) to (0.7, 0.5) to (-0.3, -0.5) to (yr2.south);
%    				\end{pgfonlayer}
%				\draw(xl1.north) to (xr1.north);
%				\draw(xl2.south) to (xr2.south);
%				\draw(yl1.north) to (yr1.north);
%				\draw(yl2.south) to (yr2.south);
%			\end{tikzpicture} & \quad \quad \quad \quad \quad \quad \quad &
%			\begin{tikzpicture}[baseline]
%				\node(xl1) at (-0.2,1){};
%				\node(xr1) at (0.2,1){};	
%				\node(xl2) at (-0.2, -1){};
%				\node(xr2) at (0.2, -1){};
%				\draw[rounded corners](xl1.north) to (-0.2,0.4) to (0.2, 0.3) to (0.2, -0.3) to (-0.2, -0.4) to (xl2.south);	
%       				\draw[rounded corners](xr1.north) to (0.2,0.4) to (-0.2, 0.3) to (-0.2, -0.3) to (0.2, -0.4) to (xr2.south);
%				\draw(xl1.north) to (xr1.north);
%				\draw(xl2.south) to (xr2.south);	
%			\end{tikzpicture} \\
%			$b$ & & $t$ 
%\end{tabular} \end{center}
% This operad $RB$ is also clearly an action operad, since we can just define $\pi^{RB}_n \colon RB_{n} \rightarrow \Sigma_n$ to act like $\pi^B_n$ on any braids, at which point the fact that $\pi(t) \in S_1 = \{e_1\}$ will automatically take care of the twists.

%\begin{enumerate}
%\item We now describe two less trivial action operads, those given by the braid groups (\cref{ex:braid_operad_B}), $\Lambda = B$, and the ribbon braid groups, $\Lambda = RB$. In each case, the homomorphism $\pi$ is given by taking underlying permutations, and the operad structure is given geometrically by using the procedure explained after \cref{broperad}. For $RB$, the fact that $\pi(t) \in \Sigma_1 = \{e_1\}$ automatically takes care of the twists. We refer the reader to \cite{fie-br} for more information about braided operads, and to \cite{sal-wahl, wahl-thesis} for information about the ribbon case.

 % as permutations $\pi$ of the set $\{-n, 1-n, \ldots, -1, 1, \ldots, n-1, n\}$ such that $\pi(i) = -\pi(-i)$ for all $i$, as the subgroup of $O(n)$ consisting of those matrices with all integer coefficients, or as invertible $n \times n$-matrices whose entries consist of $-1$, $0$, or $1$ and in which each row and column has exactly one non-zero entry. 
%E.g., we can consider an element of $H_3$ as a permutation matrix and a $3$-tuple of elements of $C_2 = \{-1,1\}$, or simply as a signed permutation matrix:
%  \[
%    \left(\trans{2}{3}
%    ;
%    -1,1,-1\right)
%    =
%    \left(\begin{bmatrix}
%    1 & 0 & 0 \\
%    0 & 0 & 1 \\
%    0 & 1 & 0
%    \end{bmatrix}
%    ;
%    -1,1,-1\right)
%    =
%    \begin{bmatrix}
%    -1 & 0 & 0 \\
%    0 & 0 & -1 \\
%    0 & 1 & 0
%    \end{bmatrix}
%    \in H_3.
%  \]
%E.g.,
%  \[
%    \beta\left(
%      \begin{bmatrix}
%      -1 & 0 & 0 \\
%      0 & 0 & -1 \\
%      0 & 1 & 0
%      \end{bmatrix},
%      \begin{bmatrix}
%      0 & 1 \\
%      -1 & 0
%      \end{bmatrix}
%    \right)
%    =
%    \begin{bmatrix}
%      -1 &  0 &  0 &  0 & 0 \\
%      0  &  0 & -1 &  0 & 0 \\
%      0  &  1 &  0 &  0 & 0 \\
%      0  &  0 &  0 &  0 & 1 \\
%      0  &  0 &  0 & -1 & 0
%    \end{bmatrix}.
%  \]
        % \[
        %   \delta_{1;3}([-1])\beta(\sigma) = r_n \cdot \sigma =
        %   \begin{bmatrix}
        %   0 & -1 & 0 \\
        %   0 & 0 & 1 \\
        %   1 & 0 & 0
        %   \end{bmatrix}
        % \]
  % \[
  %   \beta(\sigma)\delta_{1;3}([-1]) = \sigma \cdot r_n =
  %   \begin{bmatrix}
  %   0 & 0 & 1 \\
  %   1 & 0 & 0 \\
  %   0 & -1 & 0
  %   \end{bmatrix}.
  % \]

Extensions:
%We have the following corollary.
%\begin{cor}\label{corZ}
%For an action operad $\Lambda$, the sets
%  \[
%    \mathrm{Ker}\,\pi_n = \{g \in \Lambda(n)~\colon~\pi_{n}(g) = e_{n} \}
%  \]
%form an action operad for which the inclusion $\mathrm{Ker}\,\pi \hookrightarrow \Lambda$ is a map of action operads.
%\end{cor}
%\begin{proof}
%For $\textrm{Ker}\,\pi \hookrightarrow \Lambda$ to be a map of action operads, we must define the map $\textrm{Ker}\,\pi \rightarrow \Sigma$ to be zero,  and we must check that the operadic multiplication of elements in the kernel is also in the kernel. This last fact is a trivial consequence of $\pi$ being an operad map.
%\end{proof}
%
%\begin{cor}\label{image}
%For an action operad $\Lambda$, the sets
%  \[
%    \mathrm{Im}\,\pi_n = \{\pi_n(g)~\colon~g \in \Lambda(n)\}
%  \]
%form an action operad for which the inclusion $\mathrm{Im}\,\pi \hookrightarrow \Sigma$ is a map of action operads.
%\end{cor}
%\begin{proof}
%The operad multiplication of elements in the image is also in the image, as is the unit element in $\mathrm{Im}\,\pi_1$, following again as a consequence of $\pi$ being an operad map. The map $\mathrm{Im}\,\pi \hookrightarrow \Sigma$ is an inclusion and so is immediately seen to be a map of action operads.
%\end{proof}

Presentations:
%A useful method for constructing new examples of some given algebraic structure is through the use of presentations. A presentation consists of generating data together with relations between generators using the operations of the algebra involved. In categorical terms, the generators and relations are both given as free gadgets on some underlying data, and the presentation itself is a coequalizer. This section will establish the categorical structure necessary to give presentations for action operads, and then explain how such a presentation is reflected in the associated club and $2$-monad. The most direct route to the desired results uses the theory of locally finitely presentable categories. We recall the main definitions briefly, but refer the reader to \cite{ar} for additional details.
%
%\begin{Defi}\label{def:filtered}
%  A \textit{filtered category} is a nonempty category $C$ such that
%    \begin{itemize}
%      \item if $a,b$ are objects of $C$, then there exists another object $c \in C$ and morphisms $a \rightarrow c, b \rightarrow c$; and
%      \item if $f,g \colon a \rightarrow b$ are parallel morphisms in $C$, then there exists a morphism $h \colon b \rightarrow c$ such that $hf = hg$.
%    \end{itemize}
%\end{Defi}
%
%\begin{Defi}
%  A \emph{filtered colimit} is a colimit over a filtered category.
%\end{Defi}
%
%\begin{Defi}
%  Let $C$ be a category with all filtered colimits. An object $x \in C$ is \textit{finitely presentable} if the representable functor $C(x, -) \colon C \rightarrow \mb{Sets}$ preserves filtered colimits.
%\end{Defi}
%
%\begin{Defi}
%  A \textit{locally finitely presentable category} is a category $C$ such that
%  \begin{itemize}
%    \item $C$ is cocomplete and
%    \item there exists a small subcategory $C_{fp} \subseteq C$ of finitely presentable objects such that any object $x \in C$ is the filtered colimit of some diagram in $C_{fp}$.
%  \end{itemize}
%\end{Defi}
%
%The definition of a locally finitely presentable category has many equivalent variants, but we find this one most practicable to work with in this setting.
%
%\begin{thm}
%The category $\mb{AOp}$ is locally finitely presentable.
%\end{thm}
%\begin{proof}
%First note that we can define a category $\mb{Op}^{g}$, whose objects are operads $P$ in which each $P(n)$ also carries a group structure. This is an equational theory using equations with only finitely many elements, so $\mb{Op}^{g}$ is locally finitely presentable \cite[Corollary 3.7]{ar}. A slice category of a locally finitely presentable category is itself locally finite presentable \cite[Proposition 1.57]{ar} and since the symmetric operad is an object of $\mb{Op}^{g}$, the slice category $\mb{Op}^{g}/\Sigma$ is locally finitely presentable.
%
%There is an obvious inclusion functor $\mb{AOp} \hookrightarrow \mb{Op}^{g}/\Sigma$. Now $\mb{AOp}$ is a full subcategory of $\mb{Op}^{g}/\Sigma$ which is closed under products, subobjects. Since any object of $\mb{Op}^{g}/\Sigma$ isomorphic to an action operad is in fact an action operad, the inclusion   $\mb{AOp} \hookrightarrow \mb{Op}^{g}/\Sigma$ is actually the inclusion of a reflective subcategory. One can easily check that $\mb{AOp}$ is in fact closed under all limits and filtered colimits in $\mb{Op}^{g}/\Sigma$, so by the Reflection Theorem (2.48 in \cite{ar}), $\mb{AOp}$ is locally finitely presentable.
%\end{proof}
%
%
%
%\begin{rem}
%In standard presentations of the theory of operads (see, for example, \cite{mss-op}), a nonsymmetric operad will have an underlying collection (or $\mathbb{N}$-indexed collection of sets) while a symmetric operad will have an underlying symmetric collection (or $\mathbb{N}$-indexed collection of sets in which the $n$th set has an action of $\Sigma_{n}$). Our collections over $\SS$ more closely resemble the former as there is no group action present.
%\end{rem}
%
%\begin{example}
%One can easily form new action operads from old ones by taking limits. To take a limit of a diagram in $\mb{AOp}$, one forgets down to the category of operads over $\Sigma$ and takes the limit there. Concretely, products in $\mb{AOp}$ are computed as products in $\mb{Op}/\Sigma$ which themselves are (possibly wide) pullbacks in the category of operads. This pullback will then be computed levelwise, showing that at each dimension there is a group structure with a group homomorphism to the appropriate $\Sigma_{n}$ and that the final action operad axiom holds since it does in each component. The equalizer of a pair of maps will then just be the levelwise equalizer. This shows that the pointwise product of an action operad $P$ with an action operad of the form $Z(Q)$ (as in \cref{Z}) is again an action operad, but the pointwise product of two arbitrary action operads might not be.
%\end{example}
%\begin{thm}\label{underlyingSS}
%There exists a forgetful functor $U \colon \mb{AOp} \rightarrow \mb{Sets}/\SS$ which preserves all limits and filtered colimits.
%\end{thm}
%
%\begin{proof}
%For a given action operad $\Lambda$, we put $U(\Lambda) = \left(\coprod_{\mathbb{N}} \Lambda(n), \coprod_{\mathbb{N}} \pi_n \right)$ and this easily extends to a mapping on morphisms using the universal property of the coproduct. The preservation of filtered colimits follows from the fact that these are computed pointwise, together with the fact that every map between action operads preserves underlying permutations.
%As equalizers are computed levelwise, and the product $\Lambda \times \Lambda'$ has underlying operad the pullback $\Lambda \times_{\Sigma} \Lambda'$; this pullback is itself computed levelwise. Together, these imply that $U$ also preserves all limits.
%\end{proof}
%
%\begin{proof}
%The category $\mb{Sets}/\SS$ is locally finitely presentable as it is equivalent to the functor category $[\SS, \mb{Sets}]$ (here $\SS$ is treated as a discrete category) and any presheaf category is locally finitely presentable. The functor $U$ preserves limits and filtered colimits between locally finitely presentable categories, so has a left adjoint (see Theorem 1.66 in \cite{ar}).
%\end{proof}

Lambda operads:
%Just as we had the definitions of operad, symmetric operad, and braided operad, we now come to the general definition of a $\Lambda$-operad,  Thereafter we begin to delve into the theory around $\Lambda$-operads. Beginning by recasting familiar operads as $\Lambda$-operads for particular choices of $\Lambda$, we then proceed to shadow standard operadic definitions for algebras and pseudoalgebras in this setting, characterising algebras using endomorphism $\Lambda$-operads, and deriving monads from $\Lambda$-operads. We show that there is an adjunction between $\Lambda\mbox{-}\mb{Op}$ and $\Sigma\mbox{-}\mb{Op}$, and prove some technical results regarding maps between monads induced by $\Lambda$-operads.
 	% QQQ chapter: operads and algebras with general group actions
 	% \begin{itemize}
 	% 	\item defines $\Lambda$-operads
 	% 	\item goes over familiar examples
 	% 	\item recasts familar operad algebra stuff for $\Lambda$-operads
 	% 	\begin{itemize}
	%  		\item defines algebras and pseudoalgebras for $\Lambda$-operads
	%  		\item characterises algebras using the endomorphism operads
	%  		\item monads via operads
	%  	\end{itemize}
	%  	\item theorem about the adjunction between $\Lambda-\bf{Op}$ and $\bf{\Sigma}-\bf{Op}$
	%  	\item technical results about monad maps
 	% \end{itemize}
 	
 	  %%Expanded diagram
  % \[
  %   \xy
  %     (0,0)*+{O(n) \times O(k_{1}) \times X^{k_{1}} \times \cdots \times O(k_{n}) \times X^{k_{n}}}="ul";
  %     (75,0)*+{O(n) \times X^{n}}="ur";
  %     (0,-12)*+{O(n) \times O(k_{1}) \times \cdots \times O(k_{n}) \times X^{k_{1}} \times \cdots \times X^{k_{n}}}="ml";
  %     (0,-24)*+{O(\sum k_{i}) \times X^{\sum k_{i}}}="bl";
  %     (75,-24)*+{X}="br";
  %     {\ar^>>>>>>>>>>>>>>{1 \times \alpha_{k_{1}} \times \cdots \alpha_{k_{n}}} "ul"; "ur"};
  %     {\ar^{\alpha_{n}} "ur"; "br"};
  %     {\ar_{\cong} "ul"; "ml"};
  %     {\ar_{\mu \times 1} "ml"; "bl"};
  %     {\ar_{\alpha_{\sum k_{i}}} "bl"; "br"};
  %   \endxy
  % \]
  %\begin{Defi}\label{opalgax}
%Let $O$ be a non-symmetric operad. An \textit{algebra} for $O$ consists of a set $X$ together with maps $\alpha_{n} \colon O(n) \times X^{n} \rightarrow X$ such that the following axioms hold.
%
%\end{Defi}
%
%$\coeq{P}{X}{\Lambda}{n}$
%
%Moving on to algebras for a $\Lambda$-operad, let $P$ be a $\Lambda$-operad and let $X$ be any set. Write $\coeq{P}{X}{\Lambda}{n}$ for the coequalizer of the pair of maps
%  \[
%    P(n) \times \Lambda(n) \times X^{n} \rightrightarrows P(n) \times X^{n}
%  \]
%of which the first map is the action of $\Lambda(n)$ on $P(n)$ and the second map is
%  \[
%    P(n) \times \Lambda(n) \times X^{n} \rightarrow P(n) \times \Sigma_{n} \times X^{n} \rightarrow P(n) \times X^{n}
%  \]
%using $\pi_{n} \colon \Lambda(n) \rightarrow \Sigma_{n}$ together with the canonical action of $\Sigma_{n}$ on $X^{n}$ by permutation of coordinates: $\sigma \cdot (x_{1}, \ldots, x_{n}) = (x_{\sigma^{-1}(1)}, \ldots, x_{\sigma^{-1}(n)})$. By the universal property of the coequalizer, a function $f \colon \coeq{P}{X}{\Lambda}{n} \rightarrow Y$ can be identified with a function $\tilde{f} \colon P(n) \times X^{n} \rightarrow Y$ such that
%  \[
%    \tilde{f}(p\cdot g; x_{1}, \ldots, x_{n}) = \tilde{f}\left(p; x_{g^{-1}(1)}, \ldots, x_{g^{-1}(n)}\right).
%  \]
  %
%
%\ngnoteil{Moved from a proof that was below, needs rewriting:}
%Given any map of monoids $f \colon M \rightarrow N$ in a monoidal category, there exists an adjunction between right $M$-modules and right $N$-modules given by $f^{*}$ as the right adjoint and $A \mapsto A \otimes_{M} N$ as the left adjoint. Thus we define
%  \[
%    S(P)(n) = \coequ{P}{\Sigma}{\Lambda}{n},
%  \]
%and this inherits a right $\Sigma_{n}$-action by multiplication. The unit of $S(P)$ is
%  \[
%    * \stackrel{\eta}{\longrightarrow} P(1) \longrightarrow \coequ{P}{\Sigma}{\Lambda}{1} \cong P(1)/\Lambda(1).
%  \]
%For the multiplication, let $K = k_1 + \cdots + k_n$, so we must define
%% \[
%% \mu \colon (\coequ{P}{\Sigma}{\Lambda}{n}) \times (P(k_1) \times_{\Lambda(k_1)} \Sigma_{k_1}) \times \cdots \times (P(k_n) \times_{\Lambda(k_n)} \Sigma_{k_n}) \rightarrow P(K) \times_{\Lambda(K)} \Sigma_{K}.
%% \]
%  \[
%    \mu \colon \left(\coequ{P}{\Sigma}{\Lambda}{n}\right) \times \prod_{i=1}^n \left(\coequ{P}{\Sigma}{\Lambda}{k_i}\right) \rightarrow \coequ{P}{\Sigma}{\Lambda}{K}.
%  \]
%Using the universal property of the coequalizer, this is induced by the following composite.
%
%  \begin{align*}
%    (P(n) \times \Sigma_{n}) \times \prod_{i=1}^n \left(P(k_i) \times \Sigma_{k_i}\right) &\cong \left(P(n) \times \prod_{i=1}^n P(k_{i}) \right) \times \left(\Sigma_{n} \times \prod_{i=1}^n \Sigma_{k_{i}} \right)\\
%    &\xrightarrow{\mu^P \times \mu^{\Sigma}} P(K) \times \Sigma_{K}\\
%    &\longrightarrow  \coequ{P}{\Sigma}{\Lambda}{K}
%  \end{align*}
%
%We leave verification of the associativity, unit, and equivariance axioms to the reader; they are simple applications of the same axioms for $P$ and $\Sigma$ together with some colimit universal properties and the $\Lambda$-operad axioms for $P$. It is then straightforward to check the bijection between $\Lambda$-operad maps $P \rightarrow \pi^{*}Q$ and symmetric operad maps $S(P) \rightarrow Q$, thus establishing the adjunction.
%\begin{lem}
%Let $\Lambda$ be an action operad, and let $X$ be a set. Then $\mathcal{E}_{X}$ carries a canonical $\Lambda$-operad structure.
%\end{lem}
%\begin{proof}
%$\mathcal{E}_{X}$ is a symmetric operad, so we define the actions by
%  \[
%    \mathcal{E}_{X}(n) \times \Lambda(n) \stackrel{1 \times \pi_{n}}{\longrightarrow} \mathcal{E}_{X}(n) \times \Sigma_{n} \rightarrow \mathcal{E}_{X}(n).
%  \]
%\end{proof}
%
%The previous result is really a change-of-structure-groups result. We record the general result as the following proposition.
%
%
%
%We can now use endomorphism operads to characterize algebra structures.
%
%\begin{prop}\label{endoalg}
%Let $X$ be a set, and $P$ a $\Lambda$-operad. Then algebra structures on $X$ are in 1-to-1 correspondence with $\Lambda$-operad maps $P \rightarrow \mathcal{E}_{X}$.
%\end{prop}
%\begin{proof}
%A map $P(k) \rightarrow \mathcal{E}_{X}(k)$ corresponds, using the closed structure on $\mb{Sets}$, to a map $P(k) \times X^{k} \rightarrow X$. The monoid homomorphism axioms give the unit and associativity axioms, and the requirement that $P \rightarrow \mathcal{E}_{X}$ be a map of $\Lambda$-operads gives the equivariance condition.
%\end{proof}


%\section{\texorpdfstring{$\Lambda$}{L}-Operads as Monads}\label{sec:lop-monad}
%
%This section revisits the theory of monads associated to operads, now in the context of $\Lambda$-operads.
%For many purposes, the monad associated to a $\Lambda$-operad contains all the information that is needed, as we shall see below.
%
%The key observation here is that, for each $n \in \mathbb{N}$,
%	\[
%		\coeq{\Lambda}{X}{\Lambda}{n} \cong X^n.
%	\]
%We will describe this bijection and leave the rest of the proof to the reader, which falls out of the various axioms either for being a monoid or for being a $\Lambda$-algebra.
%
%Recall that the elements of $\coeq{\Lambda}{X}{\Lambda}{n}$ are equivalence classes of the form $[g ; x_1, \ldots, x_n]$ for which
%	\[
%		(gh; x_1, \ldots, x_n) \sim \left(g; x_{h^{-1}(1)}, \ldots, x_{h^{-1}(n)}\right).
%	\]
%There is an obvious map $X^n \rightarrow \coeq{\Lambda}{X}{\Lambda}{n}$ sending $(x_1, \ldots, x_n)$ to the equivalence class $[e; x_1, \ldots, x_n]$. The inverse to this map is given by the map $\coeq{\Lambda}{X}{\Lambda}{n} \rightarrow X^n$ sending $[g;x_1,\ldots,x_n]$ to the element $(x_{g^{-1}(1)},\ldots,x_{g^{-1}(n)})$. It is then clear that these are inverses, relying on the equivalence relation to see that
%	\[
%		[g;x_1,\ldots,x_n] = \left[e;x_{g^{-1}(1)},\ldots,x_{g^{-1}(n)}\right].
%	\]
%That the second map is well-defined is simple to show.
%\ngnoteil{actually show the calculation, plus fix notation since $S$ is now $\pi_{!}$}
%The first claim is a simple calculation using the coequalizer that defines $S(\pi^{*}Q)$, using that $Q(n)$ is itself the coequalizer of the obvious pair of maps $Q(n) \times \Sigma_{n} \times \Sigma_{n}$.
%
%\ngnoteil{actually show the calculation, again!}
%For the second claim, we find a natural isomorphism
%  \[
%    \coeq{P}{X}{\Lambda}{n} \cong (\coequ{P}{\Sigma}{\Lambda}{n}) \times_{\Sigma_{n}} X^{n}
%  \]
%by the universal property of the colimits involved, so as functors $\underline{P} \cong \underline{S(P)}$. One can then easily verify that this isomorphism commutes with the unit and multiplication of the two monads involved using calculations similar to those used to establish the adjunction.

All that monad map stuff:
%\begin{rem}
%\begin{enumerate}
%\item The adjunction alone is enough to establish that $P$ and $S(P)$ have isomorphic categories of algebras using \cref{endoalg}.
%\item This theorem shows that semantically, one need never consider any kind of operad aside from symmetric operads: any other kind of operad can be symmetrized without altering the algebras. But as the operad should be considered a finer level of detail than the monad, restricting to symmetric operads misses the structure present in the more nuanced group actions.
%
%\item Furthermore, there is clearly an artifact left from only considering the algebras themselves as objects in a symmetric monoidal category. It is well-known that a braided structure is all that is required for non-symmetric operads, and so one is left to consider that the natural home for algebras over a $\Lambda$-operad might be a type of monoidal structure other than symmetric in which case the theorem above gives no insight.
%\end{enumerate}
%\end{rem}

%\ngnoteil{I can't find where we use any of this stuff. I will leave it in for now, but am tempted to delete it.}
%
%We end this section by presenting some results which allow us to transfer operad or algebra structures to other categories. We will use the following standard definitions of monad maps and transformations, as per \cite{street-formal}.
%
%\begin{Defi}\label{defi:monad_map}
%Let $S$ be a monad on a category $C$ and $T$ be a monad on a category $D$. A \emph{monad map} of from $S$ to $T$ is a functor $F \colon C \rightarrow D$ together with a natural transformation $\alpha \colon TF \Rightarrow FS$ such that the following diagrams commute.
%
% \[
%    \xy
%      (0,0)*+{FX}="a";
%      (20,10)*+{TFX}="b";
%      (20,-10)*+{FSX}="c";
%      {\ar^{\eta^T_{FX}} "a" ; "b"};
%      {\ar^{\alpha_X} "b" ; "c"};
%      {\ar_{F\eta^S_X} "a" ; "c"};
%      (40,0)*+{T^2FX}="a";
%      (60,10)*+{TFSX}="b";
%      (85,10)*+{FS^2X}="c";
%      (60,-10)*+{TFX}="d";
%      (85,-10)*+{FSX}="e";
%      {\ar^{T\alpha_X} "a" ; "b"};
%      {\ar^{\alpha_{SX}} "b" ; "c"};
%      {\ar^{F\mu^S_X} "c" ; "e"};
%      {\ar_{\mu^T_{FX}} "a" ; "d"};
%      {\ar_{\alpha_X} "d" ; "e"};
%    \endxy
%  \]
%
%A \emph{transformation} $\Gamma \colon (F, \alpha) \Rightarrow (G, \beta)$ between monad maps is a natural transformation $\Gamma \colon F \Rightarrow G$ such that the following diagram commutes.
%  
%  \[
%    \xy
%      (0,0)*+{TFX}="a";
%      (20,0)*+{TGX}="b";
%      (0,-15)*+{FSX}="c";
%      (20,-15)*+{GSX}="d";
%      %
%      {\ar^{T\Gamma_X} "a" ; "b"};
%      {\ar^{\beta_X} "b" ; "d"};
%      {\ar_{\alpha_X} "a" ; "c"};
%      {\ar_{\Gamma_{SX}} "c" ; "d"};
%    \endxy
%  \]
%\end{Defi}
%
%\begin{rem}
%Every monad map $(F,\alpha)$ induces a functor $S\Alg \rightarrow T\Alg$ on the categories of algebras. An $S$-algebra $(X,\sigma)$ is sent to the $T$-algebra $(FX,F\sigma \cdot \alpha_X)$, as we now describe. For $(FX,F\sigma \cdot \alpha_X)$ to be a $T$-algebra we require the usual diagrams to commute, shown as the outside of the diagrams below.
%
%  \[
%    \xy
%      (0,0)*+{T^2FX}="a";
%      (20,0)*+{TFSX}="b";
%      (40,0)*+{TFX}="c";
%      (0,-30)*+{TFX}="d";
%      (20,-15)*+{FS^2X}="f";
%      (20,-30)*+{FSX}="e";
%      (40,-15)*+{FSX}="g";
%      (40,-30)*+{FX}="h";
%      {\ar^{T\alpha_X} "a" ; "b"};
%      {\ar^{TF\sigma} "b" ; "c"};
%      {\ar^{\alpha_X} "c" ; "g"};
%      {\ar^{F\sigma} "g" ; "h"};
%      {\ar_{\mu^T_{FX}} "a" ; "d"};
%      {\ar_{\alpha_X} "d" ; "e"};
%      {\ar_{F\sigma} "e" ; "h"};
%      {\ar_{\alpha_{SX}} "b" ; "f"};
%      {\ar_{F\mu^S_X} "f" ; "e"};
%      {\ar^{FS\sigma} "f" ; "g"};
%      (60,0)*+{FX}="a1";
%      (85,0)*+{TFX}="b1";
%      (85,-15)*+{FSX}="c1";
%      (85,-30)*+{FX}="d1";
%      {\ar^{\eta^T_{FX}} "a1" ; "b1"};
%      {\ar^{\alpha_X} "b1" ; "c1"};
%      {\ar^{F\sigma} "c1" ; "d1"};
%      {\ar^{F\eta^S_X} "a1" ; "c1"};
%      {\ar_{\id} "a1" ; "d1"};
%    \endxy
%  \]
%
%The first diagram commutes since the left hand side is the second diagram required to commute for $(F,\alpha)$ to be a monad map, the square at the top right is an instance of naturality for $\alpha$, while the bottom right square commutes since $(X,\sigma)$ is an $S$-algebra. The second diagram commutes since the top triangle is again a requirement of $\alpha$ being a transformation, with the lower triangle commuting again as a result of $(X,\sigma)$ being an $S$-algebra.
%
%A morphism $f \colon  (X, \sigma_X) \rightarrow (Y, \sigma_Y)$ of $S$-algebras is sent to the morphism
%  \[
%    Ff \colon  (FX, F\sigma_X \cdot \alpha_X) \rightarrow (FY, F\sigma_Y \cdot \alpha_Y),
%  \]
%this being a map of $T$-algebras following from the naturality of $\alpha$ and of $f$ being an $S$-algebra map. Functoriality follows from that of $F$.
%\end{rem}
%
%Throughout the text we make reference to where results can be applied in a more general case where a symmetric monoidal category is cocomplete and for which the tensor product distributes over colimits in each variable. However, we include the following definition to be clear what is meant simply by a cocomplete symmetric monoidal category.
%\begin{Defi}\label{cocom_symm_mon_cat}
%A cocomplete symmetric monoidal category $\m{C}$ is a symmetric monoidal category for which the underlying category is cocomplete.
%% and the endofunctor $- \otimes X \colon \m{C} \rightarrow \m{C}$ preserves colimits for all objects $X \in \m{C}$.
%% QQQ Is this what we mean by a cocomplete symmetric monoidal category? The statement of the following theorem 3.3.7 then requires that the tensor product preserves colimits in each variable but the definition already includes this. QQQ
%\end{Defi}
%The following three results tie in with material in the coming chapters but are of a general nature which better in the context of this section.
%
%
%\begin{prop}\label{monoidal_to_monadmap}
%Let $C,D$ be cocomplete symmetric monoidal categories. Let $\Lambda$ be an action operad, and $P$ be a $\Lambda$-operad in $C$. Let $F \colon C \rightarrow D$ be a symmetric lax monoidal functor. Then $FP$ is a $\Lambda$-operad in $D$, and there exists a monad map $(F,\psi) \colon (C,\underline{P}) \rightarrow (D, \underline{FP})$.
%\end{prop}
%\begin{proof}
%\cref{preserveGop} describes how the functor $F$ can be used to describe a functor
%  \[
%    \Lambda\mbox{-}\mb{Op}(C) \rightarrow \Lambda\mbox{-}\mb{Op}(D),
%  \]
%from which we see that $FP$ is a $\Lambda$-operad in $D$.
%
%The functor $F$ constitutes the $1$-cell of the monad map, while $\psi$ is required to be a natural transformation as below.
%  \[
%    \xy
%      (0,0)*+{C}="a";
%      (20,0)*+{D}="b";
%      (0,-20)*+{C}="c";
%      (20,-20)*+{D}="d";
%      %
%      {\ar^{F} "a" ; "b"};
%      {\ar^{\underline{FP}} "b" ; "d"};
%      {\ar_{\underline{P}} "a" ; "c"};
%      {\ar_{F} "c" ; "d"};
%      %
%      {\ar@{=>}^{\psi} (12.5, -7.5) ; (7.5,-12.5)};
%    \endxy
%  \]
%
%We describe the components of this natural transformation at an object $X$ of $C$ below.
%  \begin{align*}
%    \underline{FP}(FX) &= \coprod_{n \in \mathbb{N}} FP(n) \otimes_{\Lambda(n)} (FX)^n \\
%    &\rightarrow \coprod_{n \in \mathbb{N}} F\left(P(n) \otimes_{\Lambda(n)} X^n\right)\\
%    &\rightarrow F\left(\coprod_{n \in \mathbb{N}} P(n) \otimes_{\Lambda(n)} X^n\right)\\
%    &=F(\underline{P}(X))
%  \end{align*}
%The first morphism is a composite of the coherence cells of the type
%  \[
%    FX \otimes FY \rightarrow F(X \otimes Y)
%  \]
%for the symmetric lax monoidal functor $F$, while the second morphism is the induced morphism out of the coproduct. Naturality follows from that of the component morphisms. It is then straightforward to see that the monad morphism diagrams commute since the diagrams involved consist of instances of the coherence axioms for $F$ along with naturality of the coherence cells.
%\end{proof}
%
%\begin{prop}\label{opmap_to_monadmap}
%Let $C$ be a cocomplete symmetric monoidal category. Let $\Lambda$ be an action operad, and $P,Q$ be  $\Lambda$-operads in $C$ with a map $\sigma \colon P \rightarrow Q$ of $\Lambda$-operads between them. Then $\sigma$ induces a monad map $(\id, \sigma^*) \colon (C,\underline{Q}) \rightarrow (C,\underline{P})$ and hence a functor on categories of algebras.
%\end{prop}
%\begin{proof}
%We will first describe the components of the natural transformation $\sigma^* \colon \underline{P} \Rightarrow \underline{Q}$. The component $\sigma^*_X$ at an object $X$ of $C$ is a morphism between the coproducts
%  \[
%    \sigma^*_X \colon \coprod_{n \in \mathbb{N}} P(n) \otimes_{\Lambda(n)} X^n \rightarrow \coprod_{n \in \mathbb{N}} Q(n) \otimes_{\Lambda(n)} X^n.
%  \]
%This is seen to be induced by the universal properties of the coequalizers and coproducts in the following diagram, where $\sigma_n$ denotes the $n$-ary component of the $\Lambda$-operad map $\sigma$ and $\lambda^P$, $\lambda^Q$, $\rho^P$, and $\rho^Q$ denote the usual left and right actions.
%
%  \[
%    \xy
%      (0,0)*+{P(n) \otimes \Lambda(n) \otimes X^n}="a";
%      (50,0)*+{Q(n) \otimes \Lambda(n) \otimes X^n}="b";
%      (0,-20)*+{P(n) \otimes X^n}="c";
%      (50,-20)*+{Q(n) \otimes X^n}="d";
%      (0,-40)*+{P(n) \otimes_{\Lambda(n)} X^n}="e";
%      (50,-40)*+{Q(n) \otimes_{\Lambda(n)} X^n}="f";
%      (0,-60)*+{\coprod_{n \in \mathbb{N}} P(n) \otimes_{\Lambda(n)} X^n}="g";
%      (50,-60)*+{\coprod_{n \in \mathbb{N}} Q(n) \otimes_{\Lambda(n)} X^n}="h";
%      %
%      {\ar^{\rho^P} (2,-3)*{}; (2,-17)*{} };
%      {\ar_{\lambda^P} (-2,-3)*{}; (-2,-17)*{} };
%      {\ar^{\rho^Q} (52,-3)*{}; (52,-17)*{} };
%      {\ar_{\lambda^Q} (48,-3)*{}; (48,-17)*{} };
%      {\ar_{c^P_n} "c" ; "e"};
%      {\ar^{c^Q_n} "d" ; "f"};
%      {\ar "e" ; "g"};
%      {\ar "f" ; "h"};
%      %
%      {\ar^{\sigma_n \otimes \id \otimes \id} "a" ; "b"};
%      {\ar^{{\sigma_n} \otimes \id} "c" ; "d"};
%      {\ar_{\exists! \sigma^*_{n,X}} "e" ; "f"};
%      {\ar^{\exists! \sigma^*_X} "g" ; "h"};      
%    \endxy
%  \]
%
%The upper square which includes $\lambda^P$ and $\lambda^Q$ commutes due to $\sigma$ being a $\Lambda$-operad map, while the square with both $\rho$ actions commutes because the $\sigma_n$ and $\rho$ do not interact. Since $c^Q_n$ coequalizes $\lambda^Q$ and $\rho^Q$, then this commutativity shows that $c^Q_n \cdot (\sigma_n \otimes \id) \cdot \lambda^P = c^Q_n \cdot (\sigma_n \otimes \id) \cdot \rho^P$, hence the morphism $\sigma^*_{n,X}$ exists. The morphism $\sigma^*_X$ is then induced by the universal property of the coproduct $\underline{P}(X)$.
%
%It is then routine to check that these components are natural in $X$ and constitute a monad map. That a functor is then induced on the category of algebras follows from Lemma 6.1.1 of \cite{leinster}; the process is described above, following \cref{defi:monad_map}.
%\end{proof}
%We can combine these two propositions.
%
%\begin{cor}\label{monoidaladj_cor}
%If $C, D, P, F$ are as in \cref{monoidal_to_monadmap}, and $F$ is part of a monoidal adjunction (i.e., an adjunction in which both functors are symmetric lax monoidal, and the unit and counit are monoidal transformations) $F \dashv U$, then $(F, \id)$ and $(U, \id)$ are both monad maps. The unit $\eta \colon 1 \Rightarrow UF$ induces an operad map $\eta \colon P \Rightarrow UFP$, and a transformation between monad maps
%  \[
%    (\id, \id) \Rightarrow (\id, \eta^*) \circ (U, \psi^U) \circ \left(F, \psi^F\right).
%  \]
%The counit $\epz \colon FU \Rightarrow 1$ induces an operad map $\epz \colon FUFP \Rightarrow FP$, and a transformation between monad maps
%  \[
%    \left(F, \psi^F\right) \circ (\id, \eta^*) \circ \left(U,\psi^U\right) \Rightarrow (\id, \id).
%  \]
%These constitute an adjunction $\left(F,\psi^F\right) \dashv (\id, \eta^*) \circ \left(U, \psi^U\right)$ in the $2$-category of monads, and hence induce an adjunction between $\underline{P}$-algebras in $C$ and $\underline{FP}$-algebras in $D$.
%\end{cor}

%\begin{Defi}
%Let $\Lambda$ be an action operad. The category $\Lambda\mb{\mbox{-}Coll}$ of $\Lambda$-collections has objects $X = \{ X(n) \}_{n \in \N}$ which consist of a set $X(n)$ for each natural number $n$ together with an action $X(n) \times \Lambda(n) \rightarrow X(n)$ of $\Lambda(n)$ on $X(n)$. A morphism $f \colon X \rightarrow Y$ in $\Lambda\mb{\mbox{-}Coll}$ consists of a $\Lambda(n)$-equivariant map $f_{n} \colon X(n) \rightarrow Y(n)$ for each natural number $n$.
%\end{Defi}
%Given any such collection of groups $\{ \Lambda(n) \}_{n \in \N}$, we can form the category $B\Lambda$ whose objects are natural numbers and whose hom-sets are given by $B\Lambda(m,n) = \emptyset$ if $m \neq n$ and $B\Lambda(n,n) = \Lambda(n)$ (where composition and units are given by group multiplication and identity elements, respectively). Then 
%with the opposite category arising from our choice of right actions. A key step in explaining how $\Lambda$-operads arise as monoids in the category of $\Lambda$-collections is to show that being an action operad endows $B\Lambda$ with a monoidal structure.
%  \begin{align*}
%   , \\
%   
%  \end{align*}
%For the first relation above, we must have that the lefthand side is an element of
%  \[
%    X(r) \times Y(k_1) \times \cdots \times Y(k_r) \times \Lambda(n)
%  \]
%while the righthand side is an element of
%  \[
%    X(r) \times Y\left(k_{h^{-1}(1)}\right) \times \cdots \times Y\left(k_{h^{-1}(r)}\right) \times \Lambda(n);
%  \]
%for the second relation, we must have . The right $\Lambda(n)$-action on $X \circ Y(n)$ is given by multiplication on the final coordinate.

%We need that $+$ is a group homomorphism, and the second part of \cref{calclem} shows that it preserves identity elements. The final action operad axiom shows that it also preserves group multiplication since $\pi_{2}(e_{2}) = e_{2}$ (each $\pi_{n}$ is a group homomorphism) and therefore
%  \begin{align*}
%    \left(+(g,h)\right) \cdot \left(+(g',h')\right) &= \mu\left(e_{2}; g,h\right) \cdot\mu\left(e_{2}; g',h'\right) \\
%    &= \mu\left(e_{2}e_{2}; gg', hh'\right) \\
%    &= +\left(gg',hh'\right).
%  \end{align*}
%For naturality of the associator, we must have $(f+g)+h = f+(g+h)$. By the operad axioms for both units and associativity, the lefthand side is given by
%  \begin{align*}
%    \mu(e_{2}; \mu(e_{2}; f,g), h) &= \mu(e_{2}; \mu(e_{2}; f,g), \mu(\id;h)) \\
%    &= \mu(\mu(e_{2}; e_{2}, \id); f,g,h),
%  \end{align*}
%while the righthand side is then
%  \[
%    \mu(e_{2}; f, \mu(e_{2}; g,h)) = \mu(\mu(e_{2}; \id, e_{2}); f,g,h).
%  \]
%By \cref{calclem}, both of these are equal to $\mu(e_{3}; f,g,h)$, proving associativity. Naturality of the unit isomorphisms follows similarly, using $e_{0}$.

  %%Expanded diagram
  % \[
  %   \xy
  %     {\ar   (0,0)*+{Y(k_{1}) \times \cdots \times Y(k_{n}) \times B\Lambda(k, k_{1} + \cdots + k_{n})}; (40,15)*+{Y(k_{1}) \times \cdots \times Y(k_{n}) \times B\Lambda(k, k_{1} + \cdots + k_{n})} };
  %     (9.5,10)*{\scriptstyle (-\cdot g_{1}, \ldots, -\cdot g_{n}) \times 1};
  %     {\ar (40,15)*+{Y(k_{1}) \times \cdots \times Y(k_{n}) \times B\Lambda(k, k_{1} + \cdots + k_{n})}; (80,0)*+{Y^{\star n}(k)} };
  %     {\ar (0,0)*+{Y(k_{1}) \times \cdots \times Y(k_{n}) \times B\Lambda(k, k_{1} + \cdots + k_{n})}; (40,-15)*+{Y(k_{1}) \times \cdots \times Y(k_{n}) \times B\Lambda(k, k_{1} + \cdots + k_{n})} };
  %     (9.5,-10)*+{\scriptstyle 1 \times \left( (g_{1} + \cdots + g_{n})\cdot - \right)};
  %     {\ar (40,-15)*+{Y(k_{1}) \times \cdots \times Y(k_{n}) \times B\Lambda(k, k_{1} + \cdots + k_{n})}; (80,0)*+{Y^{\star n}(k)} };
  %   \endxy
  % \]
  
  %\begin{align*}
%(f_2 \bullet -) \circ (f_1 \bullet -) \circ \theta_{k_1, \ldots, k_n; k} & = (f_2 \bullet -) \circ \theta_{k_{f_1^{-1}(1)}, \ldots, k_{f_1^{-1}(n)}; k} \circ f_1[k_1, \ldots, k_n] \\
%& = \theta_{k_{f_1^{-1}(f_2^{-1}(1))}, \ldots, k_{f_1^{-1}(f_2^{-1}(n))}; k} \circ f_2[k_{f_1^{-1}(1)}, \ldots, k_{f_2^{-1}(n)}] \circ f_1[k_1, \ldots, k_n] \\
%& = \theta_{k_{(f_2f_1)^{-1}(1)}, \ldots, k_{(f_2f_1)^{-1}(n)};k} \circ (f_2f_1)[k_1, \ldots, k_n] \\
%& = \big( (f_2f_1) \bullet -\big) \circ \theta_{k_1, \ldots, k_n; k}
%\end{align*}
%\acnote{this spacing any better? similar lists to the diagram above - also is there a typo in the second line above? $f_2$ at end of the middle list should be $f_1$?}

% using the collection of maps
%  \[
%    \prod_{i=1}^{n} Y(k_{i}) \times B\Lambda(k, k_{1} + \cdots + k_{n}) \rightarrow \prod_{i=1}^{n} Y(k_{\pi (f)^{-1}(i)}) \times B\Lambda(k, k_{1} + \cdots + k_{n})
%  \]
%by using the symmetry $\pi(f)$ on the first $n$ factors and left multiplication by the element $\mu(f; e_{k_{1}}, \ldots, e_{k_{n}})$ on $B\Lambda(k, k_{1} + \cdots + k_{n})$. To induce a map between the coends, we must show that these maps commute with the two lefthand maps in the diagram above. For the top map, this is merely functoriality of the product together with naturality of the symmetry. For the bottom map, this is the equation
%  \[
%    \mu(f; \overline{e}) \cdot \mu(e; g_{1}, \ldots, g_{n}) = \mu(e; g_{\pi (f)^{-1} 1}, \ldots, g_{\pi (f)^{-1} n}) \cdot \mu(f; \overline{e}).
%  \]
%Both of these are equal to $\mu(f; g_{1}, \ldots, g_{n})$ by the action operad axiom. Functoriality is then easy to check using that the maps inducing $(f_{1}f_{2}) \bullet -$ are given by the composite of the maps inducing $f_{1} \bullet (f_{2} \bullet -)$.

%For associativity, we compute $(X \circ Y) \circ Z$ and $X \circ (Y \circ Z)$.
%  \begin{align*}
%    ((X \circ Y) \circ Z) (k) &= \int^{m} X \circ Y (m) \times Z^{\star m}(k) \\
%    &= \int^{m} \left( \int^{l} X(l) \times Y^{\star l}(m) \right) \times Z^{\star m}(k) \\
%    &\cong \int^{m,l} X(l) \times Y^{\star l}(m) \times Z^{\star m}(k) \\
%    &\cong \int^{l} X(l) \times \int^{m} Y^{\star l}(m) \times Z^{\star m}(k)
%  \end{align*}
%The first isomorphism is from products distributing over colimits and hence coends, and the second is that fact plus the Fubini Theorem for coends \cite{maclane-catwork}. A similar calculation shows
%  \[
%    (X \circ (Y \circ Z))(k) \cong \int^{l} X(l) \times (Y \circ Z)^{\star l}(k).
%  \]
%Thus the associativity isomorphism will be induced once we construct an isomorphism $\int^{m} Y^{\star l}(m) \times Z^{\star m} \cong (Y \circ Z)^{\star l}$. We do this by induction, with the $l=1$ case being the isomorphism $Y^{\star 1} \cong Y$ together with the definition of $\circ.$  Assuming true for $l$, we prove the case for $l+1$ by the calculations below.
%  \begin{align*}
%    (Y \circ Z)^{\star (l+1)} &\cong (Y \circ Z) \star (Y \circ Z)^{\star l} \\
%    &\cong (Y \circ Z) \star \left( \int^{m} Y^{\star l}(m) \times Z^{\star m} \right) \\
%    &= \left( \int^{n} Y(n) \times Z^{\star n} \right) \star \left( \int^{m} Y^{\star l}(m) \times Z^{\star m} \right) \\
%    &= \int^{a,b} \left( \int^{n} Y(n) \times Z^{\star n}(a) \right)  \times \left( \int^{m} Y^{\star l}(m) \times Z^{\star m}(b) \right) \times B\Lambda(-, a+b) \\
%    &\cong \int^{a,b,n,m} Y(n) \times Y^{\star l}(m) \times Z^{\star n}(a) \times Z^{\star m}(b) \times  B\Lambda(-, a+b) \\
%    &\cong \int^{n,m} Y(n) \times Y^{\star l}(m) \times Z^{\star (n+m)} \\
%    &\cong \int^{j} \int^{n,m} Y(n) \times Y^{\star l}(m) \times B\Lambda(j, n+m) \times Z^{\star j} \\
%    &\cong \int^{j} Y^{\star (l+1)}(j) \times Z^{\star j}
%  \end{align*}
%Each isomorphism above arises from the symmetric monoidal structure on $\mb{Sets}$ using products, the monoidal structure on presheaves using $\star$, the properties of the coend, or the fact that products distribute over colimits.

%For the monoidal category axioms on $\widehat{B\Lambda}$, we only need to note that the unit and associativity isomorphisms arise, using the universal properties of the coend, from the unit and associativity isomorphisms on the category of sets together with the interaction between products and colimits. Hence the monoidal category axioms follow by those same axioms in $\mb{Sets}$ together with the universal property of the coend.

Group actions:
%
%\ngnoteil{this is the old proof, rewrite:}
%The category $A \times_G B$ is defined as the coequalizer
%    \[
%        \xy
%            (0,0)*+{A \times G \times B}="00";
%            (30,0)*+{A \times B}="10";
%            (60,0)*+{\coeqb{A}{B}{G}}="20";
%            {\ar@<1ex>^{\lambda} "00" ; "10"};
%            {\ar@<-1ex>_{\rho} "00" ; "10"};
%            {\ar^{\varepsilon} "10" ; "20"};
%        \endxy
%    \]
%where $\lambda(a,g,b) = (a \cdot g, b)$ and $\rho(a,g,b) = (a, g \cdot b)$. However, the map $A \times B \rightarrow (A \times B)/G$, sending $(a,b)$ to the equivalence class $[a,b] = [a \cdot g, g^{-1} \cdot b]$, also coequalizes $\lambda$ and $\rho$ since
%    \[
%        [a \cdot g, b] = \left[(a \cdot g) \cdot g^{-1}, g \cdot b\right] = [a, g \cdot b].
%    \]
%
%Given any other category $X$ and a functor $\chi \colon A \times B \rightarrow X$ which coequalizes $\lambda$ and $\rho$, we define a functor $\phi \colon (A \times B)/G \rightarrow X$ by $\phi[a,b] = \chi(a,b)$. That this is well-defined is clear, since
%    \[
%        \phi\left[a \cdot g, g^{-1} \cdot b\right] = \chi\left(a \cdot g, g^{-1} \cdot b\right) = \chi\left(a \cdot \left(gg^{-1}\right), b\right) = \chi(a, b) = \phi[a,b].
%    \]
%This is also unique and so we find that $(A \times B)/G$ satisfies the universal property of the coequalizer.

%\ngnoteil{old:}
%
%\begin{lem}\label{lem:coeq-lem}
%Let $G$ be a group and let $A$, $B$ be categories for which $A$ has a right action by $G$ and $B$ has a left action by $G$. An action of $G$ on the product $A \times B$ can then be defined by
%    \[
%        (a,b) \cdot g \colon = \left(a \cdot g, g^{-1} \cdot b\right).
%    \]
%If this action of $G$ on $A \times B$ is free, then the category $(A \times B)/G$, consisting of the equivalence classes of this action, is isomorphic to the coequalizer $\coeqb{A}{B}{G}$.
%\end{lem}
%\begin{proof}
%The category $A \times_G B$ is defined as the coequalizer
%    \[
%        \xy
%            (0,0)*+{A \times G \times B}="00";
%            (30,0)*+{A \times B}="10";
%            (60,0)*+{\coeqb{A}{B}{G}}="20";
%            {\ar@<1ex>^{\lambda} "00" ; "10"};
%            {\ar@<-1ex>_{\rho} "00" ; "10"};
%            {\ar^{\varepsilon} "10" ; "20"};
%        \endxy
%    \]
%where $\lambda(a,g,b) = (a \cdot g, b)$ and $\rho(a,g,b) = (a, g \cdot b)$. However, the map $A \times B \rightarrow (A \times B)/G$, sending $(a,b)$ to the equivalence class $[a,b] = [a \cdot g, g^{-1} \cdot b]$, also coequalizes $\lambda$ and $\rho$ since
%    \[
%        [a \cdot g, b] = \left[(a \cdot g) \cdot g^{-1}, g \cdot b\right] = [a, g \cdot b].
%    \]
%
%Given any other category $X$ and a functor $\chi \colon A \times B \rightarrow X$ which coequalizes $\lambda$ and $\rho$, we define a functor $\phi \colon (A \times B)/G \rightarrow X$ by $\phi[a,b] = \chi(a,b)$. That this is well-defined is clear, since
%    \[
%        \phi\left[a \cdot g, g^{-1} \cdot b\right] = \chi\left(a \cdot g, g^{-1} \cdot b\right) = \chi\left(a \cdot \left(gg^{-1}\right), b\right) = \chi(a, b) = \phi[a,b].
%    \]
%This is also unique and so we find that $(A \times B)/G$ satisfies the universal property of the coequalizer.
%\end{proof}

Algebras etc over operads in Cat:


%\begin{rem}
%\ngnoteil{reread this, and resolve whatever issue it is addressing}
%  The requirement in \cref{def:ps-alg} of a natural isomorphism $\varphi_\eta$ is to induce a natural isomorphism $\tilde{\varphi}_\eta$. This requirement is really of a natural isomorphism
%    \[
%      \xy
%        (0,0)*+{1 \times_{\Lambda(1)} X}="a";
%        (0,-20)*+{P(1) \times_{\Lambda(1)} X}="b";
%        (25,-20)*+{X}="c";
%        %
%        {\ar_{\eta^P \times_{\Lambda(1)} 1} "a" ; "b"};
%        {\ar_<<<<<{\alpha_1} "b" ; "c"};
%        {\ar "a" ; "c"};
%        %
%        {\ar@{=>}^{\varphi_\eta} (10,-11) ; (7,-14)};
%      \endxy
%    \]
%  where $1 \times_{\Lambda(1)} X$ is the coequalizer of the trivial right action of $\Lambda(1)$ on $1$ and the usual left action of $\Lambda(1)$ on $X$. This induces a natural isomorphism
%    \[
%      \xy
%        (0,0)*+{1 \times X}="a";
%        (0,-20)*+{P(1) \times X}="b";
%        (25,-20)*+{X}="c";
%        %
%        {\ar_{\eta^P \times 1} "a" ; "b"};
%        {\ar_<<<<<<{\tilde{\alpha}_1} "b" ; "c"};
%        {\ar "a" ; "c"};
%        %
%        {\ar@{=>}^{\tilde{\varphi}_\eta} (10,-11) ; (7,-14)};
%      \endxy
%    \]
%  which can be whiskered with the isomorphism $X \rightarrow 1 \times X$. We make the convention of referring to this whiskered natural isomorphism as $\tilde{\varphi}_\eta$, since no confusion will arise in practice.
%\end{rem}


%Another interpretation of pseudoalgebras can be given in terms of pseudomorphisms of operads. Algebras for an operad $P$ can be identified with a morphism of operads $F \colon P \rightarrow \mathcal{E}_X$, where $\mathcal{E}_X$ is the endomorphism operad (\cref{endoalg}). We can similarly define pseudomorphisms for a $\cat$-enriched $\Lambda$-operad and identify pseudoalgebras with pseudomorphisms into the endomorphism operad.
%
%If $P$, $Q$ are $\Lambda$-operads then a \textit{pseudomorphism} of $\Lambda$-operads $F \colon P \rightarrow Q$ consists of a family of $\Lambda$-equivariant functors
%            \[
%                \left(F_n \colon P(n) \rightarrow Q(n)\right)_{n \in \mathbb{N}}
%            \]
%together with isomorphisms instead of the standard algebra axioms. For example, the associativity isomorphism has the following form.
%            \[
%                \xy
%                    (0,0)*+{\scriptstyle P(n) \times \prod_i P(k_i)}="00";
%                    (35,0)*+{\scriptstyle Q(n) \times \prod_i Q(k_i)}="10";
%                    (0,-15)*+{\scriptstyle P(\Sigma k_i)}="01";
%                    (35,-15)*+{\scriptstyle Q(\Sigma k_i)}="11";
%                    {\ar^{F_n \times \prod_i F_{k_i}} "00" ; "10"};
%                    {\ar^{\mu^Q} "10" ; "11"};
%                    {\ar_{\mu^P} "00" ; "01"};
%                    {\ar_{F_{\Sigma k_i}} "01" ; "11"};
%                    {\ar@{=>}^{\psi_{k_1,\ldots,k_n}} (15,-5.5) ; (15,-9.5)};
%                \endxy
%            \]
%
%These isomorphisms are then required to satisfy their own axioms, and these ensure that we have a weak map of $2$-monads $\underline{P} \Rightarrow \underline{Q}$. In particular, one can show that a pseudomorphism from $P$ into the endomorphism operad $\mathcal{E}_X$ produces pseudoalgebras for the operad $P$ using the closed structure on $\mb{Cat}$. While abstractly pleasing, we do not pursue this argument any further here.

%\section{Transferring Operads and \texorpdfstring{$\Lambda$}{L}-Monoidal Categories}
%
%\ngnoteil{start here}
%
%\begin{rem}
%I moved this here for
%now.
%\end{rem}
%
%We have seen (\cref{op=monad1}) that given any $\Lambda$-operad $P$ there is an induced monad $\underline{P} \colon \m{C} \rightarrow \m{C}$ and that the category of algebras for the operad $P$ is isomorphic to the category of algebras for the monad $\underline{P}$, following \cite{maygeom}. Now we are considering $\Lambda$-operads in $\cat$, the induced monad associated to an operad of this sort can be shown to be a $2$-monad (see \cite{KS} for background on $2$-monads) and we will proceed to show that the notions of pseudoalgebra for both the operad and the associated $2$-monad correspond precisely, i.e., there is an isomorphism of $2$-categories between the $2$-category with either strict or pseudo-level cells defined operadically and the $2$-category with either strict or pseudo-level cells defined $2$-monadically.
%
%
%
%The associated monad $\underline{P}$ acquires the structure of a $2$-functor as follows. We define $\underline{P}$ on categories much like before as  the coproduct
%    \[
%        \underline{P}(X) = \coprod_n \coeq{P}{X}{\Lambda}{n},
%    \]
%whose objects will be written as equivalence classes $[p;x_1,\ldots,x_n]$ where $p \in P(n)$ and each $x_i \in X$, sometimes written as $[p;\underline{x}]$ when there is no confusion. On functors we define $\underline{P}$ in a similar way, exactly as with functions of sets. Given a natural transformation $\alpha \colon f \Rightarrow g$ we define a new natural transformation $\underline{P}(\alpha)$ as follows. The component of $\underline{P}(\alpha)$ at the object
%    \[
%        [p;x_1,\ldots,x_n]
%    \]
%is given by the morphism
%    \[
%        [1_p;\alpha_{x_1},\ldots,\alpha_{x_n}]
%    \]
%in $\underline{P}(X)$.
%It is a simple observation that this constitutes a $2$-functor, and that the components of the unit and multiplication are functors and are $2$-natural.
%
%
%
%
%First we will set out some conventions and definitions.
%\begin{conv}\label{conv_coeq}
%We will identify maps $\alpha_n \colon \coeq{P}{X}{\Lambda}{n} \rightarrow X$ with maps $\tilde{\alpha}_n \colon P(n) \times X^n \rightarrow X$ which are equivariant with respect to the $\Lambda$-actions via the universal property of the coequalizer. The coequalizer in $\mb{Cat}$ also has a $2$-dimensional aspect to its universal property, so that a natural transformation $\Gamma \colon \alpha_{n} \Rightarrow \beta_{n}$ between functors as above determines and is determined by a transformation $\tilde{\Gamma} \colon \tilde{\alpha}_{n} \Rightarrow \tilde{\beta}_{n}$ with the property that the two possible whiskerings of $\tilde{\Gamma}$ with the two functors $P(n) \times \Lambda(n) \times X^{n} \rightarrow P(n) \times X^{n}$ are equal.
%
%Note also that in the following definitions we will often write the composite
%    \[
%        P(n) \times \prod_{i=1}^n \left(P(k_i) \times X^{k_i}\right) \rightarrow P(n) \times \prod_{i=1}^n P(k_i) \times X^{\Sigma k_i} \xrightarrow{\mu^P \times 1} P(\Sigma_{k_i}) \times X^{\Sigma k_i}
%    \]
%simply abbreviated as $\mu^P \times 1$. Furthermore, instead of using an element $\id \in P(1)$ as the operadic unit, we will now denote this as $\eta^{P} \colon 1 \rightarrow P(1)$.
%\end{conv}

%\begin{Defi}
%A functor $F \colon X \rightarrow Y$ is an \emph{isofibration} if given $x \in X$ and an isomorphism $f\colon y \xrightarrow{\cong} F(x)$ in $Y$, then there exists an isomorphism $g \colon y \cong x$ in $X$ such that $F(g) = f$.
%\end{Defi}
%
%\begin{prop}
%There exists a natural transformation $p \colon EU \Rightarrow B$, where $U$ is the underlying set of a group, which is pointwise an isofibration. Applying the classifying space functor to the component $p_{G}$ gives a universal principal $G$-bundle.
%\end{prop}
%\begin{proof}
%Given a group $G$, $p_{G} \colon EUG \rightarrow BG$ sends every object of $EUG$ to the unique object of $BG$. The unique isomorphism $g \rightarrow  h$ in $EUG$ is mapped to $hg^{-1} \colon * \rightarrow *$. It is easy to directly check that this is an isofibration, as well as to see that the classifying spaces of $EUG$ and $BG$ are the spaces classically known as $EG,BG$, with $|p_{G}|$ being the standard universal principal $G$-bundle.
%\end{proof}
%
%We will also need the functors $E, B$ defined for more than just a single set or group, in particular for the sets or groups which make up an operad and are indexed by the natural numbers.
%
%\begin{nota}\label{nota:e_b}
%Let $S$ be a set which we view as a discrete category.
%  \begin{enumerate}
%    \item For any functor $F \colon S \rightarrow \mb{Sets}$, let $EF$ denote the composite $E \circ F \colon S \rightarrow \mb{Cat}$; we often view $F$ as an indexed set $\{ F(s) \}$, in which case $EF$ is the indexed category $\{ EF(s) \}$.
%    \item For any functor $F \colon S \rightarrow \mb{Grp}$, let $BF$ denote the composite $B \circ F \colon S \rightarrow \mb{Cat}$; we often view $F$ as an indexed group $\{ F(s) \}$, in which case $BF$ is the indexed category $\{ BF(s) \}$.
%  \end{enumerate}
%\end{nota}
%

Transfer of operads stuff for EL:

%
%We are additionally interested in $\Lambda$-operads in $\mb{Cat}$ (or other cocomplete symmetric monoidal categories in which the tensor product preserves colimits in each variable). While the definition above gives the correct notion of a $\Lambda$-operad in $\mb{Cat}$ if we interpret the two equivariance axioms to hold for both objects and morphisms, it is useful to give a purely diagrammatic expression of these axioms. In the diagrams below, expressions of the form $G \times C$ for a group $G$ and category $C$ mean that the group $G$ is to be treated as a discrete category. This follows the standard method of how one expresses group actions in categories other than $\mb{Sets}$ using a copower. Thus the diagrams below are the two equivariance axioms given in \cref{Defi:lamop} expressed diagrammatically, where $K = k_1 + \cdots + k_n$.
%    %% Expanded diagram
%  % \[
%  %   \xy
%  %     (0,0)*+{\scriptstyle P(n) \times P(k_{1}) \times \cdots \times P(k_{n}) \times \Lambda(k_{1}) \times \cdots \times \Lambda(k_{n}) } ="00";
%  %     (0,-15)*+{\scriptstyle P(\underline{k}) \times \Lambda(\underline{k}) } ="01";
%  %     (60,0)*+{\scriptstyle P(n) \times P(k_{1}) \times \Lambda(k_{1}) \times \cdots \times P(k_{n}) \times  \Lambda(k_{n}) } ="20";
%  %     (60,-15)*+{\scriptstyle P(n) \times P(k_{1}) \times \cdots \times P(k_{n}) } ="21";
%  %     (30, -25)*+{\scriptstyle P(\underline{k}) } ="12";% diagram
%  %     {\ar^{\cong} "00" ; "20"};
%  %     {\ar^{1 \times \alpha_{k_1} \times \cdots \times \alpha_{k_n}} "20" ; "21"};
%  %     {\ar^{\mu^P} "21" ; "12"};
%  %     {\ar_{\mu^P \times \mu^\Lambda(e;-)} "00" ; "01"};
%  %     {\ar_{\alpha_{\underline{k}}} "01" ; "12"};
%  %   \endxy
%  % \]
%  \[
%    \xy
%      (0,0)*+{\scriptstyle P(n) \times \left(\prod_{i=1}^n P(k_i)\right) \times \left(\prod_{i=1}^n \Lambda(k_i)\right)} ="00";
%      (0,-15)*+{\scriptstyle P(K) \times \Lambda(K) } ="01";
%      (60,0)*+{\scriptstyle P(n) \times \left(\prod_{i=1}^n \left(P(k_{i}) \times \Lambda(k_{i})\right)\right)} ="20";
%      (60,-15)*+{\scriptstyle P(n) \times \left(\prod_{i=1}^n P(k_i)\right) } ="21";
%      (30, -25)*+{\scriptstyle P(K) } ="12";% diagram
%      {\ar^{\cong} "00" ; "20"};
%      {\ar^{1 \times \prod_{i=1}^n \alpha_i} "20" ; "21"};
%      {\ar^{\mu^P} "21" ; "12"};
%      {\ar_{\mu^P \times \mu^\Lambda(e;-)} "00" ; "01"};
%      {\ar_{\alpha_{K}} "01" ; "12"};
%    \endxy
%  \]
%    %% Expanded diagram
%  % \[
%  %   \xy
%  %     (0,0)*+{\scriptstyle P(n) \times \Lambda(n) \times P(k_{1}) \times \cdots \times P(k_{n}) } ="00";
%  %     (0,-10)*+{\scriptstyle P(n) \times \Lambda(n) \times \Lambda(n) \times P(k_{1}) \times \cdots \times P(k_{n}) } ="01";
%  %     (0,-20)*+{\scriptstyle P(n) \times \Lambda(n) \times P(k_{1}) \times \cdots \times P(k_{n}) \times \Lambda(n) } ="02";
%  %     (0,-30)*+{\scriptstyle P(n) \times \Sigma_{n} \times P(k_{1}) \times \cdots \times P(k_{n}) \times \Lambda(n) } ="03";
%  %     (55,-30)*+{\scriptstyle P(\underline{k}) \times \Lambda(\underline{k}) } ="13";
%  %     (70,0)*+{\scriptstyle P(n) \times P(k_{1}) \times \cdots \times P(k_{n}) } ="20";
%  %     (70,-18)*+{\scriptstyle P(\underline{k}) } ="21";
%  %     {\ar_{1 \times \Delta \times 1} "00" ; "01"};
%  %     {\ar^{\cong} "01" ; "02"};
%  %     {\ar_{1 \times \pi_{n} \times 1} "02" ; "03"};
%  %     {\ar^{} "03" ; "13"};
%  %     (35,-33)*{\scriptstyle \tilde{\mu}^P \times \mu^\Lambda(-;\underline{e})};
%  %     {\ar_{\alpha_{\underline{k}}} "13" ; "21"};
%  %     {\ar^{\alpha_{n} \times 1} "00" ; "20"};
%  %     {\ar^{\mu^P} "20" ; "21"};
%  %   \endxy
%  % \]
%  \[
%    \xy
%      (0,0)*+{\scriptstyle P(n) \times \Lambda(n) \times \prod_{i=1}^n P(k_i) } ="00";
%      (0,-10)*+{\scriptstyle P(n) \times \Lambda(n) \times \Lambda(n) \times \prod_{i=1}^n P(k_i) } ="01";
%      (0,-20)*+{\scriptstyle P(n) \times \Lambda(n) \times \prod_{i=1}^n P(k_i) \times \Lambda(n) } ="02";
%      (0,-30)*+{\scriptstyle P(n) \times \Sigma_{n} \times \prod_{i=1}^n P(k_i) \times \Lambda(n) } ="03";
%      (55,-30)*+{\scriptstyle P(K) \times \Lambda(K) } ="13";
%      (70,0)*+{\scriptstyle P(n) \times \prod_{i=1}^n P(k_i) } ="20";
%      (70,-18)*+{\scriptstyle P(K) } ="21";
%      {\ar_{1 \times \Delta \times 1} "00" ; "01"};
%      {\ar_{\cong} "01" ; "02"};
%      {\ar_{1 \times \pi_{n} \times 1} "02" ; "03"};
%      {\ar^{} "03" ; "13"};
%      (35,-33)*{\scriptstyle \tilde{\mu}^P \times \mu^\Lambda(-;\underline{e})};
%      {\ar_{\alpha_{K}} "13" ; "21"};
%      {\ar^{\alpha_{n} \times 1} "00" ; "20"};
%      {\ar^{\mu^P} "20" ; "21"};
%    \endxy
%  \]
%In the second diagram, the morphism
%    \[
%        \tilde{\mu}^P \colon P(n) \times \Sigma_n \times \prod_{i=1}^n P(k_i) \rightarrow P(K) \times \Lambda(K)
%    \]
%is first the left action of $\Sigma_n$ on the product followed by the operad multiplication, and $\underline{e}$ is $e_{k_{1}}, \ldots, e_{k_{n}}$.
%
%\begin{Defi}\label{Defi:actop_to_cat}
%Let $\Lambda$ be an action operad. Then $B\Lambda$ (see \cref{nota:e_b}) is the category with objects the natural numbers and
%  \[
%    B\Lambda(m,n) = \left\{ \begin{array}{lc}
%    \Lambda(n), & m = n \\
%    \emptyset, & m \neq n,
%    \end{array} \right.
%  \]
%where composition is given by group multiplication and the identity morphism is the unit element $e_n \in \Lambda(n)$.
%\end{Defi}
%
%\begin{thm}\label{preserveGop}
%Let $M,N$ be cocomplete symmetric monoidal categories in which the tensor product preserves colimits in each variable, and let $F \colon M \rightarrow N$ be a symmetric lax monoidal functor with unit constraint $\varphi_{0}$ and tensor constraint $\varphi_{2}$. Let $\Lambda$ be an action operad, and $P$ a $\Lambda$-operad in $M$. Then $FP = \{ F(P(n)) \}$ has a canonical $\Lambda$-operad structure, giving a functor
%  \[
%    F_{*} \colon \Lambda\mbox{-}Op(M) \rightarrow \Lambda\mbox{-}Op(N)
%  \]
%from the category of $\Lambda$-operads in $M$ to the category of $\Lambda$-operads in $N$.
%\end{thm}
%\begin{proof}
%The category of $\Lambda$-operads in $M$ is the category of monoids for the composition product $\circ_{M}$ on $[B\Lambda^{\textrm{op}}, M]$ constructed in \cref{sec:opasmon}. Composition with $F$ gives a functor
%  \[
%    F_{*} \colon  [B\Lambda^{\textrm{op}}, M] \rightarrow [B\Lambda^{\textrm{op}}, N],
%  \]
%  and to show that it gives a functor between the categories of monoids we need only prove that $F_{*}$ is lax monoidal with respect to $\circ_{M}$ and $\circ_{N}$. In other words, we must construct natural transformations with components $F_{*}X \circ_{N} F_{*}Y \rightarrow F_{*}(X \circ_{M} Y)$ and $I_{Op(N)} \rightarrow F_{*}(I_{Op(M)})$ and then verify the lax monoidal functor axioms. We note that in the calculations below, we often write $F$ instead of $F_{*}$, but it should be clear from context when we are applying $F$ to objects and morphisms in $M$ and when we are applying $F_{*}$ to a functor $ B\Lambda^{\textrm{op}} \rightarrow M$.
%
%We first remind the reader about copowers in cocomplete categories. For an object $X$ and set $S$, the copower $X \odot S$ is the coproduct $\coprod_{s \in S} X$. We have natural isomorphisms $(X \odot S) \odot T \cong X \odot (S \times T)$ and $X \odot 1 \cong X$, and using these we can define an action of a group $G$ on an object $X$ using a map $X \odot G \rightarrow X$. Any functor $F$ between categories with coproducts is lax monoidal with respect to those coproducts:  the natural map $FA \coprod FB \rightarrow F(A \coprod B)$ is just the map induced by the universal property of the coproduct using $F$ applied to the coproduct inclusions $A \hookrightarrow A \coprod B, B \hookrightarrow A \coprod B$. In particular, for any functor $F$ there exists an induced map $FX \odot S \rightarrow F(X \odot S)$.
%
%The unit object in $[B\Lambda^{\textrm{op}}, M]$ for $\circ_{M}$ is the copower $I_{M} \odot B\Lambda(-,1)$. Thus the unit constraint for $F_{*}$ is the composite
%  \[
%    I_{N} \odot B\Lambda(-,1) \stackrel{\varphi_{0} \odot 1}{\longrightarrow} FI_{M} \odot B\Lambda(-,1) \rightarrow F(I_{M} \odot B\Lambda(-,1) ).
%  \]
%
%For the tensor constraint, we will require a map
%  \[
%    t \colon (FY)^{\star n}(k) \rightarrow F\left(Y^{\star n}(k)\right)
%  \]
% where $\star$ is the Day convolution product; having constructed one, the tensor constraint is then the following composite.
%% \[
%% \begin{array}{rcl}
%% (FX \circ FY)(k) & \cong & \int^{n} FX(n) \otimes (FY)^{\star n}(k) \\
%% & \stackrel{ \int 1 \otimes t}{\longrightarrow}  & \int^{n} FX(n) \otimes F(Y^{\star n}(k)) \\
%% & \stackrel{\int \varphi_{2}}{\longrightarrow}  & \int^{n} F(X(n) \otimes Y^{\star n}(k)) \\
%% & \longrightarrow & F (\int^{n} X(n) \otimes Y^{\star n}(k)) \\
%% & \cong & F(X \circ Y)(k),
%% \end{array}
%% \]
%  \begin{align*}
%    (FX \circ FY)(k)  &\xrightarrow{\cong} \int^{n} FX(n) \otimes (FY)^{\star n}(k) \\
%    &\xrightarrow{\int 1 \otimes t} \int^{n} FX(n) \otimes F(Y^{\star n}(k)) \\
%    &\xrightarrow{\int \varphi_2} \int^{n} F(X(n) \otimes Y^{\star n}(k)) \\
%    &\longrightarrow F \left(\int^{n} X(n) \otimes Y^{\star n}(k)\right) \\
%    &\xrightarrow{\cong} F(X \circ Y)(k)
%  \end{align*}
%
%Both isomorphisms in the composite above are induced by universal properties (see \cref{section:operads_in_Cat} for more details) and the unlabeled arrow is induced by the same argument as that for coproducts above but this time using coends. The arrow $t$ is constructed in a similar fashion, and is the composite below.
%% \[
%% \begin{array}{rcl}
%% (FY)^{\star n}(k) & = & \int^{k_{1}, \ldots, k_{n}} FY(k_{1}) \otimes \cdots \otimes FY(k_{n}) \odot B\Lambda(k, \sum k_{i}) \\
%% & \rightarrow &  \int^{k_{1}, \ldots, k_{n}} F(Y(k_{1}) \otimes \cdots \otimes Y(k_{n})) \odot B\Lambda(k, \sum k_{i}) \\
%% & \rightarrow & \int^{k_{1}, \ldots, k_{n}} F(Y(k_{1}) \otimes \cdots \otimes Y(k_{n}) \odot B\Lambda(k, \sum k_{i}) ) \\
%% & \rightarrow & F\int^{k_{1}, \ldots, k_{n}} Y(k_{1}) \otimes \cdots \otimes Y(k_{n}) \odot B\Lambda(k, \sum k_{i})  \\
%% & = & F(Y^{\star n}(k))
%% \end{array}
%% \]
%  \begin{align*}
%    (FY)^{\star n}(k) & =  \int^{k_{1}, \ldots, k_{n}} FY(k_{1}) \otimes \cdots \otimes FY(k_{n}) \odot B\Lambda(k, \Sigma k_{i}) \\
%    & \rightarrow   \int^{k_{1}, \ldots, k_{n}} F(Y(k_{1}) \otimes \cdots \otimes Y(k_{n})) \odot B\Lambda(k, \Sigma k_{i}) \\
%    & \rightarrow  \int^{k_{1}, \ldots, k_{n}} F(Y(k_{1}) \otimes \cdots \otimes Y(k_{n}) \odot B\Lambda(k, \Sigma k_{i}) ) \\
%    & \rightarrow  F\int^{k_{1}, \ldots, k_{n}} Y(k_{1}) \otimes \cdots \otimes Y(k_{n}) \odot B\Lambda(k, \Sigma k_{i})  \\
%    & =  F(Y^{\star n}(k))
%  \end{align*}
%Checking the lax monoidal functor axioms is tedious but entirely routine using the lax monoidal functor axioms for $F$ together with various universal properties of colimits, and we leave the details to the reader.
%\end{proof}
%
%Combining \cref{preserveGop} and \cref{gisgop} with \cref{symmoncor}, we immediately obtain the following.

%\ngnoteil{I moved a bunch of stuff here, then commented it out. If something used to be here but is now gone, check the file.}

%where the action of $\Lambda(n)$ on $\ELn$ is given by the obvious multiplication action on the right, and the action of $\Lambda(n)$ on $X^{n}$ is given using $\pi_{n} \colon \Lambda(n) \rightarrow \Sigma_{n}$ together with the standard left action of $\Sigma_{n}$ on $X^{n}$ in any symmetric monoidal category. The $2$-monad $\underline{E\Sigma}$ is that for symmetric strict monoidal categories (see \cref{sec:examples} for this and further examples).

% QQQ Here is probably a natural place to split into another chapter? Previous stuff "Abstract categorical properties of action operads", later stuff "Monoidal structures and multicategories". Or after these results/the next section.
% QQQ We came to the conclusion just to cite Yau's results in Chapter 19. `A strict $G$-monoidal category is..., etc.'

%Strict $\Lambda$-monoidal categories, in the sense of being strict algebras for the monad $\underline{\EL}$, can be characterised in more familiar terms by specifying a monoidal structure with appropriate equivariant interaction with the $\Lambda(n)$-actions. Explicit proofs of such can be found in , along with similar characterisations for strict $\Lambda$-monoidal functors and $\Lambda$-monoidal categories whose underlying monoidal structure is weak.

% \begin{prop}\label{el_via_moncats}
% A strict $\Lambda$-monoidal category structure on $X$ determines and is determined by
% \begin{itemize}
% \item a strict monoidal category structure on $X$: $(X, \otimes, I)$,
% \item for each $\sigma \in \Lambda(n)$ and $x_1$, $\ldots$, $x_n \in X$, an isomorphism
%   \[
%     \sigma_{x_1,\ldots,x_n} \colon x_1 \otimes \ldots x_n \rightarrow x_{\pi(\sigma)^{-1}(1)} \otimes \ldots \otimes x_{\pi(\sigma)^{-1}(n)}
%   \]
% which is natural in each $x_i$,
% \item such that for any $\tau_i \in \Lambda(k_i)$ and $\sigma \in \Lambda(n)$, where $1 \leq i \leq n$, the following diagram commutes
% \[
%   \xy
%     (0,0)*+{\bigotimes_{i=1}^n \bigotimes_{j=1}^{k_i} x_{i,j}}="a";
%     (50,0)*+{\bigotimes_{i=1}^n \bigotimes_{j=1}^{k_i} x_{i,\pi(\tau_i)^{-1}(j)}}="b";
%     (25,-20)*+{\bigotimes_{i=1}^n \bigotimes_{j=1}^{k_i} x_{\pi(\sigma)^{-1}(i),\pi\left(\tau_{\pi(\sigma)^{-1}(i)}\right)^{-1}(j)}}="c";
%     %
%     {\ar^(.4){\tau_1 \otimes \cdots \otimes \tau_n} "a" ; "b"};
%     {\ar^{\sigma^{+}} "b" ; "c"};
%     {\ar_{\mu^{\Lambda}(\sigma;\tau_1,\ldots,\tau_n)} "a" ; "c"};
%   \endxy
% \]
% \item QQQ needs to be some equivariance condition in here, otherwise appropriate axioms (e.g., hexagon identities) don't come out of this (can't remember why I came to this conclusion...)
% \end{itemize}
% \end{prop}
% \begin{proof}


% First we shall show an $\EL$-algebra structure on $X$ can be used to specify the monoidal structure described above. If $X$ is an $\EL$-algebra then we also have the family of morphisms $\alpha_n \colon \ELn \times X^n \rightarrow X$ satisfying the usual axioms. We define the monoidal product of two objects $x$, $y \in X$ to be
%   \[
%     x \otimes y = \alpha_2(e_2;x,y)
%   \]
% and the unit object to be
%   \[
%     I = \alpha_0(e_0;).
%   \]
% Strict associativity follows from the the $\EL$-algebra axioms and \cref{calclem} as follows:
%   \begin{align*}
%     (x_1 \otimes x_2) \otimes x_3 &= \alpha_2(e_2;\alpha_2(e_2;x_1,x_2),x_3)\\
%     &= \alpha_2(e_2;\alpha_2(e_2;x_1,x_2),\alpha_1(e_1;x_3))\\
%     &= \alpha_3(\mu(e_2;e_2,e_1);x_1,x_2,x_3)\\
%     &= \alpha_3(e_3;x_1,x_2,x_3)\\
%     &= \alpha_3(\mu(e_2;e_1,e_2);x_1,x_2,x_3)\\
%     &= \alpha_2(e_2;\alpha(e_1;x_1),\alpha(e_2;x_2,x_3))\\
%     &= \alpha_2(e_2;x_1,\alpha_2(e_2;x_2,x_3))\\
%     &= x_1 \otimes (x_2 \otimes x_3).
%   \end{align*}
% Due to this we can then write $n$-fold monoidal products as
%   \[
%     x_1 \otimes \ldots \otimes x_n = \alpha_n(e_n;x_1,\ldots,x_n)
%   \]
% and for any $\sigma \in \Lambda(n)$ we now define
%   \[
%     x_{\pi(\sigma)^{-1}(1)} \otimes \ldots \otimes x_{\pi(\sigma)^{-1}(n)} = \alpha_n(\sigma;x_1,\ldots,x_n).
%   \]
% For the unit object we have
%   \begin{align*}
%     I \otimes x &= \alpha_2(e_2;I,x)\\
%     &= \alpha_2(e_2;\alpha_0(e_0;),\alpha_1(e_1;x))\\
%     &= \alpha_1(\mu^\Lambda(e_2;e_0,e_1);x)\\
%     &= \alpha_1(e_1;x)\\
%     &= x,
%   \end{align*}
% with similar working to show that $x \otimes I = x$.

% Next we specify the permutation isomorphisms. Given $\sigma \in \Lambda(n)$, there exists a unique isomorphism
%   \[
%     ! \colon e_n \rightarrow \sigma
%   \]
% in $\ELn$, which we use to define the isomorphism
%   \[
%     \tilde{\sigma} = \alpha_n(!;\underline{\id}) \colon \alpha_n(e_n;x_1,\ldots,x_n) \rightarrow \alpha_n(\sigma;x_1,\ldots,x_n).
%   \]
% That these are natural follows simply from the functoriality of each $\alpha_n$, giving
%   \[
%   \alpha_n(!;\underline{\id})\alpha(\id;f_1,\ldots,f_n) = \alpha_n(!;f_1,\ldots,f_n) = \alpha_n(\id;f_1,\ldots,f_n)\alpha_n(!;\underline{\id})
%   \]
% for $f_i \colon x_i \rightarrow x_i'$. The isomorphisms also satisfy the axioms specified in the diagram above as a direct result of the $\EL$-algebra axioms.

% Conversely, we begin with a strict monoidal category $(X,\otimes,I)$, along with isomorphisms
%   \[
%     \tilde{\sigma} \colon x_1 \otimes \ldots \otimes x_n \rightarrow x_{\pi(\sigma)^{-1}(1)} \otimes \ldots \otimes \pi(\sigma)^{-1}(n)
%   \]
% satisfying the axioms specified above. We need to describe morphisms
%   \[
%     \alpha_n \colon \ELn \times X^n \rightarrow X
%   \]
% which satisfy the axioms of an $\EL$-algebra. To begin with we first define
%   \[
%     \alpha_0(e_0;) = I.
%   \]
% To define each $\alpha_n$ on objects, we then put
%   \[
%     \alpha_n(\sigma;x_1,\ldots,x_n) = x_{\pi(\sigma)^{-1}(1)} \otimes \ldots \otimes x_{\pi(\sigma)^{-1}(n)}.
%   \]
% This is easily checked to be well-defined.

% A morphism in $\ELn \times_{\Lambda(n)} X^n$ from $[\sigma;x_1,\ldots,x_n]$ to $[\tau;y_1,\ldots,y_n]$ consists of a unique isomorphism $! \colon \sigma \cong \tau$ along with morphisms $f_i \colon x_i \rightarrow y_i$ in $X$. We define $\alpha_n(!;f_1,\ldots,f_n)$ to be the following composite.
%     \[
%         \alpha_n(\sigma;x_1,\ldots,x_n) \xrightarrow{\tilde{\sigma}^{-1}} \alpha_n(e_n;x_1,\ldots,x_n) \xrightarrow{\otimes f_i} \alpha_n(e_n;y_1,\ldots,y_n) \xrightarrow{\tilde{\tau}} \alpha_n(\tau;y_1,\ldots,y_n)
%     \]
% Again, well-definedness and functoriality conditions are easily checked.

% The unit axiom is satisfied immediately since we then have $\alpha_1(e_1;x) = x$. The other $\EL$-algebra axiom then follows, with some careful consideration of indices, from two applications of the diagram described in the statement of the proposition.
% \end{proof}

% \begin{prop}\label{el_strictmap}
% Let $X,Y$ be strict $\Lambda$-monoidal categories, and $F \colon X \rightarrow Y$ be a functor. $F$ is a strict $\Lambda$-monoidal functor if and only if
% \end{prop}
% \begin{proof}

% \end{proof}

% \begin{prop}\label{el_weakmap}
% Let $X,Y$ be strict $\Lambda$-monoidal categories, and $F \colon X \rightarrow Y$ be a functor. The structure of a weak $\Lambda$-monoidal functor determines and is determined by some stuff.
% \end{prop}
% \begin{proof}

% \end{proof}

% \begin{prop}\label{el_2cells}
% Let $X,Y$ be strict $\Lambda$-monoidal categories, and $F, G \colon X \rightarrow Y$ be strict $\Lambda$-monoidal functors. The structure of an $\EL$-algebra $2$-cell $\alpha \colon F \Rightarrow G$ determines and is determined by some stuff.
% \end{prop}
% \begin{proof}

% \end{proof}


%\ngmpar{Rework into a full description of the free algebra $\EL(X)$}For an action operad $\Lambda$ and any category $X$, the set of morphisms from $[e; x_1, \ldots, x_n]$ to $[e; y_1, \ldots, y_n]$ in $\coeq{\EL}{X}{\Lambda}{n}$ is
%  \[
%    \coprod_{g \in \Lambda(n)} \prod_{i=1}^{n} X\left(x_i, y_{g(i)}\right).
%  \]
%
%\begin{rem}
%
%\ngnoteil{copied from exposition}
%It will be useful for our calculations later to give an explicit description of the categories $\coeq{\EL}{X}{\Lambda}{n}$. Objects are equivalence classes of tuples $(g; x_1, \ldots, x_n)$ where $g \in \Lambda(n)$ and the $x_{i}$ are objects of $X$, with the equivalence relation given by
%  \[
%    (gh; x_1, \ldots, x_n) \sim \left(g; x_{h^{-1}(1)}, \ldots, x_{h^{-1}(n)}\right);
%  \]
%  we write these classes as $[g; x_1, \ldots, x_n]$. Morphisms are then equivalence classes of morphisms
%  \[
%    (!; f_1, \ldots, f_n) \colon  (g; x_1, \ldots, x_n) \rightarrow \left(h; x_1', \ldots, x_n'\right).
%  \]
%We have two distinguished classes of morphisms, one for which the map $! \colon  g \rightarrow h$ is the identity and one for which all the $f_{i}$'s are the identity. Every morphism in $\ELn \times X^{n}$ is uniquely a composite of a  morphism of the first type followed by one of the second type. Now $\coeq{\EL}{X}{\Lambda}{n}$ is a quotient of $\ELn \times X^{n}$ by a free group action, so every morphism of $\coeq{\EL}{X}{\Lambda}{n}$ is in the image of the quotient map. Using this fact, we can prove the following useful lemma.
%
%\ngnoteil{old proof}
%A morphism with source $(e; x_1, \ldots, x_n)$ in $\ELn \times X^{n}$ is uniquely a composite
%  \[
%    (e; x_1, \ldots, x_n) \stackrel{(\id; f_{1}, \ldots, f_{n})}{\longrightarrow} \left(e; x_1', \ldots, x_n'\right) \stackrel{(!; \id, \ldots, \id)}{\longrightarrow} \left(g; x_1', \ldots, x_n'\right).
%  \]
%Descending to the quotient, this becomes a morphism
%  \[
%    [e; x_1, \ldots, x_n] \rightarrow \left[g; x_1', \ldots, x_n'\right] = \left[e; x_{g^{-1}(1)}', \ldots, x_{g^{-1}(n)}'\right],
%  \]
%and therefore is a morphism $[e; x_1, \ldots, x_n] \rightarrow [e; y_1, \ldots, y_n]$ precisely when $y_i = x_{g^{-1}(i)}'$, and so $f_i \in   X(x_i, y_{g(i)})$.
%
%The material in this section can be given a rather more abstract interpretation, in the sense of \cite{KL97}. The idea here is that the category of $\Lambda$-collections acts on the category $\cat$ via a functor $\diamond \colon \Lambda\text{-}\mathbf{Coll} \times \cat \rightarrow \cat$ which sends $(P,X)$ to $\underline{P}(X)$ as described above. Fixing a $\Lambda$-collection $P$ produces an endofunctor $\underline{P} \colon \cat \rightarrow \cat$ which is then a monad when $P$ is a $\Lambda$-operad, just as monoids in $\Lambda\text{-}\mathbf{Coll}$ are precisely $\Lambda$-operads.
%\end{rem}

Coherence:

% For a $2$-monad $T$, the $2$-categories $\mb{Ps}\mbox{-}T\mbox{-}\mb{Alg}$ (of pseudoalgebras and weak morphisms) and $T\mbox{-}\mb{Alg}_s$ (of strict algebras and strict morphisms) are of particular interest. The behavior of colimits in both of these $2$-categories can often be deduced from properties of $T$, the most common being that $T$ is finitary. In practice, one thinks of a finitary monad as one in which all operations take finitely many inputs as variables. 
% If $T$ is finitary, then $T\mbox{-}\mb{Alg}_s$ will be cocomplete by standard results given in \cite{BKP}. There are additional results of a purely $2$-dimensional nature concerning finitary $2$-monads, detailed in \cite{lack-cod} and extending those in \cite{BKP}, namely the existence of a left adjoint
%    \[
%        \mb{Ps}\mbox{-}T\mbox{-}\mb{Alg} \rightarrow T\mbox{-}\mb{Alg}_s
%    \]
%to the forgetful $2$-functor which regards a strict algebra as a pseudoalgebra with identity structure isomorphisms.

%To show that $\underline{P}$ is finitary we must show that it preserves filtered colimits or, equivalently, that it preserves directed colimits (see \cite{ar}). Consider some directed colimit, $\text{colim}X_{i}$ say, in $\cat$. Then consider the following sequence of isomorphisms:
%    \begin{align*}
%      \underline{P}(\text{colim}X_{i}) &= \coprod_n \coeq{P}{(\text{colim}X_{i})}{\Lambda}{n} \\
%      &\cong \coprod_n P(n) \times_{\Lambda(n)} \text{colim}(X_{i}^n) \\
%      &\cong \coprod_n \text{colim}(\coeq{P}{X_i}{\Lambda}{n}) \\
%      &\cong \text{colim}\coprod_n \coeq{P}{X_i}{\Lambda}{n} \\
%      &= \text{colim}\underline{P}(X_{i}).
%    \end{align*}
%Since $\cat$ is locally finitely presentable then directed colimits commute with finite limits, giving the first isomorphism. The second isomorphism follows from this fact as well as that colimits commute with coequalizers. The third isomorphism is simply coproducts commuting with other colimits.

Cartesian stuff:

%The pullback of the functors $\coeqb{1}{H^n}{\Sigma_{n}}$ and $\coeqb{1}{S^n}{\Sigma_{n}}$ is a category consisting of pairs of objects $[p;\underline{c}]$ and $[q;\underline{b}]$, where $\underline{b}$ and $\underline{c}$ represent lists of elements in $B$ and $C$, respectively, that satisfy the property that
%    \[
%        \left[p;\underline{H(c)}\right] = \left[q; \underline{S(b)}\right]
%    \]
%    in $\coeqsig{P}{D}{n}$.
%Since the action is free by hypothesis, Lemma \ref{lem:coeq-lem} implies that a pair
%    \[
%        \left(\left[p;\underline{c}\right], \left[q;\underline{b}\right]\right)
%    \]
%is in the pullback if and only if there exists a necessarily unique element $g \in \Sigma_n$ such that $p \cdot g = q$ and $Hc_i = Sb_{g^{-1}(i)}$. 
%The morphisms in the pullback admit a similar description.
%
%Using this we can define mutual inverses between $\coeqsig{P}{A}{n}$ and the pullback $Q'$. Considering the category $A$ as the pullback of the diagram we started with, we can consider objects of $\coeqsig{P}{A}{n}$ as being equivalence classes
%    \[
%        [p;(b_1,c_1),\ldots,(b_n,c_n)]
%    \]
%where $p \in P(n)$ and $Hc_i = Sb_i$ for all $i$.
%
%Taking such an object, we send it to the pair
%    \[
%        \left(\left[p;c_1,\ldots,c_n\right],[p;b_1,\ldots,b_n]\right),
%    \]
%and note that it lies in the pullback since the identity in $\Sigma_n$ satisfies the condition given earlier. An inverse to this sends a pair of equivalence classes in $Q'$ to the single equivalence class
%    \[
%        \left[p;\left(c_1,b_{g^{-1}(1)}\right),\ldots,\left(c_n,b_{g^{-1}(n)}\right)\right]
%    \]
%in $\coeqsig{P}{A}{n}$. If we apply the map into $Q'$ we get the pair
%    \[
%        \left(\left[p;c_1,\ldots,c_n\right],\left[p;b_{g^{-1}(1)},\ldots,b_{g^{-1}(n)}\right]\right)
%    \]
%which is equal to the original pair since $p \cdot g = q$; the other composite is trivially an identity. A similar calculation on morphisms finishes the proof that $\coeqsig{P}{A}{n}$ is the pullback as required.

Clubs:

%Since the monads $\underline{\EL}$ and $\underline{E\Sigma}$ both decompose into a disjoint union of functors, we only have to show that, for any $n$, the square below is a pullback.
%  \[
%    \xy
%      (0,0)*+{\coeq{\EL}{X}{\Lambda}{n}} ="00";
%      (0,-10)*+{B\Lambda(n)} ="01";
%      (35,0)*+{E\Sigma_{n} \otimes_{\Sigma_{n}} X^n} ="10";
%      (35,-10)*+{B\Sigma_{n}} ="11";
%      {\ar^{} "00" ; "10"};
%      {\ar^{} "10" ; "11"};
%      {\ar^{} "00" ; "01"};
%      {\ar^{} "01" ; "11"};
%    \endxy
%  \]
%By \cref{lem:coeq-lem}
%%(QQQ happy with this reference? QQQ Seems like the right thing?)
%, this amounts to showing that the square below is a pullback.
%  \[
%    \xy
%      (0,0)*+{\left(\EL(n) \times X^n\right)/\Lambda(n)} ="00";
%      (0,-10)*+{B\Lambda(n)} ="01";
%      (35,0)*+{\left(E\Sigma_{n} \times X^n\right)/\Sigma_{n}} ="10";
%      (35,-10)*+{B\Sigma_{n}} ="11";
%      {\ar^{} "00" ; "10"};
%      {\ar^{} "10" ; "11"};
%      {\ar^{} "00" ; "01"};
%      {\ar^{} "01" ; "11"};
%    \endxy
%  \]
%Now the bottom map is clearly bijective on objects since these categories only have one object. An object in the top right is an equivalence class
%  \[
%    [\sigma; x_{1}, \ldots, x_{n}] = \left[e; x_{\sigma^{-1}(1)}, \ldots, x_{\sigma^{-1}(n)}\right].
%  \]
%A similar description holds for objects in the top left, with $g \in \Lambda(n)$ replacing $\sigma$ and $\pi(g)^{-1}$ replacing $\sigma^{-1}$ in the subscripts. The map along the top sends $[g; x_{1}, \ldots, x_{n}]$ to $[\pi(g); x_{1}, \ldots, x_{n}]$, and thus sends $[e; x_{1}, \ldots, x_{n}]$ to $[e; x_{1}, \ldots, x_{n}]$, giving a bijection on objects.
%
%Now a morphism in $(\ELn \times X^{n})/\Lambda(n)$ can be given as
%  \[
%    [e; x_{1}, \ldots, x_{n}] \stackrel{[!; f_{i}]}{\longrightarrow} [g; y_{1}, \ldots, y_{n}].
%  \]
%Mapping down to $B\Lambda(n)$ gives $ge^{-1} = g$, while mapping over to $(E\Sigma_{n} \times X^{n})/\Sigma_{n}$ gives $[!; f_{i}]$ where $! \colon e \rightarrow \pi(g)$ is now a morphism in $E\Sigma_{n}$. In other words, a morphism in the upper left corner of our putative pullback square is determined completely by its images along the top and lefthand functors. Furthermore, given $g \in \Lambda(n)$, $\tau = \pi(g)$, and morphisms $f_{i} \colon x_{i} \rightarrow y_{i}$ in $X$, the morphism $[! \colon e \rightarrow g; f_{i}]$ maps to the pair $(g, [! \colon e \rightarrow \tau; f_{i}])$, completing the proof that this square is indeed a pullback.

%\ngmpar{new stuff starting here, we need to make sure I didn't miss anything}
%Recall from \cref{Defi:coll-over-SS} that a collection over $\SS$ is a set $X$ and a function $f \colon X \to \SS$ from $X$ to $\SS = \coprod \Sigma_n$. The function $f$ assigns to $x \in X$ its arity, consisting of both a natural number $n$ and a symmetry $\sigma \in \Sigma_n$.
%
%\begin{Defi}\label{Defi:formal-comps}
%Let $(X, f)$ be a collection over $\SS$. We define the \emph{collection of formal composites of $(X,f)$} as follows.
%\begin{enumerate}
%\item $\textrm{Comp}^0(X,f)$ is the collection $\{ e_n \}_{n \in \mathbb{N}} \subseteq \SS$ consisting of each identity permutation $e_n \in \Sigma_n$; we denote the inclusion function above by $f_0$.
%\item $\textrm{Comp}^1(X,f)$ is the collection $(X,f) \to \SS$, and we define $f_1 = f$. 
%\item Given $\textrm{Comp}^k(X,f)$ as a collection over $\SS$ via $f_k \colon \textrm{Comp}^k(X,f) \to \SS$ for all $k \leq j$, the elements of $\textrm{Comp}^{j+1}(X,f)$ are tuples $(x; y_1, \ldots, y_n)$ where
%\begin{itemize}
%\item $x \in \textrm{Comp}^i(X,f)$ for $i=0, 1$;
%\item each $y_i \in \textrm{Comp}^k(X,f)$ for $i=1, \ldots, n$; and
%\item $f(x) \in \Sigma_n$.
%\end{itemize}
%The function $f_{k+1} \colon \textrm{Comp}^{k+1}(X,f) \to \SS$ is defined by
%\[
%f_{k+1}(x; y_1, \ldots, y_n) = \mu^{\Sigma}\big( f(x); f_k(y_1), \ldots, f_k(y_n) \big),
%\]
%where $\mu^{\Sigma}$ is operadic composition in $\Sigma$.
%\end{enumerate}
%The collection of formal composites of $(X,f)$, denoted $\textrm{Comp}(X,f)$, is 
%\[
%\coprod_{k \in \mathbb{N}} \textrm{Comp}^k(X,f),
%\]
%computed as the coproduct in the category of collections over $\SS$.
%\end{Defi}
%
%\begin{Defi}\label{Defi:mon-schema}
%A \emph{monoidal schema} $\mathbb{S}$ consists of 
%\begin{itemize}
%\item a collection over $\SS$, $\pi \colon \mathcal{G} \to \SS$ (see \cref{Defi:coll-over-SS}), and
%\item a subset $\mathcal{R} \subseteq \textrm{Comp}(\mathcal{G}, \pi) \times_{\SS} \textrm{Comp}(\mathcal{G}, \pi)$ of the pullback below,
% \begin{equation}\label{eqn:pb-schema}
%    \xy
%      (0,0)*+{\textrm{Comp}(\mathcal{G}, \pi) \times_{\SS} \textrm{Comp}(\mathcal{G}, \pi)} ="00";
%      (0,-15)*+{\textrm{Comp}(\mathcal{G}, \pi)} ="01";
%      (50,0)*+{\textrm{Comp}(\mathcal{G}, \pi)} ="10";
%      (50,-15)*+{\SS} ="11";
%      {\ar^{} "00" ; "10"};
%      {\ar^{} "10" ; "11"};
%      {\ar^{} "00" ; "01"};
%      {\ar^{} "01" ; "11"};
%    \endxy
%  \end{equation}
%\end{itemize}
%where both functions $\textrm{Comp}(\mathcal{G}, \pi) \to \SS$ are the coprod $\coprod \pi_k$ as in \cref{Defi:formal-comps}.
%\end{Defi}
%
%A monoidal schema is meant to encapsulate a type of strict monoidal category equipped with isomorphism indexed by the elements of $\mathcal{G}$.
%
%\begin{Defi}\label{Defi:S-premoncat}
%Let $\mathbb{S} = (\mathcal{G}, \mathcal{R})$ be a monoidal schema. An \emph{$\mathbb{S}$-premonoidal category} consists of
%\begin{itemize}
%\item a strict monoidal category $(M, \otimes, I)$ and
%\item for each $g \in \mathcal{G}$ with $\pi(g) \in \Sigma_n$, a natural isomorphism $[g] \colon \otimes_n \cong \otimes_n \circ \pi(g)$.
%\end{itemize}
%
%
%such that\ngnote{make sure identities work correctly} for each $(r,r') \in \mathcal{R}$ SOMETHINGSOMETHINGSOMETHING.
%\end{Defi}
%
%\begin{Defi}\label{Defi:S-premoncat-realize}
%Let $(\mathcal{G}, \pi)$ be a monoidal schema, $(M, \otimes, I)$ be an $\mathbb{S}$-premonoidal category via the assignment $g \mapsto [g]$, and $f \in \textrm{Comp}(\mathcal{G}, \pi)$. The \emph{realization} of $f$ in $M$, denoted $\textrm{r}(f)$, is defined inductively as follows.
%\begin{enumerate}
%\item For the unique element $e_0 \in \textrm{Comp}^0(\mathcal{G}, \pi)$, define $\textrm{r}(e_0) = I$.
%\item For an element $g \in \mathcal{G} \subseteq \textrm{Comp}^1(\mathcal{G}, \pi)$, define $\textrm{r}(g) = [g]$. For the element $\id \in \textrm{Comp}^1(\mathcal{G}, \pi)$, define $\textrm{r}(\id) = 1_M$\ngnote{or identity transformation on $1_M$?}.
%\end{enumerate} 
%\end{Defi}
%
%
%\begin{Defi}\label{Defi:S-moncat}
%An $\mathbb{S}$-premonoidal category is an \emph{$\mathbb{S}$-monoidal category} if...
%\end{Defi}

%\ngnoteil{insert summarizing comment}
%
%\begin{cor}\label{cor:pres2}
%\ngnoteil{rewrite: some club generated by $\mathbb{S}$ is the same as the action operad generated by $\mathbb{S}$}
%Assume we have a notion $\mathcal{M}$ of strict monoidal category that is given by  a set natural isomorphisms
%  \[
%    \mathcal{G} = \left\{ (f, \pi_{f}) \, | \,  x_{1} \otimes \cdots \otimes x_{n} \stackrel{f}{\longrightarrow} x_{\pi_{f}^{-1}(1)} \otimes \cdots \otimes x_{\pi_{f}^{-1}(n)} \right\}
%  \]
%subject to a set $\mathcal{R}$ of axioms. Each such axiom is given by the data
%\begin{itemize}
%  \item two finite sets $f_{1}, \ldots, f_{n}$ and $f_{1}', \ldots, f_{m}'$ of elements of $\mathcal{G}$; and
%  \item two formal composites $F,F'$ using only composition and tensor product operations and the $f_{i}$, respectively $f_{i}'$, 
%\end{itemize}
%such that the underlying permutation of $F$ equals the underlying permutation of $F'$ (we compute the underlying permutations using the functions $\beta, \delta$ of \cref{thm:charAOp}). The element $\left(\underline{f}, \underline{f}', F, F'\right)$ of the set $\mathcal{R}$ of axioms corresponds to the requirement that the composite of the morphisms $f_{i}$ using $F$ equals the composite of the morphisms $f_{j}'$ using $F'$ in any strict monoidal category of type $\mathcal{M}$. Then strict monoidal categories of type $\mathcal{M}$ are given as the algebras for the club $\EL$ where $\Lambda$ is the action operad with
%\begin{itemize}
%  \item $\mathbf{g} = \mathcal{G}$,
%  \item $\mathbf{r} = \mathcal{R}$,
%  \item $s_{1}$ given by mapping the generator $\left(\underline{f}, \underline{f}', F, F'\right)$ to the operadic composition of the $f_{i}$ using $F$ via $\beta, \delta$, and
%  \item $s_{2}$ given by mapping the generator $\left(\underline{f}, \underline{f}', F, F'\right)$ to the operadic composition of the $f_{i}'$ using $F'$ via $\beta, \delta$.
%\end{itemize}
%\end{cor}
%
%\ngnoteil{these are now redundant, but remark how the earlier calculation is just checking this. it is still interesting though bc translating between the group presentations and the aop presentation is nontrivial}
%
%\begin{example}
%\ngnoteil{shorten?}
%The $2$-monad for symmetric strict monoidal categories (or permutative categories, as they are known in the topological literature) is given by $E\Sigma$, so the notion of symmetric strict monoidal categories corresponds to the symmetric operad. While this example is well-known, we go into further detail to set the stage for less common examples.
%
%The $2$-monad $\underline{E\Sigma}$ on $\mb{Cat}$ is given by
%  \[
%    \underline{E\Sigma} (X) = \coprod E\Sigma_{n} \times_{\Sigma_{n}} X^{n}.
%  \]
%An object of $E\Sigma_{n} \times_{\Sigma_{n}} X^{n}$ is an equivalence class of the form $[\sigma; x_{1}, \ldots, x_{n}]$ where $\sigma \in \Sigma_{n}$ and $x_{i} \in X$. The equivalence relation gives
%  \[
%    [\sigma; x_{1}, \ldots, x_{n}] = \left[e; x_{\sigma^{-1}(1)}, \ldots, x_{\sigma^{-1}(n)}\right],
%  \]
%so objects can be identified with finite strings of objects of $X$. Morphisms are given by equivalence classes of the form
%  \[
%    [\sigma; x_{1}, \ldots, x_{n}] \stackrel{[!; f_{1}, \ldots, f_{n}]}{\longrightarrow} [\tau; y_{1}, \ldots, y_{n}].
%  \]
%Here $! \colon \sigma \cong \tau$ is the unique isomorphism in $E \Sigma_{n}$, and $f_{i} \colon x_{i} \rightarrow y_{i}$ in $X$. Using the equivalence relation, we find that morphisms between finite strings
%  \[
%    x_{1}, \ldots, x_{n} \rightarrow y_{1}, \ldots, y_{n}
%  \]
%are given by a permutation $\rho \in \Sigma_{n}$ together with maps $f_{i} \colon x_{i} \rightarrow y_{\rho(i)}$ in $X$ (note that there are no morphisms between strings of different length); this is a special case of the calculation in \cref{hom-set-lemma}. Thus $E \Sigma(X)$ is easily seen to be the free permutative category generated by $X$, and therefore $\Sigma$-monoidal categories are permutative categories.
%\end{example}
%
%\begin{example}
%\ngnoteil{shorten?}
%The template above can be used to show that the braid operad $B$ corresponds to the $2$-monad for braided strict monoidal categories. The details are almost exactly the same, only we use braids instead of permutations. The equivalence relation on objects gives
%  \[
%    [\gamma; x_{1}, \ldots, x_{n}] = \left[e; x_{\pi(\gamma)^{-1}(1)}, \ldots, x_{\pi(\gamma)^{-1}(n)}\right],
%  \]
%where $\gamma \in B_{n}$ and $\pi(\gamma)$ is its underlying permutation; thus objects of $EB(X)$ are once again finite strings of objects of $X$. A morphism
%  \[
%    x_{1}, \ldots, x_{n} \rightarrow y_{1}, \ldots, y_{n}
%  \]
%is then given by a braid $\gamma \in B_{n}$ together with maps $f_{i} \colon x_{i} \rightarrow y_{\pi(\gamma)(i)}$ in $X$. Thus one should view a morphism as given by
%\begin{itemize}
%\item a finite ordered set $x_{1}, \ldots, x_{n}$ of objects of $X$ as the source,
%\item another such finite ordered set (of the same cardinality) $y_{1}, \ldots, y_{n}$ of objects of $X$ as the target,
%\item a geometric braid $\gamma \in B_{n}$ on $n$ strands, and
%\item for each strand, a morphism in $X$ from the object labeling the source of that strand to the object labeling the target.
%\end{itemize}
%This is precisely Joyal and Street's \cite{js} construction of the free braided strict monoidal category generated by a category $X$, and thus $B$-monoidal categories are braided strict monoidal categories.
%
%This example can be extended to include ribbon braided categories as well. A \textit{ribbon braid} is given, geometrically, in much the same way as a braid except that instead of paths $[0,1] \rightarrow \mathbb{R}^{3}$ making up each individual strand, we use ribbons
%$[0,1] \times [-\varepsilon, \varepsilon] \rightarrow \mathbb{R}^{3}$. This introduces the possibility of performing a full twist on a ribbon, and one can describe ribbon braided categories using generators and relations by introducing a natural twist isomorphism $\tau_{A} \colon A \rightarrow A$ and imposing one relation between the twist and the braid $\gamma_{A,B} \colon A \otimes B \rightarrow B \otimes A$. In \cite{sal-wahl}, the authors show that the ribbon braid groups give an action operad $RB$, and that (strict) ribbon braided categories are precisely the algebras for $ERB$.
%\end{example}

Coboundary:

%\ngnoteil{replace what follows with a discussion/reference to }
%
%Our interest is in strict coboundary categories by which we mean coboundary categories with strict underlying monoidal category. Under the assumption of strictness, the second axiom above does not include associations for the tensor product and reduces to a square. To show that every coboundary category is equivalent to a strict coboundary category, we must introduce the $2$-category $\mb{CobCat}$ of coboundary categories.
%
%\begin{Defi}
%Let $(C,\sigma), (C', \sigma')$ be coboundary categories. A \emph{coboundary functor} $F \colon C \rightarrow C'$ is a strong monoidal functor (with invertible constraints $\varphi_{0}$ for the unit and $\varphi_{x,y}$ for the tensor product) such that the following diagram commutes for all objects $x$, $y \in \m{C}$.
%  \[
%    \xy
%      (0,0)*+{Fx \otimes Fy}="00";
%      (25,0)*+{F(x \otimes y)}="10";
%      (0,-20)*+{Fy \otimes Fx}="01";
%      (25,-20)*+{F(y \otimes x)}="11";
%      %
%      {\ar^{\varphi_{x,y}} "00";"10"};
%      {\ar^{F\sigma_{x,y}} "10";"11"};
%      {\ar_{\sigma_{Fx,Fy}} "00";"01"};
%      {\ar_{\varphi_{y,x}} "01";"11"};
%    \endxy
%  \]
%  % \[
%  %   F\sigma_{x,y} \circ \varphi_{x,y} = \varphi_{y,x} \circ \sigma_{Fx,Fy}'
%  % \]
%\end{Defi}
%
%Coboundary functors are composed just as strong monoidal functors are, giving the following.
%
%\begin{lem}
%Coboundary categories, coboundary functors, and monoidal transformations form a $2$-category, which we denote $\mb{CobCat}$.
%\end{lem}
%
%
%\begin{prop}
%Let $(C, \sigma)$ be a coboundary category. Then there exists a strict coboundary category $(C', \sigma')$ which is equivalent to $(C, \sigma)$ in $\mb{CobCat}$.
%\end{prop}
%\begin{proof}
%Consider the underlying monoidal category of $(C, \sigma)$, which we will just write as $C$. We can find a strict monoidal category $C'$ by coherence for monoidal categories together with an equivalence, as monoidal categories, between $C$ and $C'$. By standard methods \cite{maclane-catwork}, this can be improved to an adjoint equivalence between $C$ and $C'$ in the $2$-category of monoidal categories, strong monoidal functors, and monoidal transformations. Let $F \colon  C \rightarrow C', G \colon C' \rightarrow C$ be the functors in this adjoint equivalence, and $\eta \colon 1 \Rightarrow FG$ the unit (which we note for emphasis is invertible since it the unit of an adjoint equivalence). For objects $x,y \in C'$, we define a commutor $\sigma'$ for $C'$ as the following composite.
%  % \begin{align*}
%  %   xy & \stackrel{\eta \otimes \eta}{\longrightarrow} FGxFGy \\
%  %   & \stackrel{\cong}{\longrightarrow} F(GxGy) \\
%  %   & \stackrel{F\sigma}{\longrightarrow} F(GyGx) \\
%  %   & \stackrel{\cong}{\longrightarrow}  FGyFGx \\
%  %   & \stackrel{\eta^{-1} \otimes \eta^{-1}}{\longrightarrow} yx.
%  % \end{align*}
%  % \begin{align*}
%  %   xy &\xrightarrow{\eta \otimes \eta} FGxFGy \\
%  %   &\xrightarrow{\cong} F(GxGy) \\
%  %   &\xrightarrow{F\sigma} F(GyGx) \\
%  %   &\xrightarrow{\cong} FGyFGx \\
%  %   &\xrightarrow{\eta^{-1} \otimes \eta^{-1}} yx
%  % \end{align*}
%  \[
%    xy \xrightarrow{\eta \otimes \eta} FGxFGy
%    \xrightarrow{\cong} F(GxGy)
%    \xrightarrow{F\sigma} F(GyGx)
%    \xrightarrow{\cong} FGyFGx
%    \xrightarrow{\eta^{-1} \otimes \eta^{-1}} yx
%  \]
%We then leave to the reader, if they wish, the computations to show that $\sigma'$ is a commutor for $C'$ and that $F,G$ become coboundary functors using $\sigma'$.
%\end{proof}

  % \[
  %   \resizebox{\textwidth}{!}{$m_{P_{1} + \cdots + P_{n}} \cdot \beta(m_{p_{11}}, \ldots, m_{p_{n,m_{n}}}) \cdot \beta \left( m_{P_{1}}\cdot \beta(m_{p_{11}}, \ldots, m_{p_{1,m_{1}}}), \ldots,  m_{P_{n}}\cdot \beta(m_{p_{n1}}, \ldots, m_{p_{1,m_{n}}}) \right).$}
  % \]
  % \[
  %   m_{P_{1} + \cdots + P_{n}} \cdot \beta(m_{p_{11}}, \ldots, m_{p_{n,m_{n}}}) \cdot \beta \left( m_{P_{1}}\cdot \beta(m_{p_{11}}, \ldots, m_{p_{1,m_{1}}}), \ldots,  m_{P_{n}}\cdot \beta(m_{p_{n1}}, \ldots, m_{p_{1,m_{n}}}) \right).

PS-comm:

%it is some additional structure that we briefly explain here. For a field $k$, the category $\mb{Vect}$ of vector spaces over $k$ has many nice features. Of particular interest to us are the following three structures. First, the category $\mb{Vect}$ is monoidal using the tensor product $\otimes_{k}$. Second, the set of linear maps $V \rightarrow W$ is itself a vector space which we denote $[V,W]$. Third, there is a notion of multilinear map $V_{1} \times \cdots \times V_{n} \rightarrow W$, with linear maps being the $1$-ary version. While these three structures are each useful in isolation, they are tied together by natural isomorphisms
%  \[
%    \mb{Vect}(V_{1} \otimes V_{2}, W) \cong \mb{Vect}(V_{1}, [V_{2}, W]) \cong \mb{Bilin}(V_{1} \times V_{2}, W)
%  \]
%expressing that $\otimes$ gives a closed monoidal structure which represents the multicategory of multilinear maps. Moreover, the adjunction between $\mb{Vect}$ and $\mb{Sets}$ respects all of this structure in the appropriate way. 

        % \item\label{axiom:t_equiv} QQQ Equivariance axiom - we decided this was just naturality in the end! (See Remark 11.2 in \cite{guillou_multiplicative}.) QQQ It's basically `compose, act, switch' is the same as `compose, switch, act', but can't actually see where it's used in the proof:
        %   \[
        %     \lambda_{p \cdot g, q \cdot h} \circ \mu^P\left(\id_p \cdot g; \underline{\id_q \cdot h}\right) = \mu^P\left(\id_q \cdot h; \underline{\id_p \cdot g}\right) \circ \lambda_{p,q}.
        %   \]
        % For each $p \in P(n)$, $q \in P(m)$, $g \in \Lambda(n)$, and $h \in \Lambda(m)$, the following diagram commutes:
        %   \[
        %     \xy
        %       (-20,10)*+{\mu^P\left(p;\underline{q}\right) \cdot t_{m,n}}="a";
        %       (20,10)*+{\mu^P\left(q;\underline{p}\right)}="b";
        %       (-20,-10)*+{\mu^P\left(p \cdot g; \underline{q \cdot h}\right) \cdot t_{m,n}}="c";
        %       (20,-10)*+{\mu^P\left(q \cdot h; \underline{p \cdot g}\right)}="d";
        %       %
        %       {\ar^{\lambda_{p,q}} "a" ; "b"};
        %       {\ar^{\mu^P\left(\id_q \cdot h; \underline{\id_p \cdot g}\right)} "b" ; "d"};
        %       {\ar_{\mu^P\left(\id_p \cdot g; \underline{\id_q \cdot h}\right) \cdot t_{m,n}} "a" ; "c"};
        %       {\ar_{\lambda_{p \cdot g, q \cdot h}} "c" ; "d"};
        %     \endxy
        %   \]
        
   %            \[
%                \mu^{\Lambda}\left(e_l; t_{m_1,n}, \ldots, t_{m_l,n}\right) \cdot \mu^{\Lambda}\left(t_{l,n};\underline{e_{m_1},\ldots,e_{m_l}}\right) = t_{n,M}.
%            \]
%            \acnoteil{check below is same as above, then delete above}

%            \[
%                \mu^{\Lambda}\left(t_{m,l};\underline{e_{n_1}},\ldots,\underline{e_{n_m}}\right) \cdot \mu^{\Lambda}\left(e_m;t_{n_1,l},\ldots,t_{n_m,l}\right) = t_{N,l}.
%            \]
%            \acnoteil{check below is same as above, then delete above}

     % \begin{thm}\label{pscomm}
% Let $P$ be a $\Lambda$-operad. Then the following equip $\underline{P}$ with a pseudo-commutative structure.
%     \begin{itemize}
%         \item For each pair $(m,n) \in \mathbb{N}_{+}^2$, we are given an element $t_{m,n} \in \Lambda(mn)$ such that $\pi(t_{m,n}) = \tau_{m,n}$.
%         \item For each $p \in P(n)$, $q \in P(m)$, we are given a natural isomorphism
%             \[
%                 \lambda_{p,q} \colon \mu(p;q,\ldots,q) \cdot t_{m,n} \cong \mu(q;p,\ldots,p).
%             \]
%             We write this as $\lambda_{p,q}\colon \mu(p; \underline{q}) \cdot t_{m,n} \cong \mu(q; \underline{p})$.
%     \end{itemize}
% These must satisfy the following:  
%     \begin{enumerate}
%         \item For all $n \in \mathbb{N}_+$\nomenclature[N]{$\mathbb{N}_+$}{the set of positive natural numbers},
%             \[
%                 t_{1,n} = e_n = t_{n,1}
%             \]
%              and for all $p \in P(n)$, the isomorphism $\lambda_{p, \id}\colon p \cdot e_n \cong p$ is the identity map.
%         \item For all $l, m_1, \ldots, m_l, n \in \mathbb{N}_+$, with $M = \Sigma m_i$,
%             \[
%                 \mu^{\Lambda}\left(e_l; t_{m_1,n}, \ldots, t_{m_l,n}\right) \cdot \mu^{\Lambda}\left(t_{l,n};\underline{e_{m_1},\ldots,e_{m_l}}\right) = t_{n,M}.
%             \]
%             Here $\underline{e_{m_1},\ldots,e_{m_l}}$ is the list $e_{m_{1}}, \ldots, e_{m_{l}}$ repeated $n$ times.
%         \item For all $l, m, n_1,\ldots, n_m \in \mathbb{N}_+$, with $N = \Sigma n_i$,
%             \[
%                 \mu^{\Lambda}\left(t_{m,l};\underline{e_{n_1}},\ldots,\underline{e_{n_m}}\right) \cdot \mu^{\Lambda}\left(e_m;t_{n_1,l},\ldots,t_{n_m,l}\right) = t_{N,l}.
%             \]
%             Here $\underline{e_{n_{i}}}$ indicates that each $e_{n_{i}}$ is repeated $l$ times.
%         \item For any $l, m_i, n \in \mathbb{N}_+$, with $1 \leq i \leq n$, and $p \in P(l)$, $q_i \in P(m_i)$ and $r \in P(n)$, the following diagram commutes. (Note that we maintain the convention that anything underlined indicates a list, and double underlining indicates a list of lists. Each instance should have an obvious meaning from context and the equations appearing above.)
%           \[
%             \xy
%                 (0,0)*+{\mu\left(p;\underline{\mu(q_i;\underline{r})}\right) \cdot \mu(e_l;\underline{t_{n,m_i}})\mu(t_{n,l};\underline{\underline{e_{m_i}}})}="00";
%                 (60,0)*+{\mu\left(p;\underline{\mu(q_i;\underline{r})}\right) \cdot t_{n,M}}="10";
%                 (0,-15)*+{\mu\left(p;\underline{\mu(q_i;\underline{r})\cdot t_{n,m_i}}\right) \cdot \mu(t_{n,l};\underline{e_{m_1},\ldots,e_{m_l}})}="01";
%                 (60,-20)*+{\mu\left(\mu(p;q_1,\ldots,q_n);\underline{\underline{r}}\right)\cdot t_{n,M}}="11";
%                 (0,-30)*+{\mu\left(p;\underline{\mu(r;\underline{q_i})}\right) \cdot \mu(t_{n,l};\underline{e_{m_1},\ldots,e_{m_l}})}="02";
%                 (60,-40)*+{\mu\left(\mu(p;q_1,\ldots,q_n);\underline{\underline{r}}\right)}="12";
%                 (0,-45)*+{\mu\left(\mu(p;\underline{r}) \cdot t_{n,l} ; \underline{q_1,\ldots,q_n}\right)}="03";
%                 (60,-60)*+{\mu\left(r;\underline{\mu(p;q_1,\ldots,q_n)}\right)}="13";
%                 (0,-60)*+{\mu\left(\mu(r,\underline{p});\underline{q_1,\ldots,q_n}\right)}="04";
%                 {\ar@{=} "00" ; "10"};
%                 {\ar@{=} "00" ; "01"};
%                 {\ar@{=} "10" ; "11"};
%                 {\ar_{\mu(1;\underline{\lambda_{q_i,r}}) \cdot 1} "01" ; "02"};
%                 {\ar@{=} "02" ; "03"};
%                 {\ar@{=} "04" ; "13"};
%                 {\ar_{\mu(\lambda_{p,r};1)} "03" ; "04"};
%                 {\ar^{\lambda_{\mu(p;q_1,\ldots,q_n),r}} "11" ; "12"};
%                 {\ar@{=} "12" ; "13"};
%             \endxy
%           \]
%         \item For any $l,m, n_i \in \mathbb{N}_+$, with $1 \leq i \leq m$, and $p \in P(l)$, $q \in P(m)$ and $r_i \in P(n_i)$, the following diagram commutes.
%           \[
%             \xy
%                 (0,0)*+{\mu\left(\mu(p;\underline{q}) \cdot t_{m,l} ; \underline{\underline{r_i}}\right) \cdot \mu(e_m;\underline{t_{n_i,l}})}="00";
%                 (60,0)*+{\mu\left(\mu(p;\underline{q});\underline{\underline{r_i}}\right) \cdot \mu(t_{m,l};\underline{\underline{e_{n_i}}})\mu(e_{m};\underline{t_{n_i,l}})}="10";
%                 (60,-15)*+{\mu\left(p;\underline{\mu(q;\underline{r_i})}\right) \cdot \mu(t_{m,l};\underline{\underline{e_{n_i}}})\mu(e_{m};\underline{t_{n_i,l}})}="11";
%                 (0,-20)*+{\mu\left(\mu(q;\underline{p}); \underline{r_1},\ldots,\underline{r_m}\right) \cdot \mu(e_m;\underline{t_{n_i,l}})}="01";
%                 (0,-40)*+{\mu\left(q;\underline{\underline{\mu(p;r_i)}}\right) \cdot \mu(e_m;\underline{t_{n_i,l}})}="02";
%                 (0,-60)*+{\mu\left(q;\underline{\mu(p;\underline{r_i}) \cdot t_{n_i,l}}\right)}="03";
%                 (60,-30)*+{\mu\left(p;\underline{\mu(q;r_1,\ldots,r_m)}\right) \cdot t_{N,l}}="12";
%                 (60,-45)*+{\mu\left(\mu(q;r_1,\ldots,r_m); \underline{\underline{p}}\right)}="13";
%                 (60,-60)*+{\mu\left(q;\underline{\mu(r_i;\underline{p})}\right)}="14";
%                 {\ar@{=} "00" ; "10"};
%                 {\ar@{=} "10" ; "11"};
%                 {\ar@{=} "11" ; "12"};
%                 {\ar^{\lambda_{p,\mu(q;r_1,\ldots,r_m)}} "12" ; "13"};
%                 {\ar@{=} "13" ; "14"};
%                 {\ar_{\mu(\lambda_{p,q};1) \cdot 1} "00" ; "01"};
%                 {\ar@{=} "01" ; "02"};
%                 {\ar@{=} "02" ; "03"};
%                 {\ar_{\mu(1;\underline{\lambda_{p,r_i}})} "03" ; "14"};
%             \endxy
%           \]
%     \end{enumerate}
% \end{thm}

%In the case that either $p$ or $q$ is an identity then we choose the component of $\gamma$ to be the isomorphism involving the appropriate identity element using Axiom \eqref{axiom:t_id} above.

%There are two things to note about the definition above before we continue. First, it is easy to check that
%  \[
%    t_{m,n}^{-1} \cdot \underline{\left(a, \underline{b}\right)} = \underline{\left(\underline{a},b\right)}
%  \]
%since $\pi(t_{m,n}) = \tau_{m,n}$; this ensures that $\gamma$ has the correct target. Second, the morphism above has second component the identity. This is actually forced upon us by the requirement that $\gamma$ be a modification:  in the case that $A,B$ are discrete categories, the only possible morphism is an identity, and the modification axiom then forces that statement to be true for general $A,B$ by considering the inclusion $A_{0} \times B_{0} \hookrightarrow A \times B$ where $A_{0}, B_{0}$ are the discrete categories with the same objects as $A, B$.

% \[
% \xy
% {\ar^{\scriptstyle \mu\left(1; \lambda_{q_{1}, r} t^{-1}_{n,m_{1}}, \ldots, \lambda_{q_{1}, r} t^{-1}_{n,m_{l}}\right)} (0,0)*+{\scriptstyle \mu\left(p; \mu(q_{1}; \un{r}), \ldots, \mu(q_{n}; \un{r})\right)}; (75,0)*+{\scriptstyle \mu\left(p; \mu(r; \un{q_{1}}) t^{-1}_{n,m_{1}}, \ldots, \mu(r; \un{q_{l}}) t^{-1}_{n,m_{l}} \right)} }
% \endxy
% \]
% on the first component. 


%The category $\coequ{P}{\Sigma}{\Lambda}{n}$ will necessarily be a groupoid as it is a colimit of groupoids: contractible categories are always groupoids, and both $\Lambda(n)$ and $\Sigma_{n}$ are discrete. Let $g \in \textrm{ker} \, \pi_{n}$ be any non-identity element, and let $p \in P(n)$ be any object. Then
%  \[
%    [p \cdot g, e] = [p, \pi(g)e] = [p,e],
%  \]
%but unless $p\cdot g = p$ in $P(n)$, there will be a unique isomorphism between them that will not be the identity, and hence will define a nontrivial automorphism of $[p,e]$ in  $\coequ{P}{\Sigma}{\Lambda}{n}$. The existence of such ensures that $\coequ{P}{\Sigma}{\Lambda}{n}$ is not contractible.

Braided monoidal pscomm that is wrong:

%\section{Extended Example: Braided Monoidal Categories}
%
%We conclude with a computation using \cref{def:ps-comm_operad} and \cref{thm:pscomm}. This result (\ref{braidpscomm} below) was only conjectured in \cite{HP}, but we are able to prove it quite easily with the machinery developed thus far. Our strategy is to construct a $\Lambda$-operad which is contractible together with the group elements required in \cref{def:ps-comm_operad}. Note that the symmetrized version of this operad will not be contractible, and we do not know of a proof using the structure of the symmetrized operad.
%
%\begin{thm}\label{braidpscomm}
%The $2$-monad $\underline{B}$ for braided strict monoidal categories on $\mb{Cat}$ has two pseudo-commutative structures on it, neither of which are symmetric.
%\end{thm}
%
%In order to apply our theory, the $2$-monad $\underline{B}$ must arise from a $\Lambda$-operad. The following proposition describes it as such, and can largely be found as Example 3.2 in the work of Fiedorowicz \cite{fie-br}\ngnote{I think we should also rework this using our presentations stuff}.
%
%\begin{prop}
%The $2$-monad $\underline{B}$ is the $2$-monad associated to the $B$-operad $B$ with the category $B(n)$ having objects the elements of the $n$th braid group $B_{n}$ and a unique isomorphism between any pair of objects; the action of $B_{n}$ on $B(n)$ is given by right multiplication on objects and is then uniquely determined on morphisms.
%\end{prop}
%
%\ngmpar{Yeah this paragraph should go, it doesn't really help that much. Make sure to keep the references though}The interested reader could easily verify that algebras for the $B$-operad $B$ are braided strict monoidal categories. The objects of $\underline{B}(X)$ can be identified with finite lists of objects of $X$, and morphisms are generated by the morphisms of $X$ together with new isomorphisms
%  \[
%    x_{1}, \ldots, x_{n} \stackrel{\gamma}{\longrightarrow} x_{\gamma^{-1}(1)}, \ldots, x_{\gamma^{-1}(n)}
%  \]
%where $\gamma \in B_{n}$ and the notation $\gamma^{-1}(i)$ means, as before, that we take the preimage of $i$ under the permutation $\pi(\gamma)$ associated to $\gamma$. This shows that $\underline{B}(X)$ is the free braided strict monoidal category generated by $X$ according to \cite{js}, and it is easy to verify that the $2$-monad structure on $\underline{B}$ arising from the $B$-operad structure on $B$ is the correct one to produce braided strict monoidal categories as algebras.
%
%\begin{Defi}
%A braid $\gamma \in B_{n}$ is \textit{positive} if it is an element of the submonoid of $B_{n}$ generated by the elements $\sigma_{1}, \sigma_{2}, \ldots, \sigma_{n-1}$.
%\end{Defi}
%
%\begin{Defi}
% A braid $\gamma \in B_{n}$ is \textit{minimal} if no pair of strands in $\gamma$ cross twice.
%\end{Defi}
%
%For our purposes, we are interested in braids which are both positive and minimal. A proof of the following lemma can be found in \cite[Theorem 2.6]{EM2} \ngnote{specific ref}\acnote{correct ref?}.
%
%\begin{lem}\label{pmlem}
%Let $PM_{n}$ be the subset of $B_{n}$ consisting of positive, minimal braids. Then the function sending a braid to its underlying permutation is a bijection of sets $PM_{n} \rightarrow \Sigma_{n}$.
%\end{lem}
%
%\begin{rem}\label{pmrem}
%It is worth noting that this bijection is not an isomorphism of groups, since $PM_{n}$ is not a group or even a monoid. The element $\sigma_{1} \in B_{n}$ is certainly in $PM_{n}$, but $\sigma_{1}^{2}$ is not as the first two strands cross twice. Thus we see that the product of two minimal braids is generally not minimal, but by definition the product of positive braids is positive.
%\end{rem}
%
%\begin{proof}[Proof of \cref{braidpscomm}]
%We refer to the Axioms of \cref{def:ps-comm_operad} throughout the proof. In order to use \cref{thm:pscomm} with the action operad being the braid operad $B$, we must first construct elements $t_{m,n} \in B_{mn}$ satisfying certain properties\ngnote{new axioms! check them!!} as in \cref{def:ps-comm_operad}. Using \cref{pmlem}, we define $t_{m,n}$ to be the unique positive minimal braid such that $\pi(t_{m,n}) = \tau_{m,n}$. Since $\tau_{1,n} = e_{n} = \tau_{n,1}$ in $\Sigma_{n}$ and the identity element $e_{n} \in B_{n}$ is positive and minimal, we find that $t_{1,n} = e_{n} = t_{n,1}$ in $B_{n}$, satisfying Axiom \eqref{axiom:t_id}. Thus in order to verify the remaining hypotheses, we must check two equations, each of which states that some element $t_{m,n}$ can be expressed as a product of operadic compositions of other elements.
%
%Let $l, m_{1}, \ldots, m_{l}, n$ be natural numbers, and let $N = \sum m_{i}$. We must check Axiom \eqref{axiom:t_sumL} is satisfied, i.e., that
%  \[
%    \beta(t_{n, m_{1}}, \ldots, t_{n, m_{l}}) \cdot \delta(t_{n,l}) = t_{N, l}
%  \]
%  \[
%    \mu(e_{l}; t_{n, m_{1}}, \ldots, t_{n, m_{l}}) \mu\left(t_{n,l}; \underline{e_{m_{1}}, \ldots, e_{m_{l}}}\right) = t_{N, l}
%  \]
%in $B_{lN}$. These braids have the same underlying permutations by construction, and both are positive since each operadic composition on the left is positive. The braid on the right is minimal by definition, so if we prove that the braid on the left is also minimal, they are necessarily equal. Now $\mu\left(t_{n,l}; \underline{e_{m_{1}}, \ldots, e_{m_{l}}}\right)$ is given by the braid for $t_{n,l}$ but with the first strand replaced by $m_{1}$ strands, the second strand replaced by $m_{2}$ strands, and so on for the first $l$ strands of $t_{n,l}$, and then repeating for each group of $l$ strands. In particular, since strands $i, i+l, i+2l, \ldots, i + (n-1)l$ never cross in $t_{n,l}$, none of the $m_{i}$ strands that each of these is replaced with cross. The braid $\mu(e_{l}; t_{n, m_{1}}, \ldots, t_{n, m_{l}})$ consists of the disjoint union of the braids for each $t_{n,m_{i}}$, so if two strands cross in $\mu(e_{l}; t_{n, m_{1}}, \ldots, t_{n, m_{l}})$ then they must both cross in some $t_{n,m_{i}}$. The strands in $t_{n,m_{i}}$ are those numbered from $n(m_{1} + \cdots + m_{i-1}) + 1$ to $n(m_{1} + \cdots + m_{i-1} + m_{i})$. This is a consecutive collection of $nm_{i}$ strands, and it is simple to check that these strands are precisely those connected (via the group operation in $B_{Nl}$, concatenation) to the duplicated copies of strands $i, i+l, i+2l, \ldots, i + (n-1)l$ in $t_{n,l}$. Thus if a pair of strands were to cross in
%% $\mu(e_{l}; t_{n, m_{1}}, \ldots, t_{n, m_{l}})$
%$\beta(t_{n, m_{1}}, \ldots, t_{n, m_{l}})$, that pair cannot also have crossed in
%% $\mu\left(t_{n,l}; \underline{e_{m_{1}}, \ldots, e_{m_{l}}}\right)$
%$\delta(t_{n,l})$, showing that the resulting product braid
%  \[
%    \beta(t_{n, m_{1}}, \ldots, t_{n, m_{l}}) \cdot \delta(t_{n,l})
%  \]
%  \[
%    \mu(e_{l}; t_{n, m_{1}}, \ldots, t_{n, m_{l}}) \mu\left(t_{n,l}; \underline{e_{m_{1}}, \ldots, e_{m_{l}}}\right)
%  \]
%is minimal. The calculation for Axiom \eqref{axiom:t_sumR}, showing that
%  \[
%    \delta(t_{m,l}) \cdot \beta(t_{n_{1}, l}, \ldots, t_{n_{m}, l})
%  \]
%  \[
%    \mu\left(t_{m,l}; \underline{e_{1}}, \ldots, \underline{e_{n_{m}}}\right) \mu\left(e_{m}; t_{n_{1}, l}, \ldots, t_{n_{m}, l}\right)
%  \]
%is also minimal, follows from the same argument, showing that it is equal to $t_{N, l}$ (here $N$ is the sum of the $n_{i}$, where once again $i$ ranges from 1 to $l$).
%
%\acnoteil{Where are Axioms \ref{axiom:t_diagR} and \ref{axiom:t_diagL} checked?}
%
%These calculations show, using \cref{thm:pscomm}, that the $B$-operad $B$ induces a $2$-monad which has a pseudo-commutative structure. As noted before, $B$-algebras are precisely braided strict monoidal categories. The second pseudo-commutative structure arises by using negative, minimal braids instead of positive ones, and proceeds using the same arguments. This finishes the first part of the proof of \cref{braidpscomm}.
%
%We will now show that neither of these pseudo-commutative structures is symmetric. The symmetry axiom in this case reduces to the fact that, in some category which is given as a coequalizer, the morphism with first component
%  \[
%    f\colon \mu\left(p; \underline{q}\right) \cdot t_{n,m}t_{m,n} \rightarrow \mu\left(q; \underline{p}\right) \cdot t_{m,n} \rightarrow \mu\left(p; \underline{q}\right)
%  \]
%is the identity. Now it is clear that $t_{n,m}$ is not equal to $t_{m,n}^{-1}$ in general: taking $m=n=2$, we note that $t_{2,2} = \sigma_{2}$, and this element is certainly not of order two in $B_{4}$. $B(4)$ is the category whose objects are the elements of $B_{4}$ with a unique isomorphism between any two pair of objects, and $B_{4}$ acts by multiplication on the right; this action is easily shown to be free and transitive. We recall (see \cref{coeq-lem}) that in a coequalizer of the form $\coeqb{A}{B}{G}$, a morphism $[f_{1}, f_{2}]$ equals $[g_{1}, g_{2}]$ if and only if there exists an $x \in G$ such that
%  \begin{align*}
%    f_{1} \cdot x &= g_{1}, \\
%    x^{-1} \cdot f_{2} &= g_{2}.
%  \end{align*}
%For the coequalizer in question, for $f$ to be the first component of an identity morphism would imply that $f \cdot x$ would be a genuine identity in $B(4)$ for some $x$. But $f \cdot x$ would have source $\mu\left(p; \underline{q}\right) t_{n,m}t_{m,n}x$ and target $\mu\left(p; \underline{q}\right)x$, which requires $t_{n,m}t_{m,n}$ to be the identity group element for all $n,m$. In particular, this would force $t_{2,2}$ to have order two, which as noted above does not hold in $B_{4}$, thus giving a contradiction.
%\end{proof}
%
%\begin{rem}
%The pseudo-commutativities given above are not necessarily the only ones that exist for the $B$-operad $B$, but we do not know a general strategy for producing others.
%\end{rem}
%
%\begin{example}
%\ngnoteil{Move this into coboundary section, reword this example somewhat}
%Non-Example: Cactus operad.
%
%Begin by defining $t_{2,2} = s_{2,3}$, which has underlying permutation $\pi_4(t_{2,2}) = \trans{2}{3} = \tau_{2,3}$, as required. This seems to be a sensible choice to then demonstrate that we can describe all other $t_{m,n}$ required for a pseudo-commutativity on $J$. In particular, we should be able to describe $t_{2,4}$ which would have underlying permutation $\tau_{2,4} = (2 \,\, 3 \,\, 5)(4 \,\, 7 \,\, 6)$. If the required elements $t_{m,m}$ existed for the cactus operad $J$, then we would be able to apply the axioms from \cref{def:ps-comm_operad} to the element $t_{2,4}$ in order to see how it is constructed from $t_{2,2} = s_{2,3}$.
%
%By Axiom \eqref{axiom:t_sumR} of \cref{def:ps-comm_operad} we should be able to split the element $t_{2,4}$ up as follows.
%  \begin{align*}
%    t_{2,4} &= t_{2,2+2} \\
%    &= \beta(t_{2,2},t_{2,2}) \cdot \delta_{2,2,2,2}(t_{2,2}) \\
%    &= s_{2,3} \cdot s_{5,7} \cdot \delta_{2,2,2,2}(s_{2,3}) \\
%    &= s_{2,3} \cdot s_{5,7} \cdot s_{2,6} \cdot \beta(e_2,s_{1,2},s_{1,2},e_2) \\
%    &= s_{2,3} \cdot s_{5,7} \cdot s_{2,6} \cdot s_{3,4} \cdot s_{5,6}.
%  \end{align*}
%Here we have used $\delta$ as defined for generators $s_{p,q}$ in \cref{J_aop}. This element has underlying permutation
%  \[
%      \trans{2}{3}\trans{5}{7}\trans{2}{6}\trans{3}{5}\trans{3}{4}\trans{5}{6} = \trans{2}{6}(3 \,\, 4 \,\, 7 \,\,5)
%  \]
%which is not equal to $\tau_{2,4} = (2 \,\, 3 \,\, 5)(4 \,\, 7 \,\, 6)$. Hence if $J$ were to have a pseudo-commutative structure, then it cannot arise in this way.
%\end{example}
%
%
%\begin{rem}
%I commented out the profunctors stuff, but it is still in the file.
%\end{rem}
%%\subsection{Profunctors and multicategories}
%%In this section we generalize from operads to multicategories (or colored operads). The notions of plain and symmetric multicategories are standard \cite{bd_hda3}, but in fact there is a corresponding notion of $\Lambda$-multicategory for any action operad $\Lambda$. We will give the basic definition and then show that it arises abstractly from a lifting of $\underline{\EL}$ as a $2$-monad  on $\mb{Cat}$ to a pseudomonad on $\mb{Prof}$, the bicategory of categories, profunctors, and transformations. A quick treatment of similar material but restricted to the symmetric case can be found in \cite{garner_poly}.
%%
%%\begin{Defi}\label{lambda_multicat}
%%Let $\Lambda$ be an action operad. A \emph{$\Lambda$-multicategory} $M$ consists of the following data:
%%\begin{itemize}
%%  \item a set of objects $M_{0}$;
%%  \item for any finite list $x_{1}, \ldots, x_{n}$ of objects and any object $y$, a set
%%    \[
%%      M(x_{1}, \ldots, x_{n}; y)
%%    \]
%%  of multi-arrows (or just arrows) from $x_{1}, \ldots, x_{n}$ to $y$;
%%  \item for each $\alpha \in \Lambda(n)$, an isomorphism
%%    \[
%%      -\cdot \alpha \colon M(x_{1}, \ldots, x_{n}; y) \rightarrow M\left(x_{\pi(\alpha)(1)}, \ldots, x_{\pi(\alpha)(n)}; y\right);
%%    \]
%%  \item for each object $x$, an arrow $\id_{x} \in M(x;x)$; and
%%  \item a composition function
%%  % \[
%%  % M(y_{1}, \ldots, y_{k}; z) \times M(x_{11}, \ldots, x_{1,n_{1}}; y_{1}) \times \cdots \times M(x_{k1}, \ldots, x_{k,n_{k}}; y_{k}) \rightarrow M(\underline{x}; z)
%%  % \]
%%    \[
%%      M(y_1,\ldots,y_k;z) \times \prod_{i=1}^k M(x_{i1},\ldots,x_{in_i};y_i) \rightarrow M(\underline{x};z)
%%    \]
%%  where $\underline{x} = x_{11}, \ldots, x_{1,n_{1}}, x_{21}, \ldots, x_{k,n_{k}}$, and which we write as
%%    \[
%%      (g; f_{1}, \ldots, f_{k}) \mapsto g(f_{1}, \ldots, f_{k}).
%%    \]
%%\end{itemize}
%%These data are subject to the following axioms.
%%\begin{enumerate}
%%\item $\id$ is a two-sided unit:
%%  \begin{align*}
%%    \id(f) &= f, \\
%%    f(\id,\ldots,\id) &= f.
%%  \end{align*}
%%\item Composition is associative:
%%  \[
%%    f\left( g_{1}(h_{11}, \ldots, h_{1m_{1}}), \ldots, g_{n}(h_{n1}, \ldots, h_{nm_{n}}) \right) = f(g_{1}, \ldots, g_{n})(h_{11}, \ldots, h_{nm_{n}}).
%%  \]
%%\item Composition respects the group actions:
%%% \[
%%% \begin{array}{rcl}
%%% f(g_{1} \cdot \alpha_{1}, \ldots, g_{n} \cdot \alpha_{n}) & = & f(g_{1}, \ldots, g_{n}) \cdot \mu^{\Lambda}(e; \alpha_{1}, \ldots, \alpha_{n}), \\
%%% f\cdot \alpha (g_{1}, \ldots, g_{n}) & = & f(g_{\pi^{-1}(\alpha)(1)}, \ldots, g_{\pi^{-1}(\alpha)(n)}) \cdot \mu^{\Lambda}(\alpha; e_{1}, \ldots, e_{n}).
%%% \end{array}
%%% \]
%%  \begin{align*}
%%    f(g_1 \cdot \alpha_1,\ldots, g_n \cdot \alpha_n) &= f(g_1,\ldots,g_n) \cdot \mu^{\Lambda}(e;\alpha_1,\ldots,\alpha_n), \\
%%    (f \cdot \alpha)(g_1,\ldots,g_n) &= f\left(g_{\pi^{-1}(\alpha)(1)},\ldots,g_{\pi^{-1}(\alpha)(n)}\right) \cdot \mu^{\Lambda}(\alpha;e_1,\ldots,e_n).
%%  \end{align*}
%%\end{enumerate}
%%\end{Defi}
%%
%%\begin{Defi}
%%Let $M, N$ be $\Lambda$-multicategories. A \emph{$\Lambda$-multifunctor} $F$ consists of the following data:
%%\begin{itemize}
%%\item a function $F_{0} \colon M_{0} \rightarrow N_{0}$ on sets of objects and
%%\item functions $F \colon M(x_1, \ldots, x_n; y) \rightarrow N(F_{0}(x_1), \ldots, F_{0}(x_n); F_{0}(y))$ which are $\Lambda(n)$-equivariant in that $F(f \cdot \alpha) = F(f) \cdot \alpha$.
%%\end{itemize}
%%These data are subject to the following axioms.
%%\begin{enumerate}
%%\item $F$ preserves identities: $F(\id_x) = \id_{F_{0}(x)}$.
%%\item $F$ preserves composition: $F\left( f(g_1, \ldots, g_n) \right) = F(f) \left( F(g_1), \ldots, F(g_n) \right).$
%%\end{enumerate}
%%\end{Defi}
%%
%%
%%
%%Recall that the bicategory $\mb{Prof}$ has objects categories, $1$-cells $F \colon X \srarrow Y$ profunctors from $X$ to $Y$ or equivalently functors
%%  \[
%%    F \colon Y^{\textrm{op}} \times X \rightarrow \mb{Sets},
%%  \]
%%and $2$-cells transformations $F \Rightarrow G$. Composition of profunctors is given by the coend formula
%%  \[
%%    G \circ F (z,x) = \int^{y \in Y} G(z,y) \times F(y,x)
%%  \]
%%and hence is only unital and associative up to coherent isomorphism. There exists an embedding pseudofunctor $(-)^{+} \colon  \mb{Cat} \hookrightarrow \mb{Prof}$ which is the identity on objects and sends a functor $F \colon X \rightarrow Y$ to the profunctor $F^{+}$ defined by $F^{+}(y,x) = Y(y,Fx)$.
%%
%%
%%\begin{thm}
%%The $2$-monad $\underline{\EL}$ on the $2$-category $\mb{Cat}$ lifts to a pseudomonad $\widetilde{\underline{\EL}}$ on the bicategory $\mb{Prof}$.
%%\end{thm}
%%\begin{proof}
%%On objects, we have $\widetilde{\underline{\EL}}(X) = \underline{\EL}(X)$. Let $F \colon  X \srarrow Y$ be a profunctor given by the functor $F \colon Y^{\textrm{op}} \times X \rightarrow \mb{Sets}$. We define $\widetilde{\underline{\EL}}F$ to be the functor
%%  \[
%%    ( \underline{\EL}(Y) )^{\textrm{op}} \times \underline{\EL}(X) \rightarrow \mb{Sets}
%%  \]
%%which is defined by the formulas
%%  \[
%%    \widetilde{\Lambda}F \left( [e; x_1, \ldots, x_n], [e; y_1, \ldots, y_m] \right) = \left\{
%%    \begin{array}{lr}
%%    \varnothing & \textrm{if $n \neq m$}, \\
%%    \coprod_{g \in \Lambda(n)} \prod_{i=1}^{n} F\left(y_i, x_{\pi(g)(i)}\right) & \textrm{if $n = m$.}
%%    \end{array}
%%    \right.
%%  \]
%%For a functor $G \colon X \rightarrow Y$, it is easy to check that
%%  \[
%%    \widetilde{\underline{\EL}}\left(G^{+}\right) = \left( \underline{\EL} G \right)^{+}
%%  \]
%%using \cref{hom-set-lemma}. The same formulas define the action of  $\widetilde{\underline{\EL}}$ on $2$-cells as well. The multiplication and unit of $\widetilde{\underline{\EL}}$ are just $\mu^{+}$ and $\eta^{+}$, where $\mu, \eta$ are the multiplication and unit, respectively, of $\underline{\EL}$. The remainder of the pseudomonad data comes from the pseudofunctoriality of $(-)^{+}$, and the axioms follow from the $2$-monad axioms for $\underline{\EL}$ and the pseudofunctor axioms for $(-)^{+}$.
%%\end{proof}
%%
%%\begin{rem}
%%Since $\mb{Prof}$ is essentially the Kleisli bicategory for the free cocompletion pseudomonad, this lift corresponds to a pseudo-distributive law between $\underline{\EL}$ and the free cocompletion pseudomonad, but we do not pursue this perspective here.
%%\end{rem}
%%
%%Given a bicategory $B$ and a pseudomonad $T$ on $B$, we can form the Kleisli bicategory of $T$, $\mb{Kl}_{T}$. It has the same objects as $B$, but a $1$-cell from $a$ to $b$ in  $\mb{Kl}_{T}$ is a $1$-cell $f \colon a \rightarrow Tb$ in $B$. In the case $B = \mb{Prof}, T = \widetilde{\underline{\EL}}$, the objects of $\mb{Kl}_{T}$ are categories, the $1$-cells $X \srarrow Y$ are profunctors from $X$ to $\underline{\EL}(Y$), or alternatively a functor $(\underline{\EL}(Y))^{op} \times X \rightarrow \mb{Sets}$, and the $2$-cells are natural transformation between such.
%%
%%We now recall some standard definitions \cite{ben-bicats}.
%%
%%\begin{Defi}
%%Let $B$ be a bicategory. A \emph{monad} $(x,t,\mu,\eta)$ in $B$ consists of the following data:
%%\begin{itemize}
%%  \item an object $x$,
%%  \item a $1$-cell $t \colon  x \rightarrow x$,
%%  \item a $2$-cell $\mu \colon t^{2} \Rightarrow t$, and
%%  \item a $2$-cell $\eta \colon \id_x \Rightarrow t$.
%%\end{itemize}
%%These data are subject to the following axioms.
%%  \[
%%    \xy
%%      (0,0)*+{(t \circ t) \circ t} ="1";
%%      (25,0)*+{t \circ (t \circ t)} ="2";
%%      (40,-12)*+{t \circ t} ="3";
%%      (0,-24)*+{t \circ t} ="4";
%%      (40,-24)*+{t} ="5";
%%      {\ar^{\cong} "1";"2" };
%%      {\ar^{t * \mu} "2";"3" };
%%      {\ar^{\mu} "3";"5" };
%%      {\ar_{\mu * t} "1";"4" };
%%      {\ar_{\mu} "4";"5" };
%%      (60,0)*+{\id_{x} \circ t} ="11";
%%      (90,0)*+{t \circ t} ="12";
%%      (90,-10)*+{t} ="13";
%%      {\ar^{\eta * t} "11";"12" };
%%      {\ar^{\mu} "12";"13" };
%%      {\ar_{\cong} "11";"13" };
%%      (60,-16)*+{t \circ \id_{x}} ="11";
%%      (90,-16)*+{t \circ t} ="12";
%%      (90,-26)*+{t} ="13";
%%      {\ar^{t * \eta} "11";"12" };
%%      {\ar^{\mu} "12";"13" };
%%      {\ar_{\cong} "11";"13" };
%%    \endxy
%%  \]
%%\end{Defi}
%%
%%We have already defined monad maps in the particular case that $B = \textbf{Cat}$ (see \cref{defi:monad_map}), but we now recall a more general definition.
%%\begin{Defi}
%%Let $(x,t,\mu,\eta), (x',t',\mu',\eta')$ be monads in $B$. An \emph{oplax monad map} $(F, \alpha)$ from $t$ to $t'$ consists of the following data:
%%\begin{itemize}
%%\item a $1$-cell $F \colon x \rightarrow x'$ and
%%\item a $2$-cell $\alpha \colon F \circ t \Rightarrow t' \circ F$.
%%\end{itemize}
%%These data are subject to the following axioms, in which we suppress the constraints of the bicategory $B$.
%%  \[
%%    \xy
%%      (0,0)*+{Ft^{2}} ="1";
%%      (25,0)*+{t'Ft} ="2";
%%      (40,-12)*+{t'^{2} F} ="3";
%%      (0,-24)*+{Ft} ="4";
%%      (40,-24)*+{t'F} ="5";
%%      {\ar^{\alpha * t} "1";"2" };
%%      {\ar^{t' * \alpha} "2";"3" };
%%      {\ar^{\mu' * F} "3";"5" };
%%      {\ar_{F * \mu} "1";"4" };
%%      {\ar_{\alpha} "4";"5" };
%%      (60,0)*+{F} ="11";
%%      (90,0)*+{Ft} ="12";
%%      (90,-10)*+{t'F} ="13";
%%      {\ar^{F*\eta} "11";"12" };
%%      {\ar^{\alpha} "12";"13" };
%%      {\ar_{\eta'*F} "11";"13" };
%%    \endxy
%%  \]
%%\end{Defi}
%%
%%\begin{Defi}
%%Let $(F,\alpha), (F', \alpha')$ be oplax monad maps from $t$ to $t'$. A \emph{transformation of monad maps} $\Gamma \colon (F, \alpha) \Rightarrow (F', \alpha')$ is a $2$-cell $\Gamma \colon F \Rightarrow F'$ such that
%%  \[
%%    \xy
%%      (0,0)*+{Ft} ="1";
%%      (40,0)*+{t'F} ="2";
%%      (40,-12)*+{t'F'} ="3";
%%      (0,-12)*+{F't} ="4";
%%      {\ar^{\alpha } "1";"2" };
%%      {\ar^{t' * \Gamma} "2";"3" };
%%      {\ar_{\Gamma * t} "1";"4" };
%%      {\ar_{\alpha'} "4";"3" };
%%    \endxy
%%  \]
%%commutes.
%%\end{Defi}
%%
%%It is simple to check that monads, oplax monad maps, and transformations of monad maps form a bicategory.
%%
%%
%%\begin{thm}
%%The category $\Lambda\mbox{-}\mb{Multicat}$ of
%%\begin{itemize}
%%\item $\Lambda$-multicategories and
%%\item $\Lambda$-multifunctors
%%\end{itemize}
%%and the bicategory $\mb{Mnd}_{d}(\mb{Kl}_{\widetilde{\underline{\EL}}})$ of
%%\begin{itemize}
%%\item monads on sets (viewed as discrete categories) in $\mb{Kl}_{\widetilde{\underline{\EL}}}$,
%%\item oplax monad maps $(F, \alpha)$ between them which are isomorphic to one of the form $(f^{+}, \alpha)$ for $f \colon S \rightarrow T$ for some function of the underlying sets, and
%%\item transformations of monad maps
%%\end{itemize}
%%are biequivalent.
%%
%%Under this biequivalence, the category of $\Lambda$-operads is equivalent to the bicategory of monads on the terminal set in $\mb{Kl}_{\widetilde{\underline{\EL}}}$.
%%\end{thm}
%%\begin{proof}
%%First, we note that $\mb{Mnd}_{d}(\mb{Kl}_{\widetilde{\underline{\EL}}})$ is a locally essentially discrete bicategory, by which we mean the hom-categories are all equivalent to discrete categories. We will show there is a unique isomorphism or no $2$-cell at all between oplax monad maps of the form $(f^{+}, \alpha)$, from which the claim follows in general. A $2$-cell between such has as its data a natural transformation $\gamma \colon f^{+} \Rightarrow g^{+}$ which has components
%%  \[
%%    \gamma_{[e; t_1, \ldots, t_n], s} \colon f^{+}([e; t_1, \ldots, t_n], s) \rightarrow g^{+}([e; t_1, \ldots, t_n], s).
%%  \]
%%Both of these sets are empty unless $n=1$, and then the source is nonempty when $f(s) = t$ and the target is nonempty when $g(s)=t$; when nonempty, both of these sets are singletons. If both are nonempty for some $s$, then the functions $f,g$ agree on $s$. Assume the target is nonempty for some $([e;t], s)$ but that the source is empty, in other words that $g(s)=t$ but $f(s) \neq t$. Then consider $\gamma_{[e;f(s)], s}$. Its source is $f^{+}([e;f(s)], s)$ which is nonempty by construction, but its target is $g^{+}([e;f(s)], s)$. We know that $g(s) = t \neq f(s)$, so $g^{+}([e;f(s)], s)$ must be empty, giving a map from a nonempty set to an empty one, a contradiction. Thus there is a at most one $2$-cell from an oplax monad map $(f^{+}, \alpha)$ to another $(g^{+}, \beta)$, such a map can only exist if $f = g$, and if it does exist then it is invertible. Thus the hom-categories of $\mb{Mnd}_{d}(\mb{Kl}_{\widetilde{\underline{\EL}}})$ are essentially discrete, and this bicategory is equivalent to a category.
%%
%%We begin by describing an object of $\mb{Mnd}_{d}(\mb{Kl}_{\widetilde{\underline{\EL}}})$ which is a monad in $\mb{Kl}_{\widetilde{\underline{\EL}}}$ whose underlying category is a set $S$. A $1$-cell $M \colon S \srarrow S$ is then a functor $(\underline{\EL}(S)^{op} \times S \rightarrow \mb{Sets}$ which amounts to sets $M(s_1, \ldots, s_n; s)$ for $s_1, \ldots, s_n, s \in S$ together with a right action of $\Lambda(n)$ as in \ref{lambda_multicat}. A $2$-cell $1_{S} \Rightarrow M$ consists of a $\Lambda(1)$-equivariant function $\Lambda(1) \rightarrow M(s;s)$ for each $s \in S$, in other words an element $\id_{s} \in M(s;s)$. A $2$-cell $M \circ M \Rightarrow M$ then consists of a multicategorical composition function, as in \ref{lambda_multicat}, with appropriate equivariance built in by the coend used for composition of profunctors. Associativity and unit conditions are then seen to be the same as for $\Lambda$-multicategories.
%%
%%By definition, an oplax monad map $(f^{+}, \alpha) \colon  (S,M) \rightarrow (S', M')$ consists of a function $f \colon S \rightarrow S'$ and a transformation $\alpha \colon M \circ f^{+} \Rightarrow f^{+} \circ M'$ satisfying two axioms. The transformation $\alpha$ amounts to giving $\Lambda(n)$-equivariant functions
%%  \[
%%    M(s_1, \ldots, s_n; s) \rightarrow M'\left(f(s_1), \ldots, f(s_n); f(s)\right),
%%  \]
%%and the two axioms correspond to the unit and composition axioms for a $\Lambda$-multifunctor.
%%
%%These descriptions give the action on objects and morphisms of a pseudofunctor $\Lambda\mbox{-}\mb{Multicat} \rightarrow \mb{Mnd}_{d}(\mb{Kl}_{\widetilde{\underline{\EL}}})$ with local contractibility providing the pseudofunctoriality constraints as well as showing that the axioms for a pseudofunctor hold. It is also clear that this pseudofunctor is biessentially surjective and locally essentially surjective, so it is a biequivalence once again using local contractibility.
%%
%%The final claim is then an immediate consequence of the definitions of $\Lambda$-operad and $\Lambda$-multicategory.
%%\end{proof}



Pretty pictures
%   \[
%       \begin{bmatrix}
%         \textcolor{myorange}{(a_1,b_1)} & \textcolor{myorange}{(a_1,b_2)} & \textcolor{myorange}{(a_1,b_3)} \\
%         \textcolor{mygreen}{(a_2,b_1)} & \textcolor{mygreen}{(a_2,b_2)} & \textcolor{mygreen}{(a_2,b_3)} \\
%         \textcolor{myviolet}{(a_3,b_1)} & \textcolor{myviolet}{(a_3,b_2)} & \textcolor{myviolet}{(a_3,b_3)} \\
%         \textcolor{myred}{(a_4,b_1)} & \textcolor{myred}{(a_4,b_2)} & \textcolor{myred}{(a_4,b_3)} \\
%         \textcolor{myblue}{(a_5,b_1)} & \textcolor{myblue}{(a_5,b_2)} & \textcolor{myblue}{(a_5,b_3)}                               
%       \end{bmatrix}^T
%       =
%       \begin{bmatrix}
%         \textcolor{myorange}{(a_1,b_1)} & \textcolor{mygreen}{(a_2,b_1)} & \textcolor{myviolet}{(a_3,b_1)} & \textcolor{myred}{(a_4,b_1)} & \textcolor{myblue}{(a_5,b_1)} \\ 
%         \textcolor{myorange}{(a_1,b_2)} & \textcolor{mygreen}{(a_2,b_2)} & \textcolor{myviolet}{(a_3,b_2)} & \textcolor{myred}{(a_4,b_2)} & \textcolor{myblue}{(a_5,b_2)} \\  
%         \textcolor{myorange}{(a_1,b_3)} & \textcolor{mygreen}{(a_2,b_3)} & \textcolor{myviolet}{(a_3,b_3)} & \textcolor{myred}{(a_4,b_3)} & \textcolor{myblue}{(a_5,b_3)}
%       \end{bmatrix}
%   \]
% \begin{figure}[h]
% \centering
% \includegraphics[width=0.8\columnwidth]{4-monoidal_structures/t-5-3.pdf}
% \end{figure}
% We now give sufficient conditions for equipping the $2$-monad $\underline{P}$ associated to a $\Lambda$-operad $P$ with a pseudo-commutative structure. Let $\mathbb{N}_{+}$ denote the set of positive natural numbers.


