\documentclass[11pt]{article}

\usepackage{url}
\usepackage{amssymb}
\usepackage{pdfsync}
\usepackage{bbm}
\usepackage{natbib}
\usepackage[all]{xy}
\xyoption{tips}\SelectTips{cm}{10}
\usepackage{amsmath}
\usepackage{amsthm}
\usepackage{hyperref}
\usepackage{lmodern}
\usepackage{color}
\usepackage{tikz}
\usepackage{mathtools}
\usepackage{pdflscape}

\title{Cauchy completeness of computadic $P$-algebras}

\date{}
\def\newblock{\hskip .11em plus .33em minus .07em}

\theoremstyle{definition}
\newtheorem{definition}{Definition}[section]
\newtheorem{example}[definition]{Example}
\newtheorem{construction}[definition]{Construction}
\newtheorem{remark}[definition]{Remark}

\theoremstyle{plain}
\newtheorem{proposition}[definition]{Proposition}
\newtheorem{theorem}[definition]{Theorem}
\newtheorem{lemma}[definition]{Lemma}
\newtheorem{corollary}[definition]{Corollary}
\newtheorem{conjecture}[definition]{Conjecture}


\newcommand{\m}[1]{\mathcal{#1}}
\newcommand{\amalgc}{\amalg{C_0^n}}
\newcommand{\alphast}{\alpha_{st}}
\newcommand{\lp}{\left(}
\newcommand{\rp}{\right)}
\newcommand{\lb}{\left[}
\newcommand{\rb}{\right]}
\newcommand{\ps}[1]{\text{Ps-}{#1}\text{-}\mathbf{Alg}}
\newcommand{\st}[1]{{#1}\text{-}\mathbf{Alg}_s}
\newcommand{\alg}[1]{{#1}\text{-}\mathbf{Alg}}
\newcommand{\mbn[1]}{\mathbb{#1}_n}
\newcommand{\mbg}{\mathbb{G}_n}
\newcommand{\ngset}{n\text{-}\mathbf{GSet}}
\newcommand{\cptd}{\text{-}\mathbf{Cptd}}
\newcommand{\gset}{\text{-}\mathbf{GSet}}
\newcommand{\sli}[2]{{#2}^{(#1)}}
\newcommand{\gray}{\mathbf{Gray}\text{-}\mathbf{Cat}}
\newcommand{\tricat}{\mathbf{Tricat}}
\newcommand{\bicats}{\mathbf{Bicat}_s\text{-}\mathbf{Cat}}
\newcommand{\mbf}[1]{\mathbf{#1}}
\newcommand{\fnc}{F_{n+1}}

\makeatletter
\def\slashedarrowfill@#1#2#3#4#5{%
  $\m@th\thickmuskip0mu\medmuskip\thickmuskip\thinmuskip\thickmuskip
   \relax#5#1\mkern-7mu%
   \cleaders\hbox{$#5\mkern-2mu#2\mkern-2mu$}\hfill
   \mathclap{#3}\mathclap{#2}%
   \cleaders\hbox{$#5\mkern-2mu#2\mkern-2mu$}\hfill
   \mkern-7mu#4$%
}
\def\rightslashedarrowfill@{%
  \slashedarrowfill@\relbar\relbar\mapstochar\rightarrow}
\newcommand\xslashedrightarrow[2][]{%
  \ext@arrow 0055{\rightslashedarrowfill@}{#1}{#2}}
\makeatother

\def\slashedrightarrow{\xslashedrightarrow{}}
\begin{document}
\begin{theorem}
There is a functor $C \colon \mathbf{AOp} \rightarrow \mathbf{CSGrp}$.
\end{theorem}
\begin{proof}
    Let $\Lambda$ be an action operad. Put $C(\Lambda)(n) = \Lambda_{n+1}$. We will show that this collection of groups constitutes a crossed simplicial group by following Proposition 1.6 of \cite{FL91}. The group homomorphisms into the symmetric groups are given by $\pi$ of the action operad. We define $s_i \colon \Lambda_{n} \rightarrow \Lambda_{n+1}$ by
        \[
            s_i(g) = \mu(g; e_1, \ldots, e_1, e_2, e_1, \ldots, e_1),
        \]
    where $e_2$ is in the $\pi(g)^{-1}(i)^{th}$ input. Similarly for $d_i \colon \Lambda_{n} \rightarrow \Lambda_{n-1}$, using $e_0$ rather than $e_2$.
    
    We will describe how to think of the elements $s_i(g)$ and $d_i(g)$ in terms of the diagrams that represent their permutations under $\pi$. For example, the permutation $(0 \, 2)(1 \, 3) \in \Sigma_{3}$ can be drawn as follows.
        \[
            \xy
                {\ar@{-} (0,0) ; (10,-10)};
                {\ar@{-} (5,0) ; (15,-10)};
                {\ar@{-} (10,0) ; (0,-10)};
                {\ar@{-} (15,0) ; (5,-10)};
            \endxy
        \]
    The diagram representing the permutation $s_i(g)$ is given by drawing the permutation for $g$ and adding a new string running parallel and to the right of the $i^{th}$ output string. E.g., $s_1((0 \, 2)(1 \, 3)) = (0 \, 3 \, 1 \, 4 \, 2)$ is represented by the following diagram.
        \[
             \xy
                {\ar@{-} (0,0) ; (10,-10)};
                {\ar@{-} (5,0) ; (15,-10)};
                {\ar@{-} (10,0) ; (0,-10)};
                {\ar@{-} (15,0) ; (5,-10)};
                %
                {\ar@{|->}^{s_1} (17.5,-5) ; (27.5,-5)};
                %
                {\ar@{-} (30,0) ; (40,-10)};
                {\ar@{-} (35,0) ; (45,-10)};
                {\ar@{-} (40,0) ; (30,-10)};
                {\ar@{-} (45,0) ; (35,-10)};
                {\ar@{--} (47.5,0) ; (37.5,-10)};
            \endxy
        \]
    Similarly the diagram representing the permutation $d_i(g)$ is given by drawing the permutation $g$ and deleting the $i^{th}$ output string. E.g., $d_2((0 \, 2)(1 \, 3)) = (0 \, 2 \, 1)$ is represented by the following diagram.
        \[
             \xy
                {\ar@{-} (0,0) ; (10,-10)};
                {\ar@{-} (5,0) ; (15,-10)};
                {\ar@{-} (10,0) ; (0,-10)};
                {\ar@{-} (15,0) ; (5,-10)};
                %
                {\ar@{|->}^{d_2} (17.5,-5) ; (27.5,-5)};
                %
                {\ar@{.} (30,0) ; (40,-10)};
                {\ar@{-} (35,0) ; (45,-10)};
                {\ar@{-} (40,0) ; (30,-10)};
                {\ar@{-} (45,0) ; (35,-10)};
            \endxy
        \]
    It is then simple to check that the simplicial identities hold for these maps. For example when $i < j$ we require that $d_i \cdot d_j = d_{j-1} \cdot d_i$. Following the method above we see that $d_i(d_j(g))$ is obtained by first writing out the diagram representing $g$, deleting the $j^{th}$ output string, and renumbering the strings above this. We then follow this by deleting the $i^{th}$ output string and renumbering the strings above this. If instead we consider $d_{j-1}(d_i(g))$ we see that the permutation is obtained by first writing out the diagram representing $g$, deleting the ${i}^{th}$ output string, and renumbering the strings above this. Since we have renumbered the strings above the $i^{th}$ output, the removal the ${j-1}^{th}$ output string in the resulting diagram is the same as removing the $j^{th}$ output string as we did before. The identity follows.
    
    We now require that these maps satisfy two further conditions as detailed in \cite{FL91}. The first is that $s_i (gh) = s_i(g) s_{\pi(g)^{-1}(i)}(h)$ and similarly for the $d_i$. Now
        \[
            s_i(gh) = \mu(gh; e_1, \ldots, e_1, e_2, e_1, \ldots, e_1)
        \]
    with the $e_2$ in the $\pi(h^{-1}g^{-1})(i)^{th}$ position. Compare this with
        \[
            s_i(g)s_{\pi(g)^{-1}(i)}(h) = \mu(g; e_1, \ldots, e_1, e_2, e_1, \ldots, e_2) \mu(h; e_1, \ldots, e_1, e_2, e_1, \ldots, e_1)
        \]
    where the first composite has $e_2$ in the $\pi(g)^{-1}(i)^{th}$ position and the second has $e_2$ in the $\pi(h)^{-1}(\pi(g)^{-1}(i))^{th}$ position. The action operad axiom tells us that we can combine the two composites, where the element from the left composite that is multiplied with $e_2$ in the right composite is that which is in the $\pi(h)(\pi(h^{-1}g^{-1})(i))^{th}$ position, i.e., the $e_2$ from the left composite. Hence the identity is satisfied. Since it did not matter what the elements were we can repeat the argument for the $d_i$ with $e_0$ in place of $e_2$.
    
    The final condition of the proposition is that the following two diagrams commute.
        \[
            \xymatrix{
                [n + 1] \ar[rr]^{\sigma_{\pi(g)^{-1}(i)}} \ar[d]_{\pi(s_i(g))} & & [n] \ar[d]^{\pi(g)} & & [n - 1] \ar[rr]^{\delta_{\pi(g)^{-1}(i)}} \ar[d]_{\pi(d_i(g))} & & [n] \ar[d]^{\pi(g)} \\
                [n+1] \ar[rr]_{\sigma_i} && [n] & & [n-1] \ar[rr]_{\delta_i} & & [n]
            }
        \]
    This can easily be seen as a consequence of how $s_i(g)$ and $d_i(g)$ are described in terms of the diagrams that represent their underlying permutations. Following the bottom path of the first diagram, the elements $1$ to $g^{-1}(i)$ and $g^{-1}(i)+2$ to $n+1$ are permuted according to $g$, while $g^{-1}(i) + 1$ is sent to $i +1$. This is followed by every $k \leq i$ being mapped to $k$, $i +1$ also being mapped to $i$, and every $k \geq i + 1$ being mapped to $k-1$. The top path of the diagram maps each $k \leq g^{-1}(i)$ to $k$, maps $g^{-1}(i)+1$ to $g^{-1}(i)$ as well, and every $k \geq g^{-1}(i) + 2$ to $k-1$. This is followed by permuting the elements according to $g$. A similar argument shows that the second diagram commutes as well.
    
    On morphisms we simply send a map $F \colon \Lambda \rightarrow \Gamma$ of action operads to the collection of morphisms $F_n \colon \Lambda_n \rightarrow \Gamma_n$. That these interact appropriately with the face and degeneracy maps follows immediately from fact that $F$ is a map of operads. To be a map of crossed simplicial groups we also require that $\pi^{\Gamma} \cdot F_n = \pi^{\Lambda}$ for all $n \in \mathbb{N}$ but this is simply part of the requirement of $F$ being a map of action operads.
\end{proof}
\bibliographystyle{alpha}
\bibliography{biblio}
\end{document} 