

We conclude with a computation using Theorem \ref{pscomm}.  This result was only conjectured in \cite{HP}, but we are able to prove it quite easily with the machinery developed thusfar.

\begin{thm}\label{braidpscomm}
The 2-monad $\underline{Br}$ for strict braided monoidal categories on $\mb{Cat}$ has two pseudo-commutative structures on it, neither of which are symmetric.
\end{thm}

%\begin{rem}
%We believe the pseudo-commutative structures we will construct for the proof of Theorem \ref{braidpscomm} are in fact the only pseudo-commutative structures, but have not verified this.
%\end{rem}

\begin{Defi}
A braid $\gamma \in B_{n}$ is \textit{positive} if it is an element of the submonoid of $B_{n}$ generated by the elements $\sigma_{1}, \sigma_{2}, \ldots, \sigma_{n-1}$.
\end{Defi}

\begin{Defi}
 A braid $\gamma \in B_{n}$ is \textit{minimal} if no pair of strands in $\gamma$ cross twice.
\end{Defi}

For our purposes, we are interested in braids which are both positive and minimal.  A proof of the following lemma can be found in \cite{EM2}.

\begin{lem}\label{pmlem}
Let $PM_{n}$ be the subset of $B_{n}$ consisting of positive, minimal braids.  Then the function sending a braid to its underlying permutation is a bijection of sets $PM_{n} \rightarrow \Sigma_{n}$.
\end{lem}

\begin{rem}\label{pmrem}
It is worth noting that this bijection is not an isomorphism of groups, since $PM_{n}$ is not a group or even a monoid.  The element $\sigma_{1} \in B_{n}$ is certainly in $PM_{n}$, but $\sigma_{1}^{2}$ is not as the first two strands cross twice.  Thus we see that the product of two minimal braids is generally not minimal, but by definition the product of positive braids is positive.
\end{rem}

\begin{proof}[Proof of Theorem \ref{braidpscomm}]
In order to use Theorem \ref{pscomm} with the operad of structure groups being the braid operad $\mb{B} = \{B_{n} \}$, we must first construct elements $t_{m,n} \in B_{mn}$ satisfying certain properties.  Using Lemma \ref{pmlem}, we define $t_{m,n}$ to be the unique positive braid such that $\pi(t_{m,n}) = \tau_{m,n}$.  Since $\tau_{1,n} = e_{n} = \tau_{n,1}$ in $\Sigma_{n}$ and the identity element $e_{n} \in B_{n}$ is positive and minimal, we have that $t_{1,n} = e_{n} = t_{n,1}$ in $B_{n}$.  Thus in order to verify the remaining hypotheses, we must check two equations, each of which states that some element $t_{m,n}$ can be expressed as a product of operadic compositions of other elements.

Let $l, m_{1}, \ldots, m_{l}, n$ be natural numbers, and let $M = \sum m_{i}$.  We must check that
\[
\mu(e_{l}; t_{n, m_{1}}, \ldots, t_{n, m_{l}}) \mu(t_{n,l}; \underline{e_{m_{1}}, \ldots, e_{m_{l}}}) = t_{N, l}
\]
in $B_{lN}$.  These braids have the same underlying permutations by construction, and both are positive since each operadic composition on the left is positive.  The braid on the right is minimal by definition, so if we prove that the braid on the left is also minimal, they are necessarily equal.  Now $\mu(t_{n,l}; \underline{e_{m_{1}}, \ldots, e_{m_{l}}})$ is given by the braid for $t_{n,l}$ but with the first strand replaced by $m_{1}$ strands, the second strand replaced by $m_{2}$ strands, and so on for the first $l$ strands of $t_{n,l}$, and then repeating for each group of $l$ strands.  In particular, since strands $i, i+l, i+2l, \ldots, i + (n-1)l$ never cross in $t_{n,l}$, none of the $m_{i}$ strands that each of these is replaced with cross.  The braid $\mu(e_{l}; t_{n, m_{1}}, \ldots, t_{n, m_{l}})$ consists of the disjoint union of the braids for each $t_{n,m_{i}}$, so if two strands cross in $\mu(e_{l}; t_{n, m_{1}}, \ldots, t_{n, m_{l}})$ then they must both cross in some $t_{n,m_{i}}$.  The strands in $t_{n,m_{i}}$ are those numbered from $n(m_{1} + \cdots + m_{i-1}) + 1$ to $n(m_{1} + \cdots + m_{i-1} + m_{i})$.  This is a consecutive collection of $nm_{i}$ strands, and it is simple to check that these strands are precisely those connected (via the group operation in $B_{Nl}$, concatenation) to the duplicated copies of strands $i, i+l, i+2l, \ldots, i + (n-1)l$ in $t_{n,l}$.  Thus if a pair of strands were to cross in $\mu(e_{l}; t_{n, m_{1}}, \ldots, t_{n, m_{l}})$, that pair cannot also have crossed in $\mu(t_{n,l}; \underline{e_{m_{1}}, \ldots, e_{m_{l}}})$, showing that the resulting product braid
\[
\mu(e_{l}; t_{n, m_{1}}, \ldots, t_{n, m_{l}}) \mu(t_{n,l}; \underline{e_{m_{1}}, \ldots, e_{m_{l}}})
\]
is minimal.  The calculation showing that
\[
\mu(t_{m,l}; \underline{e_{1}}, \ldots, \underline{e_{n_{m}}}) \mu(e_{m}; t_{n_{1}, l}, \ldots, t_{n_{m}, l})
\]
is also minimal follows from the same argument, showing that it is equal to $t_{N, l}$ (here $N$ is the sum of the $n_{i}$, where once again $i$ ranges from 1 to $l$).

Now strict braided monoidal categories are the algebras for a contractible $\mb{B}$-operad $Br$ following Fiedorowicz \cite{fie-br}, therefore the construction above gives a pseudo-commutative structure.  The second pseudo-commutative structure arises by using negative, minimal braids instead of positive ones, and proceeds using the same arguments.

We will now show that neither of these pseudo-commutative structures is symmetric.  The symmetry axiom in this case reduces to the fact that, in some category which is given as a coequalizer, the morphism with first component
\[
f:\mu(p; \underline{q}) \cdot t_{n,m}t_{m,n} \rightarrow \mu(q; \underline{p}) \cdot t_{m,n} \rightarrow \mu(p; \underline{q})
\]
is the identity.  Now it is clear that that $t_{n,m}$ is not equal to $t_{m,n}^{-1}$ in general: taking $m=n=2$, we note that $t_{2,2} = \sigma_{2}$, and this element is certainly not of order two in $B_{4}$.  $Br(4)$ is the category $EB_{4}$ (the category whose elements are the elements of $B_{4}$ with a unique isomorphism between any two pair of objects), and $B_{4}$ acts by multiplication on the right; this action is clearly free and transitive.  We recall (see Lemma \ref{coeq-lem}) that in a coequalizer of the form $A \times_{G} B$, we have that a morphism $[f_{1}, f_{2}]$ equals $[g_{1}, g_{2}]$ if and only if there exists an $x \in G$ such that
\[
\begin{array}{rcl}
f_{1} \cdot x & = & g_{1}, \\
x^{-1} \cdot f_{2} & = & g_{2}.
\end{array}
\]
For the coequalizer in question, for $f$ to be the first component of an identity morphism would imply that $f \cdot x$ would be a genuine identity in $EB_{4}$ for some $x$.  But $f \cdot x$ would have source $\mu(p; \underline{q}) t_{n,m}t_{m,n}x$ and target $\mu(p; \underline{q})x$, which requires $t_{n,m}t_{m,n}$ to be the identity group element for all $n,m$, a contradiction.
\end{proof}

\begin{rem}
The pseudo-commutativities given above are not necessarily the only ones that exist for the $\mb{B}$-operad $Br$, but we do not know a general strategy for producing others.
\end{rem}

%One can see from the two axioms concerning $t_{n,M}$ and $t_{N,l}$ that any given $t_{n,m}$ is determined, via the group structure, operadic composition, and these equations together with the requirement that if $n$ or $m$ is 1 then the corresponding $t_{n,m}$ is the identity element, by the $t_{p,q}$ for $p<n, q<m$.  So in order to use Theorem \ref{...}, one must only choose $t_{2,2}$, and the rest will be determined.  As in our proof above, we must still check that the axioms are satisfied, though, as these observations only amount to saying that a pseudo-commutative structure constructed in this fashion is unique once $t_{2,2}$ is chosen, not that an arbitrary choice of $t_{2,2}$ will give such a structure.  Preliminary calculations, though, show that other choices of $t_{2,2}$ might also produce pseudo-commutativities.  The authors have briefly investigated $\sigma_{2}^{3}$ and $\sigma_{2} \sigma_{1}^{2}$, and while low-dimensional calculations have not given any insight into why these elements might fail to produce pseudo-commutativities, they have also not shown a general strategy of proof for the whole structure as in the proof of \ref{braidpscomm} above.
