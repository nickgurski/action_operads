Operads were defined by May \cite{maygeom} in the early 70's to provide a convenient tool to approach problems in algebraic topology, notably the question of when a space $X$ admits an $n$-fold delooping $Y$ so that $X \simeq \Omega^{n}Y$.  An operad, like an algebraic theory \cite{lawvere-thesis}, is something like a presentation for a monad or algebraic structure.  The theory of operads has seen great success, and we would like to highlight two reasons.  First, operads can be defined in any suitable symmetric monoidal category, so that there are operads of topological spaces, of chain complexes, of simplicial sets, and of categories, to name a few examples.  Moreover, symmetric (lax) monoidal functors carry operads to operads, so we can use operads in one category to understand objects in another via transport by such a functor.  Second, operads in a fixed category are highly flexible tools.  In particular, the categories listed above all have some inherent notion of ``homotopy equivalence'' which is weaker than that of isomorphism, so we can study operads which are equivalent but not isomorphic.  These tend to have algebras which have similar features in an ``up-to-homotopy'' sense but very different combinatorial or geometric properties arising from the fact that different objects make up these equivalent but not isomorphic operads.

Operads in the category $\mb{Cat}$ of small categories have a unique flavor arising from the fact that $\mb{Cat}$ is not just a category but a 2-category.  These 2-categorical aspects have not been widely treated in the literature, although a few examples can be found.  Lack \cite{lack-cod} mentions braided $\mb{Cat}$-operads (the reader new to braided operads should refer to the work of Fiedorowicz \cite{fie-br}) in his work on coherence for 2-monads, and Batanin \cite{bat-eh} uses lax morphisms of operads in $\mb{Cat}$ in order to define the notion of an internal operad.  But aside from a few appearances, the basic theory of operads in $\mb{Cat}$ and their 2-categorical properties seems missing.  This paper was partly motivated by the need for such a theory to be explained from the ground up.

There were two additional motivations for the work in this paper.  In thinking about coherence for monoidal functors, the first author was led to a general study of algebras for multicategories internal to $\mb{Cat}$.  These give rise to 2-monads (or perhaps pseudomonads, depending on how the theory is set up), and checking abstract properties of these 2-monads prompts one to consider the simpler case of operads in $\mb{Cat}$ instead of multicategories.  The other motivation was from the second author's attempt to understand the interplay between operads in $\mb{Cat}$, operads in $\mb{Top}$, and the passage from (bi)permutative categories to $E_{\infty}$ (ring) spaces.  The first of these motivations raised the issue of when operads in $\mb{Cat}$ are cartesian, while the second led us to consider when an operad in $\mb{Cat}$ possesses a pseudo-commutative structure.

While considering how to best tackle a general discussion of operads in $\mb{Cat}$, it became clear that restricting attention to the two most commonly used types of operads, symmetric and non-symmetric operads, was both short-sighted and unnecessary.  Many theorems apply to both kinds of operads at once, with the difference in proofs being negligible; in fact, most of the arguments which applied to the symmetric case seemed to apply to the case of braided operads as well.   This led us to the notion of an action operad $\mb{G}$, and then to a definition of $\mb{G}$-operads.  In essence, this is merely the general notion of what it means for an operad $P = \{ P(n) \}_{n \in \N}$ to have groups of equivariance $\mb{G} = \{ G(n) \}_{n \in \N}$ such that $G(n)$ acts on $P(n)$.  Choosing different natural families of groups $\mb{G}$, we recover known variants of the definition of operad. \\ \begin{center}
\begin{tabular}{c|c}
Groups $\mb{G}$ & Type of operad  \\ \hline
Terminal groups & Non-symmetric operad \\
Symmetric groups & Symmetric operad \\
Braid groups & Braided operad \\
\end{tabular} \\ \end{center}
These definitions have appeared, with minor variations, in two sources of which we are aware.  In Wahl's thesis \cite{wahl-thesis}, the essential definitions appear but not in complete generality as she requires a surjectivity condition.  Zhang \cite{zhang-grp} also studies these notions\footnote{Zhang calls our action operad a \textit{group operad}.  We dislike this terminology as it seems to imply that we are dealing with an operad in the category of groups, which is not the case unless all of the maps $\pi_{n}:G(n) \rightarrow \Sigma_{n}$ are zero maps.}, once again in the context of homotopy theory, but requires the  superfluous condition that $e_{1} = \textrm{id}$ (see Lemma \ref{calclem}).

This paper consists of the following.  In Section 1, we give the definition of an action operad $\mb{G}$ and a $\mb{G}$-operad.  We develop this definition abstractly so as to apply it in any suitable symmetric monoidal category.   It is standard to express operads as monoids in a particular functor category using a composition tensor product.  In order to show that our $\mb{G}$-operads fit into this philosophy, we must work abstractly and use the calculus of coends together with the Day convolution product \cite{day-thesis}.  The reader uninterested in these details can happily skip them, although we find the route taken here to be quite satisfactory in justifying the axioms for an action operad $\mb{G}$ and the accompanying notion of $\mb{G}$-operad.  Many of our calculations are generalizations of those appearing in work of Kelly \cite{kelly-op}, although there are slight differences in flavor between the two treatments.
%Kelly:  On the operads of JP May

Section 2 works through the basic 2-categorical aspects of operads in $\mb{Cat}$.  We explain how every operad gives rise to a 2-monad, and show that all of the various 1-cells between algebras of the associated 2-monad correspond to the obvious sorts of 1-cells one might define between algebras over an operad in $\mb{Cat}$.  Similarly, we show that the algebra 2-cells, using the 2-monadic approach, correspond to the obvious notion of transformation one would define using the operad.

Section 3 studies three basic 2-categorical properties of an operad, namely the property of being finitary, the property of being 2-cartesian, and the coherence property.  The first of these always holds, as a simple calculation shows.  The second of these turns out to be equivalent to the action of $G(n)$ on $P(n)$ being free for all $n$, at least up to a certain kernel.  In particular, our characterization clearly shows that every non-symmetric operad is 2-cartesian, and that a symmetric operad is 2-cartesian if and only if the symmetric group actions are all free.  (It is useful to note that a 2-monad on $\mb{Cat}$ is 2-cartesian if and only if the underlying monad on the category of small categories is cartesian in the usual sense as the (strict) 2-pullback of a diagram is the same as its pullback.)  The third property is also easily shown to hold for any $\mb{G}$-operad on $\mb{Cat}$ using a factorization system argument due to Power \cite{power-gen}.

Section 4 then goes on to study the question of when a $\mb{G}$-operad $P$ admits a pseudo-commutative structure.  Such a structure provides the 2-category of algebras with a richer structure that includes well-behaved notions of tensor product, internal hom, and multilinear map that fit together much as the analogous notions do in the category of vector spaces.  When $P$ is contractible (i.e., each $P(n)$ is equivalent to the terminal category), this structure can be obtained from a collection of elements $t_{m,n} \in G(mn)$ satisfying certain properties.  In particular, we show that every contractible symmetric operad is pseudo-commutative, and we prove that there exist such elements $t_{m,n} \in Br_{mn}$ so that every contractible braided operad is pseudo-commutative as well (in fact in two canonical ways).  Thus Section 4 can be seen as a continuation, in the operadic context, of the work in \cite{HP}, and in particular the ``geometric'' proof of the existence of a pseudo-commutative structure for braided strict monoidal categories demonstrates the power of being able to change the groups of equivariance.

The authors would like to thank John Bourke, Martin Hyland, Tom Leinster, and Peter May for various conversations which led to this paper.  While conducting this research, the second author was supported by an EPSRC Early Career Fellowship. 