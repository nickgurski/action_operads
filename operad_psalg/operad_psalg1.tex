\section{Operads in symmetric monoidal categories}

In this section, we will explore the general definition of an operad $P$ which is equipped with groups of equivariance $G(n)$.  The group $G(n)$ will act on the right on the object $P(n)$, and the operad structure of $P$ will be required to respect this action.  For certain choices of the groups $G(n)$, we will recover standard notions of operads such as symmetric operads, non-symmetric operads, and braided operads.  The definitions here will, unless otherwise stated, apply in any symmetric monoidal category $\mathcal{V}$ in which the functors $X \otimes -, - \otimes X$ preserve colimits for every object $X \in \mathcal{V}$.

\begin{conv} 
We adopt the following conventions throughout.
\begin{itemize}
\item $\Sigma_{n}$ is the symmetric group on $n$ letters, and $Br_{n}$ is the braid group on $n$ strands.
\item For a group $G$, a right $G$-action on a set $X$ will be denoted $(x,g) \mapsto x \cdot g$.  We will use both $\cdot$ and concatenation to represent multiplication in a group.
\item The symbol $e$ will generically represent an identity element in a group.  If we have a set of groups $\{ G(n) \}_{n \in \N}$ indexed by the natural numbers, then $e_{n}$ is the identity element in $G(n)$.
\item We will often be interested in elements of a product of the form
\[
A \times B_{1} \times \cdots \times B_{n} \times C
\]
(or similar, for example without $C$).  We will write elements of this set as $(a; b_{1}, \ldots, b_{n}; c)$, and in the case that we need equivalence classes of such elements they will be written as $[a; b_{1}, \ldots, b_{n}; c]$.
\end{itemize}
\end{conv}

We begin with the basic definitions.

\begin{Defi}
An \textit{operad} $O$ (in the category of sets) consists of
\begin{itemize}
\item sets $O(n)$ for each natural number $n$,
\item an element $\textrm{id} \in O(1)$, and
\item functions
\[
\mu: O(n) \times O(k_{1}) \times \cdots \times O(k_{n}) \rightarrow O(k_{1} + \cdots + k_{n}),
\]
\end{itemize}
satisfying the following two axioms.
\begin{enumerate}
\item The element $\textrm{id} \in O(1)$ is a two-sided unit for $\mu$ in the sense that
\[
\begin{array}{rcl}
\mu(\textrm{id}; x) & = & x \\
\mu(x; \textrm{id}, \ldots, \textrm{id}) & = & x
\end{array}
\]
for any $x \in O(n)$.
\item The functions $\mu$ (called operadic multiplication or operadic composition) are associative in the sense that the diagram below commutes.
\[
\xy
% var defs
% 00    20
% 	    21
% 02    22
(0,0)*+{\scriptstyle O(n) \times O(k_{1}) \times \cdots \times O(k_{n}) \times O(l_{1,1}) \times \cdots \times O(l_{1},k_{1}) \times \cdots \times O(l_{n,1}) \times \cdots \times O(l_{n},k_{n})} ="00";
(0,-50)*+{\scriptstyle O(k_{1} + \cdots + k_{n}) \times O(l_{1,1}) \times \cdots \times O(l_{1},k_{1}) \times \cdots \times O(l_{n,1})\times \cdots \times O(l_{n},k_{n})} ="02";
(55,-10)*+{\scriptstyle O(n) \times \prod_{i=1}^n O(k_{1}) \times O(l_{i,1}) \times \cdots \times O(l_{i}, k_{i}) } ="20";
(55,-25)*+{\scriptstyle O(n) \times O(\sum l_{1,-}) \times \cdots \times O(\sum l_{n,-})} ="21";
(55, -40)*+{\scriptstyle  O(\sum l_{-,-})} ="22";
% diagram
{\ar_{\scriptstyle \mu \times 1} "00" ; "02"};
{\ar_{\mu} "02" ; "22"};
{\ar^{\cong} "00" ; "20"};
{\ar^{1 \times \prod \mu} "20" ; "21"};
{\ar^{\mu} "21" ; "22"};
\endxy
\]
\end{enumerate}
\end{Defi}


\begin{rem}
\begin{enumerate}
\item One can change from operads in $\mb{Sets}$ to operads in another symmetric monoidal category $\mathcal{V}$ by requiring each $O(n)$ to be an object of $\mathcal{V}$ and replacing all instances of cartesian product with the appropriate tensor product in $\mathcal{V}$.  This includes replacing the element $\textrm{id} \in O(1)$ with a map $I \rightarrow O(1)$ from the unit object of $\mathcal{V}$ to $O(1)$.
\item Every operad has an underlying \textit{collection} which consists of the natural number-indexed set $\{ O(n) \}_{n \in \N}$, but without a chosen identity element or composition maps.  The category of collections is a presheaf category (in this case on the discrete category of natural numbers), and we will equip it with a monoidal structure in which monoids are precisely operads in Theorem \ref{operad=monoid}.
\end{enumerate}
\end{rem}

One is intended to think that $x \in O(n)$ is a function with $n$ inputs and a single output, as below.
\[
\xy
{\ar@{-} (0,0)*{}; (25,-10)*{} };
{\ar@{-} (0,-20)*{}; (25,-10)*{} };
{\ar@{-} (0,0)*{}; (0,-20)*{} };
{\ar@{-} (25,-10)*{}; (35,-10)*{} };
{\ar@{-} (0,0)*{}; (-10,0)*{} };
{\ar@{-} (0,-3)*{}; (-10,-3)*{} };
{\ar@{-} (0,-17)*{}; (-10,-17)*{} };
{\ar@{-} (0,-20)*{}; (-10,-20)*{} };
(11,-10)*{x}; (-5,-10)*{\vdots}
\endxy
\]
Operadic composition is then a generalization of function composition, with the pictorial representation below being $\mu(x; y_{1}, y_{2})$ for $\mu:O(2) \times O(2) \times O(3) \rightarrow O(5)$.
\[
\xy
{\ar@{-} (0,0)*{}; (25,-10)*{} };
{\ar@{-} (0,-20)*{}; (25,-10)*{} };
{\ar@{-} (0,0)*{}; (0,-20)*{} };
{\ar@{-} (25,-10)*{}; (35,-10)*{} };
{\ar@{-} (0,-3)*{}; (-10,-3)*{} };
{\ar@{-} (0,-17)*{}; (-10,-17)*{} };
(11,-10)*{x};
{\ar@{-} (-25,2)*{}; (-10,-3)*{} };
{\ar@{-} (-25,-8)*{}; (-10,-3)*{} };
{\ar@{-} (-25,2)*{}; (-25,-8)*{} };
{\ar@{-} (-25,1)*{}; (-30,1)*{} };
{\ar@{-} (-30,-7)*{}; (-25,-7)*{} };
(-19,-3)*{y_{1}};
{\ar@{-} (-25,-12)*{}; (-10,-17)*{} };
{\ar@{-} (-25,-22)*{}; (-10,-17)*{} };
{\ar@{-} (-25,-12)*{}; (-25,-22)*{} };
{\ar@{-} (-25,-13)*{}; (-30,-13)*{} };
{\ar@{-} (-25,-17)*{}; (-30,-17)*{} };
{\ar@{-} (-25,-21)*{}; (-30,-21)*{} };
(-19,-17)*{y_{2}};
\endxy
\]

\begin{example}
The canonical example of an operad is the \textit{symmetric operad} which we write as $\Sigma$.  The set $\Sigma(n)$ is the set of elements of the symmetric group $\Sigma_{n}$.  The identity element $\textrm{id} \in \Sigma(1)$ is just the identity permutation on a one-element set.  Operadic composition in $\Sigma$ will then be given by a function
\[
\Sigma(n) \times \Sigma(k_{1}) \times \cdots \times \Sigma(k_{n}) \rightarrow \Sigma(k_{1} + \cdots + k_{n})
\]
which takes permutations $\sigma \in \Sigma_{n}, \tau_{i} \in \Sigma_{k_{i}}$ and produces the following permutation in $\Sigma_{k_{1} + \cdots + k_{n}}$.  First we form the block sum permutation $\tau_{1} \oplus \cdots \oplus \tau_{n}$ which permutes the first $k_{1}$ elements according to $\tau_{1}$, the next $k_{2}$ elements according to $\tau_{2}$ and so on; this is an element of $\Sigma_{k_{1} + \cdots + k_{n}}$.  Then we take the permutation $\sigma^+ \in \Sigma_{k_{1} + \cdots + k_{n}}$ which permutes the $n$ different blocks $1$ through $k_{1}$, $k_{1}+1$ through $k_{1} + k_{2}$, and so on, according to the permutation $\sigma \in \Sigma_{n}$.  Operadic composition in $\Sigma$ is then given by the formula
\[
\mu(\sigma; \tau_{1}, \ldots, \tau_{n}) = \sigma^+ \cdot (\tau_{1} \oplus \cdots \oplus \tau_{n}).
\]
Below we have drawn the permutation for the composition
\[
\mu:\Sigma(3) \times \Sigma(2) \times \Sigma(4) \times \Sigma(3) \rightarrow \Sigma(9)
\]
evaluated on the element $\Big( (123); (12), (12)(34), (13) \Big)$.
\[
\xy
{\ar@{-} (0,0)*{}; (5,-5)*{} };
{\ar@{-} (5,0)*{}; (0,-5)*{} };
{\ar@{-} (12,0)*{}; (17,-5)*{} };
{\ar@{-} (17,0)*{}; (12,-5)*{} };
{\ar@{-} (22,0)*{}; (27,-5)*{} };
{\ar@{-} (27,0)*{}; (22,-5)*{} };
{\ar@{-} (34,0)*{}; (44,-5)*{} };
{\ar@{-} (39,0)*{}; (39,-5)*{} };
{\ar@{-} (44,0)*{}; (34,-5)*{} };
{\ar@{-} (0,-5)*{}; (17,-13)*{} };
{\ar@{-} (5,-5)*{}; (22,-13)*{} };
{\ar@{-} (12,-5)*{}; (29,-13)*{} };
{\ar@{-} (17,-5)*{}; (34,-13)*{} };
{\ar@{-} (22,-5)*{}; (39,-13)*{} };
{\ar@{-} (27,-5)*{}; (44,-13)*{} };
{\ar@{-} (34,-5)*{}; (0,-13)*{} };
{\ar@{-} (39,-5)*{}; (5,-13)*{} };
{\ar@{-} (44,-5)*{}; (10,-13)*{} };
\endxy
\]
Note that $(12)(34) \in \Sigma(4)$ is actually $\mu(e_{2}; (12), (12))$, where $e_{2} \in \Sigma_{2}$ is the identity permutation.  Using this and operad associativity, one can easily check that
\[
\mu \Big( (123); (12), (12)(34), (13) \Big) = \mu \Big( (1234); (12), (12), (12), (13) \Big),
\]
where now the composition on the right side uses the function
\[
\mu:\Sigma(4) \times \Sigma(2) \times \Sigma(2) \times \Sigma(2) \times \Sigma(3) \rightarrow \Sigma(9).
\]
This equality is obvious using the picture above, but verifiable directly using only the algebra of the symmetric operad.
\end{example}

The definition we have given above is for what some might call a \textit{plain} or \textit{non-symmetric operad}.  In many applications, something more sophisticated is required.

\begin{Defi}
A \textit{symmetric operad} consists of
\begin{itemize}
\item an operad $O$ and
\item for each $n$, a right $\Sigma_{n}$-action on $O(n)$,
\end{itemize}
satisfying the following axioms.
\[
\begin{array}{rcl}
\mu(x; y_{1} \cdot \tau_{1}, \ldots, y_{n} \cdot \tau_{n}) & = & \mu(x; y_{1}, \ldots, y_{n}) \cdot (\tau_{1} \oplus \cdots \oplus \tau_{n}) \\
\mu(x \cdot \sigma; y_{1}, \ldots, y_{n}) & =  & \mu(x; y_{\sigma^{-1}(1)}, \ldots, y_{\sigma^{-1}(n)}) \cdot \sigma^{+}
\end{array}
\]
For the above equations to make sense, we must have
\begin{itemize}
\item $x \in O(n)$,
\item $y_{i} \in O(k_{i})$ for $i=1, \ldots, n$,
\item $\tau_{i} \in \Sigma_{k_{i}}$, and
\item $\sigma \in \Sigma_{n}$.
\end{itemize}
\end{Defi}

The operad $\Sigma$ detailed above is a symmetric operad, with the symmetric group action being given by right multiplication.  We leave it to the reader to check the axioms, but in each case they are entirely straightforward.  In the original topological applications \cite{maygeom}, symmetric operads were the central figures while plain operads were generally not as useful.  A further kind of operad was studied by Fiedorowicz in \cite{fie-br}; we give the definition below in analogy with that for symmetric operads, with interpretation to follow afterwards to make it entirely rigorous.  We do this to emphasize the key features that we will generalize in Definition \ref{aoperad}.

\begin{Defi}\label{broperad}
A \textit{braided operad} consists of
\begin{itemize}
\item an operad $O$ and
\item for each $n$, a right action of the $n$th braid group $Br_{n}$ on $O(n)$,
\end{itemize}
satisfying the following axioms.
\[
\begin{array}{rcl}
\mu(x; y_{1} \cdot \tau_{1}, \ldots, y_{n} \cdot \tau_{n}) & = & \mu(x; y_{1}, \ldots, y_{n}) \cdot (\tau_{1} \oplus \cdots \oplus \tau_{n}) \\
\mu(x \cdot \sigma; y_{1}, \ldots, y_{n}) & =  & \mu(x; y_{\sigma^{-1}(1)}, \ldots, y_{\sigma^{-1}(n)}) \cdot \sigma^{+}
\end{array}
\]
For the above equations to make sense, we must have
\begin{itemize}
\item $x \in O(n)$,
\item $y_{i} \in O(k_{i})$ for $i=1, \ldots, n$,
\item $\tau_{i} \in Br_{k_{i}}$, and
\item $\sigma \in Br_{n}$.
\end{itemize}
\end{Defi}

In order to make sense of this definition, we must define $\tau_{1} \oplus \cdots \oplus \tau_{n}$ and $\sigma^{+}$ in the context of braids.  The first is the block sum in the obvious sense:  given $n$ different braids on $k_{1}, \ldots, k_{n}$ strands, respectively, we form a new braid on $k_{1} + \cdots + k_{n}$ strands by taking a disjoint union where the braid $\tau_{i}$ is to the left of $\tau_{j}$ if $i < j$.  The braid $\sigma^{+}$ is obtained by replacing the $i$th strand with $k_{i}$ consecutive strands, all of which are braided together according to $\sigma$.  Finally, the notation $\sigma^{-1}(i)$ should be read as $\pi(\sigma)^{-1}(i)$, where $\pi:Br_{n} \rightarrow \Sigma_{n}$ is the underlying permutation map.

We require one final preparatory definition.

\begin{Defi}
Let $O,O'$ be operads.  Then an \textit{operad map} $f:O \rightarrow O'$ consists of functions $f_{n}:O(n) \rightarrow O'(n)$ for each natural number such that the following axioms hold.
\[
\begin{array}{rcl}
f(\textrm{id}_{O}) & = & \textrm{id}_{O'} \\
f(\mu^{O}(x; y_{1}, \ldots, y_{n})) & = & \mu^{O'}(f(x); f(y_{1}), \ldots, f(y_{n}))
\end{array}
\]
\end{Defi}

\begin{conv}
In the above definition and below, we adopt the convention that if an equation requires using operadic composition in more than one operad, we will indicate this by a superscript on each instance of $\mu$ unless it is entirely clear from context.
\end{conv}

\begin{example}
One can form an operad $Br$ where $Br(n)$ is the underlying set of the $n$th braid group, $Br_{n}$.  This is done in much the same way as we did for the symmetric operad, and the collection of maps $\pi_{n}:Br_{n} \rightarrow \Sigma_{n}$ giving the underlying permutations constitutes an operad map $Br \rightarrow \Sigma$.
\end{example}

One should note that the axioms for symmetric and braided operads each use the fact that the groups of equivariance themselves form an operad.  This is what we call an action operad.


\begin{Defi}\label{aoperad}
An \textit{action operad} $\mb{G}$ consists of
\begin{itemize}
\item an operad $G = \{ G(n) \}_{n \in \N}$ in the category of sets such that each $G(n)$ is equipped with the structure of a group and
\item a map $\pi:G \rightarrow \Sigma$ which is simultaneously a map of operads and a group homomorphism $\pi_{n}:G(n) \rightarrow \Sigma_{n}$ for each $n$
\end{itemize}
such that
\[
\mu(g; f_1, \ldots f_n)  \mu(g'; f_1', \ldots, f_n') = \mu (gg'; f_{\pi(g')(1)}f_{1}', \ldots, f_{\pi(g')(n)}f_{n}')
\]
holds in the group $G(k_{1} + \cdots + k_{n})$, provided both sides make sense.  This occurs precisely when
\begin{itemize}
\item $g, g' \in G(n)$,
\item $f_{i} \in G(k_{\pi(g')^{-1}(i)})$, and
\item $f_{i}' \in G(k_{i})$.
\end{itemize}
%one additional axiom holds.  Write
%\[
%\mu: G(n) \times G(k_{1}) \times \cdots \times G(k_{n}) \rightarrow G(k_{1} + \cdots + k_{n})
%\]
%for the multiplication in the operad $G$.  Let $(g; f_1, \ldots f_n)$ be an element of the product $G(n) \times G(k_{\pi(g')^{-1}(1)}) \times \cdots \times G(k_{\pi(g')^{-1}(1)})$ and $(g'; f_1', \ldots, f_n')$ be an element of the product $G(n) \times G(k_{1}) \times \cdots \times G(k_{n})$.  We require that
%\[
%\mu(g; f_1, \ldots f_n)  \mu(g'; f_1', \ldots, f_n') = \mu (gg'; f_{\pi(g')(1)}f_{1}', \ldots, f_{\pi(g')(n)}f_{n}')
%\]
\end{Defi}


\begin{rem}
\begin{itemize}
\item The final axiom is best explained using the operad $\Sigma$ of symmetric groups.  Reading symmetric group elements as permutations from top to bottom, below is a pictorial representation of the final axiom for the map $\mu:\Sigma_{3} \times \Sigma_{2} \times \Sigma_{2} \times \Sigma_{2} \rightarrow \Sigma_{6}.$
\[
\xy
{\ar@{-} (0,0)*{}; (5,-5)*{} };
{\ar@{-} (5,-5)*{}; (29,-10)*{} };
{\ar@{-} (5,0)*{}; (0,-5)*{} };
{\ar@{-} (0,-5)*{}; (24,-10)*{} };
{\ar@{-} (12,0)*{}; (12,-5)*{} };
{\ar@{-} (12,-5)*{}; (0,-10)*{} };
{\ar@{-} (17,0)*{}; (17,-5)*{} };
{\ar@{-} (17,-5)*{}; (5,-10)*{} };
{\ar@{-} (24,0)*{}; (29,-5)*{} };
{\ar@{-} (29,-5)*{}; (17,-10)*{} };
{\ar@{-} (29,0)*{}; (24,-5)*{} };
{\ar@{-} (24,-5)*{}; (12,-10)*{} };
{\ar@{-} (0,-10)*{}; (5,-15)*{} };
{\ar@{-} (5,-10)*{}; (0,-15)*{} };
{\ar@{-} (12,-10)*{}; (17,-15)*{} };
{\ar@{-} (17,-10)*{}; (12,-15)*{} };
{\ar@{-} (24,-10)*{}; (24,-15)*{} };
{\ar@{-} (29,-10)*{}; (29,-15)*{} };
{\ar@{-} (0,-15)*{}; (0,-20)*{} };
{\ar@{-} (5,-15)*{}; (5,-20)*{} };
{\ar@{-} (12,-15)*{}; (24,-20)*{} };
{\ar@{-} (17,-15)*{}; (29,-20)*{} };
{\ar@{-} (24,-15)*{}; (12,-20)*{} };
{\ar@{-} (29,-15)*{}; (17,-20)*{} };
{\ar@{-} (40,0)*{}; (45,-5)*{} };
{\ar@{-} (45,0)*{}; (40,-5)*{} };
{\ar@{-} (52,0)*{}; (52,-5)*{} };
{\ar@{-} (57,0)*{}; (57,-5)*{} };
{\ar@{-} (64,0)*{}; (69,-5)*{} };
{\ar@{-} (69,0)*{}; (64,-5)*{} };
{\ar@{-} (40,-5)*{}; (40,-10)*{} };
{\ar@{-} (45,-5)*{}; (45,-10)*{} };
{\ar@{-} (52,-5)*{}; (57,-10)*{} };
{\ar@{-} (57,-5)*{}; (52,-10)*{} };
{\ar@{-} (64,-5)*{}; (69,-10)*{} };
{\ar@{-} (69,-5)*{}; (64,-10)*{} };
{\ar@{-} (40,-10)*{}; (64,-15)*{} };
{\ar@{-} (45,-10)*{}; (69,-15)*{} };
{\ar@{-} (52,-10)*{}; (40,-15)*{} };
{\ar@{-} (57,-10)*{}; (45,-15)*{} };
{\ar@{-} (64,-10)*{}; (52,-15)*{} };
{\ar@{-} (69,-10)*{}; (57,-15)*{} };
{\ar@{-} (40,-15)*{}; (40,-20)*{} };
{\ar@{-} (45,-15)*{}; (45,-20)*{} };
{\ar@{-} (52,-15)*{}; (64,-20)*{} };
{\ar@{-} (57,-15)*{}; (69,-20)*{} };
{\ar@{-} (64,-15)*{}; (52,-20)*{} };
{\ar@{-} (69,-15)*{}; (57,-20)*{} };
(34.5,-10)*+{=};
(4.5,-25)*{\scriptstyle \mu\big((23);(12),(12), \textrm{id}\big) \cdot \mu\big((132); (12), \textrm{id}, (12)\big) };
(64.5,-25)*{\scriptstyle \mu\big((23)\cdot (132); \textrm{id} \cdot (12), (12) \cdot \textrm{id}, (12) \cdot (12)\big)} ;
\endxy
\]
\item Our definition of an action operad is the same as that appearing in Wahl's thesis \cite{wahl-thesis}, but without the condition that each $\pi_{n}$ is surjective.  It is also the same as that appearing in work of Zhang \cite{zhang-grp}, although we prove later (see Lemma \ref{calclem}) that the condition $e_{1} = \textrm{id}$ in Zhang's definition follows from the rest of the axioms.
\end{itemize}
\end{rem}

\begin{example}
\begin{enumerate}
\item There are two trivial examples of action operads.  The first is the symmetric operad  $\mb{G} = \mb{\Sigma}$ with the identity map; this is the terminal object in the category of action operads (see Definition \ref{mapaop}).  The second is $\mb{G} = \mb{T}$ consisting of the terminal groups $T(n) = *$.  Here the $\pi_{n}$'s are given by the inclusion of identity elements; this is the initial object in the category of action operads.
\item Two less trivial examples are given by the braid groups, $\mb{G} = \mb{Br}$, and the ribbon braid groups, $\mb{G} = \mb{RBr}$.  (A ribbon braid is given, geometrically, as a braid with strands replaced by ribbons in which we allow full twists.  The actual definition of the ribbon braid groups is as the fundamental group of a configuration space in which points have labels in the circle, $S^{1}$; see \cite{sal-wahl}.)  In each case, the homomorphism $\pi$ is given by taking underlying permutations, and the operad structure is given geometrically by using the procedure explained after Definition \ref{broperad}.  We refer the reader to \cite{fie-br} for more information about braided operads, and to \cite{sal-wahl, wahl-thesis} for information about the ribbon case.
\end{enumerate}
\end{example}

Action operads are themselves the objects of a category, $\mb{AOp}$.  The morphisms of this category are defined below.
\begin{Defi}\label{mapaop}
A \textit{map of action operads} $f: \mb{G} \rightarrow \mb{G}'$ consists of a map $f:G \rightarrow G'$ of the underlying operads such that
\begin{enumerate}
\item $\pi^{G'} \circ f = \pi^{G}$ (i.e., $f$ is a map of operads over $\Sigma$) and
\item each $f_{n}:G(n) \rightarrow G'(n)$ is a group homomorphism.
\end{enumerate}
\end{Defi}


%Salvatore-Wahl:  Framed disks operads...

Just as we had the definitions of operad, symmetric operad, and braided operad, we now come to the general definition of a $\mb{G}$-operad, where $\mb{G}$ is an action operad.

\begin{Defi}
Let $\mb{G}$ be an action operad.  A \textit{$\mb{G}$-operad} $P$ (in $\mb{Sets}$) consists of
\begin{itemize}
\item an operad $P$ in $\mb{Sets}$ and
\item for each $n$, an action $P(n) \times G(n) \rightarrow P(n)$ of $G(n)$ on $P(n)$
\end{itemize}
such that the following two equivariance axioms hold.
\[
\begin{array}{c}
\mu^{P}(x; y_{1} \cdot g_{1}, \ldots, y_{n} \cdot g_{n}) =\mu^{P}(x; y_{1}, \ldots, y_{n}) \cdot \mu^{G}(e; g_{1}, \ldots, g_{n})  \\
\mu^{P}(x \cdot g; y_{1}, \ldots, y_{n})  =  \mu^{P}(x; y_{\pi(g)^{-1}(1)}, \ldots, y_{\pi(g)^{-1}(n)}) \cdot \mu^{G}(g; e_{1}, \ldots, e_{n})
\end{array}
\]
\end{Defi}

\begin{example}
\begin{enumerate}
\item Let $\mb{T}$ denote the terminal operad in $\mb{Sets}$ equipped with its unique action operad structure.  Then a $\mb{T}$-operad is just a non-symmetric operad in $\mb{Sets}$.
\item Let $\mb{\Sigma}$ denote the operad of symmetric groups with $\pi:\Sigma \rightarrow \Sigma$ the identity map.  Then a $\mb{\Sigma}$-operad is a symmetric operad in the category of sets.
\item Let $\mb{Br}$ denote the operad of braid groups with $\pi_{n}:Br_{n} \rightarrow \Sigma_{n}$ the canonical projection of a braid onto its underlying permutation.  Then a $\mb{Br}$-operad is a braided operad in the sense of Fiedorowicz \cite{fie-br}.
\end{enumerate}
\end{example}

\begin{rem}
It is possible to consider $\mb{G}$-operads in categories other than the category of sets.  In this case we still use the notion of an action operad given above, but then take the operad $P$ to have objects $P(n)$ which are the objects of some closed symmetric monoidal category $\mathcal{V}$.  We will rarely use anything that might require the closed structure as such, only the fact that the tensor product distributes over colimits in each variable.  This is a consequence of the fact that both $X \otimes -$ and $- \otimes X$ are left adjoints in the case of a closed symmetric monoidal category.  Thus while we set up the foundations using only operads in $\mb{Sets}$, the diligent reader can easily modify this theory for their closed symmetric monoidal category of choice.  In fact, we will use the same theory in $\mb{Cat}$ with its cartesian structure, noting only that the same arguments work in $\mb{Cat}$ with essentially no modification.
\end{rem}

\begin{Defi}
Let $\mb{G}$ be an action operad.  The category $\mb{G\mbox{-}Coll}$ of $\mb{G}$-collections has objects $X = \{ X(n) \}_{n \in \N}$ which consist of a set $X(n)$ for each natural number $n$ together with an action $X(n) \times G(n) \rightarrow X(n)$ of $G(n)$ on $X(n)$.  A morphism $f:X \rightarrow Y$ in $\mb{G\mbox{-}Coll}$ consists of a $G(n)$-equivariant map $f_{n}:X(n) \rightarrow Y(n)$ for each natural number $n$.
\end{Defi}

\begin{rem}
The definition of $\mb{G\mbox{-}Coll}$ does not require that $\mb{G}$ be an action operad, only that one has a natural number-indexed set of groups.  Given any such collection of groups $\{ G(n) \}_{n \in \N}$, we can form the category $\mathbb{G}$ whose objects are natural numbers and whose hom-sets are given by $\mathbb{G}(m,n) = \emptyset$ if $m \neq n$ and $\mathbb{G}(n,n) = G(n)$ (where composition and units are given by group multiplication and identity elements, respectively).  Then $\mb{G\mbox{-}Coll}$ is the presheaf category
\[
\hat{\mathbb{G}} = [\mathbb{G}^{\textrm{op}}, \mb{Sets}],
 \]
with the opposite category arising from our choice of right actions.  A key step in explaining how $\mb{G}$-operads arise as monoids in the category of $\mb{G}$-collections is to show that being an action operad endows $\mathbb{G}$ with a monoidal structure.
\end{rem}

\begin{Defi}
Let $\mb{G}$ be an action operad, and let $X, Y$ be $\mb{G}$-collections.  We define the $\mb{G}$-collection $X \circ Y$ to be
\[
X \circ Y (n) = \Big( \coprod_{k_{1} + \cdots + k_{r} = n} X(r) \times Y(k_{1}) \times \cdots \times Y(k_{r}) \Big) \times G(n) / \sim
\]
where the equivalence relation is generated by
\[
\begin{array}{rcl}
(xh; y_{1}, \ldots, y_{r}; g) & \sim & (x; y_{\pi(h)^{-1}(1)}, \ldots, y_{\pi(h)^{-1}(r)}; \mu(h; e, \ldots, e)g), \\
(x; y_{1}, \ldots, y_{r}; \mu(e; g_{1}, \ldots, g_{r})g) & \sim & (xe; y_{1}g_{1}, \ldots, y_{r}g_{r}; g).
\end{array}
\]
For the first relation above, we must have that the lefthand side is an element of
\[
X(r) \times Y(k_1) \times \cdots \times Y(k_r) \times G(n)\]
while the righthand side is an element of
\[
X(r) \times Y(k_{\pi(h)^{-1}(1)}) \times \cdots \times Y(k_{\pi(h)^{-1}(r)}) \times G(n);
\]
for the second relation, we must have $x \in X(r)$, $y_{i} \in Y(k_{i})$, $f \in G(r)$, $g_{i} \in G(k_{i})$, and $g \in G(n)$.  The right $G(n)$-action on $X \circ Y(n)$ is given by multiplication on the final coordinate.
\end{Defi}


We will now develop the tools to prove that the category $\mb{G\mbox{-}Coll}$ has a monoidal structure given by $\circ$, and that operads are the monoids therein.

\begin{thm}\label{operad=monoid}
Let $\mb{G}$ be an action operad.
\begin{enumerate}
\item The category $\mb{G\mbox{-}Coll}$ has a monoidal structure with tensor product given by $\circ$ and unit given by the collection $I$ with $I(n) = \emptyset$ when $n \neq 1$, and $I(1) = G(1)$ with the $\mb{G}$-action given by multiplication on the right.
\item The category $\mb{Mon}(\mb{G\mbox{-}Coll})$ of monoids in $\mb{G\mbox{-}Coll}$ is equivalent to the category of $\mb{G}$-operads with morphisms being those operad maps $P \rightarrow Q$ for which each $P(n) \rightarrow Q(n)$ is $G(n)$-equivariant.
\end{enumerate}
\end{thm}

While this theorem can be proven by direct calculation using the equivalence relation given above, such a proof is unenlightening.  Furthermore, we want to consider $\mb{G}$-operads in categories other than sets, so an element-wise proof might not apply.  Instead we now develop some general machinery that will apply to $\mb{G}$-operads in any cocomplete symmetric monoidal category in which each of  the functors $X \otimes -, - \otimes X$ preserve colimits (as is the case if the monoidal structure is closed).  This theory also demonstrates the importance of the final axiom in the definition of an action operad.  We begin with a calculational lemma.

\begin{lem}\label{calclem}
Let $\mb{G}$ be an action operad, and write $e_{n}$ for the unit element in the group $G(n)$.
\begin{enumerate}
\item In $G(1)$, the unit element $e_{1}$ for the group structure is equal to the identity element for the operad structure, $\textrm{id}$.
\item The equation
\[
\mu(e_{n}; e_{i_{1}}, \ldots, e_{i_{n}}) = e_{I}
\]
holds for any natural numbers $n, i_{j}, I = \sum i_{j}$.
\item The group $G(1)$ is abelian.
\end{enumerate}
\end{lem}
\begin{proof}
For the first claim, let $g \in G(1)$.  Then
\[
\begin{array}{rcl}
g & = & g \cdot e_{1} \\
& = & \mu(g; \textrm{id}) \cdot \mu(\textrm{id}; e_{1}) \\
& = & \mu(g \cdot \textrm{id}; \textrm{id} \cdot e_{1}) \\
& = & \mu(g \cdot \textrm{id}; \textrm{id}) \\
& = & g \cdot \textrm{id}
\end{array}
\]
using that $e_{1}$ is the unit element for the group structure, that $\textrm{id}$ is a two-sided unit for operad multiplication, and the final axiom for an action operad together with the fact that the only element of the symmetric group $\Sigma_{1}$ is the identity permutation.  Thus $g = g \cdot \textrm{id}$, so $\textrm{id} = e_{1}$.

For the second claim, write the operadic product as $\mu(e; \underline{e})$, and consider the square of this element. We have
\[
\begin{array}{rcl}
\mu(e; \underline{e}) \cdot \mu(e; \underline{e}) & = & \mu(e \cdot e; \underline{e} \cdot \underline{e}) \\
&= & \mu(e; \underline{e})
\end{array}
\]
where the first equality follows from the last action operad axiom together with the fact that $e$ gets mapped to the identity permutation; here $\underline{e} \cdot \underline{e}$ is the sequence $e_{i_{1}} \cdot e_{i_{1}}, \ldots, e_{i_{n}} \cdot e_{i_{n}}$.  Thus $\mu(e; \underline{e})$ is an idempotent element of the group $G(I)$, so must be the identity element $e_{I}$.

For the final claim, note that operadic multiplication $\mu:G(1) \times G(1) \rightarrow G(1)$ is a group homomorphism by the action operad axioms, and $\textrm{id} = e_{1}$ is a two-sided unit, so the Eckmann-Hilton argument shows that $\mu$ is actually group multiplication and that $G_{1}$ is abelian.
\end{proof}

Our construction of the monoidal structure on the category of $\mb{G}$-collections will require the Day convolution product \cite{day-thesis}.  This is a general construction which produces a monoidal structure on the category of presheaves $[\mathcal{V}^{\textrm{op}}, \mb{Sets}]$ from a monoidal structure on the category $\mathcal{V}$.  Since the category of $\mb{G}$-collections is the presheaf category $[\mathbb{G}^{\textrm{op}}, \mb{Sets}]$, we need to show that $\mathbb{G}$ has a monoidal structure.

\begin{prop}\label{Gmonoidal}
The action operad structure of $\mb{G}$ gives $\mathbb{G}$ a strict monoidal structure.
\end{prop}
\begin{proof}
The tensor product on $\mathbb{G}$ is given by addition on objects, with unit object 0.  The only thing to do is define the tensor product on morphisms and check naturality for the associativity and unit isomorphisms, which will both be identities.  On morphisms, $+$ must be given by a group homomorphism
\[
+:G(n) \times G(m) \rightarrow G(n+m),
\]
 and this is given by the formula
\[
+(g,h) = \mu(e_{2}; g,h).
\]
We need that $+$ is a group homomorphism, and the second part of Lemma \ref{calclem} shows that it preserves identity elements.  The final action operad axiom shows that it also preserves group multiplication since $\pi_{2}(e_{2}) = e_{2}$ (each $\pi_{n}$ is a group homomorphism) and therefore
\[
\begin{array}{rcl}
\Big(+(g,h)\Big) \cdot \Big(+(g',h')\Big) & = & \mu(e_{2}; g,h) \cdot \mu(e_{2}; g',h') \\
 & = & \mu(e_{2}e_{2}; gg', hh') \\
& = & +(gg',hh').
\end{array}
\]
We now write $+(g,h)$ as $g+h$.

For naturality of the associator, we must have $(f+g)+h = f+(g+h)$.  By the operad axioms for both units and associativity, the lefthand side is given by
\[
\begin{array}{rcl}
\mu(e_{2}; \mu(e_{2}; f,g), h) & = & \mu(e_{2}; \mu(e_{2}; f,g), \mu(\textrm{id};h)) \\
& = & \mu(\mu(e_{2}; e_{2}, \textrm{id}); f,g,h),
\end{array}
\]
while the righthand side is then
\[
\mu(e_{2}; f, \mu(e_{2}; g,h)) = \mu(\mu(e_{2}; \textrm{id}, e_{2}); f,g,h).
\]
By Lemma \ref{calclem}, both of these are equal to $\mu(e_{3}; f,g,h)$, proving associativity.  Naturality of the unit follows similarly, using $e_{0}$.
\end{proof}

Now that $\mathbb{G}$ has a monoidal structure, we get a monoidal structure on the category of $\mathbb{G}$-collections
\[
[\mathbb{G}^{\textrm{op}}, \mb{Sets}] = \hat{\mathbb{G}}
\]
using Day convolution, denoted $\star$.  Given collections $X, Y$, their convolution product $X \star Y$ is given by the coend formula
\[
X \star Y (k) = \int^{m,n \in \mathbb{G}} X(m) \times Y(n) \times \mathbb{G}(k, m+n)
\]
We refer the reader to \cite{day-thesis} for further details.  We do note, however, that the $n$-fold Day convolution product of a presheaf $Y$ with itself is given by the following coend formula.
\[
Y^{\star n}(k) = \int^{(k_{1}, \ldots, k_{n}) \in \mathbb{G}^{n}} Y(k_{1}) \times \cdots \times Y(k_{n}) \times \mathbb{G}(k, k_{1} + \cdots + k_{n})
\]
Computations with Day convolution will necessarily involve heavy use of the calculus of coends, and we refer the reader in need of a refresher course on coends to \cite{maclane-catwork}.  Our goal is to express the substitution tensor product as a coend just as in \cite{kelly-op}, and to do that we need one final result about the Day convolution product.

\begin{lem}\label{calclem2}
Let $\mb{G}$ be an action operad, let $Y \in \hat{\mathbb{G}}$, and let $k$ be a fixed natural number.  Then the assignment
\[
n \mapsto Y^{\star n}(k)
\]
can be given the structure of a functor $\mathbb{G} \rightarrow \mb{Sets}$.
\end{lem}
\begin{proof}
Since the convolution product is given by a coend, it is the universal object with maps
\[
Y(k_{1}) \times \cdots \times Y(k_{n}) \times \mathbb{G}(k, k_{1} + \cdots + k_{n}) \rightarrow Y^{\star n}(k)
\]
such that the following diagram commutes for every $g_{1} \in G(k_{1}), \ldots, g_{n} \in G(k_{n})$.
\[
\xy
{\ar   (0,0)*+{Y(k_{1}) \times \cdots \times Y(k_{n}) \times \mathbb{G}(k, k_{1} + \cdots + k_{n})}; (40,15)*+{Y(k_{1}) \times \cdots \times Y(k_{n}) \times \mathbb{G}(k, k_{1} + \cdots + k_{n})} };
(9.5,10)*{\scriptstyle (-\cdot g_{1}, \ldots, -\cdot g_{n}) \times 1};
{\ar (40,15)*+{Y(k_{1}) \times \cdots \times Y(k_{n}) \times \mathbb{G}(k, k_{1} + \cdots + k_{n})}; (80,0)*+{Y^{\star n}(k)} };
{\ar (0,0)*+{Y(k_{1}) \times \cdots \times Y(k_{n}) \times \mathbb{G}(k, k_{1} + \cdots + k_{n})}; (40,-15)*+{Y(k_{1}) \times \cdots \times Y(k_{n}) \times \mathbb{G}(k, k_{1} + \cdots + k_{n})} };
(9.5,-10)*+{\scriptstyle 1 \times \big( (g_{1} + \cdots + g_{n})\cdot - \big)};
{\ar (40,-15)*+{Y(k_{1}) \times \cdots \times Y(k_{n}) \times \mathbb{G}(k, k_{1} + \cdots + k_{n})}; (80,0)*+{Y^{\star n}(k)} };
\endxy
\]
The first map along the top acts using the $g_{i}$'s, while the first map along the bottom is given by
\[
h \mapsto \mu(e_{n}; g_{1}, \ldots, g_{n}) \cdot h
\]
in the final coordinate.

Let $f \in G(n)$, considered as a morphism $n \rightarrow n$ in $\mathbb{G}$.  We induce a map $f \bullet -:Y^{\star n}(k) \rightarrow Y^{\star n}(k)$ using the collection of maps
\[
\prod_{i=1}^{n} Y(k_{i}) \times \mathbb{G}(k, k_{1} + \cdots + k_{n}) \rightarrow \prod_{i=1}^{n} Y(k_{\pi (f)^{-1}(i)}) \times \mathbb{G}(k, k_{1} + \cdots + k_{n})
\]
by using the symmetry $\pi(f)$ on the first $n$ factors and left multiplication by the element $\mu(f; e_{k_{1}}, \ldots, e_{k_{n}})$ on $\mathbb{G}(k, k_{1} + \cdots + k_{n})$.  To induce a map between the coends, we must show that these maps commute with the two lefthand maps in the diagram above.  For the top map, this is merely functoriality of the product together with naturality of the symmetry.  For the bottom map, this is the equation
\[
\mu(f; \overline{e}) \cdot \mu(e; g_{1}, \ldots, g_{n}) = \mu(e; g_{\pi (f)^{-1} 1}, \ldots, g_{\pi (f)^{-1} n}) \cdot \mu(f; \overline{e}).
\]
Both of these are equal to $\mu(f; g_{1}, \ldots, g_{n})$ by the action operad axiom.  Functoriality is then easy to check using that the maps inducing $(f_{1}f_{2}) \bullet -$ are given by the composite of the maps inducing $f_{1} \bullet (f_{2} \bullet -)$.
\end{proof}

We are now ready for the abstract description of the substitution tensor product.  The following proposition is easily checked directly using the definition of the coend; in fact, the righthand side below should be taken as the definition of $X \circ Y$ as both sides are really the result of some colimiting process.

\begin{prop}
Let $X, Y \in \hat{\mathbb{G}}$.  Then
\[
(X \circ Y)(k) \cong \int^{n} X(n) \times Y^{\star n}(k).
\]
\end{prop}

Finally we are in a position to prove Theorem \ref{operad=monoid}.  We make heavy use of the following consequence of the Yoneda lemma:  given any functor $F:\mathbb{G} \rightarrow \mb{Sets}$ and a fixed object $a \in \mathbb{G}$, we have a natural isomorphism
\[
\int^{n \in \mathbb{G}} \mathbb{G}(n,a) \times F(n) \cong F(a);
\]
there is a corresponding result for $F:\mathbb{G}^{\textrm{op}} \rightarrow \mb{Sets}$ using representables of the form $\mathbb{G}(a,n)$ instead.

\begin{proof}[Proof of \ref{operad=monoid}]
First we must show that $\mb{G}\mbox{-}\mb{Coll}$ has a monoidal structure using $\circ$.  To prove this, we must give the unit and associativity isomorphisms and then check the monoidal category axioms.  First, note that the unit object is given as $I = \mathbb{G}(-,1)$.  Then for the left unit isomorphism, we have
\[
\begin{array}{rcl}
I \circ Y (k) & \cong & \int^{n} \mathbb{G}(n,1) \times Y^{\star n}(k) \\
& \cong & Y^{\star 1}(k) \\
& \cong & Y(k)
\end{array}
\]
using only the properties of the coend.  For the right unit isomorphism, we have
\[
\begin{array}{rcl}
X \circ I (k) & \cong & \int^{n} X(n) \times I^{\star n}(k) \\
& \cong & \int^{n} X(n) \times \int^{k_{1}, \ldots, k_{n}} \mathbb{G}(k_{1},1) \times \cdots \times \mathbb{G}(k_{n},1) \times \mathbb{G}(k, k_{1} + \cdots + k_{n}) \\
& \cong & \int^{n} X(n) \times \mathbb{G}(k,1+ \cdots +1) \\
& = & \int^{n} X(n) \times \mathbb{G}(k,n) \\
& \cong & X(k)
\end{array}
\]
using the same methods.

For associativity, we compute $(X \circ Y) \circ Z$ and $X \circ (Y \circ Z)$.
\[
\begin{array}{rcl}
(X \circ Y) \circ Z (k) & = & \int^{m} X \circ Y (m) \times Z^{\star m}(k) \\
& = & \int^{m} \big( \int^{l} X(l) \times Y^{\star l}(m) \big) \times Z^{\star m}(k) \\
& \cong & \int^{m,l} X(l) \times Y^{\star l}(m) \times Z^{\star m}(k) \\
& \cong & \int^{l} X(l) \times \int^{m} Y^{\star l}(m) \times Z^{\star m}(k)
\end{array}
\]
The first isomorphism is from products distributing over colimits and hence coends, and the second is that fact plus the Fubini Theorem for coends \cite{maclane-catwork}.  A similar calculation shows
\[
X \circ (Y \circ Z)(k) \cong \int^{l} X(l) \times (Y \circ Z)^{\star l}(k).
\]
Thus the associativity isomorphism will be induced once we construct an isomorphism $\int^{m} Y^{\star l}(m) \times Z^{\star m} \cong (Y \circ Z)^{\star l}$.  We do this by induction, with the $l=1$ case being the isomorphism $Y^{\star 1} \cong Y$ together with the definition of $\circ.$  Assuming true for $l$, we prove the case for $l+1$ by the calculations below.
\[
\begin{array}{rcl}
(Y \circ Z)^{\star l+1} & \cong & (Y \circ Z) \star (Y \circ Z)^{\star l} \\
& \cong & (Y \circ Z) \star \big( \int^{m} Y^{\star l}(m) \times Z^{\star m} \big) \\
& = & \big( \int^{n} Y(n) \times Z^{\star n} \big) \star \big( \int^{m} Y^{\star l}(m) \times Z^{\star m} \big) \\
& = & \int^{a,b} \big( \int^{n} Y(n) \times Z^{\star n}(a) \big)  \times \big( \int^{m} Y^{\star l}(m) \times Z^{\star m}(b) \big) \times \mathbb{G}(-, a+b) \\
& \cong & \int^{a,b,n,m} Y(n) \times Y^{\star l}(m) \times Z^{\star n}(a) \times Z^{\star m}(b) \times  \mathbb{G}(-, a+b) \\
& \cong & \int^{n,m} Y(n) \times Y^{\star l}(m) \times Z^{\star (n+m)} \\
& \cong & \int^{j} \int^{n,m} Y(n) \times Y^{\star l}(m) \times \mathbb{G}(j, n+m) \times Z^{\star j} \\
& \cong & \int^{j} Y^{\star (l+1)}(j) \times Z^{\star j}
\end{array}
\]
Each isomorphism above arises from the symmetric monoidal structure on $\mb{Sets}$ using products, the monoidal structure on presheaves using $\star$, the properties of the coend, or the fact that products distribute over colimits.

For the monoidal category axioms on $\hat{\mathbb{G}}$, we only need to note that the unit and associativity isomorphisms arise, using the universal properties of the coend, from the unit and associativity isomorphisms on the category of sets together with the interaction between products and colimits.  Hence the monoidal category axioms follow by those same axioms in $\mb{Sets}$ together with the universal property of the coend.

Now we must show that monoids in $(\hat{\mathbb{G}}, \circ)$ are operads.  By the Yoneda lemma, a map of $\mb{G}$-collections $\eta: I \rightarrow X$ corresponds to an element $\textrm{id} \in X(1)$ since $I = \mathbb{G}(-,1)$.  A map $\mu:X \circ X \rightarrow X$ is given by a collection of $G(k)$-equivariant maps $X \circ X (k) \rightarrow X(k)$.  By the universal property of the coend, this is equivalent to giving maps
\[
\mu_{n, \underline{k}}:X(n) \times X(k_{1}) \times \cdots \times X(k_{n}) \times \mathbb{G}(k, k_{1}+\cdots +k_{n}) \rightarrow X(k)
\]
which are compatible with the actions of $G(k)$ (using the hom-set in the source, and the standard right action in the target) as well as each of $G(n), G(k_{1}), \ldots, G(k_{n})$.  The hom-set in $\mathbb{G}$ is nonempty precisely when $k=k_{1} + \cdots + k_{n}$, so we define the operad multiplication $\mu$ for $X$ to be
\[
\mu (x; y_{1}, \ldots, y_{n}) = \mu_{n, \underline{k}}(x; y_{1}, \ldots, y_{n}; e_{k}).
\]
Compatibility with the actions of the  $G(n), G(k_{1}), \ldots, G(k_{n})$ give the equivariance axioms, and the unit and associativity for the monoid structure give the unit and associativity axioms for the operad structure.  Finally, it is easy to check that a map of monoids is nothing more than an operad map which is appropriately equivariant for each $n$.
\end{proof}

\begin{rem}
The above result can be interpreted for $\mb{G}$-operads in an arbitrary cocomplete symmetric monoidal category $\mathcal{V}$ in which tensor distributes over colimits in each variable.  In order to do so, the following changes must be made.  First, cartesian products of objects $X(k)$ must be replaced by the tensor product in $\mathcal{V}$ of the same objects.  Second, any product with a hom-set from $\mathbb{G}$ must be replaced by a copower with the same set (recall that the copower of a set $S$ with an object $X$ is given by the formula $S \odot X = \coprod_{S} X$).  The same changes also allow one to interpret the results below about algebras in such a category, unless noted otherwise.
\end{rem}

An operad is intended to be an abstract description of a certain type of algebraic structure, and the particular instances of that structure are the algebras for that operad.  We begin with the definition of an algebra over a plain operad.

\begin{Defi}\label{opalgax}
Let $O$ be an operad.  An \textit{algebra} for $O$ consists of a set $X$ together with maps $\alpha_{n}:O(n) \times X^{n} \rightarrow X$ such that the following axioms hold.
\begin{enumerate}
\item The element $\textrm{id} \in O(1)$ is a unit in the sense that
\[
\alpha_{1}(\textrm{id}; x) = x
\]
for all $x \in X$.
\item The maps $\alpha_{n}$ are associative in the sense that the diagram
\[
\xy
(0,0)*+{O(n) \times O(k_{1}) \times X^{k_{1}} \times \cdots \times O(k_{n}) \times X^{k_{n}}}="ul";
(75,0)*+{O(n) \times X^{n}}="ur";
(0,-12)*+{O(n) \times O(k_{1}) \times \cdots \times O(k_{n}) \times X^{k_{1}} \times \cdots \times X^{k_{n}}}="ml";
(0,-24)*+{O(\sum k_{i}) \times X^{\sum k_{i}}}="bl";
(75,-24)*+{X}="br";
{\ar^>>>>>>>>>>>>>>{1 \times \alpha_{k_{1}} \times \cdots \alpha_{k_{n}}} "ul"; "ur"};
{\ar^{\alpha_{n}} "ur"; "br"};
{\ar_{\cong} "ul"; "ml"};
{\ar_{\mu \times 1} "ml"; "bl"};
{\ar_{\alpha_{\sum k_{i}}} "bl"; "br"};
\endxy
\]
commutes.
\end{enumerate}
\end{Defi}


Moving on to algebras for a $\mb{G}$-operad, let $P$ be a $\mb{G}$-operad and let $X$ be any set.  Write $P(n) \times_{G(n)} X^{n}$ for the coequalizer of the pair of maps
\[
P(n) \times G(n) \times X^{n} \rightrightarrows P(n) \times X^{n}
\]
of which the first map is the action of $G(n)$ on $P(n)$ and the second map is
\[
P(n) \times G(n) \times X^{n} \rightarrow P(n) \times \Sigma_{n} \times X^{n} \rightarrow P(n) \times X^{n}
\]
using $\pi_{n}:G(n) \rightarrow \Sigma_{n}$ together with the canonical action of $\Sigma_{n}$ on $X^{n}$ by permutation of coordinates: $\sigma \cdot (x_{1}, \ldots, x_{n}) = (x_{\sigma^{-1}(1)}, \ldots, x_{\sigma^{-1}(n)})$.  By the universal property of the coequalizer, a function $f:P(n) \times_{G(n)} X^{n} \rightarrow Y$ can be identified with a function $\tilde{f}:P(n) \times X^{n} \rightarrow Y$ such that
\[
\tilde{f}(p\cdot g; x_{1}, \ldots, x_{n}) = \tilde{f}(p; x_{\pi(g)^{-1}(1)}, \ldots, x_{\pi(g)^{-1}(n)}).
\]

\begin{Defi}
Let $P$ be a $\mb{G}$-operad.  An \textit{algebra} for $P$ consists of a set $X$ together with maps $\alpha_{n}:P(n) \times_{G(n)} X^{n} \rightarrow X$ such that the maps $\tilde{\alpha}_{n}$ satisfy the usual operad algebra axioms given in Definition \ref{opalgax}.
\end{Defi}

\begin{rem}
It is worth noting that the equivariance required for a $P$-algebra is built into the definition above by requiring the existence of the maps $\alpha_{n}$ to be defined on coequalizers, even though the algebra axioms then only use the maps $\tilde{\alpha}_{n}$.  Since every $\mb{G}$-operad has an underlying plain operad (see \ref{pbaop}, applied to the unique map $\mb{T} \rightarrow \mb{G}$), this reflects the fact that the algebras for the $\mb{G}$-equivariant version are always algebras for the plain version, but not conversely.
\end{rem}

\begin{Defi}
The category of algebras for $P$, $P\mbox{-}\mb{Alg}$, has objects the $P$-algebras $(X, \alpha)$ and morphisms $f: (X, \alpha) \rightarrow (Y, \beta)$ those functions $f:X \rightarrow Y$ such that the following diagram commutes for every $n$.
\[
\xy
{\ar^{1 \times f^{n}} (0,0)*+{P(n) \times X^{n}}; (50,0)*+{P(n) \times Y^{n}} };
{\ar^{\tilde{\beta}_{n}} (50,0)*+{P(n) \times Y^{n}}; (50,-15)*+{Y} };
{\ar_{\tilde{\alpha}_{n}} (0,0)*+{P(n) \times X^{n}}; (0,-15)*+{X} };
{\ar_{f} (0,-15)*+{X}; (50,-15)*+{Y} };
\endxy
\]
\end{Defi}

Let $X$ be a set.  Then the endomorphism operad of $X$, denoted $\mathcal{E}_{X}$, is given by the sets $\mathcal{E}_{X}(n) = \mb{Sets}(X^{n}, X)$ with the identity function in $\mathcal{E}_{X}(1)$ giving the unit element and composition of functions giving the composition operation.  Concretely, composition is given by the formula
\[
\mu(f; g_{1}, \ldots, g_{n}) = f \circ (g_{1} \times \cdots \times g_{n}).
\]

\begin{lem}
Let $G$ be an action operad, and let $X$ be a set.  Then $\mathcal{E}_{X}$ carries a canonical $\mb{G}$-operad structure.
\end{lem}
\begin{proof}
$\mathcal{E}_{X}$ is a symmetric operad, so we define the actions by
\[
\mathcal{E}_{X}(n) \times G(n) \stackrel{1 \times \pi_{n}}{\longrightarrow} \mathcal{E}_{X}(n) \times \Sigma_{n} \rightarrow \mathcal{E}_{X}.
\]
\end{proof}

The previous result is really a change-of-structure groups result.  We record the general result as the following proposition and note that the proof is essentially the same as that for the lemma.

\begin{prop}\label{pbaop}
Let $f:\mb{G} \rightarrow \mb{G'}$ be a map of action operads.  Then $f$ induces a functor $f^{*}$ from the category of $\mb{G'}$-operads to the category of $\mb{G}$-operads.
\end{prop}

We can now use endomorphism operads to characterize algebra structures.

\begin{prop}\label{endoalg}
Let $X$ be a set, and $P$ a $\mb{G}$-operad.  Then algebra structures on $X$ are in 1-to-1 correspondence with $\mb{G}$-operad maps $P \rightarrow \mathcal{E}_{X}$.
\end{prop}
\begin{proof}
A map $P(k) \rightarrow \mathcal{E}_{X}(k)$ corresponds, using the closed structure on $\mb{Sets}$, to a map $P(k) \times X^{k} \rightarrow X$.  The monoid homomorphism axioms give the unit and associativity axioms, and the requirement that $P \rightarrow \mathcal{E}_{X}$ be a map of $\mb{G}$-operads gives the equivariance condition.
\end{proof}

\begin{rem}
The proposition above holds for $P$-algebras in any closed symmetric monoidal category.  Having a closed structure (in addition to all small colimits) is a stronger condition than the tensor preserving colimits in each variable, but it is a natural one that arises in many examples.
\end{rem}

\begin{Defi}
Let $P$ be a $\mb{G}$-operad.  Then $P$ induces an endofunctor of $\mb{Sets}$, denoted $\underline{P}$, by the following formula.
 \[
	\underline{P}(X)	 =  \coprod_n P(n) \times_{G(n)} X^n
\]
\end{Defi}

We now have the following proposition; its proof is standard \cite{maygeom}, and we leave it to the reader.

\begin{prop}\label{op=monad1}  Let $P$ be a $\mb{G}$-operad.
\begin{enumerate}
\item The $\mb{G}$-operad structure on $P$ induces a monad structure on $\underbar{P}$.
\item The category of algebras for the operad $P$ is isomorphic to the category of algebras for the monad $\underbar{P}$.
\end{enumerate}
\end{prop} 